\documentclass{article}
\pagestyle{empty}
\setlength{\textwidth}{6in}
\setlength{\oddsidemargin}{0in}
\setlength{\topmargin}{-.5in}
\setlength{\textheight}{9in}
\def\given{\,|\,}
\def\Var{\mathop{\rm Var\,}\nolimits}
\def\Cov{\mathop{\rm Cov\,}\nolimits}
\newcommand{\beaa}{\begin{eqnarray*}}
\newcommand{\eeaa}{\end{eqnarray*}}
\newcommand{\bea}{\begin{eqnarray}}
\newcommand{\eea}{\end{eqnarray}}


\begin{document}

\begin{center}
{\bf STAT/MATH 414, section 001}

{\bf Homework \#8, due Friday, Mar. 27}
\end{center}

{\em Show all work.  Exercise numbers refer to Hogg and Tanis, 7th edition, unless otherwise stated.}

\begin{enumerate}

\item Exercise 4.1-6.  In addition to parts (a) and (b), also find the correlation of $X$ and $Y$.

{\bf Hint:\ }
Logically, the correlation should be negative because if $X$ happens to be really large, this
means $Y$ is more likely to be small and vice versa.  This will give you one way to check
your work (if you get a positive correlation, you did something wrong).  Also, you might find
the ``Note'' in Exercise 4.2-6 helpful.

\item Exercise 4.1-8.

\item Exercise 4.1-11.   Instead of merely answering the yes/no question in the exercise, explain how you
can tell at a single glance that $X$ and $Y$ must be independent.  


\item Suppose that $(X,Y)$ is uniformly distributed on a circle with center at the origin and
radius 1.  What is $P(2X+Y>1)$?

{\bf Hint:\ } Sketch the region over which you must integrate.  Also, an antiderivative of 
$\sqrt{1-y^2}$ is $\frac12y\sqrt{1-y^2} + \frac12\arcsin(y)$.  

\item
A device runs until either of two components fails, at which point the device stops
running. The joint density function of the lifetimes of the two components, both
measured in hours, is
\beaa
f(x,y) = c(x+y) \quad\mbox{for $0<x<3$ and $0<y<3$.}
\eeaa
Calculate the probability that the device fails during its first hour of operation.


\item Prove that for any two random variables $X$ and $Y$ for which $\Var(X)$ and $\Var(Y)$
are finite,
\beaa
\Var(X+Y) = \Var(X) + \Var(Y) + 2\Cov(X,Y).
\eeaa
{\bf Hint:\ } Don't worry about the ``finite'' part --- it just means that you don't need to be nervous
about the case $\Var(X)=\infty$ or $\Var(Y)=\infty$, when the above addition can be troublesome.
You are free to use the facts $\Var(X)=E(X^2)-[E(X)]^2$ and 
$\Cov(X,Y)=E(XY) - E(X)E(Y)$ without proving them.

\item Exercise 4.2-1.  Also find $\Var (X+Y)$.

\item 
Two insurers provide bids on an insurance policy to a large company. The bids must be
between 2000 and 2200. The company decides to accept the lower bid if the two bids
differ by 20 or more. Otherwise, the company will consider the two bids further.
Assume that the two bids are independent and are both uniformly distributed on the
interval from 2000 to 2200.
Determine the probability that the company considers the two bids further.

\item 
An insurance company sells two types of auto insurance policies: Basic and Deluxe. The
time until the next Basic Policy claim is an exponential random variable with mean two
days. The time until the next Deluxe Policy claim is an independent exponential random
variable with mean three days.
What is the probability that the next claim will be a Deluxe Policy claim?

{\bf Note:\ }  This scenario, in which waiting times are exponential and independent and you
wish to calculate the wait until the next event, is {\em very} common --- not just in this
insurance setting!

\end{enumerate}


\end{document}

