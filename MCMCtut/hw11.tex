\documentclass[11pt]{article}
\usepackage[sort,longnamesfirst]{natbib}
\usepackage{psfig}
\usepackage{amsmath}%
\newcommand{\bthet}{ \mbox{\boldmath $\theta$}}
\newcommand{\bOne}{ {\bf 1} }
\newcommand{\lambdaT}{\lambda_{\theta}}
\newcommand{\bTheta}{ \mbox{\boldmath $\Theta$}}
\begin{document}
\pagestyle{empty}
\begin{center}
\Large
{\bf  Markov chain Monte Carlo exercise}\\
\end{center}
% Suggestion: please start this homework as early as possible as it is
% likely to be more time consuming than your previous assignments.\\
\normalsize
Consider the following hierarchical changepoint model for the number
of occurrences $Y_i$ of some event during time interval $i$ with change point $k$.
\begin{equation*}
\begin{split}
Y_i |k,\theta,\lambda \sim & \mbox{Poisson}(\theta) \mbox{ for } i=1,\dots,k\\
Y_i |k,\theta,\lambda \sim & \mbox{Poisson}(\lambda) \mbox{ for } i=k+1,\dots,n
\end{split}
\end{equation*}
Assume the following prior distributions:
$$\theta|b_1 \sim  G(0.5,b_1), \lambda|b_2 \sim  G(0.5,b_2)$$
$$b_1 \sim  G(1,1), b_2 \sim  G(1,1)$$
$$ k \sim \mbox{Discrete Uniform}(1,\dots,n)$$

$k,\theta,\lambda$ are independent and $b_1,b_2$ are independent. \\ Assume the Gamma density parameterization G$(\alpha,\beta) = \frac{1}{\Gamma(\alpha)\beta^{\alpha}} x^{\alpha-1} e^{-x/\beta}$\\\\
Inference for this model is therefore based on the 5-dimensional
posterior distribution $f(k,\theta,\lambda,b_1,b_2| {\bf Y})$ where
{\bf Y}=$(Y_1,\dots,Y_n)$. Apply this model to the data set available from the webpage 
INSERT WEBPAGE.\\\\
(1) Use the Metropolis-Hastings algorithm for simulating from the
posterior distribution $f(k,\theta,\lambda,b_1,b_2| {\bf Y})$.
Include an estimated density plot based on your samples for each of
the 5 parameters.\\
(2) The parameters of interest here are $k,\theta,\lambda$: report
your MCMC estimates of their means.\\% and 95\% confidence intervals for these 3 parameters. One easy
% way to construct a 95\% confidence interval based on your samples is
% to simply use sample quantiles (for e.g. the ``quantile'' command in {\tt R}).\\
(3) Compute Monte Carlo standard errors for your mean estimates for
$k,\theta,\lambda$ using the batch means method (this will be covered on 4/15/05).\\\\
As before, you should use heuristics such as plots of your estimates,
autocorrelations, and acceptance rates to help guide you through the
simulation process. Include the most relevant plots in your report.\\
%$a_1=a_2=0.5,c_1=c_2=1,d_1=d_2=1$
% (Simple) Monte Carlo maximum likelihood problem:\\
\end{document}
