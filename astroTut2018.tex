\documentclass[12pt]{article}
%\usepackage{psfig}
\usepackage{amsbsy,amsmath,amsthm,amssymb,graphicx}%
\newcommand{\bX}{{\bf X}}
\newcommand{\bY}{{\bf Y}}
\newcommand{\bx}{{\bf x}}
\newcommand{\bpi}{{\mathbf \pi}}
%\newcommand\lo[1]{_{\nano{#1}}}
\newcommand\lo[1]{_{#1}}
\newcommand\hi[1]{^{#1}}
\newcommand{\field}[1]{\mathbb{#1}}
\newcommand{\R}{\field{R}}
\newcommand{\F}{\field{F}}
\newcommand{\p}{\field{P}}
\newcommand{\N}{\field{N}}
\newcommand{\Z}{\field{Z}}
\newcommand{\X}{\field{X}}
\newcommand{\E}{\field{E}}

\def\qed{\hfill$\diamondsuit$}
\def\II{I\negthinspace I}
\def\B{I\negthinspace\negthinspace B}
\def\ccirc{\negthinspace\circ}
\def\median{\mathop{\mbox{med}}}
\def\argmax{\mathop{\mbox{argmax}}}
\def\argmin{\mathop{\mbox{argmin}}}
\theoremstyle{example}
\theoremstyle{remark}
\theoremstyle{lemma}
\theoremstyle{definition}
\theoremstyle{corol}
\theoremstyle{proposition}
\theoremstyle{condition}
\newtheorem{thm}{Theorem}
\newtheorem{example}{Example}
\newtheorem{remark}{Remark}
\newtheorem{lemma}{Lemma}
\newtheorem{definition}{Definition}
\newtheorem{corol}{Corollary}
\newtheorem{condition}{Condition}
\newtheorem{proposition}{Proposition}
\newtheorem{assumption}{Assumption}


 \def\Cod{\buildrel {\cal D}\over\Longrightarrow}

\def\cod{\stackrel{d}{\longrightarrow}}
\def\cop{\stackrel{\cal P}{\longrightarrow}}
\def\eqd{\stackrel{\cal D}{=}}
\def\eqp{\stackrel{\cal P}{=}}
\def\ap#1{\smash{\mathop{\approx}\limits^{#1}}}
\def\lf{\lfloor}
\def\rf{\rfloor}
\def\lc{\lceil}
\def\rc{\rceil}
\def\trans{^{\rm T}}
\oddsidemargin 0.0in
\evensidemargin 1.0in
\textwidth 6.0in
%\headheight 1.0in
\headheight 0.3in
\topmargin -0.8in
\textheight 9.0in

%% \theoremstyle{remark}
% \newtheorem{example}{Example}
% \usepackage{geometry}
% %\geometry{hmargin=1.025in,vmargin={1.25in,0.75in},nohead,footskip=0.5in}
% \geometry{hmargin=2.5cm,vmargin={2.5cm,2.5cm},nohead,footskip=0.5in}
% \renewcommand{\baselinestretch}{1.21} 
\newcommand{\Real}{{\mathbb R}}
\newcommand{\boldY}{{\bf Y}}
\newcommand{\boldZ}{{\bf Z}}
\newcommand{\boldR}{{\bf R}}
\newcommand{\boldz}{{\bf z}}
\newcommand{\boldC}{{\bf C}}
\newcommand{\boldB}{{\bf B}}
\newcommand{\boldI}{{\bf I}}
\newcommand{\bell}{ \mbox{\boldmath $\ell$}}
\newcommand{\btheta}{ \mbox{\boldmath $\theta$}}
\newcommand{\bphi}{ \mbox{\boldmath $\phi$}}
\newcommand{\bPhi}{ \mbox{\boldmath $\Phi$}}
\newcommand{\bmu}{ \mbox{\boldmath $\mu$}}
\newcommand{\bxi}{ \mbox{\boldmath $\xi$}}
\newcommand{\boldeta}{ \mbox{\boldmath $\eta$}}
\newcommand{\beq}{\begin{equation}}
\newcommand{\eeq}{\end{equation}}
\newcommand{\beqstar}{\begin{equation*}}
\newcommand{\eeqstar}{\end{equation*}}
\newcommand{\bthet}{ \mbox{\boldmath $\theta$}}
\newcommand{\blambda}{ \mbox{\boldmath $\lambda$}}
\newcommand{\bet}{ \mbox{\boldmath $\eta$}}
\newcommand{\bepsilon}{ \mbox{\boldmath $\epsilon$}}
\newcommand{\bdelta}{ \mbox{\boldmath $\delta$}}
\newcommand{\bnu}{ \mbox{\boldmath $\nu$}}
\newcommand{\bPsi}{ \mbox{\boldmath $\Psi$}}
\newcommand{\bY}{ {\bf Y} }
\newcommand{\by}{ {\bf y} }
\newcommand{\bM}{ {\bf M} }
\newcommand{\bU}{ {\bf U} }
\newcommand{\bu}{ {\bf u} }
\newcommand{\br}{ {\bf r} }
\newcommand{\bK}{ {\bf K} }
\newcommand{\bz}{ {\bf z} }
\newcommand{\bd}{ {\bf d} }
\newcommand{\bh}{ {\bf h} }
\newcommand{\bw}{ {\bf w} }
\newcommand{\bs}{ {\bf s} }
\newcommand{\bg}{ {\bf g} }
\newcommand{\bt}{ {\bf t} }
\newcommand{\bv}{ {\bf v} }
\newcommand{\bx}{ {\bf x} }
\newcommand{\bZ}{{\bf Z}}
\newcommand{\bX}{{\bf X}}
\newcommand{\bW}{{\bf W}}
\newcommand{\bE}{{\bf E}}
\newcommand{\bbeta}{ \mbox{\boldmath $\beta$}}
\newcommand{\bmuhat}{ \mbox{\boldmath $\hat{\mu}$}}
\newcommand{\beas}{\begin{eqnarray*}}
\newcommand{\eeas}{\end{eqnarray*}}
\renewcommand{\familydefault}{\sfdefault}
\renewcommand{\baselinestretch}{1.3}
\newcommand{\Nstar}{N^{*}}
\newcommand{\taustar}{\tau^{*}}
\newcommand{\alphatilde}{\tilde{\alpha}}
\newcommand{\lambdaT}{\lambda_{\theta}}
%\newcommand{\xprime}{x^{'}}
%\newcommand{\mprime}{m^{'}}
\newcommand{\bOne}{ {\bf 1} }
\newcommand{\bZero}{ {\bf 0} }
\newcommand{\xprime}{x'}
\newcommand{\nprime}{n'}
\newcommand{\mprime}{m'}
\newcommand{\bTheta}{ \mbox{\boldmath $\Theta$}}

\newcommand{\Cset}{\cal C}
\newcommand{\taueps}{\tau_{\epsilon}}
\newcommand{\Galpha}{G_{\alpha}}
\newcommand{\boldzpr}{{\bf z^{'}}}
\newcommand{\alphapr}{\alpha^{'}}
\newcommand{\tauepspr}{\taueps^{'}}
\newcommand{\Nkappa}{N_{\kappa}}
\def\baro{\vskip  .2truecm\hfill \hrule height.5pt \vskip  .2truecm}
\def\barba{\vskip -.1truecm\hfill \hrule height.5pt \vskip .4truecm}
\newcommand{\Matern}{Mat\'{e}rn }
\newcommand{\Rcode}{{\color{green} {R command: }}}
\newcommand{\bigoh}{\mathcal{O}}
\newcommand{\figtwobytwo}[4]
{
\begin{tabular}{cc}
    {{\resizebox*{0.47\textwidth}{0.35\textheight}
        {\rotatebox{0}{\includegraphics{#1}}}} \par}&
    {{\resizebox*{0.47\textwidth}{0.35\textheight}
        {\rotatebox{0}{\includegraphics{#2}}}} \par}\\
    {{\resizebox*{0.47\textwidth}{0.35\textheight}
        {\rotatebox{0}{\includegraphics{#3}}}} \par}&
    {{\resizebox*{0.47\textwidth}{0.35\textheight}
        {\rotatebox{0}{\includegraphics{#4}}}} \par}\\
\end{tabular}
}
\newcommand{\figtwobytwoA}[4]
{
\begin{tabular}{cc}
    {{\resizebox*{0.35\textwidth}{0.3\textheight}
        {\rotatebox{0}{\includegraphics{#1}}}} \par}&
    {{\resizebox*{0.35\textwidth}{0.3\textheight}
        {\rotatebox{0}{\includegraphics{#2}}}} \par}\\
    {{\resizebox*{0.35\textwidth}{0.3\textheight}
        {\rotatebox{0}{\includegraphics{#3}}}} \par}&
    {{\resizebox*{0.35\textwidth}{0.3\textheight}
        {\rotatebox{0}{\includegraphics{#4}}}} \par}\\
\end{tabular}
}
\newcommand{\figtwo}[2]
{
\begin{tabular}{cc}
    {{\resizebox*{0.35\textwidth}{0.2\textheight}
        {\rotatebox{0}{\includegraphics{#1}}}} \par}&
    {{\resizebox*{0.35\textwidth}{0.2\textheight}
        {\rotatebox{0}{\includegraphics{#2}}}} \par}\\
\end{tabular}
}
\newcommand{\figtwoA}[2]
{
\begin{tabular}{cc}
%    {{\resizebox*{0.33\textwidth}{0.24\textheight}
    {{\resizebox*{0.45\textwidth}{0.45\textheight}
        {\rotatebox{0}{\includegraphics{#1}}}} \par}&
    {{\resizebox*{0.45\textwidth}{0.45\textheight}
        {\rotatebox{0}{\includegraphics{#2}}}} \par}\\
\end{tabular}
}
\newcommand{\figtwoB}[2]
{
\begin{tabular}{cc}
%    {{\resizebox*{0.33\textwidth}{0.24\textheight}
    {{\resizebox*{0.6\textwidth}{0.7\textheight}
        {\rotatebox{0}{\includegraphics{#1}}}} \par}&
    {{\resizebox*{0.6\textwidth}{0.7\textheight}
        {\rotatebox{0}{\includegraphics{#2}}}} \par}\\
\end{tabular}
}
\newcommand{\figtwoC}[4]
{
\begin{tabular}{cc}
%    {{\resizebox*{0.33\textwidth}{0.24\textheight}
  #3 & #4 \\
    {{\resizebox*{0.47\textwidth}{0.55\textheight}
        {\rotatebox{0}{\includegraphics{#1}}}} \par}
     &
    {{\resizebox*{0.47\textwidth}{0.55\textheight}
        {\rotatebox{0}{\includegraphics{#2}}}} \par}\\
\end{tabular}
}

%\newcommand{\figone}[1]
%{
%  \begin{center}
%    {{\resizebox*{0.95\textwidth}{0.7\textheight}
%        {\rotatebox{270}{\includegraphics{#1}}}} \par}
%  \end{center}
%  }

\newcommand{\figoneA}[1]
{
  \begin{center}
    {{\resizebox*{0.35\textwidth}{0.35\textheight}
        {\rotatebox{0}{\includegraphics{#1}}}} \par}
  \end{center}
  }

\newcommand{\figoneAA}[1]
{
  \begin{center}
    {{\resizebox*{1\textwidth}{0.3\textheight}
        {\rotatebox{0}{\includegraphics{#1}}}} \par}
  \end{center}
  }

\newcommand{\figoneAAbig}[1] % Sham's version
 {
  \begin{center}
    {{\resizebox*{0.7\textwidth}{0.7\textheight}
        {\rotatebox{0}{\includegraphics{#1}}}} \par}
  \end{center}
  }
  
 
  \newcommand{\figoneAB}[1]
{
  \begin{center}
    {{\resizebox*{0.6\textwidth}{0.6\textheight}
        {\rotatebox{0}{\includegraphics{#1}}}} \par}
  \end{center}
  }
  
    \newcommand{\figoneAC}[1]
{
  \begin{center}
    {{\resizebox*{0.65\textwidth}{0.65\textheight}
        {\rotatebox{0}{\includegraphics{#1}}}} \par}
  \end{center}
  }

    \newcommand{\figoneAD}[1]
{
  \begin{center}
    {{\resizebox*{0.70\textwidth}{0.70\textheight}
        {\rotatebox{0}{\includegraphics{#1}}}} \par}
  \end{center}
  }

\newcommand{\figonebig}[1]
{
  \begin{center}
    {{\resizebox*{0.7\textwidth}{0.7\textheight}
        {\rotatebox{0}{\includegraphics{#1}}}} \par}
  \end{center}
  }
\newcommand{\figtwoAA}[2]
{
\begin{tabular}{cc}
%    {{\resizebox*{0.33\textwidth}{0.24\textheight}
    {{\resizebox*{0.45\textwidth}{0.45\textheight}
        {\rotatebox{0}{\includegraphics{#1}}}} \par}&
    {{\resizebox*{0.45\textwidth}{0.45\textheight}
        {\rotatebox{0}{\includegraphics{#2}}}} \par}\\
\end{tabular}
}

%\newcommand{\figonesmall}[1]
%{
%  \begin{center}
%    {{\resizebox*{0.475\textwidth}{0.35\textheight}
%        {\rotatebox{0}{\includegraphics{#1}}}} \par}
%  \end{center}
%  }

%\newcommand{\figonerotate}[1]
%{
%  \begin{center}
%    {{\resizebox*{0.7\textwidth}{0.7\textheight}
%%    {{\resizebox*{0.95\textwidth}{0.7\textheight}
%        {\rotatebox{0}{\includegraphics{#1}}}} \par}
%  \end{center}
%  }
% \newcommand{\figonerotatemedium}[1]
% {
%  \begin{center}
%    {{\resizebox*{0.95\textwidth}{0.7\textheight}
%    {{\resizebox*{0.855\textwidth}{0.63\textheight}
%    {{\resizebox*{0.665\textwidth}{0.49\textheight} % with image-transform-autocrop in gimp (R map files)
%        {\rotatebox{0}{\includegraphics{#1}}}} \par}
%  \end{center}
%  }
%\newcommand{\figonerotatemediumsmall}[1]
%{
%  \begin{center}
%%    {{\resizebox*{0.95\textwidth}{0.7\textheight}
%%    {{\resizebox*{0.855\textwidth}{0.63\textheight}
%    {{\resizebox*{0.532\textwidth}{0.392\textheight} % with image-transform-autocrop in gimp (R map files)
%        {\rotatebox{0}{\includegraphics{#1}}}} \par}
%  \end{center}
%  }
%% this one preserves perspective (so circle stays a circle)
%\newcommand{\figonerotatepersp}[1]
%{
%  \begin{center}
%    {{\resizebox*{0.6\textwidth}{0.7\textheight}
%%    {{\resizebox*{0.95\textwidth}{0.7\textheight}
%        {\rotatebox{0}{\includegraphics{#1}}}} \par}
%  \end{center}
%  }

%% this one preserves perspective (so circle stays a circle)
%\newcommand{\figonerotatebig}[1]
%{
%  \begin{center}
%%    {{\resizebox*{0.6\textwidth}{0.7\textheight}
%    {{\resizebox*{0.95\textwidth}{0.9\textheight}
%        {\rotatebox{0}{\includegraphics{#1}}}} \par}
%  \end{center}
%  }

\newcommand{\head}[1]
{
  \begin{center}
      {\huge {\color{blue} #1}}
    \end{center}
  }
%\definecolor{pink}{rgb}{1,0.5,0.5}
% 2 x 2 fill up the entire slide, also allow labeling
\newcommand{\figtwobytwoB}[8]
{
\begin{tabular}{cc}
  #5 & #6\\
    {{\resizebox*{0.5\textwidth}{0.3\textheight}
        {\rotatebox{0}{\includegraphics{#1}}}} \par}&
    {{\resizebox*{0.5\textwidth}{0.3\textheight}
        {\rotatebox{0}{\includegraphics{#2}}}} \par}\\
  #7 & #8\\
    {{\resizebox*{0.5\textwidth}{0.3\textheight}
        {\rotatebox{0}{\includegraphics{#3}}}} \par}&
    {{\resizebox*{0.5\textwidth}{0.3\textheight}
        {\rotatebox{0}{\includegraphics{#4}}}} \par}\\
\end{tabular}
}
\newcommand{\framenum}{\vspace{-1.45cm}
  \begin{flushright} \textcolor{black}{\normalsize \insertframenumber{$\quad$}} \end{flushright}
  \vspace{-1 cm} }              % This adds the frame number to the upper-right corner of each slide 
\newcommand{\Kv}{\color{blue} \mathbf{K_v}}
%\newcommand{\bluebtheta}{\color{blue} \mbox{\boldmath $\theta$}}
\newcommand{\bluebtheta}{\mbox{\color{blue} \boldmath $\theta$}}


\begin{document}

\begin{center}
{\bf \large \noindent Bayesian computation lab\\ Astroinformatics Summer School, Penn State
  University, June 2018.  \\
Instructor: Murali Haran
}
\end{center}

\noindent \hrulefill
Suppose observations $Y_i,\:i=1,\dots,n$ are modeled as a linear
  function of a predictor $X_i$, with two additive sources of error,
  one Gaussian with mean 0 and variance $\tau$, and the other
  exponential with parameter $\lambda$. Given a data set
  $(X_1,Y_1), \dots, (X_n,Y_n)$, we are interested in estimating
  $\beta$ (regression coefficient relating $Y$'s to $X$'s), and the
  error parameters $\tau, \lambda$. The $Y_i$s are called
  exponentially modified Gaussian random variables (EMG). An
  EMG($\mu, \tau, \lambda$) random variable is continuous with density
$$f(x;\mu, \tau, \lambda) = \frac{\lambda}{2} \exp\left(\frac{\lambda}{2} (2\mu + \lambda\tau^2 - 2x)\right) \mbox{erfc}\left(\frac{\mu + \lambda\tau^2 - x}{\sqrt{2} \tau} \right), $$
and erfc is the complementary error function defined as
$$\mbox{erfc}(x) = \frac{2}{\pi} \int_x^{\infty} e^{-t^2} dt.$$
 Suppose we have independent observations $Y_i$ with fixed (nonrandom)
$X_i$ that satisfy %Consider a regression of a variable $Y$ on $X$ where the regression model is as follows,
$Y_i \sim EMG(\beta X, \tau, \lambda),\: i=1,..,n.$ 
These data may be obtained from\\ http://personal.psu.edu/muh10/expregdat.txt
\begin{enumerate}
\item Assume that $\tau$ is known to be 2. We are interested in Bayesian inference 
for $\beta, \lambda$. Let the independent priors for 
$\beta, \lambda$ be $p(\beta), p(\lambda)$
respectively where $p(\beta)\propto 1$  (``flat prior'') and
$p(\lambda)$ is Gamma$(a=1,b=10)$ so $p(\lambda) = \frac{1}{\Gamma(a)
  b^a} \lambda^{a-1} e^{-\lambda/b}$. The resulting posterior
distribution, 
\begin{equation*}
\begin{split}
\pi(\beta,\lambda\mid \bY) & \propto \prod_{i=1}^n f(Y_i\mid \beta,
\lambda) p(\lambda) p(\beta) \\ %p(\tau)\\
& \propto \prod_{i=1}^n \frac{\lambda}{2}
\exp\left(\frac{\lambda}{2}\left(2X_i\beta + \lambda\tau - 2Y_i 
  \right)\right)\mbox{erfc}\left(\frac{X_i\beta + \lambda\tau -
      Y_i}{\sqrt{2\tau}} \right) p(\beta) p(\lambda) \\%p(\tau) 
& \propto \lambda^n\prod_{i=1}^n \mbox{erfc}\left(\frac{X_i\beta + \lambda\tau -
  Y_i}{\sqrt{2\tau}}\right)  \times\exp\left( \frac{\lambda}{2}\sum_{i=1}^n 
  (2X_i\beta + \lambda\tau - 2Y_i) \right)p(\beta)p(\lambda)%p(\tau) 
\end{split}
\end{equation*}
Hence, full conditionals are: 
\begin{equation*}
\begin{split}
\pi(\beta\mid \lambda, \bY) & \propto \lambda^n\prod_{i=1}^n \mbox{erfc}\left(\frac{X_i\beta + \lambda\tau -
  Y_i}{\sqrt{2\tau}}\right)  \times\exp\left( \frac{\lambda}{2}\sum_{i=1}^n 
  (2X_i\beta + \lambda\tau - 2Y_i) \right)p(\beta)p(\lambda)\\%p(\tau) 
\pi(\lambda\mid \beta, \bY) & \propto \lambda^n\prod_{i=1}^n \mbox{erfc}\left(\frac{X_i\beta + \lambda\tau -
  Y_i}{\sqrt{2\tau}}\right)  \times\exp\left( \frac{\lambda}{2}\sum_{i=1}^n 
  (2X_i\beta + \lambda\tau - 2Y_i) \right)p(\beta)p(\lambda)%p(\tau) 
\end{split}
\end{equation*}
On log scale, 
\begin{equation*}
\begin{split}
\log(\pi(\beta\mid \lambda, \bY)) & =\sum_{i=1}^n \mbox{logerfc}\left(\frac{X_i\beta + \lambda\tau -
  Y_i}{\sqrt{2\tau}}\right) + \left( \frac{\lambda}{2}\sum_{i=1}^n 
  (2X_i\beta + \lambda\tau - 2Y_i) \right) + \log(p(\beta))\\%p(\tau) 
\log(\pi(\lambda\mid \beta, \bY)) & = n\log(\lambda) + \sum_{i=1}^n \mbox{logerfc}\left(\frac{X_i\beta + \lambda\tau -
  Y_i}{\sqrt{2\tau}}\right) + \left( \frac{\lambda}{2}\sum_{i=1}^n 
  (2X_i\beta + \lambda\tau - 2Y_i) \right) \\
& + \log(p(\lambda))%p(\tau) 
\end{split}
\end{equation*}

% \item Provide pseudocode for a Metropolis-Hastings algorithm to
%   construct a Markov chain with stationary distribution $p(\beta_0,
%   \beta, \lambda\mid {\mathbf Y}, {\mathbf X})$ for a data set of size $n$,
%   $(X_1,Y_1),\dots, (X_n,Y_n)$.  ${\mathbf Y}, {\mathbf X}$ are
%  $(Y_1,\dots, Y_n)$, $(X_1,\dots, X_n)$ respectively. You should provide enough detail so
%   anyone reading it should be able to write code based on your
%   description. You do not have to provide starting values or specific
%   tuning parameters since those will depend on the particulars of the
%   data. You should, however, list at the beginning of the algorithm
%   any/all tuning parameters for the algorithm that you will have to
%   adjust in order to make it work well. You should also show any
%   distributions you derive/use for constructing this algorithm. Please write the algorithm and associated distributions etc. as clearly as possible.
\item Approximate (i) $E_{\pi}(\lambda)$, (ii)
  $E_{\pi}\left(\frac{1}{\beta + \lambda}\right)$, (iii) the posterior
  correlation between $\beta$ and $\lambda$.
\item Take a look at a histogram and then a smoothed histogram
  (plot(density(x)) to get an idea of the marginal posterior distributions.
\item Use the R package {\tt mcmcse} and the command {\tt mcse} to
  find the standard errors of the Monte Carlo approximations of
  the parameters, as well as the effective sample size (ESS) for each parameter.
\item Determine an appropriate length of the Markov chain 
\item Go through the MCMC checklist from the tutorial today. You may
  have to modify and re-run the algorithm. In particular, try at least
  3 different initial values and run the chains for long enough to see
  if they produce similar results. 
\item Once you are confident you are have reliable answers, you are
  ready to also estimate $\tau$. Assume the same prior for $\tau$ as
  for $\lambda.$ For this you will need to augment the
  MCMC code you have been provided. Repeat the above exercise.
\item Suppose your collaborator changes her mind and says that the
  prior for $\lambda$ should be Gamma(2,5) instead. Without re-running
  your MCMC algorithm, use importance sampling to reweight the Markov
  chain you already have in order to approximate the new posterior
  expectations of $\beta, \lambda,\tau$.
\item Return to the two-dimensional case (where $\tau$ is known). Now
  use a Laplace approximation to approximate the expectations of each
  parameter. You may find it useful to use the {\tt optim} command in
  order to do an optimization for the Laplace approximation. Please
  note that the default of optim is to minimize, not maximize, so you
  will need to either change the controls, or use the negative of the 
  posterior distribution.
\end{enumerate}

\end{document}
