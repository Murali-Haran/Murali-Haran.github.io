\documentclass[10pts]{article}
\usepackage[sort,longnamesfirst]{natbib}
\usepackage{amsmath}%
\usepackage{hyperref}
\begin{document}
%{\bf TENTATIVE SCHEDULE}\\
\pagestyle{empty}
\Large
\begin{center}
{\bf  Project and Presentation Requirements for STAT 540, Spring 2018}\\
\end{center}
\normalsize
% The project is a very important component of your class work for this course. 
\begin{itemize}
\item Important dates: 
\begin{enumerate}
\item February 27 (in class): Submit hardcopy of your project proposal.
  This is a 1 page detailed summary of your project plan.
\item (Tentatively) April 16 - April 27 (last two weeks of classes): Project presentation,
  during class time or by appointment. More information on this will
  be provided later. Submit slides {\it by midnight the day before
    your presentation}.
\item Monday, April 30: Submit final project reports on Canvas by midnight
\item Wednesday, May 2 by midnight: Submit evaluation of other students' projects/presentations on Canvas by midnight
\end{enumerate}
\item Project expectations:
\begin{enumerate}
\item Please talk to me about potential project topics. You are
  welcome to work on something related to your dissertation {\it as
    long as it is distinct from what you were already planning to
    include in your thesis.}
\item Projects must contain some original work. They should
  include (i) the implementation and study of computer code and a
  computational algorithm {\bf and} (ii) {\it at least one} of the
  following:
\begin{itemize}
\item A thorough simulation study. This is especially appropriate if
  the focus of the project is one or more papers. Examples of
  original work here include conducting a simulation that was not done
  in the manuscript and/or a comparison with an algorithm that was not
  studied in the manuscript. Replicating a simulation study that is
  already published is a reasonable starting point but you need to do
  more.
\item An application of existing methods or models to a new data
  set. The methods must be directed towards answering a carefully
  formulated scientific question. In general it is not a good idea to
  work on a data set unless you have well defined questions and some
  knowledge of the data source.
\item An application of a new method to a data set that has already
  been studied before (via other methods).
\item Some theoretical development: either rewriting/explaining a
  proof clearly or going a little beyond what is done in the paper.
\end{itemize}
\item Attribution of sources: It is very important that you attribute
  sources correctly. Do not copy material from the web -- you can use
  web searches to find appropriate references and then cite them. Do
  not use any writing from papers without citing them. This is not
  just a guideline for this project but is also to ensure that you do
  not violate principles of academic honesty. When in doubt, cite more
  rather than less.
\item Submit {\tt R code} (or Matlab/C etc.) that you used to
  implement the project. 
\item Your writeup: 
  \begin{enumerate}
  \item Your {\bf writeup} must be typed. The main portion of the
    manuscript should be no longer than 7 pages, including
    figures and tables, but excluding references (that is, references
    do not count against your word limit). The audience for this
    writeup are your classmates and instructor. You are allowed to
    have technical supplementary material but this material must also
    be carefully edited and no longer than 5 pages.
  \item Introduction/motivation (description of problem): This should
    be very concise and clear to someone who does not know any of the
    specific terminology used in your field. Some of the main points
    to convey: (i) what is the main problem being solved? (ii) outline
    of the methodology (algorithms, models); (iii) what, in
    particular, you are investigating in this project; (iv) how the
    rest of your manuscript is organized. This may also include
    necessary background material such as related methods or previous
    analyses done on the data you are studying.
  \item Description of model and algorithm: make sure you clearly
    explain the model and/or algorithm that you are studying. Do not
    simply copy what is in a paper -- you have to translate that
    information into an easy-to-understand version. 
  \item Description of data and/or simulation study: Where were the
    data obtained and what do you know about how they were collected?
    If the data are publicly available, provide enough information
    (e.g. website) to make it easily accessible. If you are describing
    a simulation study, explain how you decided to carry it out. You
    may say that you are re-using simulation design from another
    paper, but you have to explain why those parameter setting are
    reasonable or interesting. 
  \item Results: This should consist of: (a) Conclusions
    about methodology: What, if anything, have you learned about the
    applicability of the algorithms you tried to the problem you were
    considering? (b) What are useful guidelines for someone else who
    might choose to implement the algorithm(s) presented? (c)
    Scientific conclusions: If this is relevant, describe the
    implications of your analysis in the context of your particular
    scientific domain.  Again, this should be written so a non-expert
    can understand it.
  \item Future work: As briefly as possible, explain what still
    remains unresolved. For instance, is there other statistical
    methodology or algorithm that you would like to apply to this problem? Are there other questions you
    would like to pursue with the same data set or statistical problem? 
  \item Supplementary material: This is not necessary but it is likely
    to be a useful place for you to write down technical details that
    may not fit into the main manuscript. I will look over this -- if
    it is hard to read or not edited carefully, you will lose
    points. You may also include extra figures, tables here. 
  \end{enumerate}
\item Project scoring:
  \begin{enumerate}
\item Proposal: 5
\item Organization/clarity of writing: 10
\item Correctness of work: 5
\item Depth (how deep you went into the subject): 5
\item Presentation: 10
\item Attendance/evaluation of other students' projects: 5
  \end{enumerate}
\item Presentation Guidelines: 
\begin{enumerate}
\item All talks will be for a precisely scheduled time (to be
  determined by the number of projects submitted, usually between 12-20
  minutes long). You cannot convey too many details -- your focus
  should be on explaining some basic ideas. It is fine to finish in
  slightly less time but it is not okay to go
  over. %At the end of the talk, we will have 5 minutes for discussion where I and the rest of the class will ask questions.
\item You are required to send me your slides by midnight {\it the day before your talk}. Please ensure that your slides are in pdf format. If you are going to use any other format, you need to get special permission.
\item You will be allowed a prescribed number of slides (usually 1 fpr every two minutes).  Given the short amount of time allotted to each
  student, I will delete any slides that go beyond the prescribed limit.
\item Your goal should be to communicate in as effective a way as possible what you did in the project. The presentations are primarily meant to be educational for the entire class so you will also
 be graded on the clarity of your explanation.
\item Rehearse your presentations: giving a very short talk is not
  easy. Here are some suggestions: \url{http://personal.psu.edu/muh10/540/shorttalk.html}
\item Attendance is compulsory throughout the project presentation period. I will take attendance at the beginning of each class.
\item At the end of each class, if we need extra time for discussion, it is possible that the class will go over the allotted time. Please allow for an extra 15 minutes at the end of each
 of these days. 
\item Depending on the number of projects, a few of you may present your work to me in my office at a specially scheduled time.
\end{enumerate}
\end{enumerate}
\end{itemize}
\end{document}

%%% Local Variables: 
%%% mode: latex
%%% TeX-master: t
%%% End: 
