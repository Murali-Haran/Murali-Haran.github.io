\documentclass[10pt]{article}
\usepackage[sort,longnamesfirst]{natbib}
\usepackage{color}
\usepackage{hyperref}
\usepackage{graphics,enumerate}
\usepackage{amsmath}%                                                                                                         
\newcommand{\bthet}{ \mbox{\boldmath $\theta$}}
\newcommand{\bOne}{ {\bf 1} }
\newcommand{\trace}{ \mbox{tr}}
\newcommand{\diag}{ \mbox{diag}}
\newcommand{\lambdaT}{\lambda_{\theta}}
\newcommand{\bTheta}{ \mbox{\boldmath $\Theta$}}
\begin{document}

\begin{center}
{\bf Penn State STAT 540}

{\bf Homework \#3, due Thursday, Nov 15, 2018}\\

\end{center}
\noindent What you have to submit in a Canvas submission folder: (i)
Your {\tt R} code in a file titled PSUemailidHW1.R (e.g. muh10HW1.R),
(ii) pdf file that contains a clear writeup for the questions below
named PSUemailidHW1.pdf (e.g. muh10HW1.pdf). Note that
your code should be readily usable, without any modifications.
\begin{enumerate}
%% Auxiliary variable MCMC algorithm
\item Follow the bulbs example from class. Let the number of bulbs in rooms A and B be $m=400, n=25$ respectively. You may assume that all bulb lifetimes are independent, and distributed according to an exponential distribution with expectation $\theta$. The number of bulbs that survived until time $\tau=15$ in room A, $S$, is 254. You can find the lifetime  of bulbs in Room B, $B_1,\dots, B_n$ here \url{personal.psu.edu/muh10/540/bulbsB.dat}. 
\begin{enumerate}
\item Find the maximum likelihood estimator of $\theta,
  \hat{\theta_{MLE}}$ using the Expectation-Maximization (E-M)
  algorithm. Write out clear pseudocode for this algorithm, with
  enough details that someone not familiar with the E-M algorithm
  could code it up. Explain also how you obtained starting values, and
  your stopping criteria for the algorithm. (Note that for this simple
  problem, you do not need an E-M algorithm as you can construct the
  observed data log likelihood easily. If you like, you can use the
  observed data log likelihood to find the MLE, and then compare that
  result to the answer you obtain from the E-M algorithm.)
\item Approximate the standard error of $\hat{\theta_{MLE}}$ using a
  nonparametric bootstrap, and construct a 95\% confidence interval
  for $\theta$. Write out pseudocode as above.
\item Repeat the above, but with a parametric bootstrap. 
\item Assume that $\theta$ has a prior distribution that is
  Uniform(0,100). Write out clear, detailed pseudocode (including
  conditional distributions, how to construct Metropolis-Hastings
  updates etc.) for an auxiliary variable MCMC algorithm for carrying
  out inference for the above problem. That is, this code should be
  useful for approximating $\pi(\theta| B_1,\dots, B_m)$. Before you
  provide the pseudocode, clearly show what the target posterior
  distribution is for your auxiliary variable algorithm (recall from
  class that this target will include the auxiliary random
  variables). State clearly how you would approximate
  $E_{\pi}(\theta|B_1,\dots,B_m, S)$ using the Markov chain you have
  constructed.
\item (Optional) Write code for the above auxiliary variable MCMC algorithm and report the posterior 95\% credible interval for $\theta$ and plot the marginal posterior pdf for $\theta$.
\end{enumerate}
\item Return to the above bulbs problem, except assume that the model for the bulbs is now Unif($(0,\kappa)$). The number of bulbs that survived until time $\tau=15$ in room A, $S$, is 167. You can find the lifetime  of bulbs in Room B, $B_1,\dots, B_n$ here \url{personal.psu.edu/muh10/540/bulbsBProb2.dat}. 
\begin{enumerate}
\item Find the MLE of $\kappa$ using an EM algorithm. Provide pseudocode and details of the algorithm as above. Report your estimate.
\item Find a 95\% confidence interval for $\kappa$. You may use any method you like, just explain it clearly and point out any potential problems. 
\item Compare your point estimate and 95\% confidence interval for $\kappa$ from the previous two parts with the estimates you would obtain if you just used the data from Room B (that is, if you discarded the data from Room A).
\end{enumerate}
\end{enumerate}
\end{document}
