% \theoremstyle{remark}
% \newtheorem{example}{Example}
% \usepackage{geometry}
% %\geometry{hmargin=1.025in,vmargin={1.25in,0.75in},nohead,footskip=0.5in}
% \geometry{hmargin=2.5cm,vmargin={2.5cm,2.5cm},nohead,footskip=0.5in}
% \renewcommand{\baselinestretch}{1.21} 
\newcommand{\Real}{{\mathbb R}}
\newcommand{\boldY}{{\bf Y}}
\newcommand{\boldZ}{{\bf Z}}
\newcommand{\boldR}{{\bf R}}
\newcommand{\boldz}{{\bf z}}
\newcommand{\boldC}{{\bf C}}
\newcommand{\boldB}{{\bf B}}
\newcommand{\boldI}{{\bf I}}
\newcommand{\bell}{ \mbox{\boldmath $\ell$}}
\newcommand{\btheta}{ \mbox{\boldmath $\theta$}}
\newcommand{\bphi}{ \mbox{\boldmath $\phi$}}
\newcommand{\bPhi}{ \mbox{\boldmath $\Phi$}}
\newcommand{\bmu}{ \mbox{\boldmath $\mu$}}
\newcommand{\bxi}{ \mbox{\boldmath $\xi$}}
\newcommand{\boldeta}{ \mbox{\boldmath $\eta$}}
\newcommand{\beq}{\begin{equation}}
\newcommand{\eeq}{\end{equation}}
\newcommand{\beqstar}{\begin{equation*}}
\newcommand{\eeqstar}{\end{equation*}}
\newcommand{\bthet}{ \mbox{\boldmath $\theta$}}
\newcommand{\blambda}{ \mbox{\boldmath $\lambda$}}
\newcommand{\bet}{ \mbox{\boldmath $\eta$}}
\newcommand{\bepsilon}{ \mbox{\boldmath $\epsilon$}}
\newcommand{\bdelta}{ \mbox{\boldmath $\delta$}}
\newcommand{\bnu}{ \mbox{\boldmath $\nu$}}
\newcommand{\bPsi}{ \mbox{\boldmath $\Psi$}}
\newcommand{\bY}{ {\bf Y} }
\newcommand{\by}{ {\bf y} }
\newcommand{\bM}{ {\bf M} }
\newcommand{\bU}{ {\bf U} }
\newcommand{\bu}{ {\bf u} }
\newcommand{\br}{ {\bf r} }
\newcommand{\bK}{ {\bf K} }
\newcommand{\bz}{ {\bf z} }
\newcommand{\bd}{ {\bf d} }
\newcommand{\bh}{ {\bf h} }
\newcommand{\bw}{ {\bf w} }
\newcommand{\bs}{ {\bf s} }
\newcommand{\bg}{ {\bf g} }
\newcommand{\bt}{ {\bf t} }
\newcommand{\bv}{ {\bf v} }
\newcommand{\bx}{ {\bf x} }
\newcommand{\bZ}{{\bf Z}}
\newcommand{\bX}{{\bf X}}
\newcommand{\bW}{{\bf W}}
\newcommand{\bE}{{\bf E}}
\newcommand{\bbeta}{ \mbox{\boldmath $\beta$}}
\newcommand{\bmuhat}{ \mbox{\boldmath $\hat{\mu}$}}
\newcommand{\beas}{\begin{eqnarray*}}
\newcommand{\eeas}{\end{eqnarray*}}
\renewcommand{\familydefault}{\sfdefault}
\renewcommand{\baselinestretch}{1.3}
\newcommand{\Nstar}{N^{*}}
\newcommand{\taustar}{\tau^{*}}
\newcommand{\alphatilde}{\tilde{\alpha}}
\newcommand{\lambdaT}{\lambda_{\theta}}
%\newcommand{\xprime}{x^{'}}
%\newcommand{\mprime}{m^{'}}
\newcommand{\bOne}{ {\bf 1} }
\newcommand{\bZero}{ {\bf 0} }
\newcommand{\xprime}{x'}
\newcommand{\nprime}{n'}
\newcommand{\mprime}{m'}
\newcommand{\bTheta}{ \mbox{\boldmath $\Theta$}}

\newcommand{\Cset}{\cal C}
\newcommand{\taueps}{\tau_{\epsilon}}
\newcommand{\Galpha}{G_{\alpha}}
\newcommand{\boldzpr}{{\bf z^{'}}}
\newcommand{\alphapr}{\alpha^{'}}
\newcommand{\tauepspr}{\taueps^{'}}
\newcommand{\Nkappa}{N_{\kappa}}
\def\baro{\vskip  .2truecm\hfill \hrule height.5pt \vskip  .2truecm}
\def\barba{\vskip -.1truecm\hfill \hrule height.5pt \vskip .4truecm}
\newcommand{\Matern}{Mat\'{e}rn }
\newcommand{\Rcode}{{\color{green} {R command: }}}
\newcommand{\bigoh}{\mathcal{O}}
\newcommand{\figtwobytwo}[4]
{
\begin{tabular}{cc}
    {{\resizebox*{0.47\textwidth}{0.35\textheight}
        {\rotatebox{0}{\includegraphics{#1}}}} \par}&
    {{\resizebox*{0.47\textwidth}{0.35\textheight}
        {\rotatebox{0}{\includegraphics{#2}}}} \par}\\
    {{\resizebox*{0.47\textwidth}{0.35\textheight}
        {\rotatebox{0}{\includegraphics{#3}}}} \par}&
    {{\resizebox*{0.47\textwidth}{0.35\textheight}
        {\rotatebox{0}{\includegraphics{#4}}}} \par}\\
\end{tabular}
}
\newcommand{\figtwobytwoA}[4]
{
\begin{tabular}{cc}
    {{\resizebox*{0.35\textwidth}{0.3\textheight}
        {\rotatebox{0}{\includegraphics{#1}}}} \par}&
    {{\resizebox*{0.35\textwidth}{0.3\textheight}
        {\rotatebox{0}{\includegraphics{#2}}}} \par}\\
    {{\resizebox*{0.35\textwidth}{0.3\textheight}
        {\rotatebox{0}{\includegraphics{#3}}}} \par}&
    {{\resizebox*{0.35\textwidth}{0.3\textheight}
        {\rotatebox{0}{\includegraphics{#4}}}} \par}\\
\end{tabular}
}
\newcommand{\figtwo}[2]
{
\begin{tabular}{cc}
    {{\resizebox*{0.35\textwidth}{0.2\textheight}
        {\rotatebox{0}{\includegraphics{#1}}}} \par}&
    {{\resizebox*{0.35\textwidth}{0.2\textheight}
        {\rotatebox{0}{\includegraphics{#2}}}} \par}\\
\end{tabular}
}
\newcommand{\figtwoA}[2]
{
\begin{tabular}{cc}
%    {{\resizebox*{0.33\textwidth}{0.24\textheight}
    {{\resizebox*{0.45\textwidth}{0.45\textheight}
        {\rotatebox{0}{\includegraphics{#1}}}} \par}&
    {{\resizebox*{0.45\textwidth}{0.45\textheight}
        {\rotatebox{0}{\includegraphics{#2}}}} \par}\\
\end{tabular}
}
\newcommand{\figtwoB}[2]
{
\begin{tabular}{cc}
%    {{\resizebox*{0.33\textwidth}{0.24\textheight}
    {{\resizebox*{0.6\textwidth}{0.7\textheight}
        {\rotatebox{0}{\includegraphics{#1}}}} \par}&
    {{\resizebox*{0.6\textwidth}{0.7\textheight}
        {\rotatebox{0}{\includegraphics{#2}}}} \par}\\
\end{tabular}
}
\newcommand{\figtwoC}[4]
{
\begin{tabular}{cc}
%    {{\resizebox*{0.33\textwidth}{0.24\textheight}
  #3 & #4 \\
    {{\resizebox*{0.47\textwidth}{0.55\textheight}
        {\rotatebox{0}{\includegraphics{#1}}}} \par}
     &
    {{\resizebox*{0.47\textwidth}{0.55\textheight}
        {\rotatebox{0}{\includegraphics{#2}}}} \par}\\
\end{tabular}
}

%\newcommand{\figone}[1]
%{
%  \begin{center}
%    {{\resizebox*{0.95\textwidth}{0.7\textheight}
%        {\rotatebox{270}{\includegraphics{#1}}}} \par}
%  \end{center}
%  }

\newcommand{\figoneA}[1]
{
  \begin{center}
    {{\resizebox*{0.35\textwidth}{0.35\textheight}
        {\rotatebox{0}{\includegraphics{#1}}}} \par}
  \end{center}
  }

\newcommand{\figoneAA}[1]
{
  \begin{center}
    {{\resizebox*{1\textwidth}{0.3\textheight}
        {\rotatebox{0}{\includegraphics{#1}}}} \par}
  \end{center}
  }

\newcommand{\figoneAAbig}[1] % Sham's version
 {
  \begin{center}
    {{\resizebox*{0.7\textwidth}{0.7\textheight}
        {\rotatebox{0}{\includegraphics{#1}}}} \par}
  \end{center}
  }
  
 
  \newcommand{\figoneAB}[1]
{
  \begin{center}
    {{\resizebox*{0.6\textwidth}{0.6\textheight}
        {\rotatebox{0}{\includegraphics{#1}}}} \par}
  \end{center}
  }
  
    \newcommand{\figoneAC}[1]
{
  \begin{center}
    {{\resizebox*{0.65\textwidth}{0.65\textheight}
        {\rotatebox{0}{\includegraphics{#1}}}} \par}
  \end{center}
  }

    \newcommand{\figoneAD}[1]
{
  \begin{center}
    {{\resizebox*{0.70\textwidth}{0.70\textheight}
        {\rotatebox{0}{\includegraphics{#1}}}} \par}
  \end{center}
  }

\newcommand{\figonebig}[1]
{
  \begin{center}
    {{\resizebox*{0.7\textwidth}{0.7\textheight}
        {\rotatebox{0}{\includegraphics{#1}}}} \par}
  \end{center}
  }
\newcommand{\figtwoAA}[2]
{
\begin{tabular}{cc}
%    {{\resizebox*{0.33\textwidth}{0.24\textheight}
    {{\resizebox*{0.45\textwidth}{0.45\textheight}
        {\rotatebox{0}{\includegraphics{#1}}}} \par}&
    {{\resizebox*{0.45\textwidth}{0.45\textheight}
        {\rotatebox{0}{\includegraphics{#2}}}} \par}\\
\end{tabular}
}

%\newcommand{\figonesmall}[1]
%{
%  \begin{center}
%    {{\resizebox*{0.475\textwidth}{0.35\textheight}
%        {\rotatebox{0}{\includegraphics{#1}}}} \par}
%  \end{center}
%  }

%\newcommand{\figonerotate}[1]
%{
%  \begin{center}
%    {{\resizebox*{0.7\textwidth}{0.7\textheight}
%%    {{\resizebox*{0.95\textwidth}{0.7\textheight}
%        {\rotatebox{0}{\includegraphics{#1}}}} \par}
%  \end{center}
%  }
% \newcommand{\figonerotatemedium}[1]
% {
%  \begin{center}
%    {{\resizebox*{0.95\textwidth}{0.7\textheight}
%    {{\resizebox*{0.855\textwidth}{0.63\textheight}
%    {{\resizebox*{0.665\textwidth}{0.49\textheight} % with image-transform-autocrop in gimp (R map files)
%        {\rotatebox{0}{\includegraphics{#1}}}} \par}
%  \end{center}
%  }
%\newcommand{\figonerotatemediumsmall}[1]
%{
%  \begin{center}
%%    {{\resizebox*{0.95\textwidth}{0.7\textheight}
%%    {{\resizebox*{0.855\textwidth}{0.63\textheight}
%    {{\resizebox*{0.532\textwidth}{0.392\textheight} % with image-transform-autocrop in gimp (R map files)
%        {\rotatebox{0}{\includegraphics{#1}}}} \par}
%  \end{center}
%  }
%% this one preserves perspective (so circle stays a circle)
%\newcommand{\figonerotatepersp}[1]
%{
%  \begin{center}
%    {{\resizebox*{0.6\textwidth}{0.7\textheight}
%%    {{\resizebox*{0.95\textwidth}{0.7\textheight}
%        {\rotatebox{0}{\includegraphics{#1}}}} \par}
%  \end{center}
%  }

%% this one preserves perspective (so circle stays a circle)
%\newcommand{\figonerotatebig}[1]
%{
%  \begin{center}
%%    {{\resizebox*{0.6\textwidth}{0.7\textheight}
%    {{\resizebox*{0.95\textwidth}{0.9\textheight}
%        {\rotatebox{0}{\includegraphics{#1}}}} \par}
%  \end{center}
%  }

\newcommand{\head}[1]
{
  \begin{center}
      {\huge {\color{blue} #1}}
    \end{center}
  }
%\definecolor{pink}{rgb}{1,0.5,0.5}
% 2 x 2 fill up the entire slide, also allow labeling
\newcommand{\figtwobytwoB}[8]
{
\begin{tabular}{cc}
  #5 & #6\\
    {{\resizebox*{0.5\textwidth}{0.3\textheight}
        {\rotatebox{0}{\includegraphics{#1}}}} \par}&
    {{\resizebox*{0.5\textwidth}{0.3\textheight}
        {\rotatebox{0}{\includegraphics{#2}}}} \par}\\
  #7 & #8\\
    {{\resizebox*{0.5\textwidth}{0.3\textheight}
        {\rotatebox{0}{\includegraphics{#3}}}} \par}&
    {{\resizebox*{0.5\textwidth}{0.3\textheight}
        {\rotatebox{0}{\includegraphics{#4}}}} \par}\\
\end{tabular}
}
\newcommand{\framenum}{\vspace{-1.45cm}
  \begin{flushright} \textcolor{black}{\normalsize \insertframenumber{$\quad$}} \end{flushright}
  \vspace{-1 cm} }              % This adds the frame number to the upper-right corner of each slide 
\newcommand{\Kv}{\color{blue} \mathbf{K_v}}
%\newcommand{\bluebtheta}{\color{blue} \mbox{\boldmath $\theta$}}
\newcommand{\bluebtheta}{\mbox{\color{blue} \boldmath $\theta$}}
