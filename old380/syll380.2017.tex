\documentclass[10pt]{article}
\usepackage[sort,longnamesfirst]{natbib}
%\usepackage{psfig}
\usepackage{amsmath}%
\usepackage{hyperref}
\begin{document}
%{\bf TENTATIVE SCHEDULE}\\
\pagestyle{empty}
\Large
\begin{center}
{\bf  Syllabus STAT 380 (Spring 2017)}\\
{\bf Data Science Through Statistical Reasoning and Computation}\\
\end{center}
\normalsize {\bf Instructor}: Murali Haran, Professor, Department of Statistics, Penn State University,
University Park, PA. \\ Office: 421D Thomas
Building. \:Phone: 863-8126. \\ Office
Hours: Tuesdays and Wednesdays 1:30-2:30pm\\\\
{\bf Teaching Assistant}: John Ensley, PhD Student, Statistics \\
Office Hours Location: 330A Thomas Bldg \:email: {\tt john.ensley@psu.edu}\\
Office Hours: Tuesdays and Wednesdays 3-4pm  \\\\ %or 2:30-3:30?
{\bf Communication}: You can reach me and the TA through email via Canvas.\\\\ %Wed , Thu:  \\\\ 
%\\%  Office: \:\: Thomas Building \: Phone: \:\: \:
% email:\:\:\\ Office Hours: TBA\\\\ 
{\bf Class Times}: Tue/Thu 10:35am-11:50am in 362 Willard\\\\
{\bf Textbook}: {\it Data Technologies and Computational Reasoning} by D. Nolan and D. Temple Lang (pdf files will be posted weekly online on Canvas)\\
Supplement: {\it Data Science in R: A Case Studies Approach to Computational Reasoning} by Nolan and Temple Lang.\\\\
{\bf Course Schedule}: \url{http://sites.stat.psu.edu/~mharan/380/schedule.2017.html} This will change so keep checking back for updates!\\\\
{\bf Canvas and Course Website}: The course will be run largely through Canvas but I also have a website for public access \url{http://www.stat.psu.edu/~mharan/380/380.html}\\\\
%{\bf Important}: The course schedule may change according to how the class progresses. Check the website for all announcements regularly.\\\\
%{\bf Course Requirements} 
{\bf Final Course Grades will be based on:} \\
 Weekly homework + projects: 50\%.\\ % TOO MUCH: REDUCE TO 20% NEXT TIME
% Two midterm exams: 20\% $\times$ 2.
Midterm exam: 20\%\\
{\it Tentative midterm date: Thursday, March 2nd (just before spring break).}\\
Final exam: 30\%. (The final will be comprehensive.)\\\\
\noindent {\bf Homework grading:} Each homework counts for 30 points:
5 each for four selected problems (selected after submission) and 10 
for an organized and easy-to-follow completed assignment, including
{\it non-graded} problems. If there are no more than five problems in
the assignment, all of them will be graded and points assigned
appropriately. You will not receive full credit unless you show your
work when solving each problem.  You may discuss
them with others but they  must be written up independently.\\\\
 % \noindent {\bf Computing:} To supplement your understanding of
 % concepts, I will assign problems for you to do using the statistical
 % software {\tt R} which can be downloaded at no cost for any platform
 % from \url{http://www.r-project.org/}.\\
\newpage
% {\bf Final Grades:} Your final grade will be no lower than\\
% \indent A    : 93-100\%,     A- : 90-92\% \\
% \indent B+ : 87-89\%,       B  : 83-86\%,    B- : 80-82\%, \\
% \indent C+ : 77-79\%,       C  : 70-76\% \\
% \indent D   : 60-69\% ,      F   :   0-59\% \\

{\bf Course Rules:}
\begin{enumerate}
\item No make-up exams will be given for ANY reason. If you miss an
  exam, you should provide a valid reason to the instructor; once the
  instructor approves your excuse, your missed exam score will be
  replaced by some combination of your other course grade.
\item Early exams might be allowed, with prior arrangement, for
  students with direct conflicts due to other required university
  activities (chess team, field trip, Blue Band trip, etc.) The
  director of that program must provide a letter requesting that you
  be excused.
\item Homework will typically be due on Thursdays {\it in class}. You
  can submit it in my mailbox in Thomas 326 by 3:30pm on the same day
  it is due with a 20\% reduction in your score. No late homework will
  be accepted after that time under any circumstance. {\it You have 1
    week to appeal a grade. No grade changes will be made 1 week after
    a graded homework or exam is returned.}  Your two lowest homework
  scores will be dropped when computing your final grade, which should
  allow for missing homework due to unavoidable circumstances.
\item Students are responsible for all announcements and supplements
  given within any lecture and email.
\item Academic Integrity and Mutual Respect: All Penn State University, College of Science, and 
Department of Statistics policies regarding ethics, honorable behavior, and mutual respect apply in 
this course. %These can be found at the websites: 
\begin{itemize}
\item Penn State's Policies \url{http://www.psu.edu/ufs/policies/}
\item College of Science's Academic Integrity Policy \url{http://science.psu.edu/current-students/Integrity/Policy.html}
\item College of Science's Code of Mutual Respect and Cooperation
  \url{http://science.psu.edu/climate/code-of-mutual-respect-and-cooperation}
% The College of Science's Code of Mutual Respect and Cooperation embodies the values that we hope 
% our faculty, staff, and students possess and will endorse to make the College of Science a place 
% where every individual feels respected, valued, as well as
% challenged and rewarded. \\
\end{itemize}
\item If you have a disability-related need for
reasonable academic adjustments in this course, contact the Student Disability Resources (SDR) at 814-863-1807 or visit their website \url{http://equity.psu.edu/student-disability-resources}
% If you have a disability-related need for modifications in the course,
%contact the Office of Disability Services, and let the instructor know. See \url{http://www.equity.psu.edu/ods/}
% \item If you need to leave class early, please sit in the rear and
%   leave as quietly as possible. 
% \item Please turn off all electronic devices (cell-phones, pagers,
%   etc.) BEFORE you enter the classroom. 
\end{enumerate}

%  {\bf Academic Integrity:} All Penn State and Eberly College of
%  Science policies regarding
%  academic integrity apply to this course.  Please see\\
%  \url{http://www.science.psu.edu/academic/Integrity/index.html}.
% \noindent {\bf The Eberly College of Science Code of Mutual Respect
%    and Cooperation}

% \noindent
%  \url{http://www.science.psu.edu/climate/code-of-mutual-respect-and-cooperation-1} embodies
%    the values that we hope our faculty, staff, and students possess
%    and will endorse to make The Eberly College of Science a place
%    where every individual feels respected and valued, as well as
%    challenged and rewarded.

%\newpage
\end{document}
