\documentclass[10pt]{article}
\usepackage[sort,longnamesfirst]{natbib}
%\usepackage{psfig}
\usepackage{amsmath}%
\usepackage{hyperref}
\begin{document}
%{\bf TENTATIVE SCHEDULE}\\
\pagestyle{empty}
\vspace{-0.5in}
\Large
\begin{center}
{\bf Syllabus for Penn State STAT 540, Fall 2019}\\
{\bf Computationally Intensive Statistical Inference}\\
\end{center}
\normalsize {\bf Instructor}: Murali Haran, Professor, Department of Statistics, Penn State University, University Park, Pennsylvania.\\ Office: 326 Thomas
Building \\ Office
Hours: Tuesdays 1-2 pm or by appointment\\\\ 
{\bf Grader}: Vincent Pisztora, PhD Student \\  
%\hspace{2in} \:\:\:\:\:\:\:\:\\  
Office Hours: by appointment \\\\
{\bf Email communication}: You can reach me and the TA through email via Canvas.\\\\ %Wed , Thu:  \\\\ 
{\bf Class Times}: TuThu 10:35-11:50 am in Boucke 306.\\\\
{\bf Prerequisites}: STAT 513 and 514 or equivalent Casella and Berger
level mathematical statistics sequence, and at least one course in computer
programming (undergraduate level would suffice)\\\\
{\bf Textbook}: None; occasional lecture notes posted on Canvas {\it (please do 
 not distribute).} \\References: Numerical Analysis for Statisticians by K. Lange and Computational Statistics by G.H. Givens and J.A. Hoeting. \\\\
%Monte Carlo Statistical Methods by C.P. Robert and G. Casella, Springer.\\\\
%{\bf Important}: The course schedule may change according to how the class progresses. Check the website for all announcements regularly.\\\\
{\bf Coverage}: The main topics covered in the course are:
\begin{itemize}
\item Statistics-relevant computing basics; matrix computations
\item Numerical integration, Laplace approximations 
\item Monte Carlo methods:\ foundations, importance sampling, Markov
  chain Monte Carlo
\item Bootstrap
\item Optimization:\ unconstrained; second and first order methods, including stochastic gradient
\item Surrogate methods: EM/MM %algorithms
\item Advanced topics (time permitting) 
%\item Practice with: (a) programming in \href{http://www.r-project.org/}{{\bf R}}, (b) \href{http://www.python.org/}{{\bf python}}, (c) using
%  \href{http://www.stat.uni-muenchen.de/~leisch/Sweave/}{{\bf Sweave}} for
%  literate programming 
\end{itemize}
{\bf Course Website}: Main:
\url{http://personal.psu.edu/muh10/540/540.html}\\ Course schedule:
\url{http://personal.psu.edu/muh10/540/schedule540.Fall2019.html} Please bookmark these website. I will use the course
website in tandem with Canvas for course related communications. 
\newpage

\noindent 
{\bf Course Requirements:}
\begin{itemize}
\item Homework (50\%). You may discuss them but they {\it must be written
  up independently}. The homework assignments may vary in length and
  difficulty, and hence may differ in the number of points they are worth.
\item Course project (50\%: proposal + reports + presentation). I
  expect this to be a substantial project. Possibilities include:
  original research, review of existing methods, extensive simulation
  studies, or some combination of all of the above. I will determine
  whether the scope of your project is appropriate for this
  course. Important: {\bf (1) The project must be focused on
    algorithms used for statistics/probability; (2) You must obtain my
    approval for the topic; (3) You may not submit a project from a
    different class for this class. Also, your project cannot be
    thesis research that you were planning to do before you began this
    course.}
%\item Take home final (10\%) tentatively {\bf out Wednesday April 17, due Friday April 22}
%\item Final exam: in class (35\%) TBA
\end{itemize} 
{\bf Course Rules:}
\begin{enumerate}
\item Homework will be due online {\bf on Canvas}. Unless you contact me with a good reason ahead of time ({\it at least 1 day
    in advance}), the following late policies hold: submit your
  homework by midnight on the same day with
  a 20\% reduction or 10:30 am the next day with a 50\% reduction in
  your score. No late homework will be accepted after that time under
  any circumstance. {\it You have 1 week to appeal any grade. No grade
    changes will be made 1 week after a graded homework is returned.}  
\item You are welcome to use any computer language you like, as long
  as you make it easy for the TA to grade your work, and run your code. I strongly recommend you use {\tt R} since most of the class examples will be in {\tt R}.  
\item Homework submissions: All students are required to submit {\it
    typed} computing assignments. Statistics graduate students are
  required to use {\tt LaTeX} to write up assignments. % I encourage using {\tt Sweave} for assignments.
% Obviously theoretical work need not be typed up. The instructor will provide sample {\tt
  % LaTeX} files to interested students.
\item Academic Integrity and Mutual Respect: All Penn State University, College of Science, and 
Department of Statistics policies regarding ethics, honorable behavior, and mutual respect apply in 
this course. %These can be found at the websites: 
\begin{itemize}
\item Penn State's Policies \url{https://studentaffairs.psu.edu/support-safety-conduct/student-conduct/code-conduct}
\item College of Science's Academic Integrity Policy \url{http://science.psu.edu/current-students/Integrity/Policy.html}
\item College of Science's Code of Mutual Respect and Cooperation
  \url{http://science.psu.edu/climate/code-of-mutual-respect-and-cooperation}
% The College of Science's Code of Mutual Respect and Cooperation embodies the values that we hope 
% our faculty, staff, and students possess and will endorse to make the College of Science a place 
% where every individual feels respected, valued, as well as
% challenged and rewarded. \\
\end{itemize}
\item If you have a disability-related need for
reasonable academic adjustments in this course, contact the Student Disability Resources (SDR) at 814-863-1807 or visit their website \url{http://equity.psu.edu/student-disability-resources}
\end{enumerate}
% {\bf Academic Integrity:} All Penn State and Eberly
% College of Science policies regarding
% academic integrity apply to this course.  Please see\\
% {\tt http://www.science.psu.edu/academic/Integrity/index.html}.
% \\\\
% \noindent {\bf Policy on Late Homework:}\\
% Homework should be turned in in class on the day it is due.  Unless you inform
% me ahead of time ({\it at least 1 day in advance}), you will lose 50\%
% of the grade for late homework turned in within 24 hours of the due
% date/time. You will receive no credit for homework turned
% in after that.\\\\
%\noindent {\bf Important}: Check the website for all announcements regularly.\\\\
\end{document}
