%\documentclass[landscape]{article}
%\documentclass{article}
\documentclass{report}
\usepackage{amsmath,doublespace,fullpage,color,graphics,amsfonts}
\usepackage{indentfirst}
\usepackage{pstricks}
\input colordvi

\newcommand{\head}[1]
{
  \begin{center}
      {\huge {\color{blue} #1}}
    \end{center}
  }

\newcommand{\figone}[1]
{
  \begin{center}
    {{\resizebox*{0.95\textwidth}{0.7\textheight}
        {\rotatebox{360}{\includegraphics{#1}}}} \par}
  \end{center}
  }
\newcommand{\figonelabel}[2]
{
  \begin{center}
    {{\resizebox*{0.95\textwidth}{0.7\textheight}
        {\rotatebox{270}{\includegraphics{#1}}}} \par}
  \end{center}
  #2
  }

\newcommand{\figonesmall}[1]
{
  \begin{center}
    {{\resizebox*{0.475\textwidth}{0.35\textheight}
        {\rotatebox{270}{\includegraphics{#1}}}} \par}
  \end{center}
  }
%% \newcommand{\figonesmalllabel}[2]
%% {
%%   \begin{center}
%%     {{\resizebox*{0.475\textwidth}{0.35\textheight}
%%         {\rotatebox{270}{\includegraphics{#1}}}} \par}
%%     #2
%%   \end{center}
%%   }

\newcommand{\figonesmalllabel}[2]
{
  \begin{center}
    {{\resizebox*{0.57\textwidth}{0.42\textheight}
        {\rotatebox{270}{\includegraphics{#1}}}} \par}
    #2
  \end{center}
}

\newcommand{\figtwo}[4]
{
  \begin{tabular}{cc}
    {{\resizebox*{0.47\textwidth}{0.35\textheight}
        {\rotatebox{270}{\includegraphics{#1}}}} \par}&
    {{\resizebox*{0.47\textwidth}{0.35\textheight}
        {\rotatebox{270}{\includegraphics{#2}}}} \par}\\
    #3 & #4
  \end{tabular}
  }

\newcommand{\figtwosmall}[4]
{
  \begin{tabular}{cc}
    {{\resizebox*{0.4\textwidth}{0.2\textheight}
        {\rotatebox{270}{\includegraphics{#1}}}} \par}&
    {{\resizebox*{0.4\textwidth}{0.2\textheight}
        {\rotatebox{270}{\includegraphics{#2}}}} \par}\\
    #3 & #4
  \end{tabular}
  }

\newcommand{\figthree}[6]
{
\begin{tabular}{ccc}
    {{\resizebox*{0.32\textwidth}{0.23\textheight}
        {\rotatebox{270}{\includegraphics{#1}}}} \par}&
    {{\resizebox*{0.32\textwidth}{0.23\textheight}
        {\rotatebox{270}{\includegraphics{#2}}}} \par}&
    {{\resizebox*{0.32\textwidth}{0.23\textheight}
        {\rotatebox{270}{\includegraphics{#3}}}} \par}\\
    #4 & #5 & #6
\end{tabular}
}
\newcommand{\figtwobytwo}[4]
{
\begin{tabular}{cc}
    {{\resizebox*{0.47\textwidth}{0.35\textheight}
        {\rotatebox{270}{\includegraphics{#1}}}} \par}&
    {{\resizebox*{0.47\textwidth}{0.35\textheight}
        {\rotatebox{270}{\includegraphics{#2}}}} \par}\\
    {{\resizebox*{0.47\textwidth}{0.35\textheight}
        {\rotatebox{270}{\includegraphics{#3}}}} \par}&
    {{\resizebox*{0.47\textwidth}{0.35\textheight}
        {\rotatebox{270}{\includegraphics{#4}}}} \par}\\
\end{tabular}
}
\newcommand{\bThet}{{\bf \Theta}}
\newcommand{\bpi}{{\bf \pi}}
\newcommand{\datU}{{\cal U}}
\newcommand{\nodeS}{{\cal S}}
\newcommand{\nodeT}{{\cal T}}
\newcommand{\nodeD}{{\cal D}}
\newcommand{\nodeA}{{\cal A}}
%\newcommand{\Snodelist}{{\alpha_P,\alpha_C,\alpha_S,\alpha_R,\alpha_F,\alpha_{DOWN},\alpha_{TR},\alpha_{CAL},\alpha_{TFEES},\alpha_H,\alpha_X,\alpha_O}}
\newcommand{\Snodelist}{{\alpha_P,\alpha_C,\alpha_S,\alpha_R,\dots,\alpha_X,\alpha_O}}
\newcommand{\Snodetrunc}{{\alpha_P,\dots,\alpha_O}} % truncated version
\begin{document}
\thispagestyle{empty}
\pagestyle{plain}
\phantom{Blah blah blah\\}
\phantom{Blah blah blah\\}
\phantom{Blah blah blah\\}
\phantom{Blah blah blah\\}
\phantom{Blah blah blah\\}
\Huge
\begin{center}
%% PROBABILISTIC MODELING OF \\
%% BROWSING BEHAVIOR ON THE \\
%% WORLDWIDE WEB
A MODEL FOR BROWSING\\
BEHAVIOR ON THE \\
WORLDWIDE WEB
\end{center}
\huge
\phantom{Blah blah blah\\}
\begin{center}
Murali Haran\\
(joint with Alan Karr, and Ashish Sanil)\\
National Institute of Statistical Sciences.\\
\LARGE JSM Toronto, August 2004.
\end{center}
\newpage \head{THE PROBLEM}
\begin{itemize}
\item A critical problem for users of the World Wide Web is that many
  sites are difficult to navigate, hard to use and have confusing
  structure.
\item User behavior may therefore be inconsistent with its structure:
  worse yet, users may be unable to find content and abandon the site.
\item Difficult to create easy-to-use sites, and hard to understand how visitors use the site.
\item Conducting formal user studies: usually too expensive.
\item An approach: exploit the rich instrumentation in the
  on-line world.
%% \begin{itemize}
%% \item Web servers create voluminous log files that record
%%   every hit to every page on the site.
%% \item By processing the log files we
%%   can create sessions sequences of page views for each user.
%% \item Web reporting tools can produce simple reports summarizing site
%%   activity.
%% \end{itemize}
%% \item Essentially impossible to relate site activity to site structure from this summary.
%% \item Bayesian model: (a) are user transitions consistent with the Web
%%   site structure? (b) make stochastic predictions.
\end{itemize}
%\newpage \head{PRELIMINARIES: DATA}
\newpage \head{MOTIVATING EXAMPLE}
\underline{Commerce website}:\\ % information from eda.res, eda.res2 files
Web servers create large log files that record
every hit to every page on the site.\\
By processing the log files we can create sequences of page
views for each user for each session.\\
%Page sequences are associated with each session.\\
Instrumentation of user sessions for 6 months in 2002.\\
\underline{Numbers}:\\
Visitors: 238595 (820536 including length 1 sessions).\\
Sessions: 344227 (1034915 including length 1 sessions)\\
Number of pages, $n_p$: 5575 (5640 including length 1 sessions)
%Number of nodes, $n_{\tau}$: 316 (mapped from 5575 pages).  
%% How well does the website usage match the design?\\
%% Simulate page sequences (for testing purposes).  

\newpage \head{SCALABILITY, INTERPRETABILITY} Since we are interested in how
users navigate through the website, natural to look at transition
matrix.\\
%$\{M_{i,j},\: i,j \in {\cal P}\}$,=set of pages, $M_{i,j}=$P(move from $i$ to $j$)\\
$M_{i,j}=$P(move from $i$ to $j$), $i,j \in {\cal P}$,=set of pages, \\
Some problems with this:
\begin{itemize}
\item Matrix gets large very quickly (e.g. $5575 \times 5575$).
\item Hard to interpret a large transition matrix and relate it to design.
\item Difficult for website designer to describe overall design in
  terms of individual pages, particularly since individual
  pages may have been dynamically generated.
\item No idea of uncertainty associated with the estimate of each transition probability.
\end{itemize}

\newpage \head{WEBSITE DESIGN}
\begin{itemize}
\item Expert information elicited from the user provides
 list of $n_s$=316 nodes (groupings) of the web pages.
\item Design is expressed as a tree containing all such nodes.
\item The transitions consistent with the tree structure:\\
parent-child (P), child-parent (C), self (R), sibling (S).
\item Several nodes (6) considered ``special'':\\
Homepage(H), FAQ (F), Tutorial (U), Downloads (D), Fees (T), Images(I).
\item A transition to a special node is also considered to be consistent with tree structure.
\item We can now assess consistency with design.
\end{itemize}
%% Level 1 (root): never appear in data \\% 1518146, 1518151, 1518152
%% Level 2: 18 nodes: CybertraderU and FAQ.\\
%% Level 3: 96 nodes: Simulator download, Home/Welcome.\\
%% Level 4: 139 nodes: Cybertrader Pro Images.\\
%% Level 5: 38 nodes: Fees/Trading is important\\
%% Level 6: 25 nodes.
\newpage \head{PARTIAL VIEW OF NODE HIERARCHY}
\vspace{1in}
%\figone{treepic4.ps}
\figone{treepic4.jpg}

\newpage
\head{DATA REDUCTION}
\begin{itemize}
%% \item Page sequence data: $p_1,p_2,\dots,p_n$
%% \item Node sequence data: translate each page into its
%%   corresponding node, obtain: $\nu_1,\nu_2,\dots,\nu_n$.
\item Page sequence data: $p_1,p_2,\dots,p_n \Rightarrow
  \nu_1,\nu_2,\dots,\nu_n$, by translating each page into
  corresponding node.
\item Entry node, $E$, represents all pages outside the website from
  which the user first entered, $X$ represents all outside
  destinations. Single session: $E, \nu_1,\nu_2,\dots,\nu_n, X$.
\item Use node hierarchy information to classify each transition as
  parent-child (P), sibling (R), other (O) etc.
%  C, R, S, H, F, U, D, T, I, X, or O (other).
\item Can build corresponding transition count matrix, $\{TC(i,j)\}$.\\ 
$TC(i,j)=P$(move from node $i$ of transition type $j$).\\
(316 $\times$ 12 instead of 5575 $\times$ 5575).
%% \item Page hierarchy information: build minimal tree that contains all
%%   nodes that appear in the data. 
\end{itemize}
%% \newpage
%% \item Using above information: now possible to ascertain if the
%%   transition between nodes $u$ and $v$ represents a parent-child,
%%   child-parent, or sibling transition. A transition that does not
%%   qualify as one of these tree-transitions, is then categorized as
%%   either a self (repeat) transition or a transition to one of the
%%   special pages (FAQ, homepage, or exit). A transition that does not
%%   fit into any of these categories is simply categorized as 'other'.
%% \item Node transition matrix: can classify a transition from any node
%%   $\nu_i$, to another node $\nu_{i+1}$ as one of 8 possible
%%   transitions: parent-child (P), child-parent (C), self/repeat (R),
%%   sibling (S), to FAQ page (F), to homepage (H), to exit (X), to other
%%   (O). Corresponding transition matrix is of dimension: $n_{\tau}\times n_{\tau}$

\newpage \head{MODEL 1}
$\nodeS=\{206985, 1517987,\dots\}$: set of $n_s$ (316) nodes.\\
%$\nodeT=\{P,C,S,R,F,H,'DO,'TR','CA,'TF,X,O\}$ is set of $n_t$ (12) node types.\\
$\nodeT=\{P,C, R, S, H, F, U, D, T, I, X, O\}$: transition types.\\
%Let $\nodeD=\{P,C, R, S\}, \nodeA=\{H, F, U, D, T, I\}$.\\
Let $\datU$ be the data, and all parameters of the model be $\Theta$\\
$\pi_{i,j}$= probability of a type $j$ transition from node $i$.\\
$\pi_{E,i}$= transition prob. from entry node to node $i$.
\begin{equation*}
  L(\datU | \Theta)= \prod_{i \in \nodeS} \pi_{E,i}^{TC(E,i)} \prod_{i \in \nodeS,j \in \nodeT} \pi_{i,j}^{TC(i,j)}
\end{equation*}
Let $\bpi_E$=$\{\pi_{E,i}, i \in \nodeS\}$, $\bpi_i=\{\pi_{i,j}, j \in \nodeT\}$.
\\Dirichlet priors: $\bpi_i  \sim Dir (\Snodelist)$
Use Jeffrey's prior for $\Snodetrunc$ (Yang and Berger, 1996).\\
Dirichlet prior on the entry page probabilities, $\bpi_E$.

\newpage \head{SOME RESULTS} %% If at node $\nu_i$, the odds are greater than
%% 1 of taking a ``designed'' transition versus a non-designed
%% transition: user behavior at the node is according to design.\\
%\frac{\sum_{j \in {\cal D}} \pi_{ij}}{\pi_{iO}}=
MCMC to obtain posterior distribution: $\bpi_E,\bpi_i|\datU$.\\
Odds of taking a ``designed'' transition versus a non-designed
transition from node $\nu_i$ is $\frac{1-\pi_{iO}}{\pi_{iO}}$.\\
Let 95\% credible interval for $\pi_{iO}|\datU$ be $(l,u)$.\\
If $0.5 < (l,u)$, $\nu_i$ is ``bad'', if $ (l,u) < 0.5$, $\nu_i$ is ``good''.\\
%Hypothesis test is then: Does 95\% credible set for $\pi_{jO}$? $p(\pi_0<0.5|\datU)$
%95\% credible intervals for the transition probabilities to pages of type ``other'':\\
12.6\% of all nodes are bad.\\
42.4\% of all nodes are good.\\
\indent $\Rightarrow$ should redesign website, targeting good nodes.\\
%For 44.9\% of the pages (interval includes 0.5).\\
$ p\left( \sum_{k\in \{H, F, U, D, T, I, X\}}\pi_{ik}>\sum_{k\in \{P,C, R, S\}}\pi_{ik}|\datU\right)>0.9$ for around 85\% of nodes, so special nodes are more important destinations than tree-defined transitions. (Good or bad?)
%Designate nodes with largest median values for $p(\pi_{iO}|\datU)$ as ``poor'' %2.5 percentiles
%, and smallest median values as ``good''.%97.5 percentiles

\newpage \head{RESULTS: WORST AND BEST NODES}
{\LARGE
\underline{BEST NODES}
\begin{verbatim}
Download Step1, Download Step2
Cancellations, Supporting Materials (NAF)
\end{verbatim}
} 
Nodes that are related to simulation are the best (NAF-related, download Steps1,2,..).\\
Not surprising since download steps are sequential.\\
{\LARGE
\underline{WORST NODES}
\begin{verbatim}
Flash Demo - CyberX2, Account Information
CyberTrader Pro Download ,  newsletter
\end{verbatim}
} Many of the worst nodes are low in tree hierarchy, have few
children, many siblings: relationship between consistency with design
and $\{$level, children or siblings$\}$?  

\newpage \head{ALTERNATIVE
  MODEL} Let probabilities for a designed transition from $\nu_i$ be
$\theta_{i1}$, and for a non-designed transition be $\theta_{i0}$.\\
We can directly model the odds of making a designed transition versus
a non-designed transition:
\begin{equation*}
\log\left( \frac{\theta_{i1}}{\theta_{i0}}\right)=\phi_i
\end{equation*}
%% The log odds, $\phi_i$, is modeled as the sum of $\theta_i$, an
%% overall tendency for the page to encourage users to follow the design,
%% $l_i$, the level of the node in the hierarchy, $s_i$, the number of
%% siblings, and $c_i$, the number of children:
\begin{equation*}
  \phi_i = \gamma_i + \alpha_L L_i + \alpha_C C_i + \alpha_S S_i 
\end{equation*}
$L_i$ is the level of the node in the hierarchy.\\
$S_i, C_i$ are the number of siblings,children respectively.\\
$\gamma_i$: how well the node was designed (accounting for $L_i,S_i,C_i$)\\
Hyperpriors: $\theta_i
\sim N(\mu,\tau_{\theta}^{-1})$, and mean zero Gaussians for
$\alpha_L,\alpha_C,\alpha_S$ with appropriate hyperpriors, parameters.

\newpage\head{ALTERNATIVE MODEL: SOME RESULTS} %% \begin{equation*}
%%   \theta_i \sim N(\mu,\sigma^2),\:\: \beta \sim N(0,\sigma^2_{\beta}),\:\: \eta \sim N(0,\sigma^2_{\eta}),\:\: \gamma \sim N(0,\sigma^2_{\gamma})
%% \end{equation*}
%% A study of the distributions of the $\phi_i$s over all $i$: an overall
%% measure of how user behavior compares to the design. 
\begin{itemize}
\item Number of siblings of a node has a negative effect: for every additional sibling, 
 odds of moving to a non-designated node increases by a factor of 1.3.
\item Neither level of node in hierarchy, nor the number of children significantly 
affects how well the node follows the design. % means: -0.28 0.07 -1.09
\item If $\phi_i<0$, indication that node $i$ is behaving contrary to design.\\
  Forming credible intervals: 18\% of the nodes exhibit user behavior
  clearly inconsistent with design. %13\%=(42/316)
\item Best and worst nodes: roughly same as described before.
% 1/exp(-0.25)
%% Model 2 suggests there is a greater mismatch between the web design
%% and user behavior than Model 1.
\end{itemize}

%% \newpage\head{MODELS 1 AND 2} Both models are intuitive and seem fairly
%% reasonable ways to study the behavior of users on the website,
%% relative to its design.\\
%% However, they produce somewhat different conclusions: Model 1 suggests
%% over 40\% of nodes are well designed, and only 12\% are poorly
%% designed, while Model 1 suggests only 13\% of the nodes are well
%% designed.\\
%% (model choice issue)\\
%% Model 2 can be extended to answer some of the questions that Model 1
%% answers: can model the probability of picking a specific transition,
%%  given that user makes a designed transition.

\newpage \head{SUMMARY AND ISSUES}
\begin{itemize}
\item Have provided a method that uses expert information to quickly
  reduce dimensionality and increase interpretability -
  even more useful for larger websites.
\item Commerce data: browsing behavior does not appear to completely follow the
  design of the site (as narrowly defined by the node hierarchy and
  special pages).
%\item Need a more thorough analysis and comparison of the two models.
%% \item Fundamental issue: perhaps problem lies in the description
%%   of the design of the website%%rather than with the implementation and usability.
\item Description of the website design:
\begin{itemize}
\item Not clear that tree-like structure is adequate: too many
  common links appear on multiple pages - cannot all be accomodated as
  `special nodes'.
\end{itemize}
%% \item May first need alternative ways to assess both design {\it and}
%%   usability of websites.
\end{itemize}

\newpage \head{SUMMARY AND ISSUES}
\begin{itemize}
%% \item Too many dynamically generated pages using {\tt Java} scripts - users
%%   may appear to get from node A to an unrelated node C, when a java
%%   script acted as intermediate node B: important user transitions
%%   never get recorded.
\item Experts' characterization of poor design may not be best:
%%   drawing any important conclusions.
\begin{itemize}
\item Find way to define `lost' users: for eg. frequent transitions
  to homepage and exit pages - high occurrence implies poor design.
\item Try to classify whether a particular user visit/sequence was
  successful - many failures implies poor design. Easy for e-commerce,
  but very difficult to assess for others.
%\item Ultimately, data quality dictates how well we can perform.
\end{itemize}
\end{itemize}

\end{document}

%%% Local Variables: 
%%% mode: latex
%%% TeX-master: t
%%% End: 
