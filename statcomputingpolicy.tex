\documentclass[12pt]{article}
\usepackage[sort,longnamesfirst]{natbib}
\newcommand{\pcite}[1]{\citeauthor{#1}'s \citeyearpar{#1}}

\usepackage{amsbsy,amsmath,amsthm,amssymb,graphicx}
\usepackage[usenames,dvipsnames]{color}
\usepackage{subfigure}

\usepackage{geometry}
\usepackage{hyperref}
%\geometry{hmargin=2.5cm,vmargin={2.5cm,2.5cm},nohead,footskip=0.5in}
\geometry{hmargin=2.5cm,vmargin={2.5cm,2.5cm},footskip=0.5in}
\renewcommand{\baselinestretch}{1.25}
\setlength{\baselineskip}{0.3in} \setlength{\parskip}{.05in}
%\date{Draft: \today} 
\date{April 28, 2008} 
\pagestyle{empty}
\begin{document}
\begin{center}
\Huge  {\bf Computing Policy}\\
\Large  {\bf Department of Statistics, Eberly College of Science}\\
\large April 28, 2008.
\end{center}

The following list of policies apply to all faculty, staff and
students in the Department of Statistics at Penn State. Please note
that these policies are an addendum to the ECOS computing policies,
which automatically also apply to the Department of Statistics. These
policies are primarily intended to clarify ECOS computing policies in
the context of the Department of Statistics.
% clarification?
\begin{enumerate}
\item {\bf Home desktop machines:} Anyone who takes a machine home
  assumes full responsibility of the machine thereafter. In addition
  to filling out the existing Penn State Property Inventory form
%  \url{http://www.guru.psu.edu/forms/6-03RequestForAuthorizationToUseUniversityTangibleAssetsataNon-UniversityLocationFRM3.PDF}.
  \href{http://www.guru.psu.edu/forms/6-03RequestForAuthorizationToUseUniversityTangibleAssetsataNon-UniversityLocationFRM3.PDF}{{\it
      Request for Authorization To Use University Tangible Assets At a
      Non-University Location}}, an additional short form will soon be
  added which states that the individual acknowledges that by taking
  the machine home, he/she is assuming full responsibility for it.
  These forms must be submitted to Laurie Roan for approval.
\item {\bf Laptops:} System administrators will help individuals with their
  laptops under the following conditions: (i) The laptop must be
  purchased by the system administrator (on behalf of the individual),
  (ii) The system administrator will have sole root/administrator
  privileges on the laptop, (iii) Requests for help must involve
  standard, work-related software. If any of the above conditions are
  not met, the administrator is not responsible for maintenance or
  help with the laptops.
\item {\bf Portable devices} are not supported by the system administrators.
  Purchasing and maintenance are the responsibility of the user.
\item {\bf Purchases:} All devices that are part of the department
  network must be purchased by the system administrators. If an
  individual wishes to purchase a laptop and self-administer it, she
  can purchase it on her own.  Also, if a device is purchased by
  somebody other than a systems administrator, it will need approval
  before it can be connected to our network. There is no guarantee
  that such requests will be approved.
\item No Macintosh machines will be supported by the system
  administrators.  It is possible that in the future, support may be
  available, but at this time the infrastructure is not in place.
\end{enumerate} 

\noindent {\bf Information sharing:} The computing staff are currently
working on extensive documentation that will provide useful
information such as names and I.P. addresses of printers.

\end{document}
