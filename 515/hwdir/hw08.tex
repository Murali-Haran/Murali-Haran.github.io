
\documentclass{article}
\usepackage{amsmath, amsfonts,hyperref}
\usepackage{url}
\pagestyle{empty}
\setlength{\textwidth}{6in}
\setlength{\oddsidemargin}{0in}
\setlength{\topmargin}{-.5in}
\setlength{\textheight}{9in}

\begin{document}
%\pagestyle{empty}
\begin{center}
\Large
{\bf Homework 8, Stat 515, Spring 2015}\\
\normalsize
Due Wednesday, April 15, 2015 \underline{{\bf beginning of class}}\\
\end{center}
% Important things to keep in mind when writing {\tt R} code: 
% \begin{enumerate}
% \item[(a)] Organization: using functions where possible (modularization). 
% \item[(b)]  Appearance: justification (indent) (for the last, using Rstudio or emacs will be helpful). 
% \item[(c)]  Well commented code.
% \item[(d)]  Sensible variable names: generally use descriptive names
%   wherever possible.
% \item[(e)]
  {\bf Make sure you submit R code to the Angel drop folder, using
  the same naming conventions as in the last homework, e.g. your name
  should appear in the name of the R program. Please pay attention
to good programming style in order to receive full credit.}
%\end{enumerate}

\begin{enumerate}
\item Define the univariate Poisson kernel density function (Yang, 2004;
or see ``wrapped Cauchy'' in Levy (1939) and Wintner (1947)) as
follows:
\begin{equation*}
  f(\theta;\mu,\rho) = \frac{1}{2\pi} \frac{1-\rho^2}{1-2\rho \cos(\theta-\mu) + \rho^2},\:\: \mu-\pi \leq \theta \leq \mu + \pi
\end{equation*}
\begin{enumerate}
\item Simulate 10,000 draws from this density for $\mu=3, \rho=0.7$
  using a rejection sampler. Report the envelope density ($q$) and
  bounding constant ($K$) used in the rejection sampler. You may use
  any envelope density from which you are easily able to produce draws
  in {\tt R} (or the statistical programming language you are
  using). Plot the smoothed approximate density.
\item Approximate the expectation $E(\theta^2)$ using samples from (a)
  and report associated Monte Carlo standard errors.
\item Approximate $P(\theta > 4)$ and report associated Monte Carlo
standard errors.
\end{enumerate}

\item Write a program to simulate a non-homogeneous Poisson process on
  the interval $(0,10)$ where the intensity function is
  $\lambda(t)=\max(0.2,  3\sin(t))$. Simulate the process by first simulating the total number of
  counts on the interval, then use rejection sampling to sample from
  the conditional distribution of the location of the points given the
  total number of points on the interval.
\begin{enumerate}
\item Provide pseudocode for your algorithm. (A clear description of your algorithm without using specific programming language. Fully specify any distributions you use or derive.) 
\item Plot 2 realizations of this non-homogeneous Poisson process in
  separate figures. Make sure you plot the events clearly, for example
  if the vector {\tt arrivals} has all the arrival times of the
  events,  you may choose to use {\tt plot(arrivals, rep(0.2,
    length(arrivals)),    pch="x").} Of course you are free to choose better ways to draw this. 
%\item Suppose each 
\end{enumerate}

\item Let $\{X(t), t>0\}$ be Brownian motion with 0 drift and variance parameter $\sigma^2$. Suppose $X(1)=B$. 
\begin{enumerate}
\item What is the joint distribution of $X(s_1), X(s_2),\dots,X(s_n)$ for $s_1,\dots, s_n \in (0,1)$? Clearly explain the main steps in your justification.
\item Suppose $\sigma^2=3$, $B=10$, $s_1=0.1, s_2=0.3$. What is the correlation between $X(s_1)$ and $X(s_2)$? 
\item Let $0<s_1<s_2<1$ and $\sigma^2=3$, $B=10$. Plot the correlation between $X(s_1), X(s_2)$ as a function of the distance, $d=\mid s_1-s_2\mid$.
\end{enumerate}
\item For the problem above, assume $\sigma^2=3$, $B=10$. Now assume
  that $s_1,s_2$ are the order statistics of two Uniform(0,1) random
  variates. Use Monte Carlo to approximate the probability that the
  correlation between $X(s_1)$ and $X(s_2)$ is greater than
  0.1. 

\begin{enumerate}
\item Provide pseudocode for your algorithm. 
\item Report your Monte Carlo sample size and your Monte Carlo standard error.
\end{enumerate}

\end{enumerate}

\end{document}
