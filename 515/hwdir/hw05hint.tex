
\documentclass{article}
\usepackage{amsmath, amsfonts,hyperref}
\usepackage{url}
\pagestyle{empty}
\setlength{\textwidth}{6in}
\setlength{\oddsidemargin}{0in}
\setlength{\topmargin}{-.5in}
\setlength{\textheight}{9in}

\begin{document}
%\pagestyle{empty}
\begin{center}
\Large
{\bf Hint for Homework 5, Stat 515, Spring 2015}\\
\normalsize
Due Wednesday, February 25, 2015 \underline{{\bf beginning of class}}\\
\end{center}

\begin{enumerate}
\item Consider a Poisson process $N(t)$ with rate $\lambda$. Prove the
  following result: Given that $N(t)=n$, the n arrival times
  $S_1,\dots,S_n$ have the same distribution as the order statistics
  corresponding to $n$ independent random variables uniformly
  distributed on the interval $(0,t)$, i.e., $$P(S_1=t_1,\dots,S_n=t_n
  \mid N(t)=n)=\frac{n!}{t^n} I(0<t_1<\dots <t_n).  $$ Use a ``first
  principles'' argument, i.e., I would like you to do a proof that
  utilizes assumption \# 3 and \# 4 in the ``first principles'' definition
  of the Poisson process. Do not use an argument based on assuming
  that the counting process over an interval is Poisson distributed
  (the second definition discussed in class).\\\\
%Do not use the conditioning argument in the book.
  % simple problem; adapted from Problem 5 in Ross book
{\bf Hint:} For any numbers $s_i$ satisfying $0\leq s_1\leq s_2\dots \leq s_n\leq t$, show 
$$P(S_i\leq s_i, i=1,\dots, n\mid X(t)=n) = \frac{n!}{t^n} \int_0^{s_1}\hdots\int_{x_{n-2}}^{s_{n-1}} \int_{x_{n-1}}^{s_n} dx_n\hdots d x_1, $$
which is the same as the distribution of the order statistics from a
sample of $n$ random variates taken from the Uniform($0,t$)
distribution. First write each $S_i$ in terms of inter-arrival times
($T_i, i=0, \dots n$). That is, $S_1= T_0, S_2 = T_0 + T_1, \hdots,
S_n = \sum_{i=0}^{n-1} T_i$. In order to simplify things, while you
may not use the Poisson distribution (definition 2), you are welcome
to use the fact that inter-arrival times $T_i$s are independent iid
exponential random variables. You may also use known facts about
exponential r.v.s from class, e.g. the distribution of the minimum or
maximum of independent exponential r.v.s.

\end{enumerate}

% % Using first principles definition of Poisson process Ross Stoch Proc. pg.66
% \item Consider a Poisson process $N(t)$ with rate $\lambda$. Prove the
%   following result: Given that $N(t)=n$, the n arrival times
%   $S_1,\dots,S_n$ have the same distribution as the order statistics
%   corresponding to $n$ independent random variables uniformly
%   distributed on the interval $(0,t)$, i.e., $$P(S_1=t_1,\dots,S_n=t_n
%   \mid N(t)=n)=\frac{n!}{t^n} I(0<t_1<\dots <t_n).  $$
%   Use a ``first
%   principles'' argument, i.e., I would like you to do a proof that
%   utilizes Assumptions 3 and 4 in the ``first principles'' definition of the
%   Poisson process (as will be discussed in class.) Do not use the conditioning argument in the book.
% % simple problem; adapted from Problem 5 in Ross book
% \item The lifetime of a light bulb is known to be exponentially distributed with a mean of 3 years. If the lightbulb has been working for 4 years already, what is the probability it will still be working 3 years later? 
% % reworded Ross problem 9
% \item The lifetime of two machines are independent exponential($\lambda_1$) and exponential($\lambda_2$) respectively. Suppose machine 1 starts working now and the second machine is put to use $t$ units of time later. What is the probability  machine 1 will fail before machine 2? 
\end{document}
