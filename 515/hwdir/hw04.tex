
\documentclass{article}
\usepackage{amsmath, amsfonts,hyperref}
\usepackage{url}
\pagestyle{empty}
\setlength{\textwidth}{6in}
\setlength{\oddsidemargin}{0in}
\setlength{\topmargin}{-.5in}
\setlength{\textheight}{9in}

\begin{document}
%\pagestyle{empty}
\begin{center}
\Large
{\bf Homework 4, Stat 515, Spring 2015}\\
\normalsize
Due Wednesday, February 18, 2015 \underline{{\bf beginning of class}}\\
\end{center}
\begin{enumerate}%{\bf Textbook problems:} {\it Ch.4:} 2,3,5,8, 11,13.\\
\item Ehrenfest diffusion: 
\begin{enumerate}
\item Simulate the simple Ehrenfest diffusion process (following the
  description in the Ross book) with total number of particles,
  $m=50$. Start the process at $X_1=5$. Run the process for 10,000
  iterations and draw a histogram of the resulting values of $X_t$ (in
  {\tt R}, use the {\tt hist} command). You can also obtain an
  approximation to the density based on these samples by using the
  command {\tt plot(density(SampleVectorName), main="Approximate
    Density of Ehrenfest Diffusion Samples")}.
\item Read through the argument provided for deriving the stationary
  distribution of the Ehrenfest diffusion. State the stationary
  distribution $\pi$ for the particular Ehrenfest diffusion you are
  considering above. Now simulate 10,000 draws from the stationary
  distribution of the Ehrenfest diffusion (you may need to use the
  command {\tt rbinom}) and overlay a density plot of its samples on
  top of the previous plot by using {\tt
    lines(density(BinomialSampleVectorName),lty=2)}. Make sure you provide a
  clear heading for the plot 
  this plot clearly
\item State the results from class that explains what you observe from
  your comparisons above. 
\item Compare the proportion of values greater than 15 for the
  binomial distribution using {\tt pbinom}, and estimate this quantity
  using the two sets of samples you just generated. Report the true
  value along with your two estimates. Notice that you can use both
  sets of samples, the Markov chain as well as the iid sampler, to
  approximate this probability (expectation with respect to Binomial).
\item Start the process at $X_1=5$ and obtain $X_3$ for the Ehrenfest
  diffusion. Repeat this 10,000 times and obtain a histogram for
  $X_3$. Now start the process at $X_1 \sim \pi$ where $\pi$ is the
  stationary distribution above and again obtain $X_3$ for the
  Ehrenfest diffusion 10,000 times. Compare these two histograms to
  the Binomial distribution above. Briefly explain what you observe
  and why. 
\end{enumerate}
% Branching processes
\item Consider a branching process where $X_0=1$ and every individual at the $i$th generation has probability $p_j$ of having $j$ offspring, with $p_0=1/2,p_1=1/6, p_2=1/3$. Let $X_t$ be the state of the process at time $t$.
\begin{enumerate}
\item Let the probability this process will go extinct be $\pi_0$. Calculate $\pi_0$.
% Branching processes simulation exercise
\item What are the expected value and variance of $X_2$?
\item Now simulate 10,000 realizations of the process up to $t=2$, i.e., $X_0,X_1,X_2$. Plot the histogram of $X_2$, and report the sample mean and variance. Compare them to the theoretical mean and variance obtained from the previous problem. The code available here might be useful for this problem:\url{http://www.stat.psu.edu/~mharan/515/hwdir/hw04ex.R} .
\item Now simulate 10,000 realizations of the branching process. Let
  $L$ be the time taken before extinction. Show a histogram of
  $L$. Note that since a particular realization of the branching
  process may be extremely long or, for some processes, be infinite
  (since it may never go extinct), it is useful to impose
  (artificially) a very large upper limit for the length of the branching process when simulating realizations. 
\end{enumerate}

% % Using first principles definition of Poisson process Ross Stoch Proc. pg.66
% \item Consider a Poisson process $N(t)$ with rate $\lambda$. Prove the
%   following result: Given that $N(t)=n$, the n arrival times
%   $S_1,\dots,S_n$ have the same distribution as the order statistics
%   corresponding to $n$ independent random variables uniformly
%   distributed on the interval $(0,t)$, i.e., $$P(S_1=t_1,\dots,S_n=t_n
%   \mid N(t)=n)=\frac{n!}{t^n} I(0<t_1<\dots <t_n).  $$
%   Use a ``first
%   principles'' argument, i.e., I would like you to do a proof that
%   utilizes Assumptions 3 and 4 in the ``first principles'' definition of the
%   Poisson process (as will be discussed in class.) Do not use the conditioning argument in the book.
% % simple problem; adapted from Problem 5 in Ross book
% \item The lifetime of a light bulb is known to be exponentially distributed with a mean of 3 years. If the lightbulb has been working for 4 years already, what is the probability it will still be working 3 years later? 
% % reworded Ross problem 9
% \item The lifetime of two machines are independent exponential($\lambda_1$) and exponential($\lambda_2$) respectively. Suppose machine 1 starts working now and the second machine is put to use $t$ units of time later. What is the probability  machine 1 will fail before machine 2? 
\end{enumerate}
\end{document}
