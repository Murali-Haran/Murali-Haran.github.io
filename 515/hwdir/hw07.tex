\documentclass{article}
\usepackage{amsmath, amsfonts,hyperref}
\usepackage{url}
\pagestyle{empty}
\setlength{\textwidth}{6in}
\setlength{\oddsidemargin}{0in}
\setlength{\topmargin}{-.5in}
\setlength{\textheight}{9in}

\begin{document}
%\pagestyle{empty}
\begin{center}
\Large
{\bf Homework 7, Stat 515, Spring 2015}\\
\normalsize
Due Wednesday, April 8, 2015 \underline{{\bf beginning of class}}\\
\end{center}

\begin{enumerate}

%% Brownian Motion
% \item Hitting times for Brownian motion. Consider a Brownian motion with drift parameter $\mu$ and
%   variance parameter $\sigma^2$, and $a<x<b$, with $a,x,b$ fixed
%   quanties. Suppose $X(0)=x$. Let hitting time
%   $T=\min\{ t\geq 0, X(t)=a \mbox{or } X(t)=b\}$ One can show that 

%% Ross Prob Models book, 8th edition, Ch. 10: 2, 3
\item Consider $\{ W(t), t\geq 0\}$, a standard Brownian motion process. 
\begin{enumerate}
\item Find the conditional distribution of $W(s)\mid W(t_1)=A,
  W(t_2)=B,$ where $0<t_1<s<t_2$.
\item Find $E(W(t_1) W(t_2) W(t_3))$ where $0<t_1<t_2<t_3$.
\end{enumerate}

%% Ross Prob Models book, 8th edition, Ch. 10: 9
\item Let $\{X(t), t\geq 0\}$ be a Brownian motion with drift
  coefficient $\mu$ and variance parameter $\sigma^2$.  Find the
  joint distribution of $X(s), X(t)$ where $0<s<t$.

%% Ross Prob Models book, 8th edition, Ch. 10:  10
% \item Let $\{X(t), t\geq 0\}$ be a Brownian motion with drift
%   coefficient $\mu$ and variance parameter $\sigma^2$.  Find the
%   conditional distribution of $X(t)\mid X(s)=c$ when
% \begin{enumerate}
% \item $s<t$?
% \item $t<s$?
% \end{enumerate}

%% Martingales
\item Distinguishing between Markov processes and martingales: 
\begin{enumerate}
\item Provide one example of a martingale that is not a Markov process and show why this is the case.
\item Provide one example of a Markov process that is not a martingale and show why this is the case.
\end{enumerate}

% Ross, Stoch Proc 2nd edition, pg 300 Wald's Equation, but assume
% slightly stronger conditions than necessary hold here (do not need bounded stopping time)
\item Prove the following result: If $X_i, i\geq 1$, are independent
  and identically distributed (iid) %with $E(\mid X_i\mid) < \infty$
                                %for all $i\geq 1$
  and if $N$ is a bounded stopping time for $X_1,X_2,\dots$ with
  $E(N)< \infty$, then $$ E\left(\sum_{i=1}^N X_i\right) =
  E(N)E(X)$$. Hint: consider the process $Z_n=\sum_{i=1}^n (X_i -
  \mu)$.

\item Let $\{X(t), t>0\}$ be standard Brownian motion. Prove that the
  process $\{M(t), t>0\}$ where $M(t) = \exp\left(\lambda X(t) -
    \frac{1}{2} \lambda^2 t\right)$, is a martingale.

\item Simulate 3 realizations for each of the following processes on
  the interval $[0,10]$ on 20 equally spaced points on the interval.
\begin{enumerate}
\item Simulate 3 realizations of standard Brownian motion. 
\item Simulate 3 realizations of Brownian motion with variance parameter
  $\mu=0, \sigma^2=2$. 
\item Simulate 3 realizations of Brownian motion with parameters $\sigma^2=2, \mu=3$.
\end{enumerate}
Overlay the 3 realizations for each process on the same plot. Hence
you should submit 3 clearly labeled plots. You do not have to submit
your code for this problem but you have to provide pseudocode for each
 simulation algorithm above.

\item Consider a simple symmetric random walk, $S_n=\sum_{i=0}^n X_i$
  where $X_1,X_2,\dots$ are iid with $P(X_i=1)=1/2=P(X_i=-1)$ and
  define a random time, $T\in [0,3]$ at which ${S_n}$ takes on
  its maximum value $\max\{S_n : 0 \leq n \leq 3\}$. If ${S_n}$ takes its maximum
  value more than once, assume $T$ is the last such time. 
\begin{enumerate}
\item Show analytically that $E(X_T)>0$ and hence $E(X_T) \neq E(X_0)$. This
  therefore results in an ``unfair game'', as discussed in class. 
\item Find the expected value of $T$ using Monte Carlo. Write
  pseudocode for the algorithm and report your estimate along with the
  Monte Carlo sample size and Monte Carlo standard error.
\end{enumerate}

% \item Ornstein-Uhlenbeck process: Is it a martingale? Justify your answer. 
\end{enumerate}

\end{document}
