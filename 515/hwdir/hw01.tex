% \documentclass[12pt]{article}
% \usepackage[sort,longnamesfirst]{natbib}
% \usepackage{amsmath, hyperref}%
% \newcommand{\bthet}{ \mbox{\boldmath $\theta$}}
% \newcommand{\bOne}{ {\bf 1} }
% \newcommand{\lambdaT}{\lambda_{\theta}}
% \newcommand{\bTheta}{ \mbox{\boldmath $\Theta$}}

\documentclass{article}
\usepackage{amsmath, hyperref}
\usepackage{url}
\pagestyle{empty}
\setlength{\textwidth}{6in}
\setlength{\oddsidemargin}{0in}
\setlength{\topmargin}{-.5in}
\setlength{\textheight}{9in}


\begin{document}
%\pagestyle{empty}
\begin{center}
\large
{\bf Homework 1, Stat 515, Spring 2015}\\
\normalsize
Due Wednesday, January 21, 2015 \underline{{\bf beginning of class}}\\
\end{center}

\noindent Please read through Chapter 2 to review basic ideas about random
variables. Make sure you read relevant sections in Chapter 3. For {\tt
  R} code examples that might be helpful for these problems, take a
look at \url{http://www.stat.psu.edu/~mharan/515/hwdir/hw1ex.R}  

%, in particular ``the ballot problem.''
%{\bf Textbook problems:} {\it Ch.1:} 44. {\it Ch.3:} 14,24,29,42,45,80\\

\begin{enumerate}
%% conditional probability
\item Assume $X\sim$ Beta($\alpha,\beta$).  Find $E(X\mid X < t)$ where $t\in (0,1)$.

%% calculating expectations by conditioning (Chapter 3, problem 25)
\item A gambler wins each game with probability $p$. In each of the following cases, determine the expected total number of wins.
\begin{enumerate}
\item The gambler will play $n$ games; if he wins $X$ of these games, then he will play an additional $X$ games before stopping. 
\item The gambler will play until he wins; if it takes him $Y$ games to get this win, then he will play an additional $Y$ games.
\end{enumerate}

%% Chapter 3, problem 46
\item 
\begin{enumerate}
\item Show that Cov$(X,Y)$=Cov$(X, E(Y\mid X))$.
\item Suppose that for constants $a$ and $b$, $E(Y\mid X) = a + bX$. Show that $b = \mbox{Cov}(X,Y)/\mbox{Var}(X)$.
\end{enumerate}

%% Chapter 3, problem 55
\item Suppose that you arrive at a party, along with a random number
  of additional people. The number of additional people, $X\sim$
  Poisson(10). The times at which people (including you) arrive at the
  party are independent uniform(0,1) random variables.
\begin{enumerate}
\item Find the expected number of people who arrive before you.
\item Find the variance of the number of people who arrive before you.
\end{enumerate}

%% Chapter 3, problem 80
\item A coin that comes up heads with probability $p$ is flipped $n$
  consecutive times. What is the probability that starting with the
  first flip there are always more heads than tails that have appeared
  ? Hint: use the solution to the ballot problem (in Chapter 3.)

%{\bf Non-textbook problems:}
\item Toy drug trial. Consider $n$ trials, each with probability of success
$p$. Assume the trials are independent given $p$. \\Now, suppose $p
\sim $Beta($\alpha,\beta),\:\:i=1,\dots,n$. 
Recall that if $X$ is a Beta r.v.:
$$ f_X(x)=\frac{\Gamma(\alpha+\beta)}{\Gamma(\alpha)\Gamma(\beta)} x^{\alpha -1} (1-x)^{\beta-1} I(0<x<1),\:\: \alpha>0,\beta>0$$
$$ \text{E}(X)=\frac{\alpha}{\alpha + \beta},\:\:\:\: \text{Var}(X)=\frac{\alpha \beta}{(\alpha + \beta)^2 (\alpha + \beta+1)}.$$
\begin{enumerate}
\item Compute the expected value of the total number of successes.
\item Compute the variance of the total number of successes.
\end{enumerate}

\item  \noindent A short simulation exercise:
  \noindent Estimate the answers you obtained for Problem 4 above via
  simulation. You already have the answer so you can compare your
  estimates with the answer. Use 1000 replications of the
  process (one replicate of the process = one randomly sampled party.)
\begin{itemize}
\item First download, install {\tt R}; see the course webpage \url{http://www.stat.psu.edu/~mharan/515/Rlinks.html} for useful {\tt R} links.
\item You can find a simple example for random variate simulation
  here: \url{http://www.stat.psu.edu/~mharan/515/hwdir/hw1ex.R}. You
  can adapt this example to estimate the expectation and variance for this problem. 
\end{itemize}
Note: Ideally, you should also be reporting simulation (Monte Carlo) standard errors for your estimates; we will discuss this later in the course.
\end{enumerate}

Because this is your first assignment and your {\tt R} code here will
be quite short, please include a print out of your {\tt R} code with
the assignment.  You do not have to type up this assignment, but if
you want to start getting familiar with LaTeX, try looking at
\url{http://www.stat.psu.edu/~mharan/515/Latexlinks.html}

\end{document}
