
\documentclass{article}
\usepackage{amsmath, amsfonts,hyperref}
\usepackage{url}
\pagestyle{empty}
\setlength{\textwidth}{6in}
\setlength{\oddsidemargin}{0in}
\setlength{\topmargin}{-.5in}
\setlength{\textheight}{9in}

\begin{document}
%\pagestyle{empty}
\begin{center}
\Large
{\bf Homework 6 -- Part I, Stat 515, Spring 2015}\\
\normalsize
Due Wednesday, March 25, 2015 \underline{{\bf beginning of class}}\\
\end{center}

\begin{enumerate}
\item Consider a continuous-time Markov chain on $\{1,2,3 \}$ with generator
 \begin{equation*}
   G=
  \begin{bmatrix}
    -6 & 2 & 4 \\
    1 & -2 & 1\\
    3 & 1 & -4 \\
  \end{bmatrix}.
\end{equation*}
\begin{enumerate}
\item What is the distribution of the holding times? What is the transition probability matrix of the embedded ``jump chain''?
\item Simulate two realizations of this Markov chain on the interval
  (0,50). Assume it starts at 1, that is, $X(0)=1$. Provide pseudocode
  (sketch of the algorithm), {\tt R} code and a plot for each of the realizations.
  \end{enumerate}
{\bf Part II}
\item A small barbershop is operated by a single barber. It has room
  for at most two customers. Potential customers arrive as a Poisson
  process with a rate of three per hour, and the successive service times are
  independent exponential random variables with mean 0.25 hour. Find:
\begin{enumerate}
\item The average number of customers in the shop.
\item The proportion of potential customers that enter the shop (in
  the long run). 
\item If the barber could work twice as fast, how much more business
  could he do?
\end{enumerate}
\item Write {\tt R code} to simulate a single realization over a
  period of 10 days for the above barbershop. What is the average number of
  customers in the shop for this realization?
% birth-death process (Ch.6), problem 14.
% final 08 problem 7
\item Customers arrive at a small bank according to a Poisson process
  with rate 20 per hour. However, they will only enter the bank if
  there are no more than two customers (including the one being
  attended to) at the bank.  Assume that the amount of time required
  to serve a customer is exponentially distributed with mean of 5
  minutes. In the long run, what fraction of time will there be at
  least 1 customer in the bank? Solve this problem in two ways:
\begin{enumerate}
\item First write out the generator for this continous-time Markov
  chain and then find the stationary distribution using the generator.
\item Use the fact that it is a birth-death process and find the
  stationary distribution $\pi$ that satisfies detailed balance
  conditions.
\end{enumerate}
% birth-death process (Ch.6)
% P.Olofsson pg.450
% final 09 problem 
\item Consider a population where the individual death rate is $\mu>0$
  and there is constant immigration into the population (immigration
  always increases the size of the population) according to a Poisson
  process with rate $\lambda>0$.  Derive the stationary distribution of
  this process.

  \end{enumerate}

\end{document}
