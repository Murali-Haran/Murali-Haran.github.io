
\documentclass{article}
\usepackage{amsmath, amsfonts,hyperref}
\usepackage{url}
\pagestyle{empty}
\setlength{\textwidth}{6in}
\setlength{\oddsidemargin}{0in}
\setlength{\topmargin}{-.5in}
\setlength{\textheight}{9in}

\begin{document}
%\pagestyle{empty}
\begin{center}
\Large
{\bf Homework 9, Stat 515, Spring 2015}\\
\normalsize
Due Wednesday, April 22, 2015 \underline{{\bf beginning of class}}\\
\end{center}
% Important things to keep in mind when writing {\tt R} code: 
% \begin{enumerate}
% \item[(a)] Organization: using functions where possible (modularization). 
% \item[(b)]  Appearance: justification (indent) (for the last, using Rstudio or emacs will be helpful). 
% \item[(c)]  Well commented code.
% \item[(d)]  Sensible variable names: generally use descriptive names
%   wherever possible.
% \item[(e)]
  {\bf Make sure you submit R code to the Angel drop folder, using
  the same naming conventions as in the last homework, e.g. your name
  should appear in the name of the R program. Please pay attention
to good programming style in order to receive full credit.}
%\end{enumerate}

\begin{enumerate}
\item {\bf Importance Sampling}\\
Return to the univariate Poisson kernel density function (Yang,
  2004; or see ``wrapped Cauchy'' in Levy (1939) and Wintner (1947)),
  which is defined as follows:
\begin{equation}
  f(\theta;\mu,\rho) = \frac{1}{2\pi} \frac{1-\rho^2}{1-2\rho \cos(\theta-\mu) + \rho^2},\:\: \mu-\pi \leq \theta \leq \mu + \pi
\end{equation}
\begin{enumerate}
\item Approximate the expectation $E(\theta^2)$ for $\mu=3, \rho=0.7$ using importance sampling. State the importance function you used and report associated Monte Carlo standard errors.
\item Approximate the expectation $E(\theta)$ as a function of $\rho \in
  (-1,1)$, that is, find $E_{\rho}(\theta)$, with $\mu=3$ using
  importance sampling. (a) Evaluate the expectation on a grid of
  $\rho$ values ranging from -0.95 to +0.95 in increments of 0.02. You
  can use the {\tt R} function {\tt seq} to generate the grid of
  values.  For example, {\tt rhovals=seq(-.95,0.95,by=0.02)}. Estimate
  these expectations by {\it using only one set of samples from the
    importance function}. (a) State the importance function you used
  and how it satisfies requisite conditions for validity of the
  algorithm, (b) Plot the estimated expectation as a function of
  $\rho$, and (c) report Monte Carlo standard errors along with the
  estimate of the expectation at the 1st, 24th, 48th, 72nd and 96th
  value of $\rho$.  It may be best to report this information in the
  form of a table.
\end{enumerate}

\item {\bf Importance Sampling for Bayesian inference}\\ 
Suppose $Y_1,\dots,Y_n|\alpha \sim \mbox{Poisson}(\exp(\alpha))$
  (they are conditionally independent given $\alpha$), and the prior
  distribution for $\alpha$ is $\alpha \sim N(0,100)$.  The data
  $Y_1,\dots, Y_n$ are available at
  \url{http://www.stat.psu.edu/~mharan/515/hwdir/hw09.dat}. You can
  read the data by using the command: \verb+ ys = scan("http://www.stat.psu.edu/~mharan/515/hwdir/hw09.dat") + (directly from the website) OR \verb+ ys = scan("hw09.dat") + \\ We
  are interested in the posterior distribution of $\alpha$ given the
  data, i.e., the conditional distribution of $\alpha|Y$.
\begin{enumerate}
\item Estimate the posterior expectation of $\alpha$ using importance
  sampling. 
%(b) Estimate the posterior variance of $\alpha$ using importance
%sampling. \\
\item Estimate the posterior probability that $\alpha$ is greater than
  2  using importance sampling.
\item Use importance sampling to approximate the posterior expectation
  of $\alpha$ if the prior distribution for $\alpha$ is changed to
  $\alpha \sim N(0,5)$, using the same set of samples as above. What
  is a potential problem if you switched the order in which you did
  this, that is, you found an importance function that worked for
  $\alpha \sim N(0,5)$ and then tried to use the same for $\alpha \sim
  N(0,100)$?
\end{enumerate}
You need to report: (i) importance function you used (you may use more
than one and report your results for all of them if you like), (ii)
plot the estimate versus the Monte Carlo sample size to see how your
estimate converges as a function of the Monte Carlo sample size.  Do
this for at least $n=100$ to 5,000, in increments of 100, and (iii)
plot the Monte Carlo standard error as a function of $n$ in similar
fashion.
%% Univariate M-H algorithm
\item {\bf Univariate Metropolis-Hastings}\\
Return to the Bayesian inference problem above, but now change the
prior so $\alpha \sim t_3(0,100)$, a t-density with
  $\nu=3$ degrees of freedom and $\sigma^2=100$ (so
  variance=$\nu \sigma^2/(\nu-2)$).
\begin{enumerate}
\item Simulate draws from $\alpha\mid Y$ by using a random walk
  Metropolis-Hastings algorithm with a normal proposal. Try at least
  two different values for the variance of the random walk proposal
  and compare your results for the different algorithms. In
  particular:
\begin{enumerate}
\item Observe how the estimate of the expected value of $\alpha\mid Y$ and $\alpha^2\mid Y$ 
  change over time by plotting the estimate versus number of
  iterations (similar to what you did for importance
  sampling). You can plot both the estimates in the same figure by using the ``lines'' command in {\tt R}.\\
  You may choose to use the function {\tt estvssamp} in
  \url{http://www.stat.psu.edu/~mharan/batchmeans.R}.
\item Check the autocorrelation in your samples (not your
  estimates). For this you can use the {\tt acf} function in {\tt
    R}. How is the autocorrelation affected by different values of the
  variance of the random walk proposal?
\item Report the acceptance rates of the algorithm for different values of $\sigma^2$.
\end{enumerate}
\item Draw a smoothed estimated density plot based on the draws. 
\item Using these draws estimate the posterior expectation of
  $\alpha$ and $\alpha^2$. Report MCMC standard errors for your estimates.
  Draw enough samples so you feel comfortable that your estimates
  appear to be stabilizing, and that your Monte Carlo standard errors
  are reasonable. For MCMC, you can use the {\tt consistent
    batchmeans} procedure to estimate standard errors. The
  corresponding R code is available here:
  \url{http://www.stat.psu.edu/~mharan/batchmeans.R}.
\end{enumerate}
\end{enumerate}

\end{document}
