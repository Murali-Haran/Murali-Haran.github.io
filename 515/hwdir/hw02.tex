
\documentclass{article}
\usepackage{amsmath, hyperref}
\usepackage{url}
\pagestyle{empty}
\setlength{\textwidth}{6in}
\setlength{\oddsidemargin}{0in}
\setlength{\topmargin}{-.5in}
\setlength{\textheight}{9in}

\begin{document}
%\pagestyle{empty}
\begin{center}
\Large
{\bf Homework 2, Stat 515, Spring 2015}\\
\normalsize
Due Wednesday, February 4th, 2015 \underline{{\bf beginning of class}}\\
\end{center}
\begin{enumerate}%{\bf Textbook problems:} {\it Ch.4:} 2,3,5,8, 11,13.\\
%% Problem 1 from midterm for spring 2010
%% problem 2 and 3 from book 
\item Suppose that the probability of rain today depends on weather
  conditions from the previous three days. If it has rained for the
  past three days, then it will rain today with probability 0.8; if it
  did not rain for any of the past three days, then it will rain today
  with probability 0.2; and in any other case, the weather today will,
  with probability 0.6, be the same as the weather yesterday. 
\begin{enumerate}
\item Describe
  this process using a Markov chain, i.e., define a state space and
  the corresponding transition probability matrix for the process. 
\item Suppose you know that it rained the very first day but that it
  did not on the second and third days. What is the probability it will
  rain on the fifth day?
\end{enumerate}
\item Consider a Markov chain on $\Omega=\{1, 2, 3, 4, 5, 6\}$ specified by the following transition probability matrix.\\
\begin{equation*}
  P=
  \begin{bmatrix}
    \frac{1}{3} & 0   & \frac{1}{3}  & 0 & 0 & \frac{1}{3}\\
    \frac{1}{2} & \frac{1}{4}   & \frac{1}{4}  & 0 & 0 & 0\\
    0 & 0  & 0  & 0 & 1 & 0\\
    \frac{1}{4} & \frac{1}{4}   & \frac{1}{4}  & 0 & 0 & \frac{1}{4}\\
    0 & 0   & 1  & 0 & 0 & 0\\
    0 & 0   & 0  & 0 & 0 & 1\\
  \end{bmatrix},
\end{equation*}
\begin{enumerate}
\item What are the (communicating) classes of this Markov chain? Is the Markov chain irreducible ? 
\item Which states are transient and which are recurrent? Justify your answers. 
\item What is the period of each state of this Markov chain? Is the Markov chain aperiodic ? 
\item Let $X_0$ be the initial state with distribution $\mbox{\boldmath $\pi$}_0 = (0, \frac{1}{4},
\frac{3}{4}, 0, 0, 0)$ corresponding to the probability of
being in states $ 1, 2, 3,
4, 5, 6$ respectively.  Let $X_0, X_1, X_2,\dots$ be the Markov chain constructed using $P$ above. What is E$(X_1)$?
\item What is Var$(X_1)$?
\item What is E$(X_3)$? % additional problem
\end{enumerate} 
%% problem 13 from book
\item Let $P$ be the transition probability matrix of a Markov chain. Show that if, for some positive integer $r$, $P^r$ has all positive entries, then so does $P^n$, for all integers $n\geq r$.
%% computer problem using problem 1 above (from midterm spring 2010)
\item Computer problem: Simulate a realization of the random variable
  $X_3$ according to the description in Problem 2, using the initial
  distribution in 1(d). Repeat this 1000 times (generate 1000
  instances of $X_3$), and calculate the average. This is your
  estimate of $E(X_3)$. Compare it with your answer from Problem 2.
  Since this is a short program, include a printout of your code with
  your homework. {\it Please make your assignment easier to grade by
    neatly organizing your writeup and by drawing a box around and
    labeling your answers.}
% \begin{enumerate}
\item Simulate the Markov chain according to Problem 2 and run it for
  100,000 steps.  Now calculate the proportion of times the Markov
  chain was in the states 1,2,3,4,5,6 respectively. Simulate two more
  realizations, each also of length 100,000, and again record the
  proportion of times the Markov chain was in the states 1,2,3,4,5,6
  respectively.  You only have to report the proportions for each of
  the three realizations (do not print out your Markov chains or your
  code for this!)
\newpage
\item Prove that periodicity is a class property.
\item Consider two Markov chains on $\Omega=\{1, 2, 3, 4, 5\}$ specified by the following transition probability matrices.
\begin{equation*}
  P_1=
  \begin{bmatrix}
    \frac{1}{2} & 0   & \frac{1}{2}  & 0 & 0 \\
    \frac{1}{4} & \frac{1}{2}   & \frac{1}{4}  & 0 & 0\\
    \frac{1}{2} & 0  & \frac{1}{2}  & 0 & 0\\
    0 & 0   & 0  & \frac{1}{2} & \frac{1}{2}\\
    0 & 0   & 0  & \frac{1}{2} & \frac{1}{2}\\
  \end{bmatrix},
  P_2=
  \begin{bmatrix}
    \frac{1}{4} & \frac{3}{4}   & 0  & 0 & 0 \\
    \frac{1}{2} & \frac{1}{2}   & 0  & 0 & 0\\
    0 & 0  & 1  & 0 & 0\\
    0 & 0   & \frac{1}{3}  & \frac{2}{3} & 0\\
    1 & 0   & 0  & 0 & 0\\
  \end{bmatrix},
\end{equation*}
\begin{enumerate}
\item Specify the classes of the two Markov chains with these
  transition probability matrices and determine whether the states are
  transient or recurrent. %, and if they are recurrent whether they are positive or null recurrent.
\item List the periods of each state of the Markov chain. Are these Markov chains aperiodic? 
\end{enumerate}
\item Consider three urns, one colored red, one white, and one blue.
  The red urn contains 1 red and 4 blue balls; the white urn contains
  3 white balls, 2 red balls, and 2 blue balls; the blue urn contains
  4 white balls, 3 red balls, and 2 blue balls.  At the initial stage,
  a ball is randomly selected from the red urn and then returned to
  that urn. At every subsequent stage, a ball is randomly selected
  from the urn whose color is the same as that of the ball previously
  selected and is then returned to that urn. 
 Write the state space of this process and the associate initial
  distribution and transition probability matrix. Explain why this
  process is a Markov chain.
% (b) Repeat this exercise but this time use the initial distribution $P(X_0=0)=\frac{1}{8},P(X_0=1)=\frac{3}{4},P(X_0=2)=\frac{1}{8}$. Compare the proportions obtained to your results from (a). Are they similar or different? Explain why.
%\end{enumerate}
\end{enumerate}
\end{document}
