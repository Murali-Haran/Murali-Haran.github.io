
\documentclass{article}
\usepackage{amsmath, amsfonts,hyperref}
\usepackage{url}
\pagestyle{empty}
\setlength{\textwidth}{6in}
\setlength{\oddsidemargin}{0in}
\setlength{\topmargin}{-.5in}
\setlength{\textheight}{9in}

\begin{document}
%\pagestyle{empty}
\begin{center}
\Large
{\bf Homework 5, Stat 515, Spring 2015}\\
\normalsize
Due Wednesday, February 25, 2015 \underline{{\bf beginning of class}}\\
\end{center}

\begin{enumerate}%{\bf Textbook problems:} {\it Ch.4:} 2,3,5,8, 11,13.\\
  %% Using first principles definition of Poisson process Ross Stoch Proc. pg.66
\item Consider a Poisson process $N(t)$ with rate $\lambda$. Prove the
  following result: Given that $N(t)=n$, the n arrival times
  $S_1,\dots,S_n$ have the same distribution as the order statistics
  corresponding to $n$ independent random variables uniformly
  distributed on the interval $(0,t)$, i.e., $$P(S_1=t_1,\dots,S_n=t_n
  \mid N(t)=n)=\frac{n!}{t^n} I(0<t_1<\dots <t_n).  $$ Use a ``first
  principles'' argument, i.e., I would like you to do a proof that
  utilizes assumption \# 3 and \# 4 in the ``first principles'' definition
  of the Poisson process. Do not use an argument based on assuming
  that the counting process over an interval is Poisson distributed
  (the second definition discussed in class).
%Do not use the conditioning argument in the book.
  % simple problem; adapted from Problem 5 in Ross book
\item The lifetime of a light bulb is known to be exponential with mean 3 years. If the lightbulb has been working for 4 years, what is the probability it will still be working 3 years later? 
  % reworded Ross problem 9
\item The lifetime of two machines are independent exponential($\lambda_1$) and exponential($\lambda_2$) respectively. Suppose machine 1 starts working now and the second machine is put to use $t$ units of time later. What is the probability  machine 1 will fail before machine 2? 
\item Cars pass a certain street location according to a Poisson
  process with rate $\lambda$. A woman who wants to cross the street
  at that location waits until she can see that no cars will come by
  in the next $T$ time units.
  \begin{enumerate}
  \item Find the probability that her waiting time is 0.
  \item Find her expected waiting time.
  \end{enumerate}
\item Suppose $\{N(t)\}$ is a non-homogeneous Poisson process with
  intensity function $\lambda(x)$ (the intensity function is
  non-constant as it varies with $x$).
  \begin{enumerate}
  \item Derive the conditional distribution of the arrival times
    $S_1,\dots,S_n \mid N(t)=n$.
  \item For a {\it homogeneous} Poisson process, the conditional
    distribution of the arrival times $S_1,\dots,S_n \mid N(t)=n$, are
    the order statistics of $n$ i.i.d.\ Uniform($0,t$) random
    variables.  Clearly describe the steps of a general algorithm this suggests
    for simulating a Poisson process on an interval $[0,t]$. {\bf
      Hint}: you will simulate the process in two stages.
  \item Consider a {\it homogeneous} Poisson process with $\lambda=10$. Find the
    expected values of the number of events in the interval (0,1) and in
    the interval (4,5).
  \item Using the algorithm from part (b), simulate 10,000 realizations
    of the above Poisson process on the interval $[0,5]$.
    \begin{enumerate}
    \item Report the sample mean for the number of events in the interval (0,1) and the number of events in the interval (4,5). 
    \item Plot a histogram each for the distribution of the number of
      events in the interval (0,1) and the interval (4,5) respectively,
      based on the 10,000 realizations.
    \end{enumerate}
  \item Now, describe the steps of a general algorithm part (a) suggests for
    simulating a {\it non-homogeneous} Poisson process on an interval
    $[0,t]$.
    %% for interval (4,5), E(X) = 30*0.333 = 10
    %% for interval (5,6), E(X) = 30*0.333 = 10
  \end{enumerate}
\item Simulate a {\it single} realization of a Poisson process on the interval
  $(0,100)$ with $\lambda=0.2$. Do this in two ways as described
  below. For each part you need to submit your well-commented {\tt R}
  code along with a plot displaying the process. See an {\tt R}
  example for plotting here \url{http://www.stat.psu.edu/~mharan/515/hwdir/hw05ex.R} 
\begin{enumerate}
\item Simulate the process using the characterization of a Poisson process in terms of
  uniform random variates.
\item Simulate the process using the characterization of a Poisson process in terms of
  exponential random variates. (Recall: the waiting times between
  successive events are exponential random variates.)
  \end{enumerate}
  \end{enumerate}

% % Using first principles definition of Poisson process Ross Stoch Proc. pg.66
% \item Consider a Poisson process $N(t)$ with rate $\lambda$. Prove the
%   following result: Given that $N(t)=n$, the n arrival times
%   $S_1,\dots,S_n$ have the same distribution as the order statistics
%   corresponding to $n$ independent random variables uniformly
%   distributed on the interval $(0,t)$, i.e., $$P(S_1=t_1,\dots,S_n=t_n
%   \mid N(t)=n)=\frac{n!}{t^n} I(0<t_1<\dots <t_n).  $$
%   Use a ``first
%   principles'' argument, i.e., I would like you to do a proof that
%   utilizes Assumptions 3 and 4 in the ``first principles'' definition of the
%   Poisson process (as will be discussed in class.) Do not use the conditioning argument in the book.
% % simple problem; adapted from Problem 5 in Ross book
% \item The lifetime of a light bulb is known to be exponentially distributed with a mean of 3 years. If the lightbulb has been working for 4 years already, what is the probability it will still be working 3 years later? 
% % reworded Ross problem 9
% \item The lifetime of two machines are independent exponential($\lambda_1$) and exponential($\lambda_2$) respectively. Suppose machine 1 starts working now and the second machine is put to use $t$ units of time later. What is the probability  machine 1 will fail before machine 2? 
\end{document}
