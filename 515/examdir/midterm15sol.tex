\documentclass[12pt]{article}

\setlength{\topmargin}{-.75in}
\setlength{\oddsidemargin}{.0in}
\setlength{\evensidemargin}{.0in}
\setlength{\textheight}{9.5in}
\setlength{\textwidth}{6.5in}
\setlength{\parindent}{0in}

\usepackage[sort,longnamesfirst]{natbib}
\usepackage{amsmath}%
\newcommand{\suc}{\mbox{ succeeds }}
\newcommand{\fai}{\mbox{ fails }}
\begin{document}
%\pagestyle{empty}
\begin{center}
{\bf  Midterm STAT 515, Penn State Statistics}\\ 
{March 6, 2015.}
\end{center}
\begin{flushleft}
Name: %SOLUTIONS
\end{flushleft}

\noindent (1) Show all your work for full credit.\\
(2) When possible, draw a box around your final answer to each question.\\
(3) You may leave your answer unsimplified, e.g. $0.7^{4} \frac{18!}{4!8!}, 14 e^{-23}, \sum_{i=1}^n \frac{i}{3^i}$.
\begin{enumerate}

%% From Midterm 2011
%% Markov chains: time reversibility, branching processes
\item Consider a Markov chain on a finite state space
  $\Omega=\{1,2,\dots,n\}$. Now assume that the chain has a symmetric
  transition probability matrix $P$ and that $P_{ij}>0$ for all
  $i,j\in \Omega$. Hint: read all parts of this problem before working
  on it.
\begin{enumerate}
\item Show that this Markov chain has a unique stationary distribution
  $\pi$ that is (discrete) uniform on $\Omega$. [3pts]
\item Assume the Markov chain has been running for infinitely long. Is
  it time reversible?  Justify your answers, verifying
  all necessary properties. [3pts]
\item Is $\pi$ also the limiting distribution of the chain? Why? [2pts]
\end{enumerate}
  {\it Soln:} Easy to show that the chain is irreducible. Consider $i,j \in \Omega$ with $i< j$. Paths with positive probability: $i\rightarrow i+1 \rightarrow \dots j$, and $j\rightarrow j-1 \rightarrow \dots i$. \\
  Since $P_{ii}>0$ chain is aperiodic. This is a finite state,
  irreducible chain so all states are positive recurrent. Hence, chain
  is irreducible and ergodic so stationary distribution $\pi$ is
  unique and also the limiting distribution of chain. Remains to find
  $\pi$.  Easy to see that detailed balance condition holds for $\pi =
  (1/n,1/n,\dots, 1/n)$, that is $\pi_i P_{ij} = \pi_j P_{ji} \forall
  i,j \in \Omega$ when $\pi = (1/n,1/n,\dots, 1/n)$. Hence the
  discrete uniform distribution on $\Omega$ is the stationary and
  limiting distribution of the Markov chain.

\newpage

% from Midterm 06 and Final 09
% Basic Markov chain theory problem: stationarity, ergodic thm., irreducibility, periodicity, etc.
%\item An individual possesses 3 umbrellas, which she
\item A professor possesses 2 umbrellas, which she employs in going
  from her home to office and vice-versa (office to home). If she is
  at home at the beginning of a day and it is raining, then she will
  take an umbrella with her to the office, provided there is one to be
  taken.  If it is not raining, then she never takes an umbrella. She
  does the same thing when she leaves the office. Assume that the
  probability of rain at the beginning of the day (also at the end of
  the day), $p$, is independent of rain on other days.
\begin{enumerate}
\item Define a Markov chain model for the number of umbrellas she has
  at her {\it current} location. E.g. if she has 0 umbrellas
  now, at the next location she will definitely have 2 umbrellas (since both are at the next location). [3pts]
%\item Is this chain irreducible? Explain your answer. [1pts]
%\item What is the period of each state? Explain your answer. [2pts]
%\item State if each state is recurrent or transient. Explain your answer. [2pts]
\item In the long run, what fraction of the time will she get wet?
  (You may assume that the chain is irreducible, positive recurrent
  and apeiodic.) [2pts]
\end{enumerate}
Solution: (a) Each state of the Markov chain corresponds to the number of umbrellas she has at her current location so 
\begin{equation*}
  P=
  \begin{bmatrix}
    0 & 0 & 1 \\
    0 & 1-p & p \\
    1-p & p & 0
  \end{bmatrix},
\end{equation*}
% (b) Yes, it is irreducible since there is a positive probability of getting from $i$ to $j$ for any $i,j\in \{0,1,2\}$ (need to show details). 
% (c) Since chain is irreducible, and periodicity is a class property, suffices to note that state 1 is aperiodic (since we can go from 1 to 1 in a single step). Hence all states are aperiodic (they have period 1).\\
% (d) This is an irreducible finite state Markov chain. Hence all states are recurrent by a theorem from class.\\
(b) Solve ergodic equations to get stationary probability of being in state 0: $\pi_o=\frac{1-p}{3-p}.$ Hence, fraction of time she will get wet = probability it rains $\times$ probability she doesn't have an umbrella with her = $p\pi_0=\frac{p(1-p)}{3 - p}$.

\newpage

%% Poisson process problem http://wwwhome.math.utwente.nl/~litvakn/ISP/
% \item Suppose that print jobs arrive at a network printer with independent and
% exponentially distributed inter-arrival times with parameter 10 per hour. It
% takes the printer exactly 6 seconds to print each page. The sizes of the jobs
% are i.i.d.\ and have a Poisson distribution with a mean of 2 pages.
% \begin{enumerate}
% \item What is the probability that a print job arrives while the previous job is
%    not finished, when this previous job consists of 6 pages? (Assume that the
%    printing of this previous job started immediately after its arrival). [2pts]
% % \item What is the expected arrival time of the first job after 12.00 hours, when
% %    the previous job arrived at 11.58 hours?
% \item Let $M(t)$ be the number of print jobs consisting of more than 3 pages that
%    arrive in a time interval $(0,t]$. What is $P (M (t) = m)$ for $m\in \{0,1,2,\dots \}$? [3pts]
% \end{enumerate}
%% From Midterm 2011
\item A car mechanic is only able to work on one car at a time.
  Suppose the amount of time he takes to work on each car is an
  exponential random variable with mean 30 minutes, and that these
  times are independent of each other.
\begin{enumerate}
\item Suppose there are always more cars to work on than the mechanic
  has time, and that all customers bring in their cars at 9am. What is
  the expected number of cars the mechanic will have completed work on
  from 9am to 10am ? [2pts]
\item Assume the same set up as in part (a), except now you know that
  the mechanic completed work on exactly 3 cars between 9am and 10am.
  What is the expected value of the time he finished work on the first
  car ? [2pts]
\item Now suppose he schedules two appointments, car A at 9am and car
  B at 9:30am. Assume both cars arrive on time for their appointments.
  Find the expected amount of time that car B spends at the mechanic's
  shop. [3pts]
\end{enumerate}
{\it Soln:} (a) Number of cars mechanic fixes is a Poisson process with $\lambda = 1/30$. Hence, if he works from 9am to 10am (for 60 minutes), the expected number of cars he will have fixed is the expected value of a Poisson r.v. with expectation $\lambda\times 60$ = 2.\\
(b) Let times of completion for the three cars be $X_1,X_2,X_3$. The question is asking what $E(X_1 \mid N(60) = 3)$ is. Now the completion times for these 3 cars is Uniform(0,60) where units are in minutes. Hence $X_1$ is the minimum order statistics of three U($0,60$) random variables. It is easy to derive the cdf of $X_1$ and hence its pdf is $f(x) = \frac{1}{20} (1 - \frac{x}{60})^2,\: x\in (0,60)$  and $E(X_1 \mid N(60) = 3)$ = 15.\\
(c) Condition on whether car A is done by 9:30am or not. Let
completion time for car A be $X_A$. If $X_A < 30$, expected time for
car B to be done is simply 30 minutes.  If $X_A >30$, by
memorylessness of exponential r.v.s, expected additional time for car
A is 30 minutes and time for completion for car B is therefore 30
minutes + 30 minutes (expected value for car B to be done after
mechanic starts working on it.) Hence, expected amount of time car B
spends at the mechanic's shop is:
$$ 30\times P(X_A <30) + (30 +30)\times P(X_A \geq 30) = 30(1-e^{-1}) + 60e^{-1} = 30 + 30e^{-1}. $$

\newpage
%% Iterated expectations/variance (compound Poisson process e.g.) (probl. 87 Ch.5, Ross)
\item Recall that a stochastic process $\{X(t),t \geq 0 \}$ is said to be a compound Poisson process if it can be represented as $X(t)=\sum_{i=1}^{N(t)} Y_i,\: t\geq 0$ where $\{N(t),t\geq 0\}$ is a Poisson process with rate $\lambda$ and $\{Y_i,i \geq 1 \}$ is a family of independent and identically distributed random variables that is also independent of $\{N(t),t\geq 0\}$. Let $\mu_Y=E(Y_i)$  and $\sigma_Y^2 = $Var$(Y_i)$ for $i=1,2,\dots$ 
\begin{enumerate}
%\item Show that this process has independent increments.
\item Find Var($X(t)$). [2pts]
\item Find Cov$(X(t), X(t+s))$, where $t \geq 0,\: s >0$. Hint: it may be useful to use the fact that this process has independent increments. [3pts]
\end{enumerate}
{\it Soln:} (a) Use conditional variance formula, $V(X(t)) = E(V(X(t) \mid N(t)=n)) +  V(E(X(t)) \mid N(t)=n) = \mu_Y^2 \lambda t + \lambda t \sigma_Y^2 $.

(b) To use the independent increments property, rewrite Cov$(X(t), X(t+s))$ = Cov$(X(t), X(t+s)-X(t) + X(t))$ = Cov($X(t), X(t)$) + Cov($X(t), X(t+s)-X(t)$)  = Var($X(t)$) + 0 = answer from part (a).

\end{enumerate}



\end{document}

