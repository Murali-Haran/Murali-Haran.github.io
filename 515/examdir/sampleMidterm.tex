\documentclass{article}

\setlength{\topmargin}{-.5in}
\setlength{\oddsidemargin}{.0in}
\setlength{\evensidemargin}{.0in}
\setlength{\textheight}{9in}
\setlength{\textwidth}{6.5in}
\setlength{\parindent}{0in}
%\parskip=.125in

\usepackage{amsmath,bm}%
\usepackage{amsfonts}%


\newcommand{\beaa}{\begin{eqnarray*}}
\newcommand{\eeaa}{\end{eqnarray*}}
\newcommand{\bea}{\begin{eqnarray}}
\newcommand{\eea}{\end{eqnarray}}
\newcommand{\svskip}{\vspace{.2in}}
\newcommand{\mvskip}{\vspace{.25in}}
\newcommand{\lvskip}{\vspace{.5in}}
\def\E{\mathop{\rm E\,}\nolimits}
\def\Var{\mathop{\rm Var\,}\nolimits}
\def\Cov{\mathop{\rm Cov\,}\nolimits}
\def\Cor{\mathop{\rm Corr\,}\nolimits}
\def\Tr{\mathop{\rm Tr\,}\nolimits}
\def\diag{\mathop{\rm diag\,}\nolimits}
\def\midd{\mathop{\,|\,}\nolimits}
\def\cip{\mathop{\stackrel{P}{\rightarrow}}\nolimits}
\def\cid{\mathop{\stackrel{d}{\rightarrow}}\nolimits}
\def\ciqm{\mathop{\stackrel{\mbox{\scriptsize qm}}{\rightarrow}}\nolimits}
\def\defn{{\stackrel{\mbox{\scriptsize def}}{=}}}
\def\eid{{\stackrel{{\cal D}}{=}}}
\def\rvseq{\mathop{X_1, X_2, \ldots}\nolimits}
\def\rvseqn{\mathop{X_1, \ldots, X_n}\nolimits}
\def\u#1{{\underline{#1}}}
\def\o#1{{\overline{#1}}}
\def\n#1{^{(#1)}}
\newcommand{\qed}{\rule{2mm}{2mm}}


\def\cas{\mathop{\stackrel{\mbox{\scriptsize as}}{\rightarrow}}\nolimits}

\pagestyle{empty}

%%-------------------------------------------------------------------

\begin{document}
        \hrule
        \begin{center}
        \Large\bf Stat 515 \hfill Spring 2015\\
        Sample Midterm  \hfill 
        \end{center}
        \hrule

%\lvskip {\bf Name:  \rule{4in}{.01in}}

\mvskip 
This midterm is worth 20 points.  You have 60 minutes.  
{\bf For full credit, you must explain all of your work!}
Naturally, you may use any results that you know.

\lvskip 
{\bf Problem 1. [8 points]\ } 
A Markov chain $\{X_t: t=0, 1, \ldots\}$ with state space $\{0, 1, 2\}$ has transition probability matrix
\[
P=
\begin{bmatrix}
%\frac13 & \frac13 & \frac13 \\ 
%\frac13 & \frac23 & 0 \\
%0 & 1 & 0
1/3 & 1/3 & 1/3 \\ 
1/3 & 2/3 & 0 \\
0 & 1 & 0
\end{bmatrix}.
\]

\svskip
{\bf(a) [2 points]\ }
Suppose $P(X_0=0) = P(X_0=2) = \frac12$.  Find $E(X_2)$.  Show all work.

\svskip 
{\bf(b) [2 points]\ }  
Straightforward calculation shows that
$P^3$ consists only of nonzero entries.
%\[
%P^3= \frac{1}{27} \times
%\begin{bmatrix}
%8 & 17 & 2 \\ 
%8 & 16 & 3 \\
%9 & 15 & 3
%\end{bmatrix}.
%\]
Using this fact, explain how the Chapman-Kolmogorov equations (the equations that relate
$P_{ij}^{n+m}$ to $P^n$ and $P^m$ for each $i$ and $j$) imply that
 $P^n$ consists only of nonzero entries for $n\ge 3$.

\svskip 
{\bf(c) [2 points]\ }  
Prove that in the long run, no matter which state the chain starts in, the number of steps 
that the chain spends in states 0, 1, and 2, respectively, will approach the
ratios $3:6:1$.

\svskip 
{\bf(d) [2 points]\ }  
Is this Markov chain time-reversible? Explain your answer.

\lvskip 
\lvskip 
{\bf Problem 2.  [4 points]\ } 
A Markov chain has transition probability matrix
\[
P=
\begin{bmatrix}
0 & 0.5 & 0 & 0 & 0 & 0.5 \\
0 & 0 & 1 & 0 & 0 & 0 \\
0 & 0 & 0 & 1 & 0 & 0 \\
1 & 0 & 0 & 0 & 0 & 0 \\
0 & 0 & 0 & 0 & 0 & 1 \\
0.5 & 0 & 0 & 0 & 0.5 & 0 \\
\end{bmatrix}.
\]

\svskip 
{\bf(a) [2 points]\ }  
Classify each state as transient, null recurrent, or positive recurrent.  Explain.

\svskip 
{\bf(b) [2 points]\ }  
Is the chain ergodic?  Explain.

\lvskip 
\begin{center}
{\em Problem 3 is on the back of this page.}
\end{center}

\newpage
{\bf Problem 3.  [8 points]\ } 
Buses arrive at a certain bus stop according to a Poisson process with rate
2 per hour.  
%The number of hours between successive bus arrivals at a certain stop is uniformly 
%distributed on $(0,1)$.  
Passengers arrive according to an independent Poisson process with rate
10 per hour.  The instant a bus arrives, all passengers at the stop at that instant board the
bus and the bus departs.  

\svskip
{\em Fact:  An exponential random variable with rate $\lambda$ has
mean $1/\lambda$ and variance $1/\lambda^2$.}

\mvskip
{\bf(a) [2 points]\ }  
Assume that there are currently no passengers at the bus stop.
What is the probability that the next bus will pick up no passengers?  Explain.

\svskip
{\bf(b) [2 points]\ }  
If you arrive at the stop at noon, what is the expected amount of time you will have to wait until
the next arrival of any type (bus or passenger)?  Explain.

\svskip
{\bf(c) [2 points]\ }  
At 2:00, there are 2 passengers waiting for the bus.
Given this information, what is the expected arrival time of the next bus after 2:00?
Explain.

\svskip 
{\bf(d) [2 points]\ }  
Assume that there are currently no passengers at the bus stop.
Let $X$ be the number of people at the stop when the next bus arrives.
Find $E(X)$ and $\Var(X)$, showing all your work.


\end{document}
