\documentclass[11pt]{article}
\usepackage[sort,longnamesfirst]{natbib}
\usepackage{amsmath,amsthm,amssymb}%
\usepackage{hyperref}
\newcommand{\bthet}{ \mbox{\boldmath $\theta$}}
\newcommand{\bOne}{ {\bf 1} }
\newcommand{\lambdaT}{\lambda_{\theta}}
\newcommand{\bTheta}{ \mbox{\boldmath $\Theta$}}
\oddsidemargin 0.0in
\evensidemargin 1.0in
\textwidth 6.0in
%\headheight 1.0in
\headheight 0.3in
\topmargin -0.5in
\textheight 9.0in
\begin{document}
%\pagestyle{empty}
\begin{center}
\Large
{\bf  Stat 515, Spring 2015: Take home final}\\
Due Wednesday, April 29, 2015 at midnight\\
\normalsize
\end{center}
%{\bf Out} Wednesday, April 20, 201\\
%% regression with exponentially modified Gaussian distribution
\begin{enumerate}
\item Consider a regression of a variable $Y$ on $X$ where the regression model is as follows,
$Y_i \sim EMG(\beta_0 + \beta_1X, \sigma_i, \lambda),$ 
where the exponentially modified Gaussian random variable, EMG($\mu, \sigma, \lambda$), has pdf
$f(x;\mu, \sigma, \lambda) = \frac{\lambda}{2} \exp(\frac{\lambda}{2} (2\mu + \lambda\sigma^2 - 2x)) \mbox{erfc}\left(\frac{\mu + \lambda\sigma^2 - x}{\sqrt{2} \sigma} \right), $
and erfc is the complementary error function defined as
%$$\mbox{erfc} = 1 - \mbox{erf}(x) = \frac{2}{\pi} \int_x^{\infty} e^{-t^2} dt.$$
$$\mbox{erfc}(x) = \frac{2}{\pi} \int_x^{\infty} e^{-t^2} dt.$$
The EMG distribution is obtained when a normal density is convolved
with an exponential density. That is, for the regression above, the
error contains a normal error (with standard deviation $\sigma$) added
to an exponential error (with rate $\lambda$, that is, expected value
$1/\lambda$). {\tt R} code for the EMG density function is here: \url{http://sites.stat.psu.edu/~mharan/515/hwdir/emg.R}
\begin{enumerate}
\item Assume that $\beta_0=5, \lambda=0.4$, and $\sigma_i=1 $ for all
  $i$. Let the prior for $\beta_1$ be N$(0,10)$ (parameterization:
  N(mean, sd)). Write a Metropolis-Hastings algorithm to approximate
  the posterior distribution, $\pi(\beta_1\mid {\mathbf Y, X})$ for
  Dataset\#1 on Angel. Clearly and succinctly describe the algorithm
  you used, with enough detail so anyone reading it should be able to
  write code based on your description. You should also provide
  important details such as starting values (e.g. arbitrary values,
  values based on several preliminary MCMC runs, a random draw from
  some initial distribution you chose etc.) {\it Note: here, as in
    other parts, you will lose points if your answer is either unclear
    or incomplete.}
\item Report your estimate of the posterior expectation of
  $\beta_1$. This is a point estimate for $\beta_1$. Also report the
  MCMC standard error associated with this estimate.
\item Report a 95\% credible interval for $\beta_1$ based on your
  samples. Credible intervals are the Bayesian analogue of frequentist
  confidence intervals. The interpretation is that if $(L,B)$ is the
  credible interval, $P(\beta_1 \in (L,B)\mid {\mathbf Y, X})=0.95$. A
  simple 95\% credible interval may be obtained by reporting the 2.5th
  and 97.5th sample percentiles from your Markov chain. In R, if you
  have stored your Markov chain in the vector mySamples, you can use
  the command {\tt quantile(mySamples, c(0.025, 0.975))}.
\item Plot an estimate of the posterior pdf of $\beta_1$ from a
  smoothed density plot of the samples. In R, you can use the command
 {\tt plot(density(mySamples))}.
\item Describe how you determined that your approximations above were 
  accurate, along with any supporting information as discussed in
  class, e.g. plots of autocorrelations, MCMC standard errors
  etc. {\it All your plots must be clearly labelled and referenced in
    your text.}
\end{enumerate}

\item Now assume that only $\sigma_i=1$ is known. Write a
  Metropolis-Hastings algorithm to approximate the posterior
  distribution, $\pi(\beta_0, \beta_1, \lambda\mid {\mathbf Y, X})$
  for Dataset \#2 on Angel. Assume priors $\beta_0\sim N(0,10),
  \beta_1\sim N(0,10)$ (parameterization: N(mean, sd)), $\lambda \sim
  $Gamma$(0.01,100)$ (w/ Gamma parameterization such that prior expected value of $\lambda$ is 1 and variance is 100).
\begin{enumerate}
\item Clearly and succinctly describe the algorithm you used, with
  enough detail so anyone reading it should be able to write code
  based on your description.
\item Provide, preferably in a well organized table, for $\beta_0,
  \beta_1, \lambda$, the posterior mean w/ estimate of MCMC standard
  error in parentheses, posterior 95\% credible intervals.
\item Provide an approximation of the correlation between $\beta_0,
  \beta_1$.
\item Provide approximate density plots for the marginal distributions of $\beta_0, \beta_1, \lambda$. 
\item Describe how you determined that your algorithm was producing
  reliable approximations. Provide relevant plots and justifications.
\end{enumerate}

\item Repeat Problem \#2 above except do it for Dataset\#3 on
  Angel. 
\begin{enumerate}
\item Provide, preferably in a well organized table, for $\beta_0,
  \beta_1, \lambda$, the posterior mean w/ estimate of MCMC standard
  error in parentheses, posterior 95\% credible intervals.
\item Provide approximate density plots for the marginal distributions of $\beta_0, \beta_1, \lambda$. 
\item Explain (if and) how you modified your MCMC algorithm in order
  to make it work better for this problem.
\end{enumerate}
\end{enumerate}

{\bf Please read the following instructions carefully.}

Important ({\it you will lose points if you do not follow these instructions}): 
\begin{enumerate}
\item Your writeup must be {\bf no longer than 5 pages} (it may be
  shorter, but not longer) including all plots and discussions. {\it
    Anything after the 5th page will be ignored.}
\item You may not discuss the exam with anyone except the instructor,
  that is, you cannot even talk to the T.A. Please feel free to ask me
  as many questions about this as you like. I may choose not to answer
  some of the questions, but I would rather you checked with me if you
  are unsure. If I choose to answer the question, I will email my
  response to the entire class.
\item You are required to submit your R code to {\tt Angel} {\bf by
    the due date/time}. It is an electronic system so it will not
  allow you to submit it even if you are a few seconds late.  Use the
  following naming convention for the file you upload -- first
  initial, name, .R extension. For instance if your name is John
  Newman you would attach a file called jnewman.R. If you have more
  than one program file, simply give them extensions.  For e.g.:
  jnewman1.R jnewman2.R etc.  If you use some other programming
  language, say Matlab or C, use the appropriate extension for that
  language, e.g. jnewman.c or jnewman.m
\item You are required to submit your pdf code to {\tt Angel} {\bf by
    the due date/time}. Follow the same naming convention, for e.g.:
  jnewman.pdf . Please do not submit your writeup in any other format.
\item Do not include your code in the writeup.
\item Make sure you add comments to your code; this is generally a
  good practice and can also be helpful in explaining the organization
  of your program to me.
\item Use a sensible editor for your program; otherwise, your program
  will be very difficult for me to read (and you will have more
  trouble with your programming).
\item % Try to follow good programming practices.
  % (as discussed during
%   previous assignments, discussions on computing during lectures,
%   email messages throughout the course).
  {\it It should be obvious how I would run your code, if I were to
    chose to do so}. If I find the code hard to run, for e.g. if your code
  calls files in other directories (that I don't have!) you will
  lose points. Test your code carefully.
%Add comments on how to run
%  the code if necessary. 
\item Avoid having too many plots or unreadable plots, and make sure
  they are all well labeled. Note: you can produce multiple plots on
  the same page by using the {\tt par} command in {\tt R}.
%\item Assume the same parametrization for the Gamma densities as used in class before.
\end{enumerate}
\end{document}
