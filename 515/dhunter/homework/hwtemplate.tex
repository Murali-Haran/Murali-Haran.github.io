% This is a simple template you could adapt to typeset your homework assignments
\documentclass{article}

% Set the page margins a bit narrower than the defaults to save space:
\setlength{\textwidth}{6in}
\setlength{\oddsidemargin}{0in}
\setlength{\evensidemargin}{0in}
\setlength{\topmargin}{-.5in}
\setlength{\textheight}{9in}


\begin{document}

\begin{center}
{\bf STAT 515  Homework \#1 \\ Myname Goeshere}
\end{center}

\begin{enumerate} % automatically creates a numbered list

\item A fair 6-sided die is rolled repeatedly.  Let $X$ equal the number of rolls required to
obtain the first 5 and $Y$ the number required to obtain the first 6. 
% The $ symbol is to open or close math mode; all variables should be typeset this way
Here is how I will calculate the following:

  \begin{enumerate} % Nesting of enumerate environments
  \item 
  \[ % The \[ is shorthand for starting the displayed equation environment.
  E(X) = \mbox{my derivation goes here.} % \mbox is one way to typeset plain text in a math environment
  \]
  \item 
  \begin{eqnarray*}
  E(X \mid Y=1) &=& \mbox{It is possible} \\
  &=& \mbox{to typeset} \\
  &=& \mbox{multiline equations}
  \end{eqnarray*}
  \item \[ E(X \mid Y=5) = \ldots\]
  \end{enumerate}

\item Suppose that $X$ is exponentially distributed with parameter $\lambda$; i.e., 
$E(X)=1/\lambda$.  
\[
P(X-2 \le t\mid X>2) = \ldots
\]



\item Here is an exercise in which the {\tt verbatim} environment is used to typeset 
code in {\tt R}:
\begin{verbatim}
## A programming loop (to repeat some code) can be written in R 
## using a "for loop". For example:
NUMREP <- 1000
foo <- rep(0, NUMREP)
for (i in 1:NUMREP) # repeat code below NUMREP times
  {
    mysim <- runif(50) #  generate 50 random unif(0,1)
    foo[i] <- sum(mysim<0.3) # count number of values less than 0.3
  }

myestimate <- mean(foo) 
myestimate
\end{verbatim}


\end{enumerate}

\end{document}

