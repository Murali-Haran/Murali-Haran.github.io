\documentclass{article}
\usepackage{url}
\usepackage{amsmath,bm}
\usepackage{amsfonts}
\pagestyle{empty}
\setlength{\textwidth}{6in}
\setlength{\oddsidemargin}{0in}
\setlength{\topmargin}{-.5in}
\setlength{\textheight}{9.25in}

\begin{document}

\begin{center}
{\bf STAT 515}

{\bf Homework \#3, due Wednesday, Feb. 8 at 2:30pm}
\end{center}

{\it Please make every assignment easier to grade by neatly organizing your
writeup and clearly labeling your final answers when appropriate.}

\begin{enumerate}

  %% 4. 44 understanding how transience affects M.C. behavior
  \item Suppose that a population consists of a fixed number, $2m$, of genes in
  any generation. Each gene is one of two possible genetic types. If any
  generation has exactly $i$ (of its $2m$) genes of type 1, then for any $0\le
  j\le 2m$, the next generation will have exactly $j$ genes of type 1 with
  binomial probability 
  \[ 
  {{2m}\choose j} \left( \frac{i}{2m} \right)^j \left( \frac{2m-i}{2m}
  \right)^{2m-j}.  
  \]
  Let $X_n$ denote the number of type 1 genes in the $n$th generation, 
  and assume $X_0=m$.

  \begin{enumerate}

    \item Find $E(X_n)$.

    \item Suppose that $m=6$. What is the probability that $X_n=m$ for some
    $n>0$?

    \item If $m=6$, what is the expected number of generations in which all
    genes except one are of the same type?

  \end{enumerate}

  %4.20, plus an extra magic square activity
  \item A transition matrix $P$ is called {\em doubly stochastic} if each
  of its column sums equals one.  
  
  \begin{enumerate}
  
    \item If an irreducible, aperiodic Markov chain has finitely many states and
    its transition matrix is doubly stochastic, prove that its limiting probability
    distribution is discrete uniform.
    
    \item Find a doubly stochastic transition matrix $P$ for a Markov chain with
    three states such that every entry of $P$ is a different integer multiple of
    $1/12$ and such that $P_{11}=0$ and $P_{22}=1/2$. Calculate the matrix
    $P^{10}$ (i.e., the tenth power of $P$). Explain why all nine entries of
    $P^{10}$ should be nearly the same.

  \end{enumerate}

  % 4. 24 understanding stationarity, limiting, marginal distri. of M.C. states.
  \item Consider three urns, one colored red, one white, and one blue.
  The red urn contains 1 red and 3 blue balls; the white urn contains
  3 white balls, 2 red balls, and 1 blue ball; the blue urn contains
  4 white balls, 3 red balls, and 2 blue balls.  At the initial stage,
  a ball is randomly selected from the red urn and then returned to
  that (red) urn. At every subsequent stage, a ball is randomly selected
  from the urn whose color is the same as that of the ball previously
  selected and is then returned to the urn from which it was drawn. 

  \begin{enumerate}

    \item Explain why this process is a Markov chain, then define an appropriate
    transition probability matrix to describe it.

    \item Does this process have a stationary distribution? Justify your answer.

    \item Explain why this process has a limiting distribution.

    \item In the long run, what proportion of the selected balls are red? What
    proportion are white? What proportion are blue?

    \item Simulate a Markov chain of length 100,000 using the information
    provided above and count the proportion of times the chain was in each of
    the states. Compare this to your answer.

    \item Suppose you have taken 4 steps, i.e., you start with the initial
    distribution to obtain $X_0$ and use the transition probability matrix above
    to obtain state $X_4$ of the Markov chain. What proportion of times would
    you expect $X_4$ to be red, white, and blue, respectively?
 
    \item Now simulate 10,000 realizations of the random variable $X_4$ using
    the initial distribution and transition probability matrix for this process.
    Calculate the proportion of times in your simulations that $X_4$ is red,
    white, and blue. Compare these proportions to your theoretically obtained
    answers above.

  \end{enumerate} 

  % 4.66 
  \item Suppose that in a branching process, the expected number of offspring of
  a given individual equals $4/5$. Find the expected number of individuals that
  ever exist in this population, assuming that $X_0=n$.

\end{enumerate}

\end{document}

