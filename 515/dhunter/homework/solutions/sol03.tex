\documentclass{article}
\usepackage{Sweave}
\usepackage{url}
\usepackage{amsmath,bm}
\usepackage{amsfonts}
\pagestyle{empty}
\setlength{\textwidth}{6in}
\setlength{\oddsidemargin}{0in}
\setlength{\topmargin}{-.5in}
\setlength{\textheight}{9.25in}

\begin{document}

\begin{center}
{\bf STAT 515}

{\bf Homework \#3 WITH SOLUTIONS}
\end{center}

\begin{enumerate}

  \item Suppose that a population consists of a fixed number, $2m$, of genes in
  any generation. Each gene is one of two possible genetic types. If any
  generation has exactly $i$ (of its $2m$) genes of type 1, then for any $0\le
  j\le 2m$, the next generation will have exactly $j$ genes of type 1 with
  binomial probability 
  \[ 
  {{2m}\choose j} \left( \frac{i}{2m} \right)^j \left( \frac{2m-i}{2m}
  \right)^{2m-j}.  
  \]
  Let $X_n$ denote the number of type 1 genes in the $n$th generation, 
  and assume $X_0=m$.

  \begin{enumerate}

    \item Find $E(X_n)$.
    \begin{quotation}{\bf Solution:}
    Conditioning shows that
    \[
    E(X_n) = E[ E(X_n \mid X_{n-1})] = E[X_{n-1}]
    \]
    because conditional on $X_{n-1}$, $X_n$ is binomial$(2m, X_{n-1}/2m)$
    and therefore the conditional expectation of $X_n$ given $X_{n-1}$ is
    $2m\times (X_{n-1}/2m)$.  We conclude that every $X_n$ has the same expectation,
    and since $X_0$ is the constant $m$, $E(X_n)$ must be $m$.
    \end{quotation}
    
    \item Suppose that $m=6$. What is the probability that $X_n=m$ for some
    $n>0$?
    \begin{quotation}{\bf Solution:}
    This problem is quite difficult without using a computer.  The strategy is to first construct
    the $13\times 13$ transition matrix, then define $P_T$ to be the $11\times 11$ submatrix 
    that remains after removing the first and last columns and rows (leaving only the entries
    corresponding to moves from transient states to other transient states).  Then, we calculate
    $S=(I-P_T)^{-1}$, whose entries give the expected number of generations spent in each
    transient state, conditional on the starting transient state.
\begin{Schunk}
\begin{Sinput}
> P <- matrix(0,13,13) # Initialize transition matrix
> for (i in 1:13) {
+   P[i,] <- dbinom(0:12, 12, (i-1)/12) # Each row uses dbinom with a different p
+ }
> id <- diag(rep(1, 11)) # Construct identity matrix
> S <- solve(id - P[2:12, 2:12])    # S has info on transient states 1 through 11
\end{Sinput}
\end{Schunk}
    Using an argument developed in Section 4.6, 
    \[
    P(\mbox{re-entering state 6} \mid \mbox{starting in state 6}) = \frac{S_{66}-1}{S_{66}},
    \]
    So the probability in this case is given by:
\begin{Schunk}
\begin{Sinput}
> (S[6,6]-1)/S[6,6]
\end{Sinput}
\begin{Soutput}
[1] 0.6208155
\end{Soutput}
\end{Schunk}
    \end{quotation}

    \item If $m=6$, what is the expected number of generations in which all
    genes except one are of the same type?
    \begin{quotation}{\bf Solution:}
    The states corresponding to ``all genes but one of the same type'' are states 1 
    and 11.  Therefore, using the $S$ matrix calculated in part (b), we obtain 
    as an answer
\begin{Schunk}
\begin{Sinput}
> S[6,1] + S[6,11]
\end{Sinput}
\begin{Soutput}
[1] 1.816812
\end{Soutput}
\end{Schunk}
    \end{quotation}

  \end{enumerate}

  \item A transition matrix $P$ is called {\em doubly stochastic} if each
  of its column sums equals one.  
  
  \begin{enumerate}
  
    \item If an irreducible, aperiodic Markov chain has finitely many states and
    its transition matrix is doubly stochastic, prove that its limiting probability
    distribution is discrete uniform.
    \begin{quotation}{\bf Solution:}
    Since a finite, irreducible, aperiodic chain must be ergodic, we know that there
    is a unique solution to the equations $\pi_j=\sum_i\pi_iP_{ij}$.  Therefore, it suffices to
    show that if $\pi_j=\pi_i$ for all $i$ and $j$, then these equations are satisfied.  However,
    this is immediate, since if $\pi_i=\pi_j$ then we obtain
    \[
    \pi_j = \sum_i \pi_i P_{ij} = \sum_i\pi_j P_{ij} = \pi_j \sum_i P_{ij} = \pi_j
    \]
    because $\sum_iP_{ij}=1$ by the assumption of double stochasticity.
    \end{quotation}
    
    \item Find a doubly stochastic transition matrix $P$ for a Markov chain with
    three states such that every entry of $P$ is a different integer multiple of
    $1/12$ and such that $P_{11}=0$ and $P_{22}=1/2$. Calculate the matrix
    $P^{10}$ (i.e., the tenth power of $P$). Explain why all nine entries of
    $P^{10}$ should be nearly the same.
    \begin{quotation}{\bf Solution:}
    If we label the states 1, 2, and 3 (instead of 0, 1, and 2), then the only
    matrices that work here are $P$ and its transpose, where
    \[
    P = \frac{1}{12}
    \begin{bmatrix}
    0 & 5 & 7\\
    4 & 6 & 2 \\
    8 & 1 & 3 
    \end{bmatrix}.
    \]
    (Can you argue that these two are the only possibilities?)
    For $P^{10}$, we obtain
\begin{Schunk}
\begin{Sinput}
> P <- matrix(c(0, 4, 8, 5, 6, 1, 7, 2, 3), 3, 3) / 12
> P2 <- P %*% P
> P4 <- P2 %*% P2
> print(P10 <- P2 %*% P4 %*% P4)
\end{Sinput}
\begin{Soutput}
          [,1]      [,2]      [,3]
[1,] 0.3349768 0.3327671 0.3322561
[2,] 0.3330226 0.3334437 0.3335336
[3,] 0.3320006 0.3337892 0.3342103
\end{Soutput}
\end{Schunk}
    We know from part (a) that $\lim_{n\to\infty} P^n$ exists and that each row of
    this limit equals the discrete uniform $(1/3, 1/3, 1/3)$.  Here, we only take $n=10$, but
    the theory tells us that $P^n$ will get closer and closer to the limiting matrix 
    in which each entry equals $1/3$.
    \end{quotation}

  \end{enumerate}

  \item Consider three urns, one colored red, one white, and one blue.
  The red urn contains 1 red and 3 blue balls; the white urn contains
  3 white balls, 2 red balls, and 1 blue ball; the blue urn contains
  4 white balls, 3 red balls, and 2 blue balls.  At the initial stage,
  a ball is randomly selected from the red urn and then returned to
  that (red) urn. At every subsequent stage, a ball is randomly selected
  from the urn whose color is the same as that of the ball previously
  selected and is then returned to the urn from which it was drawn. 

  \begin{enumerate}

    \item Explain why this process is a Markov chain, then define an appropriate
    transition probability matrix to describe it.
    \begin{quotation}{\bf Solution:}
    Label the states $1=\mbox{red}$, $2=\mbox{white}$, and $3=\mbox{blue}$
    and let the stochastic process $X_n$ be the label of the $n$th ball drawn, 
    where $X_0$ is the first ball picked (from the red urn).  Since the color of $X_n$
    has a distribution depending only on the urn from which the ball is picked, and
    this urn is completely determined by $X_{n-1}$, the Markovian property is 
    satisfied by the process $X_0, X_1, \ldots$.
    
    The transition probabilities are given by
    \[
    P = 
    \begin{bmatrix}
    1/4 & 0 & 3/4 \\
    1/3 & 1/2 & 1/6 \\
    1/3 & 4/9 & 2/9 
    \end{bmatrix}
    = \frac{1}{36}
    \begin{bmatrix}
    9 & 0 & 27 \\
    12 & 18 & 6 \\
    12 & 16 & 8 
    \end{bmatrix}.
    \]
    \end{quotation}

    \item Does this process have a stationary distribution? Justify your answer.
    \begin{quotation}{\bf Solution:}
    This Markov chain is irreducible since there exists a path from $i$ to $j$ for any $i$ and $j$.
    It is aperiodic because it is irreducible and $P_{11}>0$. Finally, it is positive recurrent because
    it is irreducible and there are only finitely many states.  So it is irreducible and ergodic, which
    means that it has both a limiting distribution and a stationary distribution, and these two
    coincide.  (Technically, only the irreducibility and positive recurrence are needed to
    have a stationary distribution.)
    \end{quotation}

    \item Explain why this process has a limiting distribution.
    \begin{quotation}{\bf Solution:}
    See part (b).
    \end{quotation}

    \item In the long run, what proportion of the selected balls are red? What
    proportion are white? What proportion are blue?
    \begin{quotation}{\bf Solution:}
    One way to find $\pi$ satisfying $\pi^\top=\pi^\top P$ is to search for a left-eigenvector whose
    eigenvalue equals one (i.e., search for a right-eigenvector of $P^\top$):
\begin{Schunk}
\begin{Sinput}
> P <- matrix(c(9, 12, 12, 0, 18, 16, 27, 6, 8), 3, 3) / 36
> print(e <- eigen(t(P)))
\end{Sinput}
\begin{Soutput}
$values
[1]  1.00000000 -0.08333333  0.05555556

$vectors
           [,1]       [,2]          [,3]
[1,] -0.5314934  0.1860807  2.530214e-17
[2,] -0.5627577  0.5954583  7.071068e-01
[3,] -0.6331024 -0.7815391 -7.071068e-01
\end{Soutput}
\end{Schunk}
    Since the first eigenvalue is one, we want the first column of the eigenvector matrix, normalized
    so that its entries sum to one:
\begin{Schunk}
\begin{Sinput}
> pi <- as.real(e$vec[,1])
> print( pi <- pi/sum(pi))
\end{Sinput}
\begin{Soutput}
[1] 0.3076923 0.3257919 0.3665158
\end{Soutput}
\end{Schunk}
    These are the long-run proportions of
    red, white, and blue, respectively.
   % Incidentally, it is possible to find the exact value of this eigenvalue using, say, Mathematica.  
   % We obtain $\pi^\top = (50/131, 27/131, 54/131)$,  and we may verify using integer 
   % arithmetic that these proportions exactly solve the equation $\pi^\top=\pi^\top P$:
   %<<>>=
 %c(50, 27, 54) %*% matrix(c(9, 12, 12, 0, 18, 16, 27, 6, 8), 3, 3)
 %36*c(50, 27, 54)
 %@        
    \end{quotation}

    \item Simulate a Markov chain of length 100,000 using the information
    provided above and count the proportion of times the chain was in each of
    the states. Compare this to your answer.
    \begin{quotation}{\bf Solution:}
\begin{Schunk}
\begin{Sinput}
> X <- rep(0, 100000) # This is where the realizations will be stored
> pi0 <- P[1,] # The original proportions are in row 1 (the red urn's makeup)
> currentState <- sample(3, 1, prob=pi0)
> for(j in 1:100000) { # Sample according to the correct row in P
+   X[j] <- currentState <- sample(3, 1, prob=P[currentState,])
+ }
> # Now summarize the states visited by proportion:
> rbind (observed=table(X) / 100000, theoretical=pi)
\end{Sinput}
\begin{Soutput}
                    1         2         3
observed    0.3092900 0.3235600 0.3671500
theoretical 0.3076923 0.3257919 0.3665158
\end{Soutput}
\end{Schunk}
    Clearly the observed proportions are close to the theoretical
    stationary probabilities.
    \end{quotation}

    \item Suppose you have taken 4 steps, i.e., you start with the initial
    distribution to obtain $X_0$ and use the transition probability matrix above
    to obtain state $X_4$ of the Markov chain. What proportion of times would
    you expect $X_4$ to be red, white, and blue, respectively?
    \begin{quotation}{\bf Solution:}
    Starting from $\pi_0^\top=(1/4, 0, 3/4)$ (the proportions in the red urn), we obtain
    \[
    \pi_0^\top P^4 = 
    \begin{bmatrix}
    \frac{2067201}{6718464} & \frac{2188752}{6718464} & \frac{2462511}{6718464}
    \end{bmatrix},
    \]
    but of course the exact answers are really messy and the numerical values are 
    fine here:
\begin{Schunk}
\begin{Sinput}
> print (pi4 <- P[1,] %*% P %*% P %*% P %*% P)
\end{Sinput}
\begin{Soutput}
          [,1]      [,2]      [,3]
[1,] 0.3076895 0.3257816 0.3665289
\end{Soutput}
\end{Schunk}
    \end{quotation}
 
    \item Now simulate 10,000 realizations of the random variable $X_4$ using
    the initial distribution and transition probability matrix for this process.
    Calculate the proportion of times in your simulations that $X_4$ is red,
    white, and blue. Compare these proportions to your theoretically obtained
    answers above.
    \begin{quotation}{\bf Solution:}
\begin{Schunk}
\begin{Sinput}
> X4 <- rep(0, 10000) # This is where the realizations will be stored
> pi0 <- P[1,] # The original proportions are in row 1 (the red urn's makeup)
> for (i in 1:10000) {
+   currentState <- sample(3, 1, prob=pi0)
+   for (j in 1:4) { # Sample according to the correct row in P
+     currentState <- sample(3, 1, prob=P[currentState,])
+   }
+   X4[i] <- currentState
+ }
> # Now summarize the states visited by proportion:
> rbind (observed=table(X4) / 10000, theoretical=pi4)
\end{Sinput}
\begin{Soutput}
                 1         2         3
observed 0.3083000 0.3223000 0.3694000
         0.3076895 0.3257816 0.3665289
\end{Soutput}
\end{Schunk}
    Clearly the observed proportions are close to the theoretical
    stationary probabilities.  If we wished, it would be easy to add
    confidence intervals to our sample proportions since these
    10,000 realizations were generated independently.
    \end{quotation}

  \end{enumerate} 

  \item Suppose that in a branching process, the expected number of offspring of
  a given individual equals $4/5$. Find the expected number of individuals that
  ever exist in this population, assuming that $X_0=n$.
    \begin{quotation}{\bf Solution:}
    The results in Section~4.7 show that $E X_t = \mu^tE X_0$, which implies
    $E X_t = n(4/5)^t$ in this case.  Therefore, the expected total number of individuals is
    \[
    \sum_{t=0}^\infty E X_t = n\sum_{t=0}^\infty (4/5)^t = \frac{n}{1-\frac45} = 5n.
    \]
    \end{quotation}

\end{enumerate}

\end{document}

