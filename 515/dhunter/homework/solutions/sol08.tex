\documentclass{article}
\usepackage{url}
\usepackage{amsmath,bm}%
\usepackage{amsfonts}%
\pagestyle{empty}
\setlength{\textwidth}{7in}
\setlength{\oddsidemargin}{-.5in}
\setlength{\evensidemargin}{-.5in}
\setlength{\topmargin}{-.75in}
\setlength{\textheight}{9.4in}

\newcommand{\beaa}{\begin{eqnarray*}}
\newcommand{\eeaa}{\end{eqnarray*}}
\newcommand{\bea}{\begin{eqnarray}}
\newcommand{\eea}{\end{eqnarray}}
\def\E{\mathop{\rm E\,}\nolimits}
\def\Var{\mathop{\rm Var\,}\nolimits}
\def\Cov{\mathop{\rm Cov\,}\nolimits}
\def\Corr{\mathop{\rm Corr\,}\nolimits}
\def\logit{\mathop{\rm logit\,}\nolimits}
\newcommand{\eid}{{\stackrel{\cal{D}}{=}}}
\newcommand{\cip}{{\stackrel{{P}}{\to}}}
\def\bR{\mathbb{R}}     % real line

\usepackage{Sweave}
\begin{document}

\begin{center}
{\bf STAT 515}

{\bf Homework \#8 WITH SOLUTIONS}

\end{center}

\begin{enumerate}

  \item The matrix exponential function is defined as
  \[
  \exp\{ M \} = \sum_{i=0}^\infty \frac{M^i}{i!}.
  \]
  However, this definition does not provide a suitable method for calculating
  $\exp\{M\}$ for a given $M$. One simple alternative is to use one of the two
  formulas
  \[
  \exp \{ M \} = \lim_{n\to\infty} \left( I + \frac{M}{n} \right) ^n
  = \lim_{n\to\infty} \left[ \left(I - \frac{M}{n} \right)^{-1} \right]^n.
  \]
  For the post office problem on HW \#7, the rate matrix is given by
  \[
  R = \begin{bmatrix}
  -8 & 8 & 0 & 0 & 0 \\
  12 & -20 & 8 & 0 & 0 \\
  0 & 12 & -20 & 8 & 0 \\
  0 & 0 & 12 & -20 & 8 \\
  0 & 0 & 0 & 12 & -12
  \end{bmatrix},
  \]
  where rates are in hours. Approximate the value of $\exp \{.5R\}$, which is
  the transition probability matrix for a time step of 30 minutes, using two
  methods:

    \begin{enumerate}

    \item For successively larger powers of 2, i.e., $n=2, 4, 8, \ldots$, find
    the value of $( I + .5R/n) ^n$. Continue until the change in each entry is
    smaller than $10^{-5}$. Report your final value of $n$ and your final
    approximation of $\exp \{.5R\}$.
    \begin{quotation}{\bf Solution:}
    First, let's set up the $R$ matrix:
\begin{Schunk}
\begin{Sinput}
> R <- matrix(c(-8, 12, 0, 0, 0, 8, -20, 12, 0, 0, 0, 8, -20, 
+               12, 0, 0, 0, 8, -20, 12, 0, 0, 0, 8, -12),   5, 5)
\end{Sinput}
\end{Schunk}
    Next, we'll use a loop to continue trying larger and larger $n$,
    starting with $n=2$ and doubling it at every step,
    until the maximum change is smaller than $10^{-5}$:
\begin{Schunk}
\begin{Sinput}
> lastAnswer <- matrix(0, 5, 5)
> finished <- FALSE
> k <- 0
> n <- 1
> while (!finished) {
+   k <- k+1
+   n <- n*2 # double the n
+   answer <- diag(5) + .5*R/n 
+   for (j in 1:k) { # raise answer to the nth power by squaring repeatedly
+     answer <- answer %*% answer
+   }
+   finished <- all(abs(answer - lastAnswer) < 1e-5) # check for convergence
+   lastAnswer <- answer
+ }
> lastAnswer
\end{Sinput}
\begin{Soutput}
          [,1]      [,2]      [,3]       [,4]       [,5]
[1,] 0.4176125 0.2665951 0.1623088 0.09580071 0.05768283
[2,] 0.3998927 0.2611831 0.1668329 0.10513201 0.06695930
[3,] 0.3651949 0.2502494 0.1754179 0.12357079 0.08556707
[4,] 0.3233274 0.2365470 0.1853562 0.14607047 0.10869892
[5,] 0.2920193 0.2259876 0.1925259 0.16304838 0.12641878
\end{Soutput}
\begin{Sinput}
> n
\end{Sinput}
\begin{Soutput}
[1] 32768
\end{Soutput}
\end{Schunk}
    So the desired accuracy required $n=2^{15}= 32{,}768$.
    \end{quotation}
    
    \item Repeat the same procedure as in part (a) but use $[(I-.5R/n)^{-1}]^n$
    instead.
    \begin{quotation}{\bf Solution:}
    This requires only minor modifications to the code above:
\begin{Schunk}
\begin{Sinput}
> lastAnswer <- matrix(0, 5, 5)
> finished <- FALSE
> k <- 0
> n <- 1
> while (!finished) {
+   k <- k+1
+   n <- n*2 # double the n
+   answer <- solve(diag(5) - .5*R/n) 
+   for (j in 1:k) { # raise answer to the nth power by squaring repeatedly
+     answer <- answer %*% answer
+   }
+   finished <- all(abs(answer - lastAnswer) < 1e-5) # check for convergence
+   lastAnswer <- answer
+ }
> lastAnswer
\end{Sinput}
\begin{Soutput}
          [,1]      [,2]      [,3]       [,4]       [,5]
[1,] 0.4176172 0.2665964 0.1623075 0.09579832 0.05768059
[2,] 0.3998946 0.2611839 0.1668326 0.10513091 0.06695803
[3,] 0.3651919 0.2502489 0.1754190 0.12357216 0.08556805
[4,] 0.3233193 0.2365445 0.1853582 0.14607473 0.10870316
[5,] 0.2920080 0.2259833 0.1925281 0.16305474 0.12642582
\end{Soutput}
\begin{Sinput}
> n
\end{Sinput}
\begin{Soutput}
[1] 32768
\end{Soutput}
\end{Schunk}
    This example also required $n=2^{15}=32{,}768$.
    \end{quotation}
    

    \item Use the {\tt expm} function in R or Matlab to evaluate $\exp \{.5R\}$
    and compare with the two approximations you obtained.
    
    {\em In R, you will have to install and load the package called {\tt
    Matrix}. Do this using {\tt install.packages("Matrix")} and then {\tt
    library(Matrix)}.} 
    \begin{quotation}{\bf Solution:}
\begin{Schunk}
\begin{Sinput}
> library(Matrix)
> expm(.5*R)
\end{Sinput}
\begin{Soutput}
5 x 5 Matrix of class "dgeMatrix"
          [,1]      [,2]      [,3]       [,4]       [,5]
[1,] 0.4176149 0.2665957 0.1623082 0.09579952 0.05768171
[2,] 0.3998936 0.2611835 0.1668328 0.10513146 0.06695866
[3,] 0.3651934 0.2502491 0.1754184 0.12357147 0.08556756
[4,] 0.3233234 0.2365458 0.1853572 0.14607260 0.10870104
[5,] 0.2920136 0.2259855 0.1925270 0.16305156 0.12642230
\end{Soutput}
\end{Schunk}
    The answers in parts (a) and (b) are very close to the {\tt expm} answer.
    \end{quotation}
        
    \end{enumerate}

  \item Problem 3 in homework \#7 described two video game machines at an
  amusement park. For video game $i$, each period when it is being used is
  exponentially distributed with mean $1/\alpha_i$ hours and each period when it
  is not being used is exponentially distributed with mean $1/\beta_i$ hours,
  independent of the other machine. Furthermore, $\alpha_1=2$, $\alpha_2=3$,
  $\beta_1=5$, and $\beta_2=6$.
  
    \begin{enumerate}
    
    \item If neither machine is in use when the park opens at 8:00am, find the
    probability that both machines are in use at 9:30am.
    \begin{quotation}{\bf Solution:}
    To find the answer, we can find the probability transition matrix $P(1.5)$,
    which is given by the matrix exponential $\exp\{ 1.5 R\}$:
\begin{Schunk}
\begin{Sinput}
> R <- matrix(c(-11, 3, 2, 0, 6, -8, 0, 2, 5, 0, -8, 3, 0, 5, 6, -5), 4, 4)
> expm(1.5*R)
\end{Sinput}
\begin{Soutput}
4 x 4 Matrix of class "dgeMatrix"
           [,1]      [,2]      [,3]      [,4]
[1,] 0.09524491 0.1904890 0.2380893 0.4761767
[2,] 0.09524452 0.1904894 0.2380884 0.4761777
[3,] 0.09523573 0.1904707 0.2380985 0.4761951
[4,] 0.09523534 0.1904711 0.2380975 0.4761960
\end{Soutput}
\end{Schunk}
    The answer is the probability that starting in the first state at time zero,
    the chain is in the fourth state at time 1.5:
\begin{Schunk}
\begin{Sinput}
> expm(1.5*R)[1,4]
\end{Sinput}
\begin{Soutput}
[1] 0.4761767
\end{Soutput}
\end{Schunk}
    \end{quotation}
    
    
    \item Simulate 10,000 realizations of the Markov chain and give a 95\%
    confidence interval for the probability in part (a) based on your
    simulation. Does your empirical estimate agree with the theoretical value?
    \begin{quotation}{\bf Solution:}
    I will make some minor changes to the code used in homework \#7, problem 
    3(c):
\begin{Schunk}
\begin{Sinput}
> n <- 10000
> finalState <- rep(0, n)
> maxTime <- 1.5 # This is the cutoff time.
> for (count in 1:n) {
+   currentTime <- 0
+   currentState <- 1 # Assume that we always start in state 1 at time 0
+   finished <- FALSE # We'll set this to TRUE when it's time to stop.
+   while (!finished) {
+     deltaTime <- rexp(1, rate = -R[currentState, currentState])
+     if (currentTime + deltaTime > maxTime) {
+       # Now we need to finish this chain and declare the final state decided
+       finalState[count] <- currentState
+       deltaTime <- maxTime - currentTime
+       finished <- TRUE
+     }
+     currentTime <- currentTime + deltaTime
+     possibleMoves <- (1:4)[-currentState]
+     currentState <- sample(possibleMoves, 1, prob=R[currentState,possibleMoves])
+   }
+ }
> table(finalState) / n
\end{Sinput}
\begin{Soutput}
finalState
     1      2      3      4 
0.0986 0.1886 0.2488 0.4640 
\end{Soutput}
\end{Schunk}
    Notice how close the four probabilities are to the first row (really, all rows) of the
    matrix found in part (a).  Here is a 95\% confidence interval for the probability
    of ending in the fourth state:
\begin{Schunk}
\begin{Sinput}
> phat <- sum(finalState==4) / n
> phat + c(-1.96, 1.96) * sqrt(phat * (1-phat) / n)
\end{Sinput}
\begin{Soutput}
[1] 0.4542254 0.4737746
\end{Soutput}
\end{Schunk}
    \end{quotation}    
    
    \end{enumerate}
  
  \item Suppose that $X_1, X_2, X_3, X_4$ are i.i.d.~from a uniform$(0,1)$
  distribution. Let $S=X_1+X_2+X_3+X_4$.
  
    \begin{enumerate}
  
    \item Find $P(S<1)$ exactly using a four-dimensional integral. (Hint: This
    is not too difficult.)
    \begin{quotation}{\bf Solution:}
    The joint density on the four-dimensional hypercube $(0,1)^4$ is just the
    constant 1.  The probability of $S<1$ is found by integrating 
    this density over the region where $S<1$:
    \begin{eqnarray*}
    \int_0^1 \int_0^{1-w} \int_0^{1-w-x} \int_0^{1-w-x-y} dz\, dy\, dx\, dw &=&
    \int_0^1 \int_0^{1-w} \int_0^{1-w-x} (1-w-x-y)\, dy\, dx\, dw \\ &=&
    \frac12\int_0^1 \int_0^{1-w} \frac12 (1-w-x)^2 \, dx\, dw \\ &=&
    \frac16 \int_0^1 (1-w)^3 \, dw \\ &=&
    \frac1{24}.
    \end{eqnarray*}
    \end{quotation}
        
    \item Now consider $P(S<1.5)$. This is much more difficult to find
    analytically. Instead, use Monte Carlo simulation to approximate this
    probability. Give a 99\% confidence interval for the true probability, and
    use a large enough sample so that your interval is no wider than $0.01$.
    Report the sample size you used in addition to the interval.
    \begin{quotation}{\bf Solution:}
    Since we are finding a proportion, and the standard deviation of 
    a sample proportion is not more than $1/\sqrt{4n}$, we could 
    take $n$ large enough
    so that $2.58/\sqrt{4n} < .005$ since the 99\% confidence interval is 
    $\hat p\pm 2.58\times (\mbox{standard error})$.  This leads to $n$ larger than
    about 66,000.  Let's use 100,000 just for a nice round number:
\begin{Schunk}
\begin{Sinput}
> n <- 1e6
> x <- matrix(runif(4*n), ncol=4)
> phat <- sum(rowSums(x)<1.5) / n
> phat + c(-2.58, 2.58) * sqrt(phat * (1-phat) / n)
\end{Sinput}
\begin{Soutput}
[1] 0.1995289 0.2015951
\end{Soutput}
\end{Schunk}
    So the 99\% confidence interval obtained from a sample of size 100,000
    has a width of much less than 0.01 (the width is about 0.002).
    \end{quotation}
        
    \item The central limit theorem approximation to $P(S<1.5)$ is $P(Y<1.5)$,
    where $Y$ is a normal random variable with the same mean and variance as
    $S$. Based on your answer to part (b), how good does the central limit
    theorem approximation appear in this case?
    \begin{quotation}{\bf Solution:}
    Since $X_i$ has mean 1/2 and variance 1/12, $S$ has mean 2 and variance 
    1/3.  Thus, the central limit theorem approximation is 
\begin{Schunk}
\begin{Sinput}
> pnorm(1.5, mean=2, sd=1/sqrt(3))
\end{Sinput}
\begin{Soutput}
[1] 0.1932381
\end{Soutput}
\end{Schunk}
    \end{quotation}
        
    \end{enumerate}

\end{enumerate}

\end{document}

