\documentclass{article}
\usepackage{url}
\usepackage{amsmath,bm}%
\usepackage{amsfonts}%
\pagestyle{empty}
\setlength{\textwidth}{7in}
\setlength{\oddsidemargin}{-.5in}
\setlength{\evensidemargin}{-.5in}
\setlength{\topmargin}{-.5in}
\setlength{\textheight}{9in}

\newcommand{\beaa}{\begin{eqnarray*}}
\newcommand{\eeaa}{\end{eqnarray*}}
\newcommand{\bea}{\begin{eqnarray}}
\newcommand{\eea}{\end{eqnarray}}
\def\E{\mathop{\rm E\,}\nolimits}
\def\Var{\mathop{\rm Var\,}\nolimits}
\def\Cov{\mathop{\rm Cov\,}\nolimits}
\def\Corr{\mathop{\rm Corr\,}\nolimits}
\def\logit{\mathop{\rm logit\,}\nolimits}
\newcommand{\eid}{{\stackrel{\cal{D}}{=}}}
\newcommand{\cip}{{\stackrel{{P}}{\to}}}
\def\bR{\mathbb{R}}     % real line

\usepackage{Sweave}
\begin{document}

\begin{center}
{\bf STAT 515}

{\bf Homework \#7 WITH SOLUTIONS}

\end{center}

\begin{enumerate}

  \item Customers arrive at a post office at a Poisson rate of 8 per hour. There
  is a single person serving customers, and service times are exponentially
  distributed (and independent) with mean 5 minutes. Suppose that an arriving
  customer will decide to wait in line if and only if there are two or fewer
  people already in line.
  
    \begin{enumerate}

    \item In the long run, what fraction of the time will there be at least 1
    customer in the post office? Find the answer in two different ways:

      \begin{enumerate}
      
      \item Write out the rate matrix (or generator) for the continuous-time
      Markov chain and find the stationary distribution using the generator.
      \begin{quotation}{\bf Solution:}
      Although there was a bit of a misunderstanding about how many people
      should be allowed in the post office because of my ambiguous wording, the intent
      was to have 5 possible states, namely, 0 through 4 total people at the post office.  Here
      is the rate matrix with states 0 through 4 in order:
      \[
      R = \begin{bmatrix}
      -8 & 8 & 0 & 0 & 0 \\
      12 & -20 & 8 & 0 & 0 \\
      0 & 12 & -20 & 8 & 0 \\
      0 & 0 & 12 & -20 & 8 \\
      0 & 0 & 0 & 12 & -12
      \end{bmatrix}.
      \]
      The stationary distribution $\pi$ satisfies $\pi^\top R=0$, which means that $\pi$
      is an eigenvector of $R^\top$ whose eigenvalue equals zero:
\begin{Schunk}
\begin{Sinput}
> R <- matrix(c(-8, 12, 0, 0, 0, 8, -20, 12, 0, 0, 0, 8, -20, 
+                        12, 0, 0, 0, 8, -20, 12, 0, 0, 0, 8, -12),
+          5, 5)
> e <- eigen(t(R))
> pi <- e$vec[, abs(e$val)< 1e-15]
> pi <- pi/sum(pi)
\end{Sinput}
\end{Schunk}
     We want $1-\pi_0$:
\begin{Schunk}
\begin{Sinput}
> 1 - pi[1]
\end{Sinput}
\begin{Soutput}
[1] 0.6161137
\end{Soutput}
\end{Schunk}
     \ 
      \end{quotation}
      
      \item Use the fact that this is a birth-death process to find the
      stationary distribution that satisfies the detailed balance equations.
      \begin{quotation}{\bf Solution:}
      From detailed balance, we get the equations
      $\pi_1 = 8\pi_0/12$, 
      $\pi_2 = 8\pi_1/12$, 
      $\pi_3 = 8\pi_2/12$, and 
      $\pi_4 = 8\pi_3/12$.  Since the $\pi$ vector must sum to one, we obtain
      \[
      \pi_0 = \left( 1 + \frac23 + \frac4{9} + \frac8{27} + \frac{16}{81}\right)^{-1} = \frac{81}{211}.
      \]
      Therefore, the desired answer is $1-\pi_0 = 130/211 = 0.6161$.
      \end{quotation}
      
      \end{enumerate}
      
    \item In the long run, what is the expected number of customers in the post
    office (in line or being served) at any given time?
    \begin{quotation}{\bf Solution:}
    From part (a), we know that 
    \[
    \pi^\top = \left( \frac{81}{211}, \frac{54}{211}, \frac{36}{211}, \frac{24}{211},
    \frac{16}{211} \right). 
    \]
    Therefore, the expectation equals
    \[
    \frac{54}{211} + 2\times \frac{36}{211} + 3\times \frac{24}{211} 
    + 4\times \frac{16}{211} = \frac{262}{211} = 1.242.
    \]
    \end{quotation}
    
    \item What is the probability that an arriving potential customer will
    decide to leave because there are already 3 people in line?
    \begin{quotation}{\bf Solution:}
    Since the indicator of ``next event is a customer arrival'' is independent of the
    rest of the Markov chain, the answer to this question is simply
    $\pi_4$, the limiting probability of 3 people in line.  This equals $16/211=0.0758$.
    \end{quotation}
    
    \item If a new cash register is installed that decreases the mean service
    time to 4 minutes, how many more customers per hour, on average, can be
    served by this post office?
    \begin{quotation}{\bf Solution:}
    The wording of this question appears to be ambiguous.  In one interpretation, we recalculate
    $\pi_4$ using the new value $\mu=15$:
      From detailed balance, we get the equations
      $\pi_1 = 8\pi_0/15$, 
      $\pi_2 = 8\pi_1/15$, 
      $\pi_3 = 8\pi_2/15$, and 
      $\pi_4 = 8\pi_3/15$.  Since the $\pi$ vector must sum to one, we obtain
      \[
      \pi_0 = \left( 1 + \frac8{15} + \frac{64}{225} + \frac{512}{3375} + \frac{4096}{50625}\right)^{-1} 
      = \frac{50625}{103801}.
      \]
      This gives $\pi_4=(50625/103801)(8/15)^4 = 4096/103801$.  Since 
      customers arrive at the rate of 8 per hour, the average number of additional customers
      served is $8[(16/211) - (4096/103801)]=0.291$ per hour. 
          
       According to a different interpretation, we might consider that customers are only being 
       served when they are actually at the service window.  Under this interpretation, the number
       served per hour is exactly 12 whenever there is at least one person in the post office
       under the first scenario, so we get a mean of
       $12(130/211)$.  Under the second scenario, we obtain 
       $15(53176/103801)$, which also yields a difference of
       $0.291$ customers per hour.  
       
       At first, I was surprised that these two methods of solution give exactly the same 
       answer!  But after some more thought, this makes sense:  The long-run rate equals
       limit of the total number of customers divided by the total number of hours
       as the latter goes to infinity.  For this purpose, it does not matter whether the 
       1 to 3 people still in line at the end of the very-long time are counted among
       the total number of customers or not, since the limit will be the same whether or
       not the numerator is increased by 1 to 3 customers.     So evidently, my wording
       was not quite as ambiguous as I had thought!
      \end{quotation}
    
    \item The manager of the post office wants to be able to serve at least 95\%
    of the potential customers who arrive at the post office. What mean service
    time will attain this goal?
    \begin{quotation}{\bf Solution:}
    Here, we want $\pi_4$ to equal 0.05.  If $x$ is the number of customers per hour, then
    \[
    \pi_4 = \left( \frac8x \right)^4 \left( 1 + \frac8x + \frac{64}{x^2} +
    \frac{512}{x^3} + \frac{4096}{x^4} \right)^{-1}.
    \]
    With a bit of simplification, we find that the equation $\pi_4=1/20$ is the same as
    \[
    x^4+8x^3+64x^2+512x  = 19(4096).
    \]
    Since the left hand side is a strictly increasing function when $x$ is positive, there is
    a unique positive solution, which may be found numerically.  This is easy to do using
    simple trial-and-error, but an alternative uses
    the {\tt uniroot} function in R (and we know from parts (c) and (d) that the answer lies
    between 12 and 15):
\begin{Schunk}
\begin{Sinput}
> print(r <- uniroot(function(x) x^4+8*x^3+64*x^2+512*x-19*4096, lower=12, upper=15)$root)
\end{Sinput}
\begin{Soutput}
[1] 13.87314
\end{Soutput}
\end{Schunk}
    We now divide this number into 60 to obtain a mean service time, in minutes, of 
\begin{Schunk}
\begin{Sinput}
> 60/r
\end{Sinput}
\begin{Soutput}
[1] 4.324906
\end{Soutput}
\end{Schunk}
    \end{quotation}
    
    \end{enumerate}
    
  \item Suppose that a continuous-time Markov chain $X(t)$ has rate matrix
  \[
  R = \begin{bmatrix}
  -\alpha & \alpha \\
  \beta & -\beta
  \end{bmatrix}.  
  \]  
  Given positive times $s$ and $t$, calculate $\Corr[X(s), X(t)]$. Does your
  answer depend on the starting state of the chain (i.e., value of $X(0)$)?
  
  (NB: The formula for $\Corr(X,Y)$ is
  $\Corr(X,Y) = \Cov(X,Y)/\sqrt{\Var(X)\Var(Y)}$.
  Since the correlation is invariant to linear transformations, the two
  possible values that $X(t)$ may take do not influence your answer.)
  \begin{quotation}{\bf Solution:}
  If we let 0 and 1 be the two values that $X(t)$ can take, then the calculations will be 
  simplest.  Also, let us assume that $s<t$, since we may do so without loss of generality.
  
  In the case $X(0)=0$, using the notation in Section 6.4 of the textbook (the example
  about the two-state Markov chain), we obtain
  \[
  E X(s) = P_{01}(s), \quad \Var X(s) = P_{01}(s)P_{00}(s),  \quad
  E X(t) = P_{01}(t), \quad \Var X(t) = P_{01}(t)P_{00}(t),
  \]
  and
  \[
  E [X(s)X(t)] = P[X(t)=1 \mid X(s)=1] P[X(s)=1] = P_{11}(t-s)P_{01}(s).
  \]
  Putting everything together, we obtain
  \[
  \Corr[X(s),X(t) | X(0)=0] = \frac{P_{01}(s) [ P_{11}(t-s) - P_{01}(t)]}
  {\sqrt{P_{01}(s)P_{00}(s)P_{01}(t)P_{00}(t)}}.
  \]  
  Similarly, in the case $X(0)=1$, we obtain
  \[
  \Corr[X(s),X(t) | X(0)=1] = \frac{P_{11}(s) [ P_{11}(t-s) - P_{11}(t)]}
  {\sqrt{P_{11}(s)P_{10}(s)P_{11}(t)P_{10}(t)}}.
  \]  
  To simplify things slightly, let us define $x=(\alpha+\beta)s$ and
  $y=(\alpha+\beta)t$.   According to the example in Section~6.4, the
  two correlations are
  \begin{eqnarray*}
  \Corr[X(s),X(t) | X(0)=0] 
  &=& \frac{(\alpha-\alpha e^{-x})(\beta e^{x-y}+\alpha e^{-y})}
  {\sqrt{ (\alpha-\alpha e^{-x}) (\beta+\alpha e^{-x}) 
     (\alpha-\alpha e^{-y}) (\beta+\alpha e^{-y}) }} \\
  &=& \frac{\beta e^{x-y} + (\alpha-\beta)e^{-y} - \alpha e^{-x-y}}
  {\sqrt{ (1- e^{-x}) (\beta+\alpha e^{-x}) 
     (1-e^{-y}) (\beta+\alpha e^{-y}) }}
  \end{eqnarray*}
  and
  \begin{eqnarray*}
  \Corr[X(s),X(t) | X(0)=1] 
  &=& \frac{(\alpha+\beta e^{-x})(\beta e^{x-y}-\beta e^{-y})}
  {\sqrt{ (\beta-\beta e^{-x}) (\alpha+\beta e^{-x}) 
     (\beta-\beta e^{-y}) (\alpha+\beta e^{-y}) }} \\
  &=& \frac{\alpha e^{x-y} + (\beta-\alpha)e^{-y} - \beta e^{-x-y}}
  {\sqrt{ (1- e^{-x}) (\alpha+\beta e^{-x}) 
     (1-e^{-y}) (\alpha+\beta e^{-y}) }}.
  \end{eqnarray*}
  We see that these two expressions are identical except that the
  role of $\alpha$ and $\beta$ is switched.  This makes sense once we realize
  that an on-off process with parameters $\alpha$ and $\beta$ is the same
  as an off-on process with parameters $\beta$ and $\alpha$.  It also means
  that the two correlations are the same when $\alpha=\beta$.
  However, they are not generally the same.  One way to see this is to 
  consider the limit of each correlation as $\alpha\to0$:
  \[
  \lim_{\alpha\to0}
  \Corr[X(s),X(t) | X(0)=0] = 
  \frac{e^{x-y}-e^{-y}}{\sqrt{(1-e^{-x})(1-e^{-y})}}
  \]
  and
  \[
  \lim_{\alpha\to0}
  \Corr[X(s),X(t) | X(0)=1] = 
  \sqrt{\frac{e^{x-y}-e^{-y}}{(1-e^{-x})(1-e^{-y})}}.
  \]  
  \end{quotation}
  
  \item At an amusement park, there are two video game machines. Suppose that
  for video game $i$, each period when it is being used is exponentially
  distributed with rate $\alpha_i$ and each period when it is not being used is
  exponentially distributed with rate $\beta_i$, independent of the other
  machine.
  
    \begin{enumerate}

    \item Suppose that the vector $M(t)$ is given by
    \[
    M(t) = [ M_1(t), M_2(t) ] ^\top,
    \]
    where $M_i(t)=I\{\mbox{machine $i$ is being used at time $t$}\}$ for $i=1,
    2$. The Markov chain $M(t)$ has four states; give its rate matrix $R$.
    \begin{quotation}{\bf Solution:}
    Let $0$ denote ``not being used'' and $1$ denote ``being used''.  Then
    if we order the four states $(0,0), (0,1), (1,0), (1,1)$, we get
    \[
    R = 
    \begin{bmatrix}
    -(\beta_2+\beta_1) & \beta_2 & \beta_1 & 0 \\
    \alpha_2 & -(\alpha_2+\beta_1) & 0 & \beta_1 \\
    \alpha_1 & 0 & -(\alpha_1+\beta_2) & \beta_2 \\
    0 & \alpha_1 & \alpha_2 & -(\alpha_1+\alpha_2)
    \end{bmatrix}
    \]
    \end{quotation}
    
    \item In the long run, what proportion of time are both machines being used?
    \begin{quotation}{\bf Solution:}
    One way to find the long-run probability vector $\pi$ 
    is to solve the system of equations
    $\pi^\top R=0$.  
    However, a simpler method is to consider the two on-off processes separately:
    In the long run, machine $i$ spends a fraction $\beta_i/(\alpha_i+\beta_i)$ of
    the time being used.  Since the two machines are independent, we conclude that
    \[
    \pi_4 = \frac{\beta_1\beta_2}{(\alpha_1+\beta_1)(\alpha_2+\beta_2)}.
    \]
    Incidentally, this process may be shown to be time-reversible (can you
    see why?), which means that the detailed balance equations give yet
    another alternative for finding the $\pi$ vector.
    \end{quotation}
    
    \item When the amusement park first opens, each machine is in its unused
    state. We can express this fact by $M_1(0)=M_2(0)=0$. Simulate 10,000
    independent realizations of this chain, until time $t=3$, using
    $\alpha_1=2$, $\alpha_2=3$, $\beta_1=5$, and $\beta_2=6$. From your
    simulations, give an empirical estimate of the proportion $\mu$ of time in
    $(0, 3]$ that the machines are used. Report a 95\% confidence interval for
    $\mu$. How does this compare with the long-run value calculated in part (b)?

    \vspace{1ex}
    {\em To find an approximate 95\% confidence interval for a mean $\mu$ based
    on a i.i.d.~sample of size $n$, take
    \[
    \hat\mu \pm 1.96 \frac{s}{\sqrt{n}},
    \]
    where $\hat\mu$ is the sample mean and $s$ is the sample standard deviation.
    }
  
    \begin{quotation}{\bf Solution:}
    First, we'll set up the $R$ matrix:
\begin{Schunk}
\begin{Sinput}
> R <- matrix(c(-11, 3, 2, 0, 6, -8, 0, 2, 5, 0, -8, 3, 0, 5, 6, -5), 4.4)
\end{Sinput}
\end{Schunk}
   Next is the sample code I wrote that simulates 10,000 copies of the Markov chain:
\begin{Schunk}
\begin{Sinput}
> X <- list()  # Each item in X will consist of TWO vectors:  times and states
>              # Actually, technically, each item will be a list with two elements:
>              # A vector named times and a vector named states.
> maxTime <- 3 # This is the cutoff time.
> for (count in 1:10000) {
+   times <- 0
+   states <- 1 # Assume that we always start in state 1 at time 0
+   i <- 1 # which time/state are we currently in
+   finished <- FALSE # We'll set this to TRUE when it's time to stop.
+   while (!finished) {
+     currentState <- states[i]
+     currentTime <- times[i]
+     deltaTime <- rexp(1, rate = -R[currentState, currentState])
+     if (currentTime + deltaTime > maxTime) {
+       # Now we need to finish this chain
+       deltaTime <- maxTime - currentTime
+       finished <- TRUE
+     }
+     times <- c(times, currentTime + deltaTime)
+     possibleMoves <- (1:4)[-currentState]
+     states <- c(states, sample(possibleMoves, 1, prob=R[currentState,possibleMoves]))
+     i <- i+1
+   }
+   X[[count]] <- list(times=times, states=states)
+ }
\end{Sinput}
\end{Schunk}
   Next, we need to figure out how much time was spent in the 4th state:
\begin{Schunk}
\begin{Sinput}
> f <- function(a) { 
+   deltaTimes <- diff(a$times)
+   states <- a$states[-length(a$states)] # delete the final state, which is irrelevant
+   sum(deltaTimes[states==4])
+ }
> timesInState4 <- sapply(X, f)
\end{Sinput}
\end{Schunk}
    Here is a 95\% confidence interval for the true mean of the proportion of time 
    in $(0,3]$ that is spent in the fourth state:
\begin{Schunk}
\begin{Sinput}
> mean(timesInState4/3) + c(-1.96, 1.96)*sd(timesInState4/3)/100
\end{Sinput}
\begin{Soutput}
[1] 0.4427792 0.4479613
\end{Soutput}
\end{Schunk}
   This is a fairly narrow confidence interval, and it is not too close to the long-term mean
   $30/63=0.4762$.  This suggests that three time units is not long enough for the chain
   to get close to its equilibrium state.  
    \end{quotation}

    \end{enumerate}

\end{enumerate}

\end{document}

