\documentclass{article}
\usepackage{url}
\usepackage{amsmath,bm}%
\usepackage{amsfonts}%
\pagestyle{empty}
\setlength{\textwidth}{7in}
\setlength{\oddsidemargin}{-.5in}
\setlength{\evensidemargin}{-.5in}
\setlength{\topmargin}{-.75in}
\setlength{\textheight}{9.25in}

\def\Var{\mathop{\rm Var\,}\nolimits}
\def\Cov{\mathop{\rm Cov\,}\nolimits}

\begin{document}

\begin{center}
{\bf STAT 515}

{\bf Homework \#5, due Friday, Feb. 24 at 2:30pm}

{\bf This homework must be submitted electronically to ANGEL.  I strongly
encourage the use of \LaTeX.}

\end{center}

{\it Please make every assignment easier to grade by neatly organizing your
writeup and clearly labeling your final answers when appropriate. Try using
\LaTeX!}

\begin{enumerate}

% Ross exercise 5.18
\item Let $X_1$ and $X_2$ be independent exponential random variables with rates
$\lambda_1$ and $\lambda_2$, respectively. Let
\[
X_{(1)} = \min\{ X_1, X_2 \} \quad\mbox{and}\quad X_{(2)} = \max\{X_1, X_2 \}.
\]
We have shown in class that $X_{(1)}$ is exponential with rate
$\lambda_1+\lambda_2$.

  \begin{enumerate}
  
  \item Find $E X_{(2)}$. ({\bf Hint:} What is $E [ X_{(1)} + X_{(2)} ]$?)
  
  \item Find a probability density function for $X_{(2)}$ and use it to
  calculate $\Var X_{(2)}$.
  
  \item Find $\Cov (X_{(1)}, X_{(2)})$. ({\bf Hint:} What is $\Var[ X_{(1)} +
  X_{(2)} ]$?)
  
  \end{enumerate}

% Using first principles definition of Poisson process Ross Stoch Proc. pg.66
\item Theorem~5.2 in Section~5.3.5 states that in a Poisson process $N(t)$ with
rate $\lambda$, given that $N(t)=n$, the $n$ arrival times $S_1,\dots,S_n$ have
the same distribution as the order statistics corresponding to $n$ independent
random variables uniformly distributed on the interval $(0,t)$, i.e.,
$$P(S_1=t_1,\dots,S_n=t_n \mid N(t)=n)=\frac{n!}{t^n} I(0<t_1<\dots <t_n). $$

  \begin{enumerate}

  \item Clearly describe the general algorithm this suggests for simulating a
  Poisson process on an interval $[0,t]$. ({\bf Hint}: you will simulate the
  process in two stages.)

  \item Consider a {\it homogeneous} Poisson process with $\lambda=10$. Using
  the algorithm from part (a), simulate 10,000 realizations of the above Poisson
  process on the interval $[0,5]$.

  \item Report the sample mean for the number of events in the interval (0,1)
  and the number of events in the interval (4,5). How do these means compare
  with the corresponding theoretical expectations?

  \item Plot a histogram each for the distribution of the number of events in
  the interval (0,1) and the interval (4,5) respectively, based on the 10,000
  realizations.

  \end{enumerate}

% 5.44
\item Cars pass a certain street location according to a Poisson process with
rate $\lambda$. A woman who wants to cross the street at that location waits
until she can see that no cars will come by in the next $T$ time units.

  \begin{enumerate}

  \item Find the probability that her waiting time is 0.

  \item Find her expected waiting time.

  \end{enumerate}

\end{enumerate}

\end{document}

