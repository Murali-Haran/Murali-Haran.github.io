\documentclass{article}
\usepackage{url}
\usepackage{amsmath,bm}%
\usepackage{amsfonts}%
\pagestyle{empty}
\setlength{\textwidth}{7in}
\setlength{\oddsidemargin}{-.5in}
\setlength{\evensidemargin}{-.5in}
\setlength{\topmargin}{-.75in}
\setlength{\textheight}{9.4in}

\newcommand{\beaa}{\begin{eqnarray*}}
\newcommand{\eeaa}{\end{eqnarray*}}
\newcommand{\bea}{\begin{eqnarray}}
\newcommand{\eea}{\end{eqnarray}}
\def\E{\mathop{\rm E\,}\nolimits}
\def\Var{\mathop{\rm Var\,}\nolimits}
\def\Cov{\mathop{\rm Cov\,}\nolimits}
\def\Corr{\mathop{\rm Corr\,}\nolimits}
\def\logit{\mathop{\rm logit\,}\nolimits}
\newcommand{\eid}{{\stackrel{\cal{D}}{=}}}
\newcommand{\cip}{{\stackrel{{P}}{\to}}}
\def\bR{\mathbb{R}}     % real line

\begin{document}

\begin{center}
{\bf STAT 515}

{\bf Homework \#8, due Friday, Mar.~30 at 2:30pm}

{\bf This homework must be submitted electronically to ANGEL. I strongly
encourage the use of \LaTeX.}

\end{center}

{\it Please make every assignment easier to grade by neatly organizing your
writeup and clearly labeling your final answers when appropriate.}

\begin{enumerate}

  \item The matrix exponential function is defined as
  \[
  \exp\{ M \} = \sum_{i=0}^\infty \frac{M^i}{i!}.
  \]
  However, this definition does not provide a suitable method for calculating
  $\exp\{M\}$ for a given $M$. One simple alternative is to use one of the two
  formulas
  \[
  \exp \{ M \} = \lim_{n\to\infty} \left( I + \frac{M}{n} \right) ^n
  = \lim_{n\to\infty} \left[ \left(I - \frac{M}{n} \right)^{-1} \right]^n.
  \]
  For the post office problem on HW \#7, the rate matrix is given by
  \[
  R = \begin{bmatrix}
  -8 & 8 & 0 & 0 & 0 \\
  12 & -20 & 8 & 0 & 0 \\
  0 & 12 & -20 & 8 & 0 \\
  0 & 0 & 12 & -20 & 8 \\
  0 & 0 & 0 & 12 & -12
  \end{bmatrix},
  \]
  where rates are in hours. Approximate the value of $\exp \{.5R\}$, which is
  the transition probability matrix for a time step of 30 minutes, using two
  methods:

    \begin{enumerate}

    \item For successively larger powers of 2, i.e., $n=2, 4, 8, \ldots$, find
    the value of $( I + .5R/n) ^n$. Continue until the change in each entry is
    smaller than $10^{-5}$. Report your final value of $n$ and your final
    approximation of $\exp \{.5R\}$.

    \item Repeat the same procedure as in part (a) but use $[(I-.5R/n)^{-1}]^n$
    instead.

    \item Use the {\tt expm} function in R or Matlab to evaluate $\exp \{.5R\}$
    and compare with the two approximations you obtained.
    
    {\em In R, you will have to install and load the package called {\tt
    Matrix}. Do this using {\tt install.packages("Matrix")} and then {\tt
    library(Matrix)}.} 
    
    \end{enumerate}

  \item Problem 3 in homework \#7 described two video game machines at an
  amusement park. For video game $i$, each period when it is being used is
  exponentially distributed with mean $1/\alpha_i$ hours and each period when it
  is not being used is exponentially distributed with mean $1/\beta_i$ hours,
  independent of the other machine. Furthermore, $\alpha_1=2$, $\alpha_2=3$,
  $\beta_1=5$, and $\beta_2=6$.
  
    \begin{enumerate}
    
    \item If neither machine is in use when the park opens at 8:00am, find the
    probability that both machines are in use at 9:30am.
    
    \item Simulate 10,000 realizations of the Markov chain and give a 95\%
    confidence interval for the probability in part (a) based on your
    simulation. Does your empirical estimate agree with the theoretical value?
    
    \end{enumerate}
  
  \item Suppose that $X_1, X_2, X_3, X_4$ are i.i.d.~from a uniform$(0,1)$
  distribution. Let $S=X_1+X_2+X_3+X_4$.
  
    \begin{enumerate}
  
    \item Find $P(S<1)$ exactly using a four-dimensional integral. (Hint: This
    is not too difficult.)
    
    \item Now consider $P(S<1.5)$. This is much more difficult to find
    analytically. Instead, use Monte Carlo simulation to approximate this
    probability. Give a 99\% confidence interval for the true probability, and
    use a large enough sample so that your interval is no wider than $0.01$.
    Report the sample size you used in addition to the interval.
    
    \item The central limit theorem approximation to $P(S<1.5)$ is $P(Y<1.5)$,
    where $Y$ is a normal random variable with the same mean and variance as
    $S$. Based on your answer to part (b), how good does the central limit
    theorem approximation appear in this case?
    
    \end{enumerate}

\end{enumerate}

\end{document}

