\documentclass{article}
\usepackage{url}
\usepackage{amsmath,bm}%
\usepackage{amsfonts}%
\pagestyle{empty}
\setlength{\textwidth}{7in}
\setlength{\oddsidemargin}{-.5in}
\setlength{\evensidemargin}{-.5in}
\setlength{\topmargin}{-.75in}
\setlength{\textheight}{9.4in}

\newcommand{\beaa}{\begin{eqnarray*}}
\newcommand{\eeaa}{\end{eqnarray*}}
\newcommand{\bea}{\begin{eqnarray}}
\newcommand{\eea}{\end{eqnarray}}
\def\E{\mathop{\rm E\,}\nolimits}
\def\Var{\mathop{\rm Var\,}\nolimits}
\def\Cov{\mathop{\rm Cov\,}\nolimits}
\def\Corr{\mathop{\rm Corr\,}\nolimits}
\def\logit{\mathop{\rm logit\,}\nolimits}
\newcommand{\eid}{{\stackrel{\cal{D}}{=}}}
\newcommand{\cip}{{\stackrel{{P}}{\to}}}
\def\bR{\mathbb{R}}     % real line

\begin{document}

\begin{center}
{\bf STAT 515}

{\bf Homework \#9, due Friday, Apr.~6 at 2:30pm}

{\bf This homework must be submitted electronically to ANGEL. I strongly
encourage the use of \LaTeX.}

\end{center}

{\it Please make every assignment easier to grade by neatly organizing your
writeup and clearly labeling your final answers when appropriate. Try using
\LaTeX!}


\begin{enumerate}

  \item Customers arrive at a post office at a Poisson rate of 8 per hour. There
  is a single person serving customers, and service times are exponentially
  distributed (and independent) with mean 5 minutes. Suppose that an arriving
  customer will decide to wait in line if and only if there are three or fewer
  people already in the post office (which means two or fewer people in line).
  Of interest is $E(L)$, the expected total number of potential customers lost,
  during a single eight-hour day.
  
    \begin{enumerate}
    
    \item Simulate this eight-hour process 10,000 times, then find the sample
    mean
    \[
    \hat\mu_1 = \frac{1}{10^{4}} \sum_{i=1}^{10^4} L_i,
    \]
    where $L_i$ is the number lost in the $i$th simulation.

    \item Prove that $E(L)=8E(T)$, where $T$ is the total amount of time spent
    in the state ``four people in the post office''. (Hint: Use conditioning.)

    \item For the simulation in part (a), calculate
    \[
    \hat\mu_2 = \frac{8}{10^4} \sum_{i=1}^{10^4} T_i,
    \]
    where $T_i$ is the total time spent in the four-people state in the $i$th
    simulation.
    
    \item Both $\mu_1$ and $\mu_2$ are estimators of the same quantity, so which
    one is better? In a statistical sense, ``better'' often means ``smaller
    variance''. Use the conditional variance formula to prove that
    \[
    \Var \hat \mu_2 < \Var \hat\mu_1.
    \]

    \item Calculate two separate 95\% confidence intervals for $\mu$ based on
    $\hat\mu_1$ and $\hat\mu_2$. Do your results agree with the finding of part
    (d)?

    \end{enumerate}
    
  \item Describe two different algorithms for simulating from a beta$(3,2)$
  density, one based on an inversion method and the other based on a rejection
  method. Try simulating the same number of variables (something more than a
  million) using each method. Does one method appear to be more efficient than
  the other? Explain every step. If you are using R, you might find the timing
  function {\tt system.time} useful if you want to compare the algorithms based
  on their total time.
  
  As a check on your simulated values, make sure that the sample mean and
  variance of your variables are close to the theoretical mean and variance of
  $3/5$ and $1/25$.
  
  \item Suppose we want to conduct a simple hypothesis test to see whether the
  correlation $\rho$ between $X$ and $Y$ is greater than zero. Let us assume
  that $X$ and $Y$ come from a bivariate normal distribution, and a sample of 30
  points $(X_1, Y_1), \ldots, (X_{30}, Y_{30})$ gives a sample correlation of
  $\hat\rho=0.3$. We wish to find the p-value for this sample correlation.
  
  To find the p-value, we need to know the null distribution of the sample
  correlation. In this case, the exact null distribution (i.e., the distribution
  of $\hat\rho$ when $\rho=0$) is known, but it is quite complicated. Instead,
  we shall approximate the p-value using the fact (not proven here) that the
  distribution of $\hat\rho$ depends only on $\rho$ and the sample size $n$.

  The p-value in this problem is defined to be $P(\hat\rho>0.3 \mid \rho=0)$.
  Use $10^5$ Monte Carlo samples of size $30$ to obtain a 95\% confidence
  interval for the p-value.
  
  \item Suppose $(X_1, X_2)$ are bivariate normal with $E X_1=E X_2=0$ and
  covariance matrix 
  \[
  \Sigma = \begin{bmatrix}
  4 & 4 \\ 4 & 9
  \end{bmatrix}.
  \]
  Use a Monte Carlo method to estimate $P(\max\{ |X_1|, |X_2|\} <1)$ to such a
  precision that a 99\%
  confidence interval for the true value has a width of no more than $1/1000$.
  
\end{enumerate}

\end{document}

