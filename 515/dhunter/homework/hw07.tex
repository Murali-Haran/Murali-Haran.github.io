\documentclass{article}
\usepackage{url}
\usepackage{amsmath,bm}%
\usepackage{amsfonts}%
\pagestyle{empty}
\setlength{\textwidth}{7.5in}
\setlength{\oddsidemargin}{-.5in}
\setlength{\evensidemargin}{-.5in}
\setlength{\topmargin}{-.75in}
\setlength{\textheight}{9.5in}

\newcommand{\beaa}{\begin{eqnarray*}}
\newcommand{\eeaa}{\end{eqnarray*}}
\newcommand{\bea}{\begin{eqnarray}}
\newcommand{\eea}{\end{eqnarray}}
\def\E{\mathop{\rm E\,}\nolimits}
\def\Var{\mathop{\rm Var\,}\nolimits}
\def\Cov{\mathop{\rm Cov\,}\nolimits}
\def\Corr{\mathop{\rm Corr\,}\nolimits}
\def\logit{\mathop{\rm logit\,}\nolimits}
\newcommand{\eid}{{\stackrel{\cal{D}}{=}}}
\newcommand{\cip}{{\stackrel{{P}}{\to}}}
\def\bR{\mathbb{R}}     % real line

\begin{document}

\begin{center}
{\bf STAT 515}

{\bf Homework \#7, due Friday, Mar.~23 at 2:30pm}

{\bf This homework may be submitted electronically to ANGEL, though this
is not required.  I strongly encourage the use of \LaTeX\ in any case.}

\end{center}

{\it Please make every assignment easier to grade by neatly organizing your
writeup and clearly labeling your final answers when appropriate. Try using
\LaTeX!}


%Based on 6.13
\begin{enumerate}

  \item Customers arrive at a post office at a Poisson rate of 8 per hour. There
  is a single person serving customers, and service times are exponentially
  distributed (and independent) with mean 5 minutes. Suppose that an arriving
  customer will decide to wait in line if and only if there are two or fewer
  people already in line.
  
    \begin{enumerate}

    \item In the long run, what fraction of the time will there be at least 1
    customer in the post office? Find the answer in two different ways:

      \begin{enumerate}
      
      \item Write out the rate matrix (or generator) for the continuous-time
      Markov chain and find the stationary distribution using the generator.
      
      \item Use the fact that this is a birth-death process to find the
      stationary distribution that satisfies the detailed balance equations.
      
      \end{enumerate}
      
    \item In the long run, what is the expected number of customers in the post
    office (in line or being served) at any given time?
    
    \item What is the probability that an arriving potential customer will
    decide to leave because there are already 3 people in line?
    
    \item If a new cash register is installed that decreases the mean service
    time to 4 minutes, how many more customers per hour, on average, can be
    served by this post office?
    
    \item The manager of the post office wants to be able to serve at least 95\%
    of the potential customers who arrive at the post office. What mean service
    time will attain this goal?
    
    \end{enumerate}
    
  \item Suppose that a continuous-time Markov chain $X(t)$ has rate matrix
  \[
  R = \begin{bmatrix}
  -\alpha & \alpha \\
  \beta & -\beta
  \end{bmatrix}.  
  \]  
  Given positive times $s$ and $t$, calculate $\Corr[X(s), X(t)]$. Does your
  answer depend on the starting state of the chain (i.e., value of $X(0)$)?
  
  (NB: The formula for $\Corr(X,Y)$ is
  $\Corr(X,Y) = \Cov(X,Y)/\sqrt{\Var(X)\Var(Y)}$.
  Since the correlation is invariant to linear transformations, the two
  possible values that $X(t)$ may take do not influence your answer.)
  
  \item At an amusement park, there are two video game machines. Suppose that
  for video game $i$, each period when it is being used is exponentially
  distributed with rate $\alpha_i$ and each period when it is not being used is
  exponentially distributed with rate $\beta_i$, independent of the other
  machines.
  
    \begin{enumerate}

    \item Suppose that the vector $M(t)$ is given by
    \[
    M(t) = [ M_1(t), M_2(t) ] ^\top,
    \]
    
    where $M_i(t)=I\{\mbox{machine $i$ is being used at time $t$}\}$ for $i=1,
    2$. The Markov chain $M(t)$ has four states; give its rate matrix $R$.
    
    \item In the long run, what proportion of time are both machines being used?
    
    \item When the amusement park first opens, each machine is in its unused
    state. We can express this fact by $M_1(0)=M_2(0)=0$. Simulate 10,000
    independent realizations of this chain, until time $t=3$, using
    $\alpha_1=2$, $\alpha_2=3$, $\beta_1=5$, and $\beta_2=6$. From your
    simulations, give an empirical estimate of the proportion $\mu$ of time in
    $(0, 3]$ that the machines are used. Report a 95\% confidence interval for
    $\mu$. How does this compare with the long-run value calculated in part (b)?
    
    \vspace{1ex}
    {\em To find an approximate 95\% confidence interval for a mean $\mu$ based
    on a i.i.d.~sample of size $n$, take
    \[
    \hat\mu \pm 1.96 \frac{s}{\sqrt{n}},
    \]
    where $\hat\mu$ is the sample mean and $s$ is the sample standard deviation.
    }
    \end{enumerate}
  

\end{enumerate}

\end{document}

