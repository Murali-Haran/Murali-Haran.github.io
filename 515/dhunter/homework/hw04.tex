\documentclass{article}
\usepackage{url}
\usepackage{amsmath,bm}
\usepackage{amsfonts}
\pagestyle{empty}
\setlength{\textwidth}{7in}
\setlength{\oddsidemargin}{-.5in}
\setlength{\evensidemargin}{-.5in}
\setlength{\topmargin}{-.75in}
\setlength{\textheight}{9.25in}

\begin{document}

\begin{center}
{\bf STAT 515}

{\bf Homework \#4, due Friday, Feb. 17 at 2:30pm}
\end{center}

{\it Please make every assignment easier to grade by neatly organizing your
writeup and clearly labeling your final answers when appropriate. Try using
\LaTeX!}

\begin{enumerate}

\item Suppose that in a branching process with $X_0=1$, each individual produces
some number of offspring that is Poisson with mean 1, independently of all other
individuals.

  \begin{enumerate}

  \item What is the expected number of generations until the process either dies
  out or attains size $X_n\ge 5$?

  \item What is the probability that the process will ever attain size $X_n\ge 5$?  

  \end{enumerate}

\item Define a Markov chain on the nonnegative integers as follows:
$P_{0j}=I\{j=1\}$, and for $i>0$,
\begin{eqnarray*} 
P_{ij}&=& 
  \begin{cases} 
  i/(i+1) & \mbox{if $j=i+1$} \\ 
  1/(i+1) & \mbox{if $j=0$}\\ 0 & \mbox{otherwise.} 
  \end{cases} 
\end{eqnarray*}

  \begin{enumerate}
  
  \item Argue that this chain is irreducible and aperiodic.

  \item Prove that all states are recurrent.

  \item Prove that all states are null recurrent. (You may assume without proof
  that null recurrence is a class property.)

  \end{enumerate}
  
\item Consider the symmetric one-dimensional random walk of Example 4.15 with
$p=1/2$.

  \begin{enumerate}
  
  \item Let $T_{i}$ be the time at which the random walk first revisits state
  $i$ given that it begins in state $i$. That is, $T_{i} = \inf \{n>0: X_n=i
  \mid X_0=i\}$. For any $n>0$, find $P(T_i=2n)$.
   
  {\bf Hint:\ } Read the ballot problem example of Section~3.5 and the
  discussion following it.
   
  \item Prove that all states are null recurrent by showing that
  $E(T_i)=\infty$.
   
  {\bf Hint:\ } Read the random walk example of Section~4.3 for an idea about
  how to show this.

  \end{enumerate}

\item Read the random walk example of Section~4.8 and the discussion of the
Ehrenfest model following it.

  \begin{enumerate}

  \item Simulate the simple Ehrenfest diffusion process with total number of
  particles $M=30$. Start the process at $X_0=10$. Run the process for 100,000
  steps and draw a histogram of the resulting values of $X_t$.

  \item On the same histogram, indicate the true values that would be expected
  from a sample of size 100,000 from a binomial$(n=30, p=1/2)$ distribution.  See
  the example R code for ideas on how to do this.

  \item Explain what you observe from the comparison in part (b). Is the Markov
  chain you are simulating ergodic?

  \end{enumerate}

%% time reversibility + MCMC basics (reworded from Murail's HW 4)
\item Let $Q$ be a transition ``matrix'' for an irreducible Markov chain on the
set $\mathbb{Z}$ of all integers; i.e., $Q_{ij}=P(X_n=j \mid X_{n-1}=i)$ for all
$i,j\in\mathbb{Z}$. Assume that $Q_{ij}>0$ if and only if $Q_{ji}>0$ and that
$Q_{ii}>0$ for all $i$. Also suppose that
\[
\sum_{i \in \mathbb{Z}} \pi_i = 1 \quad \mbox{and $\pi_i>0$ for all
$i\in\mathbb{Z}$}
\]
and define
\[
\alpha(i,j) = \min\left(\frac{\pi_jQ_{ji}}{\pi_iQ_{ij}}, 1\right) \quad\mbox{for
all $i,j\in\mathbb{Z}$.}
\]
Consider a second Markov chain with transition probabilities given by
\[
P_{ij} =  
  \begin{cases}
  \alpha(i,j) Q_{ij} &  \mbox{ if  $ j \neq i$}  \\
  Q_{ii} + \sum_{k\neq i} Q_{ik}(1-\alpha(i,k)) & \mbox{if $i=j$,}
  \end{cases}
\]
Using time reversibility arguments, show that the second Markov chain has
stationary probabilities $\{\pi_i\}$ and that these stationary probabilities are
also the limiting probabilities of the Markov chain.

% reworded Ross problem 5.9
\item The lifetimes of two machines are independent with exponential
distributions with rates $\lambda_1$ and $\lambda_2$, respectively. Suppose
machine 1 starts working now and machine 2 starts working $t$ units of time
later. What is the probability that machine 1 will fail before machine 2?

\end{enumerate}

\end{document}

