\documentclass{beamer}
\usepackage{beamerthemeshadow}
\usepackage{amssymb}
\usepackage{tikz}

\newcommand{\beaa}{\begin{eqnarray*}}
\newcommand{\eeaa}{\end{eqnarray*}}
\newcommand{\bea}{\begin{eqnarray}}
\newcommand{\eea}{\end{eqnarray}}
\def\E{\mathop{\rm E\,}\nolimits}
\def\Var{\mathop{\rm Var\,}\nolimits}
\def\Cov{\mathop{\rm Cov\,}\nolimits}
\def\Corr{\mathop{\rm Corr\,}\nolimits}
\def\logit{\mathop{\rm logit\,}\nolimits}
\newcommand{\eid}{{\stackrel{\cal{D}}{=}}}
\newcommand{\cip}{{\stackrel{{P}}{\to}}}
\def\bR{\mathbb{R}}	% real line
\def\given{\,|\,}

%%%%%%%%%%
% From http://www.disc-conference.org/disc1998/mirror/llncs.sty
\def\vec#1{\mathchoice{\mbox{\boldmath$\displaystyle\bf#1$}}
{\mbox{\boldmath$\textstyle\bf#1$}}
{\mbox{\boldmath$\scriptstyle\bf#1$}}
{\mbox{\boldmath$\scriptscriptstyle\bf#1$}}}
%%%%%%%%%%

\def\choice#1#2{
\tikz {
\def \lx{-.2} \def \ux{.2} \def \ly{-.2} \def\uy{.2} 
\filldraw[color=#1] (\lx*.9,\ly*.9) rectangle (\ux*.9,\uy*.9);
\draw(\lx,\ly) rectangle (\ux,\uy); \node at (0,0)[color=white] {\bf {#2}}; \node at (0,0) [color=black] {{#2}};
} }

\def\question#1#2#3#4#5{{#1}

\begin{minipage}{2.45in}
\begin{enumerate}
\item [{\choice{red}{A}}]\rule{-.25em}{0em}{#2}
\item [{\choice{yellow}{C}}]\rule{-.25em}{0em}{#4}
\end{enumerate}
%\choice{red}{A} {#2}
%
%\choice{yellow}{C} {#4}
\end{minipage}\begin{minipage}{2.45in}
\begin{enumerate}
\item [{\choice{green}{B}}]\rule{-.25em}{0em}{#3}
\item [{\choice{blue}{D}}]\rule{-.25em}{0em}{#5}
\end{enumerate}
%\choice{green}{B} {#3}
%
%\choice{blue}{D}  {#5}
\end{minipage}
}

%
%  Simple macro to declare and use image all at once.
%
\def\figurehere#1#2{\pgfdeclareimage[width={#2}]{{#1}}{figures/#1} \pgfuseimage{{#1}}}

\def\startframe#1{\begin{frame}[t,fragile] \frametitle{\thedate \hfill {#1}} \vspace{-1ex}}


%
%  Declare images
%
%\pgfdeclareimage[width=3in]{geyser}{figures/geyser}

\mode<presentation>
{
%  \setbeamertemplate{background canvas}
  \usetheme{boxes}
  \usecolortheme{blackscreen}

%  \usetheme{default}
%  \setbeamercovered{transparent}
}
%\beamertemplatetransparentcovereddynamic

\usepackage[english]{babel}
\usepackage[latin1]{inputenc}
\usepackage{times}
\usepackage[T1]{fontenc}

\setbeamersize{text margin left=0.1in}
\setbeamersize{text margin right=0.1in}




\begin{document}


\def \thedate{Apr.~20}


%%%%%%%%%%%%%%%%%%%%%%%%%%%%%%%%%%%%%%%%

\startframe{Announcements}
\begin{itemize}
\item 
Last homework due on Monday (Apr.~23)
\item 
Take-home final exam will be available tomorrow at 5:00pm
due at 5:00pm on Wednesday, May~2.
\item
Please fill out SRTEs!
\item 
Read Section 4.9 in textbook.
\end{itemize}
\end{frame}


\startframe{Markov chain Monte Carlo}
Bayesian VAATMH example (from Gilks et al (1996), pp. 75--76):
\begin{itemize}
\item $Y_1, \ldots, Y_n \stackrel{\rm iid}{\sim} N(\mu, \tau^{-1})$
\item $\mu \sim N(0,1)$
\item $\tau \sim \mbox{gamma}(2,1)$
\end{itemize}
Devise an MCMC scheme for sampling from the posterior distribution.
\end{frame}

\startframe{Markov chain Monte Carlo}
Special cases of VAATMH:  Gibbs and Metropolis updates
\begin{itemize}
\item  Recall full MH acceptance probability
\item Metropolis:  $q(y_i\mid x_i, x_{-i}) = q(x_i\mid y_i, x_{-i})$
\item Gibbs:  $q(y_i \mid x_i, x_{-i}) = \pi(y_i \mid x_{-i})$
\end{itemize}
\[
\alpha(x_i, y_i \mid x_{-i}) = \min
\left\{ 1,
\frac{\pi(y_i \mid x_{-i}) q(x_i\mid y_i, x_{-i})}{ \pi(x_i \mid x_{-i}) q(y_i\mid x_i, x_{-i})}
\right\}
\]
%\notes{
%}
\end{frame}

\startframe{Markov chain Monte Carlo}
Why VAATMH works:
\begin{itemize}
\item Suppose $\vec x = (x_1, x_2)^\top$.
\item Take Markov transition densities 
$k_{1\mid 2} (y_1 \mid x_1, x_2)$ and
$k_{2\mid 1} (y_2 \mid x_1, x_2)$.
\item Then the transition density for both updates is
\[
k[ (y_1, y_2) \mid (x_1, x_2) ] =
k_{1\mid 2} (y_1 \mid x_1, x_2) k_{2\mid 1}(y_2 \mid y_1, x_2).
\]
\end{itemize}
\end{frame}

\startframe{Markov chain Monte Carlo}
Why VAATMH works:
\begin{itemize}
\item
$k[ (y_1, y_2) \mid (x_1, x_2) ] =
k_{1\mid 2} (y_1 \mid x_1, x_2) k_{2\mid 1}(y_2 \mid y_1, x_2)$.
\item Show $\pi(\vec y) = \int k(\vec y \mid \vec x) \pi(\vec x) \, d\vec x$.
\end{itemize}
\end{frame}

%%%%%%%%%%%%%%%%%%%%%%%%%%%%%%%%%%%%%%%%
\end{document}




