\documentclass[handout]{beamer}
\usepackage{beamerthemeshadow}
\usepackage{amssymb}
\usepackage{pgfpages}
\usepackage{tikz}
\pgfpagesuselayout{4 on 1}%[landscape]

\newcommand{\beaa}{\begin{eqnarray*}}
\newcommand{\eeaa}{\end{eqnarray*}}
\newcommand{\bea}{\begin{eqnarray}}
\newcommand{\eea}{\end{eqnarray}}
\def\E{\mathop{\rm E\,}\nolimits}
\def\Var{\mathop{\rm Var\,}\nolimits}
\def\Cov{\mathop{\rm Cov\,}\nolimits}
\def\Corr{\mathop{\rm Corr\,}\nolimits}
\def\logit{\mathop{\rm logit\,}\nolimits}
\newcommand{\eid}{{\stackrel{\cal{D}}{=}}}
\newcommand{\cip}{{\stackrel{{P}}{\to}}}
\def\bR{\mathbb{R}}	% real line
\def\given{\,|\,}

%%%%%%%%%%
% From http://www.disc-conference.org/disc1998/mirror/llncs.sty
\def\vec#1{\mathchoice{\mbox{\boldmath$\displaystyle\bf#1$}}
{\mbox{\boldmath$\textstyle\bf#1$}}
{\mbox{\boldmath$\scriptstyle\bf#1$}}
{\mbox{\boldmath$\scriptscriptstyle\bf#1$}}}
%%%%%%%%%%

\def\choice#1#2
{\tikz 
{\def \lx{-.2} \def \ux{.2} \def \ly{-.2} \def\uy{.2} \filldraw[color=#1] (\lx*.9,\ly*.9) rectangle (\ux*.9,\uy*.9);
\draw(\lx,\ly) rectangle (\ux,\uy); \node at (0,0)[color=white] {\bf {#2}}; \node at (0,0) [color=black] {{#2}};}}

\def\question#1#2#3#4#5{{#1}

\begin{minipage}{2.4in}
\choice{red}{A} {#2}

\choice{yellow}{C} {#4}
\end{minipage}\begin{minipage}{2.4in}
\choice{green}{B} {#3}

\choice{blue}{D}  {#5}
\end{minipage}
}

\def\notes#1{

\vspace{2ex}
{\em Notes: {#1}}}
%
%  Simple macro to declare and use image all at once.
%
\def\figurehere#1#2{\pgfdeclareimage[width={#2}]{{#1}}{figures/#1} \pgfuseimage{{#1}}}

\def\startframe#1{\begin{frame}[t,fragile] \frametitle{\thedate \hfill {#1}} }

%
%  Declare images
%
%\pgfdeclareimage[width=3in]{geyser}{figures/geyser}

\mode<presentation>
{
%  \setbeamertemplate{background canvas}
  \usetheme{boxes}
  \usecolortheme{notblackscreen}

%  \usetheme{default}
%  \setbeamercovered{transparent}
}
%\beamertemplatetransparentcovereddynamic

\usepackage[english]{babel}
\usepackage[latin1]{inputenc}
\usepackage{times}
\usepackage[T1]{fontenc}

\setbeamersize{text margin left=0.1in}
\setbeamersize{text margin right=0.1in}



\begin{document}


\def \thedate{Apr.~11}


%%%%%%%%%%%%%%%%%%%%%%%%%%%%%%%%%%%%%%%%

\startframe{Announcements}
\begin{itemize}
\item 
All homework must be turned in electronically from now on.
\item 
Only 2 more homeworks!  (Only best 10 out of 11 grades will count.)
\item
Tentative take-home final plan:  The exam will be available on Monday, Apr.~23 and due
at 5:00pm on Wednesday, May~2.
\end{itemize}
\notes{If anyone sees any problem with the tentative take-home final plan, please let me know.
}
\end{frame}


\startframe{Markov chain Monte Carlo}
\begin{itemize}
\item Goal:  Estimate $\mu=E_\pi g(X)$ but
cannot %use Monte Carlo directly since we cannot 
sample $X_i\sim \pi$ directly.
\item 
Importance sampling may be difficult, particularly as dimension increases.
\item 
MCMC solution:  Take $\hat\mu = \frac1n\sum g(X_i)$, where
$X_1, X_2, \ldots$ is a simulated Markov chain with stationary distribution $\pi$.
\end{itemize}
\end{frame}

\startframe{Markov chain Monte Carlo}
MCMC example (coming soon to a homework near you!):
\begin{itemize}
\item Suppose we observe $Y_1, \ldots, Y_n$
\item $Y_i \mid \theta \sim N(\theta, 1)$, conditionally independent.
\item $\theta \sim \mbox{log-$t$}(\mu, \sigma, r)$.
\item Wanted: A sample from the posterior $\pi(\theta\mid \vec Y)$.
\end{itemize}
\notes{In this case, we don't want $E_\pi g(X)$, but rather a whole sample from $\pi$.
NB:  It won't be an i.i.d.~sample since the Markov chain will have dependence.
The log-$t$ density function is proportional to
\[
\frac1\theta \left[ 1 + \frac1r \left( \frac {\log \theta - \mu}{\sigma} \right)^2 \right]
^{-(r+1)/2}. 
\]
}
\end{frame}

\startframe{Markov chain Monte Carlo}
Metropolis-Hastings algorithm (recall HW\#4, problem 5):
\begin{itemize}
\item Start with $X_0=x_0$.  
\item For $i=0, 1, \ldots$, 
generate $Y \sim q(y\mid x_0)$.
\item Define $\alpha(x,y) = \min\{1, \pi(y)q(x\mid y) / [\pi(x)q(y\mid x)]$.
\item Let $X_{i+1}=Y$ with probability $\alpha(X_i,Y)$.
\end{itemize}
\notes{
This slide corrects several errors in a similar slide from Monday.  Today,
we had time also to discuss ``reality checks'' on the formula for $\alpha$:
First, think about the case in which $q(x\mid y)=q(y\mid x)$.  Second, think
about the case in which $\pi(x)=\pi(y)$.
}
\end{frame}

\startframe{Markov chain Monte Carlo}
Metropolis algorithm (``random walk'' M-H):
\begin{itemize}
\item Take $q$ so that $q(x\mid y)=q(y\mid x)$.
\item In this case, $\alpha(x,y) = \min\{1, \pi(y)/\pi(x)\}$.
\item Example:  Take $Y \mid X_i=x \sim N(x, \tau^2)$.
\item Consider the tradeoff involved in choosing $\tau^2$.
\end{itemize}
\notes{
If $\tau^2$ is too large, we may propose a lot of steps that are far from the current
state, which might have much smaller probability as measured by $\pi$, which means 
that very few such proposals will be accepted.  On the other hand, if $\tau^2$ is too
small, we will only propose steps very close to the current state.  In either case,
mixing can be slow.
}
\end{frame}


%%%%%%%%%%%%%%%%%%%%%%%%%%%%%%%%%%%%%%%%
\end{document}




