\documentclass[handout]{beamer}
\usepackage{beamerthemeshadow}
\usepackage{amssymb}
\usepackage{pgfpages}
\usepackage{tikz}
\pgfpagesuselayout{4 on 1}%[landscape]

\newcommand{\beaa}{\begin{eqnarray*}}
\newcommand{\eeaa}{\end{eqnarray*}}
\newcommand{\bea}{\begin{eqnarray}}
\newcommand{\eea}{\end{eqnarray}}
\def\E{\mathop{\rm E\,}\nolimits}
\def\Var{\mathop{\rm Var\,}\nolimits}
\def\Cov{\mathop{\rm Cov\,}\nolimits}
\def\Corr{\mathop{\rm Corr\,}\nolimits}
\def\logit{\mathop{\rm logit\,}\nolimits}
\newcommand{\eid}{{\stackrel{\cal{D}}{=}}}
\newcommand{\cip}{{\stackrel{{P}}{\to}}}
\def\bR{\mathbb{R}}	% real line
\def\given{\,|\,}

%%%%%%%%%%
% From http://www.disc-conference.org/disc1998/mirror/llncs.sty
\def\vec#1{\mathchoice{\mbox{\boldmath$\displaystyle\bf#1$}}
{\mbox{\boldmath$\textstyle\bf#1$}}
{\mbox{\boldmath$\scriptstyle\bf#1$}}
{\mbox{\boldmath$\scriptscriptstyle\bf#1$}}}
%%%%%%%%%%

\def\choice#1#2
{\tikz 
{\def \lx{-.2} \def \ux{.2} \def \ly{-.2} \def\uy{.2} \filldraw[color=#1] (\lx*.9,\ly*.9) rectangle (\ux*.9,\uy*.9);
\draw(\lx,\ly) rectangle (\ux,\uy); \node at (0,0)[color=white] {\bf {#2}}; \node at (0,0) [color=black] {{#2}};}}

\def\question#1#2#3#4#5{{#1}

\begin{minipage}{2.4in}
\choice{red}{A} {#2}

\choice{yellow}{C} {#4}
\end{minipage}\begin{minipage}{2.4in}
\choice{green}{B} {#3}

\choice{blue}{D}  {#5}
\end{minipage}
}

\def\notes#1{

\vspace{2ex}
{\em Notes: {#1}}}
%
%  Simple macro to declare and use image all at once.
%
\def\figurehere#1#2{\pgfdeclareimage[width={#2}]{{#1}}{figures/#1} \pgfuseimage{{#1}}}

\def\startframe#1{\begin{frame}[t,fragile] \frametitle{\thedate \hfill {#1}} }

%
%  Declare images
%
%\pgfdeclareimage[width=3in]{geyser}{figures/geyser}

\mode<presentation>
{
%  \setbeamertemplate{background canvas}
  \usetheme{boxes}
  \usecolortheme{notblackscreen}

%  \usetheme{default}
%  \setbeamercovered{transparent}
}
%\beamertemplatetransparentcovereddynamic

\usepackage[english]{babel}
\usepackage[latin1]{inputenc}
\usepackage{times}
\usepackage[T1]{fontenc}

\setbeamersize{text margin left=0.1in}
\setbeamersize{text margin right=0.1in}



\begin{document}


\def \thedate{Apr.~25}


%%%%%%%%%%%%%%%%%%%%%%%%%%%%%%%%%%%%%%%%

\startframe{Announcements}
\begin{itemize}
\item 
Make sure you do not communicate with anyone (except me) 
about the take-home final.
\item 
Sorry for the delay on the solution for hw\#11; hopefully by tomorrow\ldots
\item
Office hours will take place as usual tomorrow.
\end{itemize}
\end{frame}


\startframe{Markov chain Monte Carlo}
MCMC standard errors for estimate of $E_\pi g(X)$
\begin{itemize}
\item Batch means idea (where $n=ab$):
\[
\mbox{Let}\quad
Y_1 = \frac{X_1+ \cdots + X_b}{b}, 
Y_2 = \frac{X_{b+1}+ \cdots + X_{2b}}{b}, \ldots
\]
\item Then $\hat\sigma^2 = \frac{1}{a-1} \sum_{i=1}^a b(Y_i - \hat\mu)^2$.
And $\widehat{\mbox{MCMCse}} = \hat\sigma/\sqrt{n}$.
\item Jones et al (2006):  Take, for example, $b=\sqrt{n}$.
\end{itemize}
\notes{I went through the logic of this method.  See the notes from this slide I posted
in the Apr.~23 lecture.}
\end{frame}

\startframe{Markov chain Monte Carlo}
\begin{itemize}
\item 
Effective sample size
(ESS) may be defined as 
\[
\mbox{ESS} = n \left( \frac{ \Var(\hat\mu_{\rm iid}) }
{ \Var(\hat\mu_{\rm corr}) } \right).
\]
\item For a stationary chain, we may write this as $\mbox{ESS} = n/\kappa_n$, where
\[
\kappa_n=1 + 2\sum_{i=1}^{n-1} \frac{n-i}{n}\Corr(X_1, X_{1+i}).
\]
\end{itemize}
\notes{
It's not a bad idea to make sure you could derive the form of $\kappa_n$---do
you understand where it comes from?
}
\end{frame}

\startframe{Markov chain Monte Carlo}
How to estimate $\kappa=1+2\sum_{i=1}^\infty \rho_i$, the ``autocorrelation time''.
\begin{itemize}
\item Take $\hat\kappa = 1 + \sum_{i=1}^M \hat \rho_i$.
\item How to choose $M$?  
  \begin{itemize}
  \item  Smallest $M$ with $|\hat \rho_i | < .1$
  \item IMSE (Initial Monotone Sequence Estimator) (Geyer, 1992)
  
  Take largest $M$ with $\hat\rho_{0}-\hat\rho_1 > \cdots > \hat\rho_{M-1}+\hat\rho_{M}>0$
  \end{itemize}
\end{itemize}
\notes{We did not talk about this very much.  
Some of the notation above is slightly different than 
what we saw in class.
}
\end{frame}

\startframe{Markov chain Monte Carlo}
A word on ``burnin'' and ``thinning''\ldots
\notes{
Thinning means only keeping every $s$th iteration in your Markov chain, which essentially
means that each actual step of the chain you're sampling consists of $s$ steps of the original
chain.  To do this is to throw away information.  However, sometimes there is a good reason,
namely, when storage space is limited and so only a fixed number of values can be kept.
When a chain is thinned, the values tend to be less correlated than the values of the original
chain.  One way to look at burnin:  A type of thinning that only applies at the beginning of 
the chain.
}
\end{frame}


\startframe{STAT 515 brief recap}
\begin{itemize}
\item Chapters 1 and 2:  Review of probability concepts
\item Chapter 3:  Conditional probability and conditioning (omit 3.6, 3.7)
\item Chapter 4:  Disc.-time, disc.-space Markov chains (omit 4.10, 4.11)
\item Chapter 5:  Exp. distribution, Pois. processes (omit some of 5.3, 5.4)
\item Midterm exam:  Chapters 1--5
\item Chapter 6:  Cont.-time, disc.-space Markov chains (omit 6.7)
\item Monte Carlo methods
\item Markov chain Monte Carlo
\end{itemize}
\notes{
The in-class final will be comprehensive, but it will be more heavily weighted toward the
material seen after the midterm.  On the next slide, the last two major topics (Monte Carlo 
methods and MCMC) are outlined in greater detail.  Friday's class will consist of discussing
the topics on the next slide.
}
\end{frame}

\startframe{STAT 515 brief recap}
More detail on the last two\ldots
\begin{itemize}
\item Monte Carlo methods
  \begin{itemize}
  \item Understanding what ``pseudo-random'' numbers are
  \item Inversion method
  \item Rejection method
  \item Simulating multivariate normals
  \item Importance Sampling
  \item Ratio importance sampling
  \item Standard error estimation
  \end{itemize}
\item Markov chain Monte Carlo
  \begin{itemize}
  \item Metropolis-Hastings algorithms
    \begin{itemize}
    \item Special case:  Symmetric proposal (Metropolis alg.)
    \item Special case:  Propose from full cond. (Gibbs samp.)
    \end{itemize}
  \item Disc.-time, cont.-space Markov chains
    \begin{itemize}
    \item Markov transition density $k ( y\mid x)$
    \item Stationary density $\pi(x)$.
    \item Harris ergodicity ($\pi$-irreducibility, aperiodicity, Harris recurrence)
    \end{itemize}
  \item Variable-at-a-time MH; full conditionals
  \item MCMC standard errors using batch means
  \item Effective sample size
  \end{itemize}
\end{itemize}
\end{frame}

%%%%%%%%%%%%%%%%%%%%%%%%%%%%%%%%%%%%%%%%
\end{document}




