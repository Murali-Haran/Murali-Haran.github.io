\documentclass[handout]{beamer}
\usepackage{beamerthemeshadow}
\usepackage{amssymb}
\usepackage{pgfpages}
\usepackage{tikz}
\pgfpagesuselayout{4 on 1}%[landscape]

\newcommand{\beaa}{\begin{eqnarray*}}
\newcommand{\eeaa}{\end{eqnarray*}}
\newcommand{\bea}{\begin{eqnarray}}
\newcommand{\eea}{\end{eqnarray}}
\def\E{\mathop{\rm E\,}\nolimits}
\def\Var{\mathop{\rm Var\,}\nolimits}
\def\Cov{\mathop{\rm Cov\,}\nolimits}
\def\Corr{\mathop{\rm Corr\,}\nolimits}
\def\logit{\mathop{\rm logit\,}\nolimits}
\newcommand{\eid}{{\stackrel{\cal{D}}{=}}}
\newcommand{\cip}{{\stackrel{{P}}{\to}}}
\def\bR{\mathbb{R}}	% real line
\def\given{\,|\,}

%%%%%%%%%%
% From http://www.disc-conference.org/disc1998/mirror/llncs.sty
\def\vec#1{\mathchoice{\mbox{\boldmath$\displaystyle\bf#1$}}
{\mbox{\boldmath$\textstyle\bf#1$}}
{\mbox{\boldmath$\scriptstyle\bf#1$}}
{\mbox{\boldmath$\scriptscriptstyle\bf#1$}}}
%%%%%%%%%%

\def\choice#1#2
{\tikz 
{\def \lx{-.2} \def \ux{.2} \def \ly{-.2} \def\uy{.2} \filldraw[color=#1] (\lx*.9,\ly*.9) rectangle (\ux*.9,\uy*.9);
\draw(\lx,\ly) rectangle (\ux,\uy); \node at (0,0)[color=white] {\bf {#2}}; \node at (0,0) [color=black] {{#2}};}}

\def\question#1#2#3#4#5{{#1}

\begin{minipage}{2.4in}
\choice{red}{A} {#2}

\choice{yellow}{C} {#4}
\end{minipage}\begin{minipage}{2.4in}
\choice{green}{B} {#3}

\choice{blue}{D}  {#5}
\end{minipage}
}

\def\notes#1{

\vspace{2ex}
{\em Notes: {#1}}}
%
%  Simple macro to declare and use image all at once.
%
\def\figurehere#1#2{\pgfdeclareimage[width={#2}]{{#1}}{figures/#1} \pgfuseimage{{#1}}}

\def\startframe#1{\begin{frame}[t,fragile] \frametitle{\thedate \hfill {#1}} }

%
%  Declare images
%
%\pgfdeclareimage[width=3in]{geyser}{figures/geyser}

\mode<presentation>
{
%  \setbeamertemplate{background canvas}
  \usetheme{boxes}
  \usecolortheme{notblackscreen}

%  \usetheme{default}
%  \setbeamercovered{transparent}
}
%\beamertemplatetransparentcovereddynamic

\usepackage[english]{babel}
\usepackage[latin1]{inputenc}
\usepackage{times}
\usepackage[T1]{fontenc}

\setbeamersize{text margin left=0.1in}
\setbeamersize{text margin right=0.1in}



\begin{document}


\def \thedate{Apr.~6}


%%%%%%%%%%%%%%%%%%%%%%%%%%%%%%%%%%%%%%%%

\startframe{Announcements}
\begin{itemize}
\item 
HW \#10 is due on Friday, April 13 at 2:30pm.
\item 
All homework must be turned in electronically from now on.
\item 
Only 2 more homeworks!  (Only best 10 out of 11 grades will count.)
\end{itemize}
\notes{It is not clear when HW 11 will be due---possibly Apr.~20.  
We discussed the due date of the take-home midterm, which still has
not been set.
}
\end{frame}

\startframe{Monte Carlo methods}
Importance sampling can be unstable:
\begin{itemize}
\item Suppose $q(x)$ is small for some $x$ where $g(x)f(x)$ is big.  
\item Then $g(x)f(x)/q(x)$ will have extreme outliers that are rare.
\item If these values are rare enough, we may never see them and thus
we may underestimate the M.C. standard error.
\end{itemize}
\notes{
These are facts we have already discussed.  The final point is
quite important, though, and it will effect one of the problems
in homework \#10.
}
\end{frame}

\startframe{Monte Carlo methods}
Importance sampling standard errors
\begin{itemize}
\item Plain importance sampling:  Easy
\item Ratio importance sampling:  Use delta method to derive
\[
{\Var} \left( \frac{ \frac1n\sum_i A_i}{\frac1n\sum_i B_i} \right) \approx
\frac{1}{\mu_B^2}
\begin{bmatrix}
1 & \frac{-\mu_A}{\mu_B}
\end{bmatrix}
\begin{bmatrix}
\sigma^2_A & \sigma_{AB} \\ \sigma_{AB} & \sigma^2_B
\end{bmatrix}
\begin{bmatrix}
1 \\ \frac{-\mu_A}{\mu_B}
\end{bmatrix}
\]
\end{itemize}
\notes{
The delta method expression is based on the lengthy in-class derivation we
did on Apr. 2.  It will be used in HW \#10.
}
\end{frame}

\startframe{Monte Carlo methods}
(Ratio) importance sampling application:
\begin{itemize}
\item Goal:  Estimate $\mu(\theta) = \E_{f_\theta} g(X)$ for $\theta\in\Theta$.
\item Plain Monte Carlo sampling:  Draw new sample for each $\theta\in\Theta$.
\item Importance sampling:  Draw a {\em single} sample from $q(x)$, etc.
\item Ratio importance sampling:  When normalizing ``constant'' unknown.
\end{itemize}
\notes{Since is it quite common to know $f_\theta$ only up to a constant, 
the ratio importance sampling method is often useful.  The word ``constant'' is in
quotation marks above because this constant might depend on $\theta$.  In other
words, we might have $f_\theta(x) = \alpha(\theta) r_\theta(x)$.  But the point is
that $\alpha(\theta)$ does not depend on $x$.
}
\end{frame}

\startframe{Monte Carlo methods}
Ratio importance sampling for $\mu(\theta)= \E_{f_\theta} g(X)$:
\begin{itemize}
\item Estimate:  $\tilde\mu = \sum_{i=1}^n g(X_i)w\theta(X_i)$, where
\[
w_\theta(X_i) = \frac{r_\theta(X_i)/q(X_i)}{\sum_{j=1}^n r_\theta(X_j)/q(X_j)}
\]
and $f_\theta(x) = \alpha(\theta) r_\theta(x)$.
\end{itemize}
\notes{This is simply another way to view ratio importance sampling:  A
weighted average of the $g(X_i)$ values.  In this case, the weights 
depend on $\theta$, but it is useful to realize that the same sample
of $X_i$ can be used for any $\theta$, and only the weights must
be recalculated for different $\theta$.
}
\end{frame}

\startframe{Monte Carlo methods}
Ratio importance sampling for $\mu(\theta)= \E_{f_\theta} g(X)$:
\begin{itemize}
\item Estimate:  $\tilde\mu = \sum_{i=1}^n g(X_i)w\theta(X_i)$.
\item Need $q(x)>0$ whenever $f_\theta(x)>0$ for any $\theta$.  
\item Ideally, $q(x)$ not near zero whenever $f_\theta(x)$ large for any $x$,
though this may be hard to do.
\end{itemize}
\notes{
We did not discuss this slide, but it was included just to repeat the warning
about $q$ being too near zero.  
}
\end{frame}

\startframe{Monte Carlo methods}
Related:  Monte Carlo MLE
\begin{itemize}
\item
Given $X\sim f_\theta(x) = \exp\{\theta^\top s(x)\} / c(\theta)$,
write  log-likelihood as
\[
\ell(\theta) = \ell(\theta_0) + (\theta-\theta_0)^\top s(X) - 
\log E_{\theta_0} \exp\{( \theta-\theta_0)^\top s(Y)\}.
\]
\item Use $Y_1, \ldots, Y_m\sim f_{\theta_0}$ to approximate
$\ell(\theta)-\ell(\theta_0)$.
\end{itemize}
\notes{
We spent quite a bit of time discussing this slide.  The technique of Monte Carlo
MLE is closely related to importance sampling (there is a small technical 
difference), and one of the HW\#10 problems involves approximating 
a log-likelihood function using Monte Carlo techniques.
}
\end{frame}




%%%%%%%%%%%%%%%%%%%%%%%%%%%%%%%%%%%%%%%%
\end{document}




