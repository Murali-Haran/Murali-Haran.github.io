\documentclass[handout]{beamer}
\usepackage{beamerthemeshadow}
\usepackage{amssymb}
\usepackage{pgfpages}
\usepackage{tikz}
\pgfpagesuselayout{4 on 1}%[landscape]

\newcommand{\beaa}{\begin{eqnarray*}}
\newcommand{\eeaa}{\end{eqnarray*}}
\newcommand{\bea}{\begin{eqnarray}}
\newcommand{\eea}{\end{eqnarray}}
\def\E{\mathop{\rm E\,}\nolimits}
\def\Var{\mathop{\rm Var\,}\nolimits}
\def\Cov{\mathop{\rm Cov\,}\nolimits}
\def\Corr{\mathop{\rm Corr\,}\nolimits}
\def\logit{\mathop{\rm logit\,}\nolimits}
\newcommand{\eid}{{\stackrel{\cal{D}}{=}}}
\newcommand{\cip}{{\stackrel{{P}}{\to}}}
\def\bR{\mathbb{R}}	% real line
\def\given{\,|\,}

%%%%%%%%%%
% From http://www.disc-conference.org/disc1998/mirror/llncs.sty
\def\vec#1{\mathchoice{\mbox{\boldmath$\displaystyle\bf#1$}}
{\mbox{\boldmath$\textstyle\bf#1$}}
{\mbox{\boldmath$\scriptstyle\bf#1$}}
{\mbox{\boldmath$\scriptscriptstyle\bf#1$}}}
%%%%%%%%%%

\def\choice#1#2
{\tikz 
{\def \lx{-.2} \def \ux{.2} \def \ly{-.2} \def\uy{.2} \filldraw[color=#1] (\lx*.9,\ly*.9) rectangle (\ux*.9,\uy*.9);
\draw(\lx,\ly) rectangle (\ux,\uy); \node at (0,0)[color=white] {\bf {#2}}; \node at (0,0) [color=black] {{#2}};}}

\def\question#1#2#3#4#5{{#1}

\begin{minipage}{2.4in}
\choice{red}{A} {#2}

\choice{yellow}{C} {#4}
\end{minipage}\begin{minipage}{2.4in}
\choice{green}{B} {#3}

\choice{blue}{D}  {#5}
\end{minipage}
}

\def\notes#1{

\vspace{2ex}
{\em Notes: {#1}}}
%
%  Simple macro to declare and use image all at once.
%
\def\figurehere#1#2{\pgfdeclareimage[width={#2}]{{#1}}{figures/#1} \pgfuseimage{{#1}}}

\def\startframe#1{\begin{frame}[t,fragile] \frametitle{\thedate \hfill {#1}} }

%
%  Declare images
%
%\pgfdeclareimage[width=3in]{geyser}{figures/geyser}

\mode<presentation>
{
%  \setbeamertemplate{background canvas}
  \usetheme{boxes}
  \usecolortheme{notblackscreen}

%  \usetheme{default}
%  \setbeamercovered{transparent}
}
%\beamertemplatetransparentcovereddynamic

\usepackage[english]{babel}
\usepackage[latin1]{inputenc}
\usepackage{times}
\usepackage[T1]{fontenc}

\setbeamersize{text margin left=0.1in}
\setbeamersize{text margin right=0.1in}



\begin{document}


\def \thedate{Jan.~25}


%%%%%%%%%%%%%%%%%%%%%%%%%%%%%%%%%%%%%%%%

\startframe{Announcements}
\begin{itemize}
\item 
No new reading for Friday; you should be through
section 4.4
\item Tomorrow's office hours will end a bit early, at 3:45.  
(So I'll start a bit early too.)
\end{itemize}
\end{frame}

\startframe{4.3 Classification of States}
Gambler's Ruin:
\begin{itemize}
\item Start with $i$ dollars, $0\le i\le N$.
\item Play independent games, winning each with probability $p$.
\item Bet 1 dollar per game (win \$1 if you win, lose \$1 if you lose)
\item Continue until you have \$0 or $\$N$, then stay there.  

(Modify a random walk so that
0 and $N$ become absorbing states.)
\end{itemize}
\notes{This slide was a repeat from Monday's lecture.  However, before launching
back into Gambler's ruin, let's first talk about random walks a bit.}
\end{frame}

\startframe{4.3 Classification of States}
State $i$ is recurrent if and only if, conditional on $X_0=i$, 
\begin{itemize}
\item $P(\mbox{ever revisiting $i$}) = 1$
\item $E(\#\{T>0:  X_T = i\}) = \infty$.
\item $\sum_{n=1}^\infty P_{ii}^n = \infty$.
\end{itemize}
\notes{We'll use the last of these to check for the recurrence (or transience) of
the states in a random walk.}
\end{frame}

\startframe{4.3 Classification of States}
\begin{columns}
\begin{column}{2.5in}
Random walk in one dimension:

\vspace{2ex}
\begin{tikzpicture}[scale=1]
%\tikz{
\def \lx{-3.1} \def \ux{3.1} \def \ly{-1.2} \def\uy{1.2}
%\draw (\lx,\ly) rectangle (\ux,\uy);
% grid with dot at origin for helping place objects in rectangle:
%\draw[step=.5, very thin] (\lx,\ly) grid (\ux,\uy); 
\filldraw (0,0) circle (.05);
\draw[dotted] (\lx, 0) -- (\ux, 0);
\foreach \x in {-3, -2, -1, 0, 1, 2, 3}
  {\draw (\x, -.25) -- (\x, .25);
  \draw (\x, -.75) node {$\x$};
  }
\draw[<-, very thick] (-1, 0) -- (0,0);
\draw[<-, very thick] (1, 0) -- (0,0);
\draw (.5, .5) node {$p$};
\draw (-.5, .5) node {$1-p$};
\end{tikzpicture}
\end{column}
\begin{column}{2.5in}
\begin{itemize}
\item
$P(X_{t+1}=i+1 \mid X_t=i) = p$
\item
$P(X_{t+1}=i-1 \mid X_t=i) = 1-p$
\item
What is $P_{ii}^{2n}$?  
\item
What is $\sum_{n=1}^\infty P_{ii}^{2n}$?
\end{itemize}
\end{column}
\end{columns}
\notes{Answer to first question:  
$P_{ii}^{2n} = {{2n}\choose{n}} p^n(1-p)^{n}$.
We discussed why.  Then we used Stirling's approximation to
give an approximation to $P_{ii}^{2n}$, showing that
\[
\sum_{n=1}^\infty P_{ii}^{2n} 
\begin{cases}
= \infty & \mbox{if $p=1/2$,} \\
< \infty & \mbox{if $p\ne 1/2$.}
\end{cases}
\]
Therefore, the Markov chain's states are recurrent if and only if $p=1/2$.}
\end{frame}

\startframe{4.3 Classification of States}
\begin{columns}
\begin{column}{2.5in}
Random walk in two dimensions:

\vspace{2ex}
\tikz{
\def \lx{-1.6} \def \ux{1.6} \def \ly{-1.1} \def\uy{1.1}
%\draw (\lx,\ly) rectangle (\ux,\uy);
% grid with dot at origin for helping place objects in rectangle:
\draw[step=.5, dotted] (\lx,\ly) grid (\ux,\uy); 
\filldraw (0,0) circle (.05);
\draw[dotted] (\lx, 0) -- (\ux, 0);
\draw[<-, very thick] (-.5, 0) -- (0,0);
\draw[<-, very thick] (.5, 0) -- (0,0);
\draw[<-, very thick] (0, -.5) -- (0,0);
\draw[<-, very thick] (0, .5) -- (0,0);
}
\end{column}
\begin{column}{2.5in}
\begin{itemize}
\item
Four possible choices at each step.
\item 
Consider the case when all 4 directions are equally likely.
\item
What is $P_{ii}^{2n}$?  
\item
What is $\sum_{n=1}^\infty P_{ii}^{2n}$?
\end{itemize}
\end{column}
\end{columns}
\notes{We can show that 
\[
P_{ii}^{2n} = {{2n}\choose{n}}^2 \left( \frac14 \right)^{2n}
\sim
\frac{1}{n\pi},
\]
so that $\sum_{n=1}^\infty P_{ii}^{2n}=\infty$
and the Markov chain is recurrent.
}
\end{frame}

\startframe{4.3 Classification of States}
Random walk in three dimensions:
\begin{itemize}
\item
Six possible choices at each step.
\item 
Consider the case when all 6 directions are equally likely.
\item
What is $P_{ii}^{2n}$?  
\item
What is $\sum_{n=1}^\infty P_{ii}^{2n}$?
\end{itemize}
\notes{Here, the derivation is harder, but it is possible to show that
the Markov chain is NOT recurrent.  This means that starting from state
$i$, the probability of ever returning to $i$ is strictly less than one.
(This probability is $0.340527$.)
}
\end{frame}


\startframe{4.3 Classification of States}
Gambler's ruin:
\begin{itemize}
\item
The (equivalence) classes are $\{0\}, \{1, \ldots, n-1\}, \{n\}$.
\item Which classes are recurrent, and which are transient?
\item What is the probability that we wind up at $N$ dollars
(as a function of $X_0=i$)?
\end{itemize}
\notes{Let $f_i$ denote the probability of winding up at $N$
given that you start at $i$.  We argued right at the end of class
that 
\[
f_i = (1-p)f_{i-1} + pf_{i+1},
\]
which is where we will pick up on Friday.
}
\end{frame}



%%%%%%%%%%%%%%%%%%%%%%%%%%%%%%%%%%%%%%%%
\end{document}


