\documentclass[handout]{beamer}
\usepackage{beamerthemeshadow}
\usepackage{amssymb}
\usepackage{pgfpages}
\usepackage{tikz}
\pgfpagesuselayout{4 on 1}%[landscape]

\newcommand{\beaa}{\begin{eqnarray*}}
\newcommand{\eeaa}{\end{eqnarray*}}
\newcommand{\bea}{\begin{eqnarray}}
\newcommand{\eea}{\end{eqnarray}}
\def\E{\mathop{\rm E\,}\nolimits}
\def\Var{\mathop{\rm Var\,}\nolimits}
\def\Cov{\mathop{\rm Cov\,}\nolimits}
\def\Corr{\mathop{\rm Corr\,}\nolimits}
\def\logit{\mathop{\rm logit\,}\nolimits}
\newcommand{\eid}{{\stackrel{\cal{D}}{=}}}
\newcommand{\cip}{{\stackrel{{P}}{\to}}}
\def\bR{\mathbb{R}}	% real line
\def\given{\,|\,}

%%%%%%%%%%
% From http://www.disc-conference.org/disc1998/mirror/llncs.sty
\def\vec#1{\mathchoice{\mbox{\boldmath$\displaystyle\bf#1$}}
{\mbox{\boldmath$\textstyle\bf#1$}}
{\mbox{\boldmath$\scriptstyle\bf#1$}}
{\mbox{\boldmath$\scriptscriptstyle\bf#1$}}}
%%%%%%%%%%

\def\choice#1#2
{\tikz 
{\def \lx{-.2} \def \ux{.2} \def \ly{-.2} \def\uy{.2} \filldraw[color=#1] (\lx*.9,\ly*.9) rectangle (\ux*.9,\uy*.9);
\draw(\lx,\ly) rectangle (\ux,\uy); \node at (0,0)[color=white] {\bf {#2}}; \node at (0,0) [color=black] {{#2}};}}

\def\question#1#2#3#4#5{{#1}

\begin{minipage}{2.4in}
\choice{red}{A} {#2}

\choice{yellow}{C} {#4}
\end{minipage}\begin{minipage}{2.4in}
\choice{green}{B} {#3}

\choice{blue}{D}  {#5}
\end{minipage}
}

\def\notes#1{

\vspace{2ex}
{\em Notes: {#1}}}
%
%  Simple macro to declare and use image all at once.
%
\def\figurehere#1#2{\pgfdeclareimage[width={#2}]{{#1}}{figures/#1} \pgfuseimage{{#1}}}

\def\startframe#1{\begin{frame}[t,fragile] \frametitle{\thedate \hfill {#1}} }

%
%  Declare images
%
%\pgfdeclareimage[width=3in]{geyser}{figures/geyser}

\mode<presentation>
{
%  \setbeamertemplate{background canvas}
  \usetheme{boxes}
  \usecolortheme{notblackscreen}

%  \usetheme{default}
%  \setbeamercovered{transparent}
}
%\beamertemplatetransparentcovereddynamic

\usepackage[english]{babel}
\usepackage[latin1]{inputenc}
\usepackage{times}
\usepackage[T1]{fontenc}

\setbeamersize{text margin left=0.1in}
\setbeamersize{text margin right=0.1in}



\begin{document}


\def \thedate{Jan.~23}


%%%%%%%%%%%%%%%%%%%%%%%%%%%%%%%%%%%%%%%%

\startframe{Announcements}
\begin{itemize}
\item 
Read Section 4.4 (both editions) before Wednesday's class.
\end{itemize}
\end{frame}


\startframe{4.1 Introduction to Markov Chains}
Consider the simple weather example with states ``rain'' and ``no rain'':
\[
P=
\begin{bmatrix}
\alpha & 1-\alpha \\
\beta & 1-\beta
\end{bmatrix}
=
\begin{bmatrix}
0.7 & 0.3 \\
0.4 & 0.6
\end{bmatrix}
\]
On Friday, I wrote $\pi_1 = P\pi_0$, which should have been 
$\pi_1^\top = \pi_0^\top P$.
\end{frame}

\startframe{4.1 Introduction to Markov Chains}
More complicated rain example with four states:

(0) rain, rain; (1) no rain, rain; (2) rain, no rain; (3) no rain, no rain

\[
P=
\begin{bmatrix}
0.7 & 0 & 0.3 & 0 \\
0.5 & 0 & 0.5 & 0 \\
0 & 0.4 & 0 & 0.6 \\
0 & 0.2 & 0 & 0.8
\end{bmatrix}
\]
\notes{I fixed the states!  Now state 1 and 2 agree with the example 
in the book (Example 4.4 in Section 4.1).  This example points out that even a process 
where the dependence is farther back in time than one step can be made Markovian
by broadening the notion of ``state''.}
\end{frame}

\startframe{4.1 Introduction to Markov Chains}
\question{If $P$ is the transition matrix for a finite-state Markov chain 
$X_0, X_1, X_2, \ldots$, and if the $X_i$ are i.i.d., then:}
{Each (horizontal) row of $P$ is the same.}
{Each (vertical) column of $P$ is the same.}
{Each column of $P$ sums to 1.}
{None of A, B, or C.}
\notes{Answer:  A.  Almost nobody got this at first.  We had a discussion about
why an i.i.d.\ sequence is technically a Markov chain (albeit an uninteresting one).}
\end{frame}

\startframe{4.1 Introduction to Markov Chains}
\begin{itemize}
\item
Suppose $Z_1, Z_2, \ldots$ are i.i.d.\ with $P(Z_i=j)=p_j$ for any 
integer $j\ge 0$.
\item $\sum_{j=0}^\infty p_j=1$
\item
Define $X_0=0$ and $X_n=\sum_{i=1}^n Z_i$ for $n\ge 1$.
\item 
What is $P$ for the Markov chain $X_0, X_1, \ldots$?
\end{itemize}
\notes{We derived 
\[
P = 
\begin{bmatrix}
p_0 & p_1 & p_2 & p_3 & p_4 & \ldots \\
0 & p_0 & p_1 & p_2 & p_3 & \ldots \\
0 & 0 & p_0 & p_1 & p_2 &  \ldots \\
0 & 0 & 0 & p_0 & p_1 &  \ldots \\
\vdots & && & & \ddots
\end{bmatrix}
\]
}
\end{frame}

\startframe{4.2 Chapman-Kolmogorov Equations}
Summary of Section 4.2:
\[
\pi_{t+1}^\top = \pi_{t}^\top P \quad\mbox{becomes} \quad
\pi_{t+n}^\top = \pi_t^\top P^n.
\]
\notes{This is a simplistic representation of what the Chapman-Kolmogorov equations
say.}
\end{frame}

\startframe{4.3 Classification of States}
Def'n:  An equivalence relation, say $\mbox{\textregistered}$, must satisfy three properties:
\begin{enumerate}
\item Reflexivity:  $x \mbox{\textregistered} x$ for all $x$.
\item Symmetry: If $x \mbox{\textregistered} y$, then $y \mbox{\textregistered} x$ for all $x, y$.
\item Transitivity:  If $x \mbox{\textregistered} y$ and $y \mbox{\textregistered} z$, then $x \mbox{\textregistered} z$
for all $x, y, z$.
\end{enumerate}
Fact:  $\mbox{\textregistered}=$``communicates with'' is an equivalence relation on states.
\notes{
Any equivalence relations partition a set into non overlapping classes of equivalent objects.
In Section 4.3, we learn that every class is composed of either recurrent states or transient
states; there is never a mixture of the two in the same class.}
\end{frame}

\startframe{4.3 Classification of States}
Gambler's Ruin:
\begin{itemize}
\item Start with $i$ dollars, $0\le i\le N$.
\item Play independent games, winning each with probability $p$.
\item Bet 1 dollar per game (win \$1 if you win, lose \$1 if you lose)
\item Continue until you have \$0 or $\$N$, then stay there.  

(Modify a random walk so that
0 and $N$ become absorbing states.)
\end{itemize}
\end{frame}

\startframe{4.3 Classification of States}
Gambler's ruin:
\begin{itemize}
\item
The (equivalence) classes are $\{0\}, \{1, \ldots, N-1\}, \{N\}$.
\item Which classes are recurrent, and which are transient?
\item What is the probability that we wind up at $N$
(as a function of $X_0=i$)?
\end{itemize}
\notes{We'll pick up this example on Wednesday.  In the last three minutes, we
discussed random walks briefly:  In a one-dimensional random walk, if $p=1/2$ then
all states are recurrent.  But if $p\ne 1/2$, then all states are transient.}
\end{frame}



%%%%%%%%%%%%%%%%%%%%%%%%%%%%%%%%%%%%%%%%
\end{document}


