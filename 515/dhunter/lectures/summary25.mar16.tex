\documentclass[handout]{beamer}
\usepackage{beamerthemeshadow}
\usepackage{amssymb}
\usepackage{pgfpages}
\usepackage{tikz}
\pgfpagesuselayout{4 on 1}%[landscape]

\newcommand{\beaa}{\begin{eqnarray*}}
\newcommand{\eeaa}{\end{eqnarray*}}
\newcommand{\bea}{\begin{eqnarray}}
\newcommand{\eea}{\end{eqnarray}}
\def\E{\mathop{\rm E\,}\nolimits}
\def\Var{\mathop{\rm Var\,}\nolimits}
\def\Cov{\mathop{\rm Cov\,}\nolimits}
\def\Corr{\mathop{\rm Corr\,}\nolimits}
\def\logit{\mathop{\rm logit\,}\nolimits}
\newcommand{\eid}{{\stackrel{\cal{D}}{=}}}
\newcommand{\cip}{{\stackrel{{P}}{\to}}}
\def\bR{\mathbb{R}}	% real line
\def\given{\,|\,}

%%%%%%%%%%
% From http://www.disc-conference.org/disc1998/mirror/llncs.sty
\def\vec#1{\mathchoice{\mbox{\boldmath$\displaystyle\bf#1$}}
{\mbox{\boldmath$\textstyle\bf#1$}}
{\mbox{\boldmath$\scriptstyle\bf#1$}}
{\mbox{\boldmath$\scriptscriptstyle\bf#1$}}}
%%%%%%%%%%

\def\choice#1#2
{\tikz 
{\def \lx{-.2} \def \ux{.2} \def \ly{-.2} \def\uy{.2} \filldraw[color=#1] (\lx*.9,\ly*.9) rectangle (\ux*.9,\uy*.9);
\draw(\lx,\ly) rectangle (\ux,\uy); \node at (0,0)[color=white] {\bf {#2}}; \node at (0,0) [color=black] {{#2}};}}

\def\question#1#2#3#4#5{{#1}

\begin{minipage}{2.4in}
\choice{red}{A} {#2}

\choice{yellow}{C} {#4}
\end{minipage}\begin{minipage}{2.4in}
\choice{green}{B} {#3}

\choice{blue}{D}  {#5}
\end{minipage}
}

\def\notes#1{

\vspace{2ex}
{\em Notes: {#1}}}
%
%  Simple macro to declare and use image all at once.
%
\def\figurehere#1#2{\pgfdeclareimage[width={#2}]{{#1}}{figures/#1} \pgfuseimage{{#1}}}

\def\startframe#1{\begin{frame}[t,fragile] \frametitle{\thedate \hfill {#1}} }

%
%  Declare images
%
%\pgfdeclareimage[width=3in]{geyser}{figures/geyser}

\mode<presentation>
{
%  \setbeamertemplate{background canvas}
  \usetheme{boxes}
  \usecolortheme{notblackscreen}

%  \usetheme{default}
%  \setbeamercovered{transparent}
}
%\beamertemplatetransparentcovereddynamic

\usepackage[english]{babel}
\usepackage[latin1]{inputenc}
\usepackage{times}
\usepackage[T1]{fontenc}

\setbeamersize{text margin left=0.1in}
\setbeamersize{text margin right=0.1in}



\begin{document}


\def \thedate{Mar.~16}


%%%%%%%%%%%%%%%%%%%%%%%%%%%%%%%%%%%%%%%%

\startframe{Announcements}
\begin{itemize}
\item 
For Monday, March 19:  Read Sections 6.6, 6.8
\item HW \#7 is due on Wednesday, March 21.  (Discuss?)
\end{itemize}
\notes{We decided that the due date will be Friday, March 23 instead 
because it seems to be the preference of most students and 
because I have no reason not to do this.  I have made the change on the course web site.}
\end{frame}

\startframe{6.5 Limiting Probabilities}
Recall Section 4.4:
\begin{itemize}
\item
If limiting probabilities existed, so did unique stationary (long-term) probabilities.
\item Not conversely!  (Why not?)
\item In Chapter 6, life is simpler in a certain sense.
\end{itemize}
\notes{
In the discrete-time case, a Markov chain could be periodic, which destroys the possibility
of limiting probabilities even though a unique set of stationary probabilities can exist.
In the continuous-time case, however, periodicity cannot occur, so this issue does not
arise.
}
\end{frame}


\startframe{6.5 Limiting Probabilities}
\begin{itemize}
\item Suppose that $\lim_{t\to\infty}P_{ij}(t)$ exists and is independent of $i$ for all $i, j$.
\item Denote $P^\infty= \lim_{t\to\infty}P_{ij}(t)$ with each row the same.  (How to label?)
\item Derive:
$0 = P^\infty R$.
\end{itemize}
\notes{
The derivation of $0 = P^\infty R$ begins with the forward Kolmogorov equations.
We discussed the interchange of the limit ($t\to\infty$) and the summation 
(in $P^\infty R$); I mentioned that this interchange could prove problematic
if we started with the backward Komogorov equations.  However, there is another,
more obvious,  reason that the  backward equations don't lead to a useful
equation here.  I overlooked this!  But we'll discuss it in Monday's class.
}
\end{frame}

\startframe{6.5 Limiting Probabilities}
The equations can be used to find the limiting $\pi$ vector:
\begin{itemize}
\item Another way to write $0 = P^\infty R$ is
\[
v_j\pi_j = \sum_{i\ne j} q_{ij}\pi_i
\quad\mbox{for all $j$.}
\]
\item Find $\pi$ if
\[
R=
\begin{bmatrix}
-\lambda_1 & \lambda_1 \\
\lambda_2 & -\lambda_2 
\end{bmatrix}.
\]
\end{itemize}
\notes{
We did not discuss the first equation---this will happen on Monday---but the second one 
(the rate matrix for the on-off process)
leads to 
\[
\pi_1=\frac{\lambda_2}{\lambda_1+\lambda_2} \quad\mbox{and}\quad
\pi_2=\frac{\lambda_1}{\lambda_1+\lambda_2}.
\]
}
\end{frame}



%%%%%%%%%%%%%%%%%%%%%%%%%%%%%%%%%%%%%%%%
\end{document}


