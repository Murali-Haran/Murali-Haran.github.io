\documentclass[handout]{beamer}
\usepackage{beamerthemeshadow}
\usepackage{amssymb}
\usepackage{pgfpages}
\usepackage{tikz}
\pgfpagesuselayout{4 on 1}%[landscape]

\newcommand{\beaa}{\begin{eqnarray*}}
\newcommand{\eeaa}{\end{eqnarray*}}
\newcommand{\bea}{\begin{eqnarray}}
\newcommand{\eea}{\end{eqnarray}}
\def\E{\mathop{\rm E\,}\nolimits}
\def\Var{\mathop{\rm Var\,}\nolimits}
\def\Cov{\mathop{\rm Cov\,}\nolimits}
\def\Corr{\mathop{\rm Corr\,}\nolimits}
\def\logit{\mathop{\rm logit\,}\nolimits}
\newcommand{\eid}{{\stackrel{\cal{D}}{=}}}
\newcommand{\cip}{{\stackrel{{P}}{\to}}}
\def\bR{\mathbb{R}}	% real line
\def\given{\,|\,}

%%%%%%%%%%
% From http://www.disc-conference.org/disc1998/mirror/llncs.sty
\def\vec#1{\mathchoice{\mbox{\boldmath$\displaystyle\bf#1$}}
{\mbox{\boldmath$\textstyle\bf#1$}}
{\mbox{\boldmath$\scriptstyle\bf#1$}}
{\mbox{\boldmath$\scriptscriptstyle\bf#1$}}}
%%%%%%%%%%

\def\choice#1#2
{\tikz 
{\def \lx{-.2} \def \ux{.2} \def \ly{-.2} \def\uy{.2} \filldraw[color=#1] (\lx*.9,\ly*.9) rectangle (\ux*.9,\uy*.9);
\draw(\lx,\ly) rectangle (\ux,\uy); \node at (0,0)[color=white] {\bf {#2}}; \node at (0,0) [color=black] {{#2}};}}

\def\question#1#2#3#4#5{{#1}

\begin{minipage}{2.4in}
\choice{red}{A} {#2}

\choice{yellow}{C} {#4}
\end{minipage}\begin{minipage}{2.4in}
\choice{green}{B} {#3}

\choice{blue}{D}  {#5}
\end{minipage}
}

\def\notes#1{

\vspace{2ex}
{\em Notes: {#1}}}
%
%  Simple macro to declare and use image all at once.
%
\def\figurehere#1#2{\pgfdeclareimage[width={#2}]{{#1}}{figures/#1} \pgfuseimage{{#1}}}

\def\startframe#1{\begin{frame}[t,fragile] \frametitle{\thedate \hfill {#1}} }

%
%  Declare images
%
%\pgfdeclareimage[width=3in]{geyser}{figures/geyser}

\mode<presentation>
{
%  \setbeamertemplate{background canvas}
  \usetheme{boxes}
  \usecolortheme{notblackscreen}

%  \usetheme{default}
%  \setbeamercovered{transparent}
}
%\beamertemplatetransparentcovereddynamic

\usepackage[english]{babel}
\usepackage[latin1]{inputenc}
\usepackage{times}
\usepackage[T1]{fontenc}

\setbeamersize{text margin left=0.1in}
\setbeamersize{text margin right=0.1in}



\begin{document}


\def \thedate{Apr.~4}


%%%%%%%%%%%%%%%%%%%%%%%%%%%%%%%%%%%%%%%%

\startframe{Announcements}
\begin{itemize}
\item 
HW \#9 is due on Friday, April 6 at 2:30pm.
\item 
All homework must be turned in electronically from now on.
\item 
Something I had not noticed regarding Problem \#2:
You can use 5 uniforms to simulate a beta(3,2).  
\end{itemize}
\notes{
We discussed the fact that if $U_1, \ldots, U_5 \sim \mbox{unif}(0,1)$,
then
\[
\frac{ \log(U_1U_2U_3)}{\log(U_1U_2U_3U_4U_5)} \sim \mbox{Beta}(3,2).
\]
We also discussed the fact that $\log(U_1U_2U_3)$ and $\log U_1+\log U_2
+\log U_3$ are the same thing mathematically, though they are NOT the same
thing in computer math:  In computer math, the first expression is faster to evaluate 
but less stable numerically because of the increased possibility of underflow.
}
\end{frame}


\startframe{Monte Carlo methods}
Importance sampling standard errors
\begin{itemize}
\item Plain importance sampling:  Easy
\item Ratio importance sampling:  Use delta method to derive
\[
\Var \left[ \sqrt{n} ( \tilde \mu - \mu)  \right] \approx \tilde\sigma^2,
\]
where $\tilde \sigma^2 = \alpha^2 (1,-\mu) \Sigma (1,-\mu)^\top$ and $\Sigma=\ldots$
\end{itemize}
\notes{This is simply a reminder about the delta-method result from
class on Monday.  We didn't discuss it today, though later we'll use it to approximate
Monte Carlo standard errors for ratio importance sampling estimators.
}
\end{frame}


\startframe{Monte Carlo methods}
Importance Sampling Example:  What is $P(X>4.5)$ for $X\sim N(0,1)$?
\begin{itemize}
\item Try naive approach.
\item Try importance sampling with $q(x) = \exp\{-(x-4.5)\} I\{x>4.5\}$.
\item Try ratio importance sampling.  (Careful!)
\item This type of rare event problem often good for importance sampling.
\end{itemize}
\notes{
We demonstrated in R first the naive approach:\\
\ \\  
{\tt mean(rnorm(1e6) $>$ 4.5)}\\
\ \\
and an importance sampling approach using the $q$ given above:\\
\ \\
{\tt K=4.5; x=rexp(1e4)+K; mean((x>4.5)*dnorm(x)/exp(K-x))}
}
\end{frame}

\startframe{Monte Carlo methods}
Lessons learned:
\begin{itemize}
\item Plain importance sampling needs %, $q$ must be chosen so that
$g(x)f(x)>0 \Rightarrow q(x)>0$.
\item Ratio importance sampling needs %, $q$ must be chosen so that
$f(x)>0 \Rightarrow q(x)>0$.
\item (Can use different samples for numerator 
and denominator.)
\end{itemize}
\notes{We discussed this issue at length.  It was illustrated using a value of 
{\tt K=5} in the example on the previous slide.  More generally, importance sampling
can run into trouble whenever $q(x)$ is very small for values of $x$ where 
$g(x) f(x)$ is large.
It can be very difficult to diagnose the problems this causes, so it should simply be
planned for and avoided.
}
\end{frame}

\startframe{Monte Carlo methods}
Ratio importance sampling application:
\begin{itemize}
\item Goal:  Estimate $\mu(\theta) = \E_{f_\theta} g(X)$ for $\theta\in\Theta$.
\item Plain importance sampling:  Draw new sample for each $\theta\in\Theta$.
\item Ratio importance sampling:  Draw a {\em single} sample from $q(x)$\ldots

(Need: $f_\theta(x)>0$ implies $q(x)>0$ for all $\theta$.)
\end{itemize}
\notes{
We only started to discuss this, but it is an important topic because the need to
estimate a function $\mu(\theta)$ that is difficult to evaluate is actually somewhat
common.  We'll continue this discussion on Friday.
}
\end{frame}


%%%%%%%%%%%%%%%%%%%%%%%%%%%%%%%%%%%%%%%%
\end{document}




