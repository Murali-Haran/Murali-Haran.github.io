\documentclass{beamer}
\usepackage{beamerthemeshadow}
\usepackage{amssymb}
\usepackage{tikz}

\newcommand{\beaa}{\begin{eqnarray*}}
\newcommand{\eeaa}{\end{eqnarray*}}
\newcommand{\bea}{\begin{eqnarray}}
\newcommand{\eea}{\end{eqnarray}}
\def\E{\mathop{\rm E\,}\nolimits}
\def\Var{\mathop{\rm Var\,}\nolimits}
\def\Cov{\mathop{\rm Cov\,}\nolimits}
\def\Corr{\mathop{\rm Corr\,}\nolimits}
\def\logit{\mathop{\rm logit\,}\nolimits}
\newcommand{\eid}{{\stackrel{\cal{D}}{=}}}
\newcommand{\cip}{{\stackrel{{P}}{\to}}}
\def\bR{\mathbb{R}}	% real line
\def\given{\,|\,}

%%%%%%%%%%
% From http://www.disc-conference.org/disc1998/mirror/llncs.sty
\def\vec#1{\mathchoice{\mbox{\boldmath$\displaystyle\bf#1$}}
{\mbox{\boldmath$\textstyle\bf#1$}}
{\mbox{\boldmath$\scriptstyle\bf#1$}}
{\mbox{\boldmath$\scriptscriptstyle\bf#1$}}}
%%%%%%%%%%

\def\choice#1#2{
\tikz {
\def \lx{-.2} \def \ux{.2} \def \ly{-.2} \def\uy{.2} 
\filldraw[color=#1] (\lx*.9,\ly*.9) rectangle (\ux*.9,\uy*.9);
\draw(\lx,\ly) rectangle (\ux,\uy); \node at (0,0)[color=white] {\bf {#2}}; \node at (0,0) [color=black] {{#2}};
} }

\def\question#1#2#3#4#5{{#1}

\begin{minipage}{2.45in}
\begin{enumerate}
\item [{\choice{red}{A}}]\rule{-.25em}{0em}{#2}
\item [{\choice{yellow}{C}}]\rule{-.25em}{0em}{#4}
\end{enumerate}
%\choice{red}{A} {#2}
%
%\choice{yellow}{C} {#4}
\end{minipage}\begin{minipage}{2.45in}
\begin{enumerate}
\item [{\choice{green}{B}}]\rule{-.25em}{0em}{#3}
\item [{\choice{blue}{D}}]\rule{-.25em}{0em}{#5}
\end{enumerate}
%\choice{green}{B} {#3}
%
%\choice{blue}{D}  {#5}
\end{minipage}
}

%
%  Simple macro to declare and use image all at once.
%
\def\figurehere#1#2{\pgfdeclareimage[width={#2}]{{#1}}{figures/#1} \pgfuseimage{{#1}}}

\def\startframe#1{\begin{frame}[t,fragile] \frametitle{\thedate \hfill {#1}} \vspace{-1ex}}


%
%  Declare images
%
%\pgfdeclareimage[width=3in]{geyser}{figures/geyser}

\mode<presentation>
{
%  \setbeamertemplate{background canvas}
  \usetheme{boxes}
  \usecolortheme{blackscreen}

%  \usetheme{default}
%  \setbeamercovered{transparent}
}
%\beamertemplatetransparentcovereddynamic

\usepackage[english]{babel}
\usepackage[latin1]{inputenc}
\usepackage{times}
\usepackage[T1]{fontenc}

\setbeamersize{text margin left=0.1in}
\setbeamersize{text margin right=0.1in}




\begin{document}


\def \thedate{Jan.~9}


%%%%%%%%%%%%%%%%%%%%%%%%%%%%%%%%%%%%%%%%

\startframe{Reading for next class}
\begin{itemize}
\item Read and attempt to understand Sections 2.8 and 3.1 through 3.3 in the Ross book.
\item Continue into Section~3.4 as far as you can (it's 15 pages long).
\end{itemize}
\end{frame}



\startframe{Background}

Which category best describes you as a student?

\begin{minipage}{2.5in}
\choice{red}{A} Statistics graduate student

\choice{yellow}{C} Other graduate student
\end{minipage}\begin{minipage}{2.5in}
\choice{green}{B}
Math or CSE graduate student

\choice{blue}{D}  Undergraduate student
\end{minipage}

%\includegraphics[width=7in]{questions/WhatTypeOfStudent.pdf}
\end{frame}

\startframe{1.2 Sample Space and Events}
\question{If you roll two 6-sided dice, how many outcomes are in the sample space?}
{6}{12}{21}{36}
\end{frame}

\startframe{1.3 Probabilities Defined on Events}
\question{For any events $E$ and $F$, and probability function $P$, 
$P(E\cup F) = $ ?}
{$P(E) + P(F)$}
{$P(E) + P(F) - P(E\cap F)$}
{$P(E) + P(F) - P(E) P(F)$}
{$P(E) P(F)$}
\end{frame}

\startframe{1.4 Conditional Probabilities}
\question{There are about 40 people in this room.  Which of the following is closest to
$P(\mbox{at least one pair of people in this room have matching birthdays})$?}
{0.9}{0.1}{0.09}{0.01}
\end{frame}

\startframe{1.5 Independent Events}
\question{If events $E$ and $F$ are independent, then $P(E \mid F) = $?}
{$P(E)$}
{$P(E \cap F)$}
{$P(E \cup F)$}
{$P(E) / P(F)$}
\end{frame}

\startframe{1.6 Bayes' Formula}
\question{A test for detecting a certain disease has probability 0.99 of being correct in all cases.  If 1\% of the
population has the disease and a randomly selected individual tests positive, which is closest to the probability that
he/she actually has the disease?}
{0.99}{0.5}{0.09}{0.05}
\end{frame}


\startframe{2.1 Random Variables}
\question{If you roll two 6-sided dice, and you let $X$ equal the total number of dots shown, what is 
$P(X = 4)$?}
{$\frac13$}{$\frac14$}{$\frac16$}{$\frac1{12}$}
\end{frame}


\startframe{2.2 Discrete Random Variables}
\question{A Bernoulli random variable is a special case of which of the following?}
{binomial random variable}{exponential random variable}{geometric random variable}{Poisson random variable}
\end{frame}

\startframe{2.2 Discrete Random Variables}
\question{Which of the following is the probability mass function for a Poisson random variable?}
{$p(i) = p(1-p)^i$}
{$p(i) = {n \choose p} p^i(1-p)^{n-i}$}
{$p(i) = \frac{\lambda^i}{i!}e^{-\lambda} $}
{$p(i) = \frac1n$}
\end{frame}

\startframe{2.3 ``Continuous'' Random Variables}
\question{If $X$ has a continuous distribution, what is true about its cumulative distribution function $F(x)$?}
{It is strictly increasing.}
{It is everywhere differentiable.}
{It integrates to one.}
{It is continuous.}
\end{frame}

\startframe{2.3 ``Continuous'' Random Variables}
\question{Which of the following is the probability density function for an exponential random variable?}
{$f(i) = \frac{1}{\beta-\alpha}$}
{$f(i) = \lambda e^{-\lambda x}$}
{$f(i) = \frac{1}{\sigma\sqrt{2\pi}}e^{-(x-\mu)^2/2\sigma^2} $}
{$f(i) = \frac{\lambda e^{-\lambda x} (\lambda x)^{\alpha-1}}{\Gamma(\alpha)}$}
\end{frame}

\startframe{2.4 Expectation of a Random Variable}
\question{If $X$ has a continuous distribution and a density function $f(x)$,  then $E(X)=$?} 
{$\int_{-\infty}^{\infty} \!\! f(x)\,dx$}
{$\int_{-\infty}^{\infty} \!\! xf(x)\,dx$}
{$\int_{-\infty}^{\infty} \!\! x^2f(x)\,dx$}
{$\int_{-\infty}^{\infty} \!\! (x-\mu)f(x)\,dx$}
\end{frame}


\startframe{2.4 Expectation of a Random Variable}
\question{The variance is\ldots}
{the expectation of the square.}
{the square of the expectation.}
{the expectation of the square plus the square of the expectation.}
{the expectation of the square minus the square of the expectation.}
\end{frame}

\startframe{2.5 Jointly Distributed Random Variables}
\question{Which is a simpler way to write $\Var(X) + \Var(Y) + 2\Cov(X,Y)$?}
{$\Var(X+Y)$}
{$\Corr(X,Y)$}
{$E(X+Y)$}
{$E(X+Y)^2$}
\end{frame}


%\startframe{1.2\quad Properties of Probability}
%
%\begin{columns}
%\begin{column}{4cm}
%
%\tikz{
%\def \lx{-1.4} \def \ux{2.4} \def \ly{-1.2} \def\uy{1.2}
%\draw (\lx,\ly) rectangle (\ux,\uy);
%% grid with dot at origin for helping place objects in rectangle:
%%\draw[step=.5, very thin] (\lx,\ly) grid (\ux,\uy); \filldraw (0,0) circle (.05);
%\draw (0,0) ellipse (1 and .8);
%\draw (1,0) ellipse (1 and .8);
%\draw (0,.9) node {\small$A$};
%\draw (1,.9) node {\small$B$};
%\draw (2.2,1) node{$S$};
%}
%\end{column}
%\begin{column}{6cm}
%Venn diagram depicting events $A$ and $B$.  In the picture, they appear to 
%have a nonempty intersection.
%\end{column}
%\end{columns}
%\end{frame}





%%%%%%%%%%%%%%%%%%%%%%%%%%%%%%%%%%%%%%%%
\end{document}


