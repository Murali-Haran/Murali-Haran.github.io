\documentclass[handout]{beamer}
\usepackage{beamerthemeshadow}
\usepackage{amssymb}
\usepackage{pgfpages}
\usepackage{tikz}
\pgfpagesuselayout{4 on 1}%[landscape]

\newcommand{\beaa}{\begin{eqnarray*}}
\newcommand{\eeaa}{\end{eqnarray*}}
\newcommand{\bea}{\begin{eqnarray}}
\newcommand{\eea}{\end{eqnarray}}
\def\E{\mathop{\rm E\,}\nolimits}
\def\Var{\mathop{\rm Var\,}\nolimits}
\def\Cov{\mathop{\rm Cov\,}\nolimits}
\def\Corr{\mathop{\rm Corr\,}\nolimits}
\def\logit{\mathop{\rm logit\,}\nolimits}
\newcommand{\eid}{{\stackrel{\cal{D}}{=}}}
\newcommand{\cip}{{\stackrel{{P}}{\to}}}
\def\bR{\mathbb{R}}	% real line
\def\given{\,|\,}

%%%%%%%%%%
% From http://www.disc-conference.org/disc1998/mirror/llncs.sty
\def\vec#1{\mathchoice{\mbox{\boldmath$\displaystyle\bf#1$}}
{\mbox{\boldmath$\textstyle\bf#1$}}
{\mbox{\boldmath$\scriptstyle\bf#1$}}
{\mbox{\boldmath$\scriptscriptstyle\bf#1$}}}
%%%%%%%%%%

\def\choice#1#2
{\tikz 
{\def \lx{-.2} \def \ux{.2} \def \ly{-.2} \def\uy{.2} \filldraw[color=#1] (\lx*.9,\ly*.9) rectangle (\ux*.9,\uy*.9);
\draw(\lx,\ly) rectangle (\ux,\uy); \node at (0,0)[color=white] {\bf {#2}}; \node at (0,0) [color=black] {{#2}};}}

\def\question#1#2#3#4#5{{#1}

\begin{minipage}{2.4in}
\choice{red}{A} {#2}

\choice{yellow}{C} {#4}
\end{minipage}\begin{minipage}{2.4in}
\choice{green}{B} {#3}

\choice{blue}{D}  {#5}
\end{minipage}
}

\def\notes#1{

\vspace{2ex}
{\em Notes: {#1}}}
%
%  Simple macro to declare and use image all at once.
%
\def\figurehere#1#2{\pgfdeclareimage[width={#2}]{{#1}}{figures/#1} \pgfuseimage{{#1}}}

\def\startframe#1{\begin{frame}[t,fragile] \frametitle{\thedate \hfill {#1}} }

%
%  Declare images
%
%\pgfdeclareimage[width=3in]{geyser}{figures/geyser}

\mode<presentation>
{
%  \setbeamertemplate{background canvas}
  \usetheme{boxes}
  \usecolortheme{notblackscreen}

%  \usetheme{default}
%  \setbeamercovered{transparent}
}
%\beamertemplatetransparentcovereddynamic

\usepackage[english]{babel}
\usepackage[latin1]{inputenc}
\usepackage{times}
\usepackage[T1]{fontenc}

\setbeamersize{text margin left=0.1in}
\setbeamersize{text margin right=0.1in}



\begin{document}


\def \thedate{Apr.~23}


%%%%%%%%%%%%%%%%%%%%%%%%%%%%%%%%%%%%%%%%

\startframe{Announcements}
\begin{itemize}
\item 
Take-home final is on ANGEL.  It is due at 5pm on May 2.
\item 
Please fill out SRTEs!
\end{itemize}
\notes{
Someone pointed out an error in the binomial pmf on the take-home final.  This is fixed now!
}
\end{frame}


\startframe{Markov chain Monte Carlo}
Special cases of VAATMH:  Gibbs and Metropolis updates
\begin{itemize}
\item  Recall full MH acceptance probability
\item Metropolis:  $q(y_i\mid x_i, x_{-i}) = q(x_i\mid y_i, x_{-i})$
\item Gibbs:  $q(y_i \mid x_i, x_{-i}) = \pi(y_i \mid x_{-i})$
\end{itemize}
\[
\alpha(x_i, y_i \mid x_{-i}) = \min
\left\{ 1,
\frac{\pi(y_i \mid x_{-i}) q(x_i\mid y_i, x_{-i})}{ \pi(x_i \mid x_{-i}) q(y_i\mid x_i, x_{-i})}
\right\}
\]
\notes{
This was a bit of review of the fact that when using a VAATMH, the $\pi$ in the acceptance ratio
should be the full conditional, not the joint density.
}
\end{frame}

\startframe{Markov chain Monte Carlo}
Why VAATMH works:
\begin{itemize}
\item Suppose $\vec x = (x_1, x_2)^\top$.
\item Take Markov transition densities 
$k_{1\mid 2} (y_1 \mid x_1, x_2)$ and
$k_{2\mid 1} (y_2 \mid x_1, x_2)$.
\item Then the transition density for both updates is
\[
k[ (y_1, y_2) \mid (x_1, x_2) ] =
k_{1\mid 2} (y_1 \mid x_1, x_2) k_{2\mid 1}(y_2 \mid y_1, x_2).
\]
\end{itemize}
\notes{
The equations that must be satisfied by $k_{1\mid 2}$ and $k_{2\mid 1}$ are
\begin{eqnarray*}
\pi ( y_1 \mid x_2) &=& \int k_{1 \mid 2}( y_1 \mid x_1, x_2) \pi(x_1 \mid x_2) \, dx_1, \\
\pi ( y_2 \mid x_1) &=& \int k_{2 \mid 1}( y_2 \mid x_1, x_2) \pi(x_2 \mid x_1) \, dx_2
\end{eqnarray*}
for all $y_1$ and $y_2$.
}
\end{frame}

\startframe{Markov chain Monte Carlo}
Why VAATMH works:
\begin{itemize}
\item
$k[ (y_1, y_2) \mid (x_1, x_2) ] =
k_{1\mid 2} (y_1 \mid x_1, x_2) k_{2\mid 1}(y_2 \mid y_1, x_2)$.
\item Show $\pi(\vec y) = \int k(\vec y \mid \vec x) \pi(\vec x) \, d\vec x$.
\end{itemize}
\notes{We proved this in class, using the two equations in the notes on the previous slide, as follows:
\begin{eqnarray*}
&&\int k(\vec y \mid \vec x) \pi(\vec x) \, d\vec x \\
 &=&
\int \!\! k_{2\mid 1}(y_2 \mid y_1, x_2)  \pi(x_2) 
\left[ \int \!\! k_{1\mid 2} (y_1 \mid x_1, x_2) \pi( x_1 \mid x_2 )
\, dx_1 \right] dx_2 \\
 &=&
\int \!\!  k_{2\mid 1}(y_2 \mid y_1, x_2) \pi(y_1 \mid x_2) \pi(x_2)
\, dx_2 \\
 &=&
\pi(y_1) \int \!\!  k_{2\mid 1}(y_2 \mid x_2, y_1 ) \pi(x_2 \mid y_1) 
\, dx_2 \\
&=& \pi(y_2 \mid y_1) \pi(y_1) = \pi(\vec y).
\end{eqnarray*}
}
\end{frame}

\startframe{Markov chain Monte Carlo}
MCMC standard errors for estimate of $E_\pi g(X)$
\begin{itemize}
\item Batch means idea (where $n=ab$):
\[
\mbox{Let}\quad
Y_1 = \frac{X_1+ \cdots + X_b}{b}, 
Y_2 = \frac{X_{b+1}+ \cdots + X_{2b}}{b}, \ldots
\]
\item Then $\hat\sigma^2 = \frac{1}{a-1} \sum_{i=1}^a b(Y_i - \hat\mu)^2$.
And $\widehat{\mbox{MCMCse}} = \hat\sigma/\sqrt{n}$.
\item Jones et al (2006):  Take, for example, $b=\sqrt{n}$.
\end{itemize}
\notes{
The idea here is to choose $Y_i$ to satisfy two goals:
First, the dependence between the $Y_i$ is weak; and second, the 
dependence within each of the $Y_i$ will capture most of the
dependence among the $X_i$.  If this is true, then
by a CLT-like result for Markov chains,
the variance of $\frac{1}{\sqrt b} \sum_{i=1}^b X_i$
is a reasonable estimator of the variance of $\frac1{\sqrt n}\sum_{i=1}^b X_i$.
This in turn means that $b/n$ times the sample variance of the $Y_i$ is
a reasonable estimator of $\Var \overline X_n$.
}
\end{frame}


%%%%%%%%%%%%%%%%%%%%%%%%%%%%%%%%%%%%%%%%
\end{document}




