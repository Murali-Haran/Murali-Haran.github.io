\documentclass[handout]{beamer}
\usepackage{beamerthemeshadow}
\usepackage{amssymb}
\usepackage{pgfpages}
\usepackage{tikz}
\pgfpagesuselayout{4 on 1}%[landscape]

\newcommand{\beaa}{\begin{eqnarray*}}
\newcommand{\eeaa}{\end{eqnarray*}}
\newcommand{\bea}{\begin{eqnarray}}
\newcommand{\eea}{\end{eqnarray}}
\def\E{\mathop{\rm E\,}\nolimits}
\def\Var{\mathop{\rm Var\,}\nolimits}
\def\Cov{\mathop{\rm Cov\,}\nolimits}
\def\Corr{\mathop{\rm Corr\,}\nolimits}
\def\logit{\mathop{\rm logit\,}\nolimits}
\newcommand{\eid}{{\stackrel{\cal{D}}{=}}}
\newcommand{\cip}{{\stackrel{{P}}{\to}}}
\def\bR{\mathbb{R}}	% real line
\def\given{\,|\,}

%%%%%%%%%%
% From http://www.disc-conference.org/disc1998/mirror/llncs.sty
\def\vec#1{\mathchoice{\mbox{\boldmath$\displaystyle\bf#1$}}
{\mbox{\boldmath$\textstyle\bf#1$}}
{\mbox{\boldmath$\scriptstyle\bf#1$}}
{\mbox{\boldmath$\scriptscriptstyle\bf#1$}}}
%%%%%%%%%%

\def\choice#1#2
{\tikz 
{\def \lx{-.2} \def \ux{.2} \def \ly{-.2} \def\uy{.2} \filldraw[color=#1] (\lx*.9,\ly*.9) rectangle (\ux*.9,\uy*.9);
\draw(\lx,\ly) rectangle (\ux,\uy); \node at (0,0)[color=white] {\bf {#2}}; \node at (0,0) [color=black] {{#2}};}}

\def\question#1#2#3#4#5{{#1}

\begin{minipage}{2.4in}
\choice{red}{A} {#2}

\choice{yellow}{C} {#4}
\end{minipage}\begin{minipage}{2.4in}
\choice{green}{B} {#3}

\choice{blue}{D}  {#5}
\end{minipage}
}

\def\notes#1{

\vspace{2ex}
{\em Notes: {#1}}}
%
%  Simple macro to declare and use image all at once.
%
\def\figurehere#1#2{\pgfdeclareimage[width={#2}]{{#1}}{figures/#1} \pgfuseimage{{#1}}}

\def\startframe#1{\begin{frame}[t,fragile] \frametitle{\thedate \hfill {#1}} }

%
%  Declare images
%
%\pgfdeclareimage[width=3in]{geyser}{figures/geyser}

\mode<presentation>
{
%  \setbeamertemplate{background canvas}
  \usetheme{boxes}
  \usecolortheme{notblackscreen}

%  \usetheme{default}
%  \setbeamercovered{transparent}
}
%\beamertemplatetransparentcovereddynamic

\usepackage[english]{babel}
\usepackage[latin1]{inputenc}
\usepackage{times}
\usepackage[T1]{fontenc}

\setbeamersize{text margin left=0.1in}
\setbeamersize{text margin right=0.1in}



\begin{document}


\def \thedate{Apr.~9}


%%%%%%%%%%%%%%%%%%%%%%%%%%%%%%%%%%%%%%%%

\startframe{Announcements}
\begin{itemize}
\item 
All homework must be turned in electronically from now on.
\item 
Only 2 more homeworks!  (Only best 10 out of 11 grades will count.)
\end{itemize}
\end{frame}

\startframe{Monte Carlo methods}
Recall:  Monte Carlo MLE
\begin{itemize}
\item
Given $X\sim f_\theta(x) = \exp\{\theta^\top s(x)\} / c(\theta)$,
write  log-likelihood as
\[
\ell(\theta) = \ell(\theta_0) + (\theta-\theta_0)^\top s(X) - 
\log E_{\theta_0} \exp\{( \theta-\theta_0)^\top s(Y)\}.
\]
\item Use $Y_1, \ldots, Y_m\sim f_{\theta_0}$ to approximate
$\ell(\theta)-\ell(\theta_0)$.
\end{itemize}
\notes{
We discussed this method of Monte Carlo likelihood as it relates to
problem 2 on HW \#10.  In particular, we discussed the fact that
the above framework applies to the case in which $X$ is binomial
(and $s(X)=X$).  This example is presented in Section 3 of
``Improving Simulation-Based Algorithms for Fitting ERGMs''
by Hummel et al (2012), which can be found on my website.
}
\end{frame}

\startframe{Monte Carlo methods}
\begin{itemize}
\item Regular importance sampling:  %With $q(x) = f_{\theta_0}(x)$,
Estimate $E_{f_{\theta_0}} g(X)f_\theta(X)/f_{\theta_0}(X)$ by
\[
\left( \sum_i\frac{g(X_i)r_\theta(X_i)}{r_{\theta_0}(X_i)} \right)
\Bigg/
\left( \sum_i \frac{r_\theta(X_i)}{r_{\theta_0}(X_i)} \right).
\]
\item
In the case of Monte Carlo MLE, $g$ is irrelevant and we only
need the denominator.
\end{itemize}
\notes{This is because the denominator converges almost surely to
$c(\theta)/c(\theta_0)$ by the Strong Law if we take
$r_\theta(x) = \exp\{\theta^\top s(x)\}$.
In other words, Monte Carlo is used here to approximate the
ratio of normalizing constants, which is related to importance
sampling but not exactly the same thing.
}
\end{frame}

\startframe{Markov chain Monte Carlo}
\begin{itemize}
\item Goal:  Estimate $\mu=E_\pi g(X)$ but
cannot %use Monte Carlo directly since we cannot 
sample $X_i\sim \pi$ directly.
\item 
Importance sampling may be difficult, particularly as dimension increases.
\item 
MCMC solution:  Take $\hat\mu = \frac1n\sum g(X_i)$, where
$X_1, X_2, \ldots$ is a simulated Markov chain with stationary distribution $\pi$.
\end{itemize}
\notes{
Here, we are relying on the Strong Law for Markov chains, which says that 
(under some regularity conditions) 
\[
\frac1n \sum_{i=1}^n g(X_i) \stackrel{\rm a.s.}{\to} \mu.
\]
In this case, $\{X_t\}t\ge 0$ is a discrete-time chain, but it can be either
discrete-space or continuous-space.  This is the first time we have considered the
latter case in this class.
}
\end{frame}

\startframe{Markov chain Monte Carlo}
Metropolis-Hastings algorithm (recall HW\#4, problem 5):
\begin{itemize}
\item Start with $X_0=x_0$.  
\item For $i=0, 1, \ldots$, 
Generate $Y \sim q(y\mid x_0)$.
\item Define $\alpha(x,y) = \min\{1, \pi(y)q(x\mid y) / [\pi(x)q(y\mid x)]$.
\item Let $X_{i+1}=Y$ with probability $\alpha(X_i,Y)$.
\end{itemize}
\notes{
There were at least three mistakes on this overhead and in my notes on the chalkboard!
We'll fix all of the mistakes on Wednesday.  (The version above is now correct.)
}
\end{frame}


%%%%%%%%%%%%%%%%%%%%%%%%%%%%%%%%%%%%%%%%
\end{document}




