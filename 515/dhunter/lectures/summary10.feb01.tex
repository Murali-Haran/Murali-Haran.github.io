\documentclass[handout]{beamer}
\usepackage{beamerthemeshadow}
\usepackage{amssymb}
\usepackage{pgfpages}
\usepackage{tikz}
\pgfpagesuselayout{4 on 1}%[landscape]

\newcommand{\beaa}{\begin{eqnarray*}}
\newcommand{\eeaa}{\end{eqnarray*}}
\newcommand{\bea}{\begin{eqnarray}}
\newcommand{\eea}{\end{eqnarray}}
\def\E{\mathop{\rm E\,}\nolimits}
\def\Var{\mathop{\rm Var\,}\nolimits}
\def\Cov{\mathop{\rm Cov\,}\nolimits}
\def\Corr{\mathop{\rm Corr\,}\nolimits}
\def\logit{\mathop{\rm logit\,}\nolimits}
\newcommand{\eid}{{\stackrel{\cal{D}}{=}}}
\newcommand{\cip}{{\stackrel{{P}}{\to}}}
\def\bR{\mathbb{R}}	% real line
\def\given{\,|\,}

%%%%%%%%%%
% From http://www.disc-conference.org/disc1998/mirror/llncs.sty
\def\vec#1{\mathchoice{\mbox{\boldmath$\displaystyle\bf#1$}}
{\mbox{\boldmath$\textstyle\bf#1$}}
{\mbox{\boldmath$\scriptstyle\bf#1$}}
{\mbox{\boldmath$\scriptscriptstyle\bf#1$}}}
%%%%%%%%%%

\def\choice#1#2
{\tikz 
{\def \lx{-.2} \def \ux{.2} \def \ly{-.2} \def\uy{.2} \filldraw[color=#1] (\lx*.9,\ly*.9) rectangle (\ux*.9,\uy*.9);
\draw(\lx,\ly) rectangle (\ux,\uy); \node at (0,0)[color=white] {\bf {#2}}; \node at (0,0) [color=black] {{#2}};}}

\def\question#1#2#3#4#5{{#1}

\begin{minipage}{2.4in}
\choice{red}{A} {#2}

\choice{yellow}{C} {#4}
\end{minipage}\begin{minipage}{2.4in}
\choice{green}{B} {#3}

\choice{blue}{D}  {#5}
\end{minipage}
}

\def\notes#1{

\vspace{2ex}
{\em Notes: {#1}}}
%
%  Simple macro to declare and use image all at once.
%
\def\figurehere#1#2{\pgfdeclareimage[width={#2}]{{#1}}{figures/#1} \pgfuseimage{{#1}}}

\def\startframe#1{\begin{frame}[t,fragile] \frametitle{\thedate \hfill {#1}} }

%
%  Declare images
%
%\pgfdeclareimage[width=3in]{geyser}{figures/geyser}

\mode<presentation>
{
%  \setbeamertemplate{background canvas}
  \usetheme{boxes}
  \usecolortheme{notblackscreen}

%  \usetheme{default}
%  \setbeamercovered{transparent}
}
%\beamertemplatetransparentcovereddynamic

\usepackage[english]{babel}
\usepackage[latin1]{inputenc}
\usepackage{times}
\usepackage[T1]{fontenc}

\setbeamersize{text margin left=0.1in}
\setbeamersize{text margin right=0.1in}



\begin{document}


\def \thedate{Feb.~1}


%%%%%%%%%%%%%%%%%%%%%%%%%%%%%%%%%%%%%%%%

\startframe{Announcements}
\begin{itemize}
\item 
No new reading for Friday; make sure you've read through 
Section 4.8 (both editions)
\item 
HW \#3 is due Wednesday, Feb.~8
\item
No class one week from Friday.

\end{itemize}
\end{frame}


\startframe{4.6 Mean Time Spent in Transient States}
Suppose you have \$2 and you bet on fair games of chance
until you either go broke or have \$5.  
\begin{itemize}
\item What is the expected number of time steps that you have \$2?
\item How long will this experiment last, on average?
\item What is the probability that you will at some point have \$1?
\end{itemize}
\notes{We saw questions one and three we saw on Monday; I added the middle 
question as an extra.}
\end{frame}

\startframe{4.6 Mean Time Spent in Transient States}
\begin{itemize}
\item
For transient states $i$ and $j$, let $s_{ij}$ equal the
expected number of time steps spent in $j$, given $X_0=i$.
\item Let $S=(s_{ij})$ be the matrix of $s_{ij}$ values.
\item Last time, we showed that
$S = I + P_TS$.
\item Therefore, $S=(I=P_T)^{-1}$.
\end{itemize}
\notes{This is review from Monday's class.}
\end{frame}

\startframe{4.6 Mean Time Spent in Transient States}
In our example, 
\[
P_T = \begin{bmatrix}
0 & \frac12 & 0 & 0 \\
\frac12 & 0 & \frac12 & 0 \\
0 & \frac12 & 0 & \frac12 \\
0 & 0 & \frac12 & 0 \\
\end{bmatrix}
\quad \mbox{and so } \quad
S = (I-P_T)^{-1} = 
\frac15
\begin{bmatrix}
8 & 6 & 4 & 2 \\
6 & 12 & 8 & 4 \\
4 & 8 & 12 & 6 \\
2 & 4 & 6 & 8 \\
\end{bmatrix}.
\]
\notes{Here, the theory developed so far is applied to the specific problem at hand (i.e., gambler's
ruin with $N=5$, $i=2$, and $p=1/2$).  Important note:  I did {\em not} reorder the states, as is done 
in Section 4.6, so that all of the transient ones come first.  Instead, this $P_T$ matrix is the 
$4\times4$ submatrix in the middle of the $6\times 6$ full $P$ matrix, since the 4 transient states are all except for the first state (zero) and the last state (five).
}
\end{frame}

\startframe{4.6 Mean Time Spent in Transient States}
\begin{columns}
\begin{column}{2in}
In our example, 
\[
S = 
\frac15
\begin{bmatrix}
8 & 6 & 4 & 2 \\
6 & 12 & 8 & 4 \\
4 & 8 & 12 & 6 \\
2 & 4 & 6 & 8 \\
\end{bmatrix}
\]
\end{column}
\begin{column}{2.5in}
Starting with \$2\ldots
\begin{itemize}
\item
Mean \# of steps with \$2:  \pause$\frac{12}{5}$
\pause
\item
Mean \# of steps before end: \pause 6
\end{itemize}
\end{column}
\end{columns}
\notes{These answers may be read directly from the $S$ matrix.
For the second one, it's necessary to sum the entire second row because
\[
E(\mbox{time spent in all transient states}) = \sum_{\mbox{$i$ transient}}
\!\!\!\!\!E(\mbox{time spent in state $i$}).
\]
}
\end{frame}

\startframe{4.6 Mean Time Spent in Transient States}
Let $f_{ij} \stackrel{\rm def}{=}P(\mbox{$X_t=j$ for some $t>0$} \mid X_0=t)$
\begin{itemize}
\item Demonstrate that $s_{ij} = I\{i=j\} + f_{ij} s_{jj}$.
\pause
\item Solve to find $f_{ij}$.
\end{itemize}
\notes{The key is to argue that 
\[
E( \mbox{\# steps in $j$} \mid X_0=i, \mbox{$X_t=j$ for some $t>0$}) =
\begin{cases}
s_{jj} + 1 & \mbox{if $i=j$} \cr
s_{jj} & \mbox{if $i\ne j$} 
\end{cases}
\]
and
\[
E( \mbox{\# steps in $j$} \mid X_0=i, \mbox{$X_t$ is never $j$ for any $t>0$}) =
\begin{cases}
1 & \mbox{if $i=j$} \cr
0 & \mbox{if $i\ne j$.} 
\end{cases}
\]
This leads to $s_{ij} = [ s_{jj} + I\{i=j\}]f_{ij} + I\{i=j\} (1-f_{ij})$ using
conditioning, as desired.
Solving gives $f_{ij} = [ s_{ij} - I\{i=j\} ]/s_{jj}$.
}
\end{frame}

\startframe{4.6 Mean Time Spent in Transient States}
\begin{columns}
\begin{column}{2in}
In our example, 
\[
S = 
\frac15
\begin{bmatrix}
8 & 6 & 4 & 2 \\
6 & 12 & 8 & 4 \\
4 & 8 & 12 & 6 \\
2 & 4 & 6 & 8 \\
\end{bmatrix}
\]
\end{column}
\begin{column}{2.5in}
Starting with \$2\ldots
\begin{itemize}
\item
Probability of ever having \$1:  \pause $\frac68$
\pause
\item
Probability of ever returning to \$2: \pause $\frac{7}{12}$
\end{itemize}
\end{column}
\end{columns}
\end{frame}

\startframe{4.7 Branching Processes}
\begin{itemize}
\item Particular type of Markov chain
\item $X_t=$size of $t$th generation, $t=0, 1, 2, \ldots$.
\item Assume each individual produces offspring 
independently according to some distribution
${\cal P} = \{P_0, P_1, P_2, \ldots\}$
on the nonnegative integers.
\end{itemize}
\end{frame}

\startframe{4.7 Branching Processes}
For a branching process $X_0, X_1, X_2, \ldots$,
\begin{itemize}
\item
Possible to demonstrate $E X_n = \mu^n$
\item 
Possible to demonstrate $\Var X_n = 
\begin{cases}
\sigma^2\mu^{n-1} \left( \frac{1-\mu^n}{1-\mu} \right) & \mbox{if $\mu\ne 1$}\cr
n\sigma^2 & \mbox{if $\mu=1$}
\end{cases}
$
\end{itemize}
\ldots where $\mu$ and $\sigma^2$ are the mean and variance of 
${\cal P}$.
\notes{The argument used for this derivation is simple conditioning
on $X_{n-1}$, then recursively on $X_{n-2}, X_{n-3}, \ldots$.}
\end{frame}

\startframe{4.7 Branching Processes}
Generally, the most interesting cases involve $P_0>0$.  In this case\ldots
\begin{itemize}
\item what are the classes of states?
\item which are recurrent / transient?
\item what does this mean for the Markov chain in the long term?
\end{itemize}
\notes{We argued that when $P_0>0$, all states other than zero form a single 
transient class.  Zero is its own (recurrent) class, since it is an absorbing
state.  We conclude that either the process grows without bound, or it dies 
out by eventually hitting zero.}
\end{frame}

\startframe{4.7 Branching Processes}
A commonly asked question is:  What is
\[
P( \mbox{population will die out} \mid
X_0=1 )?
\]
Call this probability $\pi_0$ (as in book).  Then
\[
\pi_0=\lim_{n\to\infty} P(X_n=0 \mid X_0=1).
\]
\end{frame}

\startframe{4.7 Branching Processes}
\begin{itemize}
\item
Well-known fact:  If $\mu<1$, then $\pi_0=1$.
\item
Can you derive the equation $\pi_0 = \sum_{j=0}^\infty \pi_0^j P_j$?
\end{itemize}
\notes{In an epidemiological context, $\mu$ is often called
$R_0$.  In a simple branching-process model, this means that
$R_0=1$ is the threshold between an epidemic possibly occurring
($X_n$ gets large with positive probability) and no epidemic possible
($X_n\to 0$ with probability one).
}
\end{frame}

%%%%%%%%%%%%%%%%%%%%%%%%%%%%%%%%%%%%%%%%
\end{document}


