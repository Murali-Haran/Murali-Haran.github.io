\documentclass[handout]{beamer}
\usepackage{beamerthemeshadow}
\usepackage{amssymb}
\usepackage{pgfpages}
\usepackage{tikz}
\pgfpagesuselayout{4 on 1}%[landscape]

\newcommand{\beaa}{\begin{eqnarray*}}
\newcommand{\eeaa}{\end{eqnarray*}}
\newcommand{\bea}{\begin{eqnarray}}
\newcommand{\eea}{\end{eqnarray}}
\def\E{\mathop{\rm E\,}\nolimits}
\def\Var{\mathop{\rm Var\,}\nolimits}
\def\Cov{\mathop{\rm Cov\,}\nolimits}
\def\Corr{\mathop{\rm Corr\,}\nolimits}
\def\logit{\mathop{\rm logit\,}\nolimits}
\newcommand{\eid}{{\stackrel{\cal{D}}{=}}}
\newcommand{\cip}{{\stackrel{{P}}{\to}}}
\def\bR{\mathbb{R}}	% real line
\def\given{\,|\,}

%%%%%%%%%%
% From http://www.disc-conference.org/disc1998/mirror/llncs.sty
\def\vec#1{\mathchoice{\mbox{\boldmath$\displaystyle\bf#1$}}
{\mbox{\boldmath$\textstyle\bf#1$}}
{\mbox{\boldmath$\scriptstyle\bf#1$}}
{\mbox{\boldmath$\scriptscriptstyle\bf#1$}}}
%%%%%%%%%%

\def\choice#1#2
{\tikz 
{\def \lx{-.2} \def \ux{.2} \def \ly{-.2} \def\uy{.2} \filldraw[color=#1] (\lx*.9,\ly*.9) rectangle (\ux*.9,\uy*.9);
\draw(\lx,\ly) rectangle (\ux,\uy); \node at (0,0)[color=white] {\bf {#2}}; \node at (0,0) [color=black] {{#2}};}}

\def\question#1#2#3#4#5{{#1}

\begin{minipage}{2.4in}
\choice{red}{A} {#2}

\choice{yellow}{C} {#4}
\end{minipage}\begin{minipage}{2.4in}
\choice{green}{B} {#3}

\choice{blue}{D}  {#5}
\end{minipage}
}

\def\notes#1{

\vspace{2ex}
{\em Notes: {#1}}}
%
%  Simple macro to declare and use image all at once.
%
\def\figurehere#1#2{\pgfdeclareimage[width={#2}]{{#1}}{figures/#1} \pgfuseimage{{#1}}}

\def\startframe#1{\begin{frame}[t,fragile] \frametitle{\thedate \hfill {#1}} }

%
%  Declare images
%
%\pgfdeclareimage[width=3in]{geyser}{figures/geyser}

\mode<presentation>
{
%  \setbeamertemplate{background canvas}
  \usetheme{boxes}
  \usecolortheme{notblackscreen}

%  \usetheme{default}
%  \setbeamercovered{transparent}
}
%\beamertemplatetransparentcovereddynamic

\usepackage[english]{babel}
\usepackage[latin1]{inputenc}
\usepackage{times}
\usepackage[T1]{fontenc}

\setbeamersize{text margin left=0.1in}
\setbeamersize{text margin right=0.1in}



\begin{document}


\def \thedate{Apr.~16}


%%%%%%%%%%%%%%%%%%%%%%%%%%%%%%%%%%%%%%%%

\startframe{Announcements}
\begin{itemize}
\item 
Last homework due Next Monday (Apr.~23)
\item 
Take-home final exam will be available on Saturday, Apr.~21 and due
at 5:00pm on Wednesday, May~2.
\end{itemize}
\notes{
NB:  There are two days of ``overlap'' after the take-home is available and
before the last homework is due.  Please plan your time accordingly.
}
\end{frame}


\startframe{Markov chain Monte Carlo}
For concise definitions of the conditions for Harris ergodicity\ldots
\begin{itemize}
\item $\pi$-irreducible
\item Aperiodic
\item Harris recurrent
\end{itemize}
\ldots see p.~1711 of Tierney (1994, {\it Annals of Statistics})
\notes{
Tierney's paper gives a terrific concise summary of what the above
conditions mean in the discrete-time, continuous-state-space case.
}
\end{frame}

\startframe{Markov chain Monte Carlo}
Strong Law result:
If a Harris-ergodic Markov chain $X_0, X_1, \ldots$ has
stationary distribution $\pi$, and if $\mu=E_\pi g(X)<\infty$, then
\[
\frac1n \sum_{i=1}^n g(X_i) \stackrel{\rm as}{\to} \mu
\quad\mbox{as $n\to\infty$.}
\]
Caution!  These random variables are not i.i.d.!  (Do we have a CLT?)
\notes{
Yes, we do have a CLT under certain additional assumptions, but this gets very
technical and we won't discuss it in this class.  (A definitive reference for Markov
chain CLTs is the book by Meyn and Tweedie, which is available online.)
It is important to remember that even when 
\[
\sqrt{n} \left(  \frac1n \sum_{i=1}^n g(X_i)  - \mu \right) 
\stackrel{d}{\to} N(0, \sigma^2),
\]
the value is $\sigma^2$ is NOT merely $\Var_\pi g(X_i)$, even in the case where
$X_1\sim \pi$ so the chain is stationary.  This is due to the dependence among
$X_1, X_2, \ldots$.
}
\end{frame}

\startframe{Markov chain Monte Carlo}
What about burnin?
\begin{itemize}
\item What is it?  
\item Is it necessary?
\end{itemize}
Check out Jones and Hobert (2001, {\it Stat. Sci.})
\notes{
We discussed the intuition behind burnin and looked at both the Jones and Hobert paper
and also Charlie Geyer's webpage about burnin (to find it, simply google ``geyer burnin").
}
\end{frame}

\startframe{Markov chain Monte Carlo}
\begin{itemize}
\item ``All at once'' MH:  Go from $\vec X^{(i)} \to \vec X^{(i+1)}$.
\item 
``Variable-at-a-time'' MH:
\[
\begin{bmatrix}
X_1^{(i)} \\ X_2^{(i)} \\ \vdots \\ X_k^{(i)}
\end{bmatrix}
\to
\begin{bmatrix}
X_1^{(i+1)} \\ X_2^{(i)} \\ \vdots \\ X_k^{(i)}
\end{bmatrix}
\to
\begin{bmatrix}
X_1^{(i+1)} \\ X_2^{(i+1)} \\ \vdots \\ X_k^{(i)}
\end{bmatrix}
\to \cdots \to
\begin{bmatrix}
X_1^{(i+1)} \\ X_2^{(i+1)} \\ \vdots \\ X_k^{(i+1)}
\end{bmatrix}
\]
\end{itemize}
\notes{See the notes on the following slide.}
\end{frame}

\startframe{Markov chain Monte Carlo}
Some notes about VAATMH:
\begin{itemize}
\item Each $X_k$ might be multivariate.
\item Avoids (or lessens) curse of dimensionality.
\item Need to define ``Full conditional distributions''
\end{itemize}
\notes{
My choice of notation here is unfortunate; on the previous slide, $x_k$ is the LAST 
element of the $\vec x$ vector, but here, $x_k$ is merely some generic element.
Sorry for the confusion.  
}
\end{frame}

\startframe{Markov chain Monte Carlo}
Full conditional distributions
\begin{itemize}
\item Let $\pi(\vec x)$ be the target density.
\item Denote by $\pi(x_k \mid x_{-k})$ the conditional density
of $x_k$ given all of $\vec x$ not including $x_k$.
\item When doing VAATMH, the update to $X_k$ should be 
an ``all-at-once'' MH algorithm with target density $\pi(x_k \mid x_{-k})$.
\end{itemize}
\notes{
This is much easier to understand if we consider an example of full conditionals,
which is done on the next slide.
}
\end{frame}

\startframe{Markov chain Monte Carlo}
Bayesian VAATMH example (from Gilks et al (1996), pp. 75--76):
\begin{itemize}
\item $Y_1, \ldots, Y_n \stackrel{\rm iid}{\sim} N(\mu, \tau^{-1})$
\item $\mu \sim N(0,1)$
\item $\tau \sim \mbox{gamma}(2,1)$
\end{itemize}
Devise an MCMC scheme for sampling from the posterior distribution.
\notes{We did not have time to explore this example yet; we'll do this 
at the start of class on Wednesday.
}
\end{frame}



%%%%%%%%%%%%%%%%%%%%%%%%%%%%%%%%%%%%%%%%
\end{document}




