\documentclass{beamer}
\usepackage{beamerthemeshadow}
\usepackage{amssymb}
\usepackage{tikz}

\newcommand{\beaa}{\begin{eqnarray*}}
\newcommand{\eeaa}{\end{eqnarray*}}
\newcommand{\bea}{\begin{eqnarray}}
\newcommand{\eea}{\end{eqnarray}}
\def\E{\mathop{\rm E\,}\nolimits}
\def\Var{\mathop{\rm Var\,}\nolimits}
\def\Cov{\mathop{\rm Cov\,}\nolimits}
\def\Corr{\mathop{\rm Corr\,}\nolimits}
\def\logit{\mathop{\rm logit\,}\nolimits}
\newcommand{\eid}{{\stackrel{\cal{D}}{=}}}
\newcommand{\cip}{{\stackrel{{P}}{\to}}}
\def\bR{\mathbb{R}}	% real line
\def\given{\,|\,}

%%%%%%%%%%
% From http://www.disc-conference.org/disc1998/mirror/llncs.sty
\def\vec#1{\mathchoice{\mbox{\boldmath$\displaystyle\bf#1$}}
{\mbox{\boldmath$\textstyle\bf#1$}}
{\mbox{\boldmath$\scriptstyle\bf#1$}}
{\mbox{\boldmath$\scriptscriptstyle\bf#1$}}}
%%%%%%%%%%

\def\choice#1#2{
\tikz {
\def \lx{-.2} \def \ux{.2} \def \ly{-.2} \def\uy{.2} 
\filldraw[color=#1] (\lx*.9,\ly*.9) rectangle (\ux*.9,\uy*.9);
\draw(\lx,\ly) rectangle (\ux,\uy); \node at (0,0)[color=white] {\bf {#2}}; \node at (0,0) [color=black] {{#2}};
} }

\def\question#1#2#3#4#5{{#1}

\begin{minipage}{2.45in}
\begin{enumerate}
\item [{\choice{red}{A}}]\rule{-.25em}{0em}{#2}
\item [{\choice{yellow}{C}}]\rule{-.25em}{0em}{#4}
\end{enumerate}
%\choice{red}{A} {#2}
%
%\choice{yellow}{C} {#4}
\end{minipage}\begin{minipage}{2.45in}
\begin{enumerate}
\item [{\choice{green}{B}}]\rule{-.25em}{0em}{#3}
\item [{\choice{blue}{D}}]\rule{-.25em}{0em}{#5}
\end{enumerate}
%\choice{green}{B} {#3}
%
%\choice{blue}{D}  {#5}
\end{minipage}
}

%
%  Simple macro to declare and use image all at once.
%
\def\figurehere#1#2{\pgfdeclareimage[width={#2}]{{#1}}{figures/#1} \pgfuseimage{{#1}}}

\def\startframe#1{\begin{frame}[t,fragile] \frametitle{\thedate \hfill {#1}} \vspace{-1ex}}


%
%  Declare images
%
%\pgfdeclareimage[width=3in]{geyser}{figures/geyser}

\mode<presentation>
{
%  \setbeamertemplate{background canvas}
  \usetheme{boxes}
  \usecolortheme{blackscreen}

%  \usetheme{default}
%  \setbeamercovered{transparent}
}
%\beamertemplatetransparentcovereddynamic

\usepackage[english]{babel}
\usepackage[latin1]{inputenc}
\usepackage{times}
\usepackage[T1]{fontenc}

\setbeamersize{text margin left=0.1in}
\setbeamersize{text margin right=0.1in}




\begin{document}


\def \thedate{Mar.~19}


%%%%%%%%%%%%%%%%%%%%%%%%%%%%%%%%%%%%%%%%

\startframe{Announcements}
\begin{itemize}
\item 
HW \#7 is now due on Friday, March 23.
\end{itemize}
\end{frame}

\startframe{6.5 Limiting Probabilities}
\begin{itemize}
\item Starting from the forward equations, we derived:
$0 = P^\infty R$.
\item Why not 
$0 = R P^\infty $ starting from the backward equations?
\item 
Notation warning:  Here, I'm using $\pi$ to denote the limiting probabilities
(the rows of $P^\infty$).  The book uses $P_1, P_2, \ldots$; the
$\pi_i$ are something different in Section 6.6!
\end{itemize}
\end{frame}

\startframe{6.5 Limiting Probabilities}
The equations can be used to find the limiting $\pi$ vector:
\begin{itemize}
\item Another way to write $0 = P^\infty R$ is
\[
v_j\pi_j = \sum_{i\ne j} q_{ij}\pi_i
\quad\mbox{for all $j$.}
\]
{\em Why is this ``balance condition'' different from ``detailed balance''?}
\end{itemize}
\end{frame}

\startframe{6.5 Limiting Probabilities}
Sufficient conditions for the existence of the limiting probabilities $\pi_j$:
\begin{itemize}
\item Irreducibility (all states communicate)
\item Positive recurrence (mean time to return to each state is finite)
\end{itemize}
When the limiting probabilities exist, we say the chain is ergodic.
\end{frame}

\startframe{6.5 Limiting Probabilities}
%\begin{itemize}
%\item
Re-interpretation of 
$v_j\pi_j = \sum_{i\ne j} q_{ij}\pi_i$
%\quad\mbox{
for all $j$:
\begin{itemize}
\item
In the long run,
the rate of leaving state $j$ is the same as the rate
of entering state $j$.
\item Apply this to birth-death processes to find $\pi$ for such a process.
\item Looking ahead to Section 6.6:  In fact, a birth-death process must be time-reversible.
\end{itemize}
\end{frame}

\startframe{6.6 Time reversibility}
\begin{itemize}
\item Analogue to Section~4.8:  A continuous-time ergodic Markov chain is
time-reversible with limiting probabilities $\pi_i$ if and only if
\[
\pi_i q_{ij} = \pi_j q_{ji} \quad\mbox{for all pairs $i,j$.}
\]
{\em (The \# of $i\to j$ and $j\to i$ transitions per unit time are the same.)}
\item This is {\em detailed balance}.
\end{itemize}
\end{frame}

%%%%%%%%%%%%%%%%%%%%%%%%%%%%%%%%%%%%%%%%
\end{document}


