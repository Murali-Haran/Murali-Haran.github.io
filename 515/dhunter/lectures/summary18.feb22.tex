\documentclass[handout]{beamer}
\usepackage{beamerthemeshadow}
\usepackage{amssymb}
\usepackage{pgfpages}
\usepackage{tikz}
\pgfpagesuselayout{4 on 1}%[landscape]

\newcommand{\beaa}{\begin{eqnarray*}}
\newcommand{\eeaa}{\end{eqnarray*}}
\newcommand{\bea}{\begin{eqnarray}}
\newcommand{\eea}{\end{eqnarray}}
\def\E{\mathop{\rm E\,}\nolimits}
\def\Var{\mathop{\rm Var\,}\nolimits}
\def\Cov{\mathop{\rm Cov\,}\nolimits}
\def\Corr{\mathop{\rm Corr\,}\nolimits}
\def\logit{\mathop{\rm logit\,}\nolimits}
\newcommand{\eid}{{\stackrel{\cal{D}}{=}}}
\newcommand{\cip}{{\stackrel{{P}}{\to}}}
\def\bR{\mathbb{R}}	% real line
\def\given{\,|\,}

%%%%%%%%%%
% From http://www.disc-conference.org/disc1998/mirror/llncs.sty
\def\vec#1{\mathchoice{\mbox{\boldmath$\displaystyle\bf#1$}}
{\mbox{\boldmath$\textstyle\bf#1$}}
{\mbox{\boldmath$\scriptstyle\bf#1$}}
{\mbox{\boldmath$\scriptscriptstyle\bf#1$}}}
%%%%%%%%%%

\def\choice#1#2
{\tikz 
{\def \lx{-.2} \def \ux{.2} \def \ly{-.2} \def\uy{.2} \filldraw[color=#1] (\lx*.9,\ly*.9) rectangle (\ux*.9,\uy*.9);
\draw(\lx,\ly) rectangle (\ux,\uy); \node at (0,0)[color=white] {\bf {#2}}; \node at (0,0) [color=black] {{#2}};}}

\def\question#1#2#3#4#5{{#1}

\begin{minipage}{2.4in}
\choice{red}{A} {#2}

\choice{yellow}{C} {#4}
\end{minipage}\begin{minipage}{2.4in}
\choice{green}{B} {#3}

\choice{blue}{D}  {#5}
\end{minipage}
}

\def\notes#1{

\vspace{2ex}
{\em Notes: {#1}}}
%
%  Simple macro to declare and use image all at once.
%
\def\figurehere#1#2{\pgfdeclareimage[width={#2}]{{#1}}{figures/#1} \pgfuseimage{{#1}}}

\def\startframe#1{\begin{frame}[t,fragile] \frametitle{\thedate \hfill {#1}} }

%
%  Declare images
%
%\pgfdeclareimage[width=3in]{geyser}{figures/geyser}

\mode<presentation>
{
%  \setbeamertemplate{background canvas}
  \usetheme{boxes}
  \usecolortheme{notblackscreen}

%  \usetheme{default}
%  \setbeamercovered{transparent}
}
%\beamertemplatetransparentcovereddynamic

\usepackage[english]{babel}
\usepackage[latin1]{inputenc}
\usepackage{times}
\usepackage[T1]{fontenc}

\setbeamersize{text margin left=0.1in}
\setbeamersize{text margin right=0.1in}



\begin{document}


\def \thedate{Feb.~22}


%%%%%%%%%%%%%%%%%%%%%%%%%%%%%%%%%%%%%%%%

\startframe{Announcements}
\begin{itemize}
\item 
HW\#5 is due {\em ELECTRONICALLY} on Friday.
\item 
Midterm exam:  7:00pm on Wednesday, Feb.~29 in 220 Willard.
\item New STAT 515 brief course description:
\begin{quotation}
Conditional probability and expectation, Markov chains, Poisson processes, Continuous-time Markov chains, Monte Carlo methods, Markov chain Monte Carlo
\end{quotation}
\end{itemize}
\notes{I'm going to see whether it's possible to move the midterm exam to a larger room. 
Obviously, I'll announce this in class if the room changes.
The midterm will cover the parts of Chapters 3, 4, and 5 that we've reviewed so far;
in other words, it will cover the first half of the brief course description above.}
\end{frame}

\startframe{5.4.1 Nonhomogeneous Poisson Processes}
Notice a slight change from the homogeneous version:
\begin{itemize}
\item Counts of events in disjoint time intervals are independent.
\item $P[N(s+h)-N(s)=1] = \lambda(s) h + o(h)$ as $h\to 0$.
\item $P[N(s+h)-N(s)\ge2] = o(h)$ as $h\to 0$.
\item Technically, we should also say $N(0)=0$.
\end{itemize}
\notes{The difference is $\lambda(s)h$ instead of $\lambda h$.  Everything else is the
same.}
\end{frame}

\startframe{5.4.1 Nonhomogeneous Poisson Processes}
\begin{itemize}
\item
In a nonhomogeneous Poisson process,
$N(s+t)-N(s)$ is Poisson distributed with mean
\[
\int_s^{s+t} \lambda(y)\,dy.
\]
\item
If we define $m(t) = \int_0^t \lambda(y)\,dy$, then this is just
$m(s+t)-m(s)$.
\item
What is the distribution of the waiting time until the first event?
\end{itemize}
\notes{Answer:  We derived the cdf of $T_1$, as follows:
\[
F_{T_1}(t) = 1-\exp\{-m(t)\}.
\]
This is not necessary a simple distribution.  
}
\end{frame}

\startframe{Midterm practice}
\question{Suppose that two light bulbs are both working today.  In the future,
\begin{itemize}
\item If 2 bulbs are working, with probability .02, the next day 1 will be.
\item If 1 bulb is working, with probability .05, the next day 0 will be.
\item If 0 bulbs are working, with probability 1, the next day 2 will be.
\end{itemize}
What is the expected wait time between replacement $(0\to 2)$ events?
}
{35 days}
{55 days}
{67 days}
{71 days}
\notes{Answer:  D.  We wrote down the transition probability matrix for
a 3-state Markov chain that can help with this solution.  Alternatively,
notice that the wait times to get from 2 to 1, 1 to 0, and 0 to 2 are all
geometric random variables in this case, and the respective means
are 50, 20, and 1.}
\end{frame}

\startframe{}
\question{The subject of Chapter 6 is \rule{1in}{0.01in}
Markov chains.}
{discrete-time, discrete-space}
{discrete-time, continuous-space}
{continuous-time, discrete-space}
{continuous-time, continuous-space}
\notes{Answer:  C.  As Dr.~Haran puts it, this
course will cover the DD, CD, and DC cases.  We have
done DD and we now move to CD.}
\end{frame}

\startframe{5.3 The Poisson Process}
{\em Review from Monday:  }  
Poisson processes add:
\begin{itemize}
\item
If $N_1(t)$ and $N_2(t)$ are Poisson processes with rates $\lambda_1$ and $\lambda_2$, 
then $M(t) = N_1(t)+N_2(t)$ is a Poisson process with rate $\lambda_1+\lambda_2$.
\item
Similarly, suppose we label each event $M(t)$ as type-1 or type-2 with probabilities
$\lambda_1/(\lambda_1+\lambda_2)$ and 
$\lambda_2/(\lambda_1+\lambda_2)$, respectively.  Then the counting process for each
type of event is separately a Poisson process.
\end{itemize}
\notes{The latter fact, which may be referred to as {\em thinning}
a Poisson process, will be especially important as we study continuous-time
Markov chains.}
\end{frame}

\startframe{6.2 Continuous-Time Markov chains}
Consider a continuous-time, discrete-space MC, $\{X(t) : t\in[0,\infty)\}$:
\begin{itemize}
\item The chain stays in each state for a random (real-valued) time.
\item When the chain switches, it picks from the possible states according to
some discrete-time Markov chain.
\item Markov property:  
Conditional on the present, what happens next is independent of the past.
\item Homogeneity:  The chain's behavior at time $t$ does not
depend on $t$.
\end{itemize}
\notes{
We discussed these assumptions in class.  The last two assumptions
imply that the time spent in any state is exponentially distributed.  
We will prove this fact in the next class.
}
\end{frame}


%%%%%%%%%%%%%%%%%%%%%%%%%%%%%%%%%%%%%%%%
\end{document}


