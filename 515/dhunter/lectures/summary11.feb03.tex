\documentclass[handout]{beamer}
\usepackage{beamerthemeshadow}
\usepackage{amssymb}
\usepackage{pgfpages}
\usepackage{tikz}
\pgfpagesuselayout{4 on 1}%[landscape]

\newcommand{\beaa}{\begin{eqnarray*}}
\newcommand{\eeaa}{\end{eqnarray*}}
\newcommand{\bea}{\begin{eqnarray}}
\newcommand{\eea}{\end{eqnarray}}
\def\E{\mathop{\rm E\,}\nolimits}
\def\Var{\mathop{\rm Var\,}\nolimits}
\def\Cov{\mathop{\rm Cov\,}\nolimits}
\def\Corr{\mathop{\rm Corr\,}\nolimits}
\def\logit{\mathop{\rm logit\,}\nolimits}
\newcommand{\eid}{{\stackrel{\cal{D}}{=}}}
\newcommand{\cip}{{\stackrel{{P}}{\to}}}
\def\bR{\mathbb{R}}	% real line
\def\given{\,|\,}

%%%%%%%%%%
% From http://www.disc-conference.org/disc1998/mirror/llncs.sty
\def\vec#1{\mathchoice{\mbox{\boldmath$\displaystyle\bf#1$}}
{\mbox{\boldmath$\textstyle\bf#1$}}
{\mbox{\boldmath$\scriptstyle\bf#1$}}
{\mbox{\boldmath$\scriptscriptstyle\bf#1$}}}
%%%%%%%%%%

\def\choice#1#2
{\tikz 
{\def \lx{-.2} \def \ux{.2} \def \ly{-.2} \def\uy{.2} \filldraw[color=#1] (\lx*.9,\ly*.9) rectangle (\ux*.9,\uy*.9);
\draw(\lx,\ly) rectangle (\ux,\uy); \node at (0,0)[color=white] {\bf {#2}}; \node at (0,0) [color=black] {{#2}};}}

\def\question#1#2#3#4#5{{#1}

\begin{minipage}{2.4in}
\choice{red}{A} {#2}

\choice{yellow}{C} {#4}
\end{minipage}\begin{minipage}{2.4in}
\choice{green}{B} {#3}

\choice{blue}{D}  {#5}
\end{minipage}
}

\def\notes#1{

\vspace{2ex}
{\em Notes: {#1}}}
%
%  Simple macro to declare and use image all at once.
%
\def\figurehere#1#2{\pgfdeclareimage[width={#2}]{{#1}}{figures/#1} \pgfuseimage{{#1}}}

\def\startframe#1{\begin{frame}[t,fragile] \frametitle{\thedate \hfill {#1}} }

%
%  Declare images
%
%\pgfdeclareimage[width=3in]{geyser}{figures/geyser}

\mode<presentation>
{
%  \setbeamertemplate{background canvas}
  \usetheme{boxes}
  \usecolortheme{notblackscreen}

%  \usetheme{default}
%  \setbeamercovered{transparent}
}
%\beamertemplatetransparentcovereddynamic

\usepackage[english]{babel}
\usepackage[latin1]{inputenc}
\usepackage{times}
\usepackage[T1]{fontenc}

\setbeamersize{text margin left=0.1in}
\setbeamersize{text margin right=0.1in}



\begin{document}


\def \thedate{Feb.~3}


%%%%%%%%%%%%%%%%%%%%%%%%%%%%%%%%%%%%%%%%

\startframe{Announcements}
\begin{itemize}
\item
Please read Sections 5.1, 5.2.1, 5.2.2, and 5.2.3 (both editions) for Monday.
\item 
No class next Friday.
\item Office hours will be moved from Thursday, Feb. 9 to Tuesday, Feb. 7 from 2:00 to 4:00.
\end{itemize}
\notes{We will not begin Chapter 5 on Monday; that will most likely happen on Wednesday.}
\end{frame}


\startframe{4.8 Time Reversible Markov Chains}
``Consider a stationary ergodic Markov chain (that is, an ergodic Markov chain
that has been in operation for a long time)\ldots''
\begin{itemize}
\item ergodic?
\end{itemize}
\notes{This is the very first line from Section 4.8, and so it's time to talk about what 
``ergodic'' means.}
\end{frame}

\startframe{4.4 Limiting Probabilities}
Some terminology (with which you've already dealt to some extent!)
\begin{itemize}
\item period
\item aperiodic
\item positive recurrent
\item ergodic
\end{itemize}
\notes{Each of the last three of these is a class property.  }
\end{frame}

\startframe{4.3 Classification of States}
Remember this?

\question{If all of the states of a Markov chain communicate with each other,
the chain is said to be}
{irreducible}
{ergodic}
{recurrent}
{transient}
\notes{Correct answer:  A.  }
\end{frame}

\startframe{4.4 Limiting Probabilities}
Theorem:  For an irreducible ergodic (time-homogeneous) 
Markov chain with transition matrix $P$,
\[
\lim_{n\to\infty} P^n_{ij} 
\]
exists and it does not depend on $i$ (it does generally depend on $j$).
\notes{This is a very important theorem!}
\end{frame}

%%%%%%%%%%%%%%%%%%%%%%%%%%%%%%%%%%%%%%%%
\end{document}


