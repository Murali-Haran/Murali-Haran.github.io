\documentclass{beamer}
\usepackage{beamerthemeshadow}
\usepackage{amssymb}
\usepackage{tikz}

\newcommand{\beaa}{\begin{eqnarray*}}
\newcommand{\eeaa}{\end{eqnarray*}}
\newcommand{\bea}{\begin{eqnarray}}
\newcommand{\eea}{\end{eqnarray}}
\def\E{\mathop{\rm E\,}\nolimits}
\def\Var{\mathop{\rm Var\,}\nolimits}
\def\Cov{\mathop{\rm Cov\,}\nolimits}
\def\Corr{\mathop{\rm Corr\,}\nolimits}
\def\logit{\mathop{\rm logit\,}\nolimits}
\newcommand{\eid}{{\stackrel{\cal{D}}{=}}}
\newcommand{\cip}{{\stackrel{{P}}{\to}}}
\def\bR{\mathbb{R}}	% real line
\def\given{\,|\,}

%%%%%%%%%%
% From http://www.disc-conference.org/disc1998/mirror/llncs.sty
\def\vec#1{\mathchoice{\mbox{\boldmath$\displaystyle\bf#1$}}
{\mbox{\boldmath$\textstyle\bf#1$}}
{\mbox{\boldmath$\scriptstyle\bf#1$}}
{\mbox{\boldmath$\scriptscriptstyle\bf#1$}}}
%%%%%%%%%%

\def\choice#1#2{
\tikz {
\def \lx{-.2} \def \ux{.2} \def \ly{-.2} \def\uy{.2} 
\filldraw[color=#1] (\lx*.9,\ly*.9) rectangle (\ux*.9,\uy*.9);
\draw(\lx,\ly) rectangle (\ux,\uy); \node at (0,0)[color=white] {\bf {#2}}; \node at (0,0) [color=black] {{#2}};
} }

\def\question#1#2#3#4#5{{#1}

\begin{minipage}{2.45in}
\begin{enumerate}
\item [{\choice{red}{A}}]\rule{-.25em}{0em}{#2}
\item [{\choice{yellow}{C}}]\rule{-.25em}{0em}{#4}
\end{enumerate}
%\choice{red}{A} {#2}
%
%\choice{yellow}{C} {#4}
\end{minipage}\begin{minipage}{2.45in}
\begin{enumerate}
\item [{\choice{green}{B}}]\rule{-.25em}{0em}{#3}
\item [{\choice{blue}{D}}]\rule{-.25em}{0em}{#5}
\end{enumerate}
%\choice{green}{B} {#3}
%
%\choice{blue}{D}  {#5}
\end{minipage}
}

%
%  Simple macro to declare and use image all at once.
%
\def\figurehere#1#2{\pgfdeclareimage[width={#2}]{{#1}}{figures/#1} \pgfuseimage{{#1}}}

\def\startframe#1{\begin{frame}[t,fragile] \frametitle{\thedate \hfill {#1}} \vspace{-1ex}}


%
%  Declare images
%
%\pgfdeclareimage[width=3in]{geyser}{figures/geyser}

\mode<presentation>
{
%  \setbeamertemplate{background canvas}
  \usetheme{boxes}
  \usecolortheme{blackscreen}

%  \usetheme{default}
%  \setbeamercovered{transparent}
}
%\beamertemplatetransparentcovereddynamic

\usepackage[english]{babel}
\usepackage[latin1]{inputenc}
\usepackage{times}
\usepackage[T1]{fontenc}

\setbeamersize{text margin left=0.1in}
\setbeamersize{text margin right=0.1in}




\begin{document}


\def \thedate{Mar.~30}


%%%%%%%%%%%%%%%%%%%%%%%%%%%%%%%%%%%%%%%%

\startframe{Announcements}
\begin{itemize}
\item 
HW \#9 is due on Friday, April 6 at 2:30pm.
\item 
All homework must be turned in electronically from now on.
\item Have a look at Section 11.6; we'll discuss parts of it.
\end{itemize}
\end{frame}


\startframe{Monte Carlo methods}
Recall:  Suppose that 
\begin{itemize}
\item
$f(x)=\alpha r(x)$ and $g(x)=\beta s(x)$ are density functions
\item
$r(x)$ and $s(x)$ are known functions (but $\alpha$ and $\beta$ might be unknown).  
\item
$K=\sup_x r(x)/s(x)$ is finite
and known.  
\item 
Let $X\sim g(x)$ and $U\sim \mbox{unif}(0,1)$ be independent. 
\end{itemize}
\[
\mbox{Then} \quad X \mid U \le \frac{r(X)}{Ks(X)} \sim f.
\]
\end{frame}

\startframe{Monte Carlo methods}
Two cases in which $f(x)$ may only be known up to a constant:
\begin{itemize}
\item $f(\vec x) \propto \exp \{- \theta \sum_{i} x_i x_{i+1}\}$, where
$x_i = \pm 1$ for all $i$.
\item Bayesian model:  
\[
\mbox{posterior density} =f(\theta\mid x) \propto L(x \mid \theta) p(\theta).
\]
\end{itemize}
\end{frame}


\startframe{Monte Carlo methods}
How to simulate from a multivariate normal:
\begin{itemize}
\item Goal: $\vec X \sim N_p(\vec\mu, \Sigma)$.
\item Use fact that $\Var (M\vec Y) = M (\Var \vec Y )M^\top$.
\end{itemize}
\end{frame}

\startframe{Monte Carlo methods}
Importance Sampling:
\begin{itemize}
\item Goal:  Find $\mu = \int g(x)f(x)\,dx = \E_f g(X)$.
\item Rewrite as 
\[
\mu = \int \frac{g(x)f(x)}{q(x)} q(x) \,dx = \E_q \frac{g(X)f(X)}{q(X)}.
\]
\end{itemize}
\end{frame}

\startframe{Monte Carlo methods}
Why implement importance sampling?
\begin{itemize}
\item Estimators of $\mu$:
\[
\hat\mu_1 = \frac1n \sum_{i=1}^n g(X_i),
\mbox{for $X_i \stackrel{\rm iid}{\sim} f$}
\quad \mbox{\ vs.\ } \quad
\hat\mu_2 = \frac1n \sum_{i=1}^n \frac{g(X_i)f(X_i)}{q(X_i)},
\mbox{for $X_i \stackrel{\rm iid}{\sim} q$}.
\]
\end{itemize}
\end{frame}


%%%%%%%%%%%%%%%%%%%%%%%%%%%%%%%%%%%%%%%%
\end{document}


