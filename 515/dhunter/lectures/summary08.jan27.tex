\documentclass[handout]{beamer}
\usepackage{beamerthemeshadow}
\usepackage{amssymb}
\usepackage{pgfpages}
\usepackage{tikz}
\pgfpagesuselayout{4 on 1}%[landscape]

\newcommand{\beaa}{\begin{eqnarray*}}
\newcommand{\eeaa}{\end{eqnarray*}}
\newcommand{\bea}{\begin{eqnarray}}
\newcommand{\eea}{\end{eqnarray}}
\def\E{\mathop{\rm E\,}\nolimits}
\def\Var{\mathop{\rm Var\,}\nolimits}
\def\Cov{\mathop{\rm Cov\,}\nolimits}
\def\Corr{\mathop{\rm Corr\,}\nolimits}
\def\logit{\mathop{\rm logit\,}\nolimits}
\newcommand{\eid}{{\stackrel{\cal{D}}{=}}}
\newcommand{\cip}{{\stackrel{{P}}{\to}}}
\def\bR{\mathbb{R}}	% real line
\def\given{\,|\,}

%%%%%%%%%%
% From http://www.disc-conference.org/disc1998/mirror/llncs.sty
\def\vec#1{\mathchoice{\mbox{\boldmath$\displaystyle\bf#1$}}
{\mbox{\boldmath$\textstyle\bf#1$}}
{\mbox{\boldmath$\scriptstyle\bf#1$}}
{\mbox{\boldmath$\scriptscriptstyle\bf#1$}}}
%%%%%%%%%%

\def\choice#1#2
{\tikz 
{\def \lx{-.2} \def \ux{.2} \def \ly{-.2} \def\uy{.2} \filldraw[color=#1] (\lx*.9,\ly*.9) rectangle (\ux*.9,\uy*.9);
\draw(\lx,\ly) rectangle (\ux,\uy); \node at (0,0)[color=white] {\bf {#2}}; \node at (0,0) [color=black] {{#2}};}}

\def\question#1#2#3#4#5{{#1}

\begin{minipage}{2.4in}
\choice{red}{A} {#2}

\choice{yellow}{C} {#4}
\end{minipage}\begin{minipage}{2.4in}
\choice{green}{B} {#3}

\choice{blue}{D}  {#5}
\end{minipage}
}

\def\notes#1{

\vspace{2ex}
{\em Notes: {#1}}}
%
%  Simple macro to declare and use image all at once.
%
\def\figurehere#1#2{\pgfdeclareimage[width={#2}]{{#1}}{figures/#1} \pgfuseimage{{#1}}}

\def\startframe#1{\begin{frame}[t,fragile] \frametitle{\thedate \hfill {#1}} }

%
%  Declare images
%
%\pgfdeclareimage[width=3in]{geyser}{figures/geyser}

\mode<presentation>
{
%  \setbeamertemplate{background canvas}
  \usetheme{boxes}
  \usecolortheme{notblackscreen}

%  \usetheme{default}
%  \setbeamercovered{transparent}
}
%\beamertemplatetransparentcovereddynamic

\usepackage[english]{babel}
\usepackage[latin1]{inputenc}
\usepackage{times}
\usepackage[T1]{fontenc}

\setbeamersize{text margin left=0.1in}
\setbeamersize{text margin right=0.1in}



\begin{document}


\def \thedate{Jan.~27}


%%%%%%%%%%%%%%%%%%%%%%%%%%%%%%%%%%%%%%%%

\startframe{Announcements}
\begin{itemize}
\item 
Read Sections 4.5.1 and 4.7 (both editions)
\item 
The next homework will not be posted until Monday; it will be due sometime the week
 of Feb.~6.
\end{itemize}
\notes{Also, I posted some more R code to serve as a companion to HW \#2. 
I will continue to do this whenever it seems necessary, so please continue
to ask R-related questions on email or in office hours so that I can tailor these
supplementary materials in the future.}
\end{frame}

\startframe{4.3 Classification of States}
\question{If all of the states of a Markov chain communicate with each other,
the chain is said to be}
{irreducible}
{ergodic}
{recurrent}
{transient}
\notes{Correct answer:  A.  Obviously this is nothing but a terminology question.}
\end{frame}

\startframe{4.3 Classification of States}
\question{If $P$ is the transition matrix for a Markov chain, what does the symbol
$P_{ij}^n$ mean?}
{$(P^n)_{ij}$}
{$(P_{ij})^n$}
{$nP_{ij}$}
{$P_{ij}$}
\notes{Correct answer:  A.  C and D are silly answers; I only put them in because 
I needed 4 choices.  The important thing to remember here is that the ambiguous notation
$P_{ij}^n$ refers to the $i,j$th entry of $P^n$, which has a very meaningful interpretation 
for a Markov chain.  On the other hand, $(P_{ij})^n$ does not have a useful interpretation.
}
\end{frame}


\startframe{4.3 Classification of States}
Gambler's ruin:
\begin{itemize}

\item Start with $i$ dollars, $0\le i\le N$.
\item Bet 1 dollar per game (win \$1 if you win, lose \$1 if you lose)
\item Continue until you have \$0 or $\$N$, then stay there.  
\item
The communicating classes are $\{0\}, \{1, \ldots, N-1\}, \{N\}$.
\item What is the probability that we wind up at $N$ dollars
(given $X_0=i$)?
\end{itemize}
\notes{
We derived the answer to the final question using essentially the
technique seen in Section 4.5.1 of the textbook.  This is a very useful
example, emphasizing the importance of conditioning and, occasionally,
recurrence relationships when analyzing Markov chain behavior.
}
\end{frame}

\startframe{4.3 Classification of States}
Theorem:  
Recurrence (or transience) is a class property.

How do we prove this?
\notes{On the next slide are three equivalent ways to characterize
recurrent states.  We decided that $\sum_{n=1}^\infty P_{ii}^n = \infty$
was the most useful of the three for this situation, and I asked you
to try to prove that if $\sum_{n=1}^\infty P_{ii}^n = \infty$ and 
there exist $k$ and $m$ with $P_{ij}^k>0$ and $P_{ji}^m>0$, then
$\sum_{n=1}^\infty P_{jj}^n = \infty$.  Check Section 4.3 if you get stuck;
we'll wrap this up next Monday.
}
\end{frame}

\startframe{4.3 Classification of States}
State $i$ is recurrent if and only if, conditional on $X_0=i$, 
\begin{itemize}
\item $P(\mbox{ever revisiting $i$}) = 1$
\item $E(\#\{T>0:  X_T = i\}) = \infty$.
\item $\sum_{n=1}^\infty P_{ii}^n = \infty$.
\end{itemize}
\end{frame}



%%%%%%%%%%%%%%%%%%%%%%%%%%%%%%%%%%%%%%%%
\end{document}


