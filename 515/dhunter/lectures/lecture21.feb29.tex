\documentclass{beamer}
\usepackage{beamerthemeshadow}
\usepackage{amssymb}
\usepackage{tikz}

\newcommand{\beaa}{\begin{eqnarray*}}
\newcommand{\eeaa}{\end{eqnarray*}}
\newcommand{\bea}{\begin{eqnarray}}
\newcommand{\eea}{\end{eqnarray}}
\def\E{\mathop{\rm E\,}\nolimits}
\def\Var{\mathop{\rm Var\,}\nolimits}
\def\Cov{\mathop{\rm Cov\,}\nolimits}
\def\Corr{\mathop{\rm Corr\,}\nolimits}
\def\logit{\mathop{\rm logit\,}\nolimits}
\newcommand{\eid}{{\stackrel{\cal{D}}{=}}}
\newcommand{\cip}{{\stackrel{{P}}{\to}}}
\def\bR{\mathbb{R}}	% real line
\def\given{\,|\,}

%%%%%%%%%%
% From http://www.disc-conference.org/disc1998/mirror/llncs.sty
\def\vec#1{\mathchoice{\mbox{\boldmath$\displaystyle\bf#1$}}
{\mbox{\boldmath$\textstyle\bf#1$}}
{\mbox{\boldmath$\scriptstyle\bf#1$}}
{\mbox{\boldmath$\scriptscriptstyle\bf#1$}}}
%%%%%%%%%%

\def\choice#1#2{
\tikz {
\def \lx{-.2} \def \ux{.2} \def \ly{-.2} \def\uy{.2} 
\filldraw[color=#1] (\lx*.9,\ly*.9) rectangle (\ux*.9,\uy*.9);
\draw(\lx,\ly) rectangle (\ux,\uy); \node at (0,0)[color=white] {\bf {#2}}; \node at (0,0) [color=black] {{#2}};
} }

\def\question#1#2#3#4#5{{#1}

\begin{minipage}{2.45in}
\begin{enumerate}
\item [{\choice{red}{A}}]\rule{-.25em}{0em}{#2}
\item [{\choice{yellow}{C}}]\rule{-.25em}{0em}{#4}
\end{enumerate}
%\choice{red}{A} {#2}
%
%\choice{yellow}{C} {#4}
\end{minipage}\begin{minipage}{2.45in}
\begin{enumerate}
\item [{\choice{green}{B}}]\rule{-.25em}{0em}{#3}
\item [{\choice{blue}{D}}]\rule{-.25em}{0em}{#5}
\end{enumerate}
%\choice{green}{B} {#3}
%
%\choice{blue}{D}  {#5}
\end{minipage}
}

%
%  Simple macro to declare and use image all at once.
%
\def\figurehere#1#2{\pgfdeclareimage[width={#2}]{{#1}}{figures/#1} \pgfuseimage{{#1}}}

\def\startframe#1{\begin{frame}[t,fragile] \frametitle{\thedate \hfill {#1}} \vspace{-1ex}}


%
%  Declare images
%
%\pgfdeclareimage[width=3in]{geyser}{figures/geyser}

\mode<presentation>
{
%  \setbeamertemplate{background canvas}
  \usetheme{boxes}
  \usecolortheme{blackscreen}

%  \usetheme{default}
%  \setbeamercovered{transparent}
}
%\beamertemplatetransparentcovereddynamic

\usepackage[english]{babel}
\usepackage[latin1]{inputenc}
\usepackage{times}
\usepackage[T1]{fontenc}

\setbeamersize{text margin left=0.1in}
\setbeamersize{text margin right=0.1in}




\begin{document}


\def \thedate{Feb.~29}


%%%%%%%%%%%%%%%%%%%%%%%%%%%%%%%%%%%%%%%%

\startframe{Announcements}
\begin{itemize}
\item 
Midterm exam:  7:00pm tonight, Feb.~29 in 105 Willard.
(NB:  Change of room!)
\item 
HW \#6 will be due on Friday (Mar.~2) at 2:30.
\end{itemize}
\end{frame}

\startframe{6.2 Continuous-Time Markov chains}
Given a continuous-time, discrete-space MC, let us define
$P(t)$ to be the implied transition probability matrix at
time $t$.  What properties can we establish for $P(t)$?
(NB:  Not easy to compute $P(t)$ in general.)
\begin{itemize}
\item $P(0) = ??$
\item $\sum_j P_{ij}(t) = ??$
\item $P(s)P(t) = ??$
\end{itemize}
\end{frame}

\startframe{6.8 Computing the Transition Probabilities}
The generator (or rate) matrix $R$:
\begin{itemize}
\item Take $r_{ij} = \mbox{rate of $i\to j$ transitions}$, $i\ne j$.
\item Take $r_{ii} = -\sum_{j\ne i} r_{ij}$.
\item This implies $\sum_{j} r_{ij} = \ldots$
\end{itemize}
What is $R$ for the on-off process?  A homogeneous Poisson process?
\end{frame}

\startframe{6.4 The Transition Probability Function}
The entries of $R$ were introduced much earlier (but not in this form).
\begin{itemize}
\item Take $q_{ij} = \mbox{rate of $i\to j$ transitions}$, $i\ne j$.
\item Take $v_{i}$ so that $P_{ij}=q_{ij}/v_{i}$.
\item Clearly this implies $v_i=\sum_{j\ne i} q_{ij}$.
\end{itemize}
\end{frame}

\startframe{6.4 The Transition Probability Function}
Lemma:
\begin{itemize}
\item $\lim_{h\to0} \frac{1-P_{ii}(h)}{h} = v_i$
\item $\lim_{h\to 0} \frac{P_{ij}(h)}{h} = q_{ij}$
\end{itemize}
\end{frame}

%%%%%%%%%%%%%%%%%%%%%%%%%%%%%%%%%%%%%%%%
\end{document}


