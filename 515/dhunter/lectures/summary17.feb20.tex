\documentclass[handout]{beamer}
\usepackage{beamerthemeshadow}
\usepackage{amssymb}
\usepackage{pgfpages}
\usepackage{tikz}
\pgfpagesuselayout{4 on 1}%[landscape]

\newcommand{\beaa}{\begin{eqnarray*}}
\newcommand{\eeaa}{\end{eqnarray*}}
\newcommand{\bea}{\begin{eqnarray}}
\newcommand{\eea}{\end{eqnarray}}
\def\E{\mathop{\rm E\,}\nolimits}
\def\Var{\mathop{\rm Var\,}\nolimits}
\def\Cov{\mathop{\rm Cov\,}\nolimits}
\def\Corr{\mathop{\rm Corr\,}\nolimits}
\def\logit{\mathop{\rm logit\,}\nolimits}
\newcommand{\eid}{{\stackrel{\cal{D}}{=}}}
\newcommand{\cip}{{\stackrel{{P}}{\to}}}
\def\bR{\mathbb{R}}	% real line
\def\given{\,|\,}

%%%%%%%%%%
% From http://www.disc-conference.org/disc1998/mirror/llncs.sty
\def\vec#1{\mathchoice{\mbox{\boldmath$\displaystyle\bf#1$}}
{\mbox{\boldmath$\textstyle\bf#1$}}
{\mbox{\boldmath$\scriptstyle\bf#1$}}
{\mbox{\boldmath$\scriptscriptstyle\bf#1$}}}
%%%%%%%%%%

\def\choice#1#2
{\tikz 
{\def \lx{-.2} \def \ux{.2} \def \ly{-.2} \def\uy{.2} \filldraw[color=#1] (\lx*.9,\ly*.9) rectangle (\ux*.9,\uy*.9);
\draw(\lx,\ly) rectangle (\ux,\uy); \node at (0,0)[color=white] {\bf {#2}}; \node at (0,0) [color=black] {{#2}};}}

\def\question#1#2#3#4#5{{#1}

\begin{minipage}{2.4in}
\choice{red}{A} {#2}

\choice{yellow}{C} {#4}
\end{minipage}\begin{minipage}{2.4in}
\choice{green}{B} {#3}

\choice{blue}{D}  {#5}
\end{minipage}
}

\def\notes#1{

\vspace{2ex}
{\em Notes: {#1}}}
%
%  Simple macro to declare and use image all at once.
%
\def\figurehere#1#2{\pgfdeclareimage[width={#2}]{{#1}}{figures/#1} \pgfuseimage{{#1}}}

\def\startframe#1{\begin{frame}[t,fragile] \frametitle{\thedate \hfill {#1}} }

%
%  Declare images
%
%\pgfdeclareimage[width=3in]{geyser}{figures/geyser}

\mode<presentation>
{
%  \setbeamertemplate{background canvas}
  \usetheme{boxes}
  \usecolortheme{notblackscreen}

%  \usetheme{default}
%  \setbeamercovered{transparent}
}
%\beamertemplatetransparentcovereddynamic

\usepackage[english]{babel}
\usepackage[latin1]{inputenc}
\usepackage{times}
\usepackage[T1]{fontenc}

\setbeamersize{text margin left=0.1in}
\setbeamersize{text margin right=0.1in}



\begin{document}


\def \thedate{Feb.~20}


%%%%%%%%%%%%%%%%%%%%%%%%%%%%%%%%%%%%%%%%

\startframe{Announcements}
\begin{itemize}
\item
For Wednesday,
read Sections 6.1 and 6.2.
\item 
HW\#5 is due {\em ELECTRONICALLY} on Friday.
\item 
See example code and .Rnw solutions files for homework
\item
Tentatively:  Midterm on Wednesday, Feb. 29 in evening.
\end{itemize}
\notes{We decided on Wednesday from 7:00-8:00 as the preferred time for the 
midterm, which will cover basically everything we've seen from Chapters 3, 4, and 5.
I will arrange for a classroom and announce it when I have reserved it.}
\end{frame}

\startframe{5.3 The Poisson Process}
Proof of Poisson approximation to binomial:
\begin{itemize}
\item Poisson MGF:  $M(t) = \exp\{\lambda e^t - \lambda\}$
\item Binomial MGF:  $M(t) = (1+pe^t -p)^n$
\item Let $p=\lambda/n$ and see what happens as $n\to\infty$.
\end{itemize}
\notes{Recall that a sequence of distributions converges to a limiting distribution if
and only if the corresponding MGFs converge to the MGF of the limit.}
\end{frame}


\startframe{5.3 The Poisson Process}
\begin{itemize}
\item
Let $S_0=0$ and $S_1, S_2, \ldots$ are the times of events in a Poisson process.
\item
Let $T_i=S_i-S_{i-1}$ for $i=1, 2, \ldots$ be the times between events.
\item 
How are the $T_i$ distributed?
\end{itemize}
\notes{We showed quickly (using stationary and independent increments characteristic
of a Poisson process) that the $T_i$ are i.i.d.~as exponential variables with rate $\lambda$.
The argument for arbitrary $T_i$ uses similar logic to the following:
\[
P(T_1>t) = P(\mbox{no events in $(0, t]$}) = \frac{e^{-\lambda t} \lambda^0}{0!} = e^{-\lambda t}.
\]
}
\end{frame}

\startframe{5.3 The Poisson Process}
\question{In a Poisson process with rate $\lambda$, what is the distribution of
the waiting time until the $n$th event?}
{Exponential}
{Gamma}
{Lognormal}
{Uniform}
\notes{Correct answer:  B.  Roughly 75\% of the class got this.  It was merely a check
to see whether you remembered that the sum of i.i.d.~exponential variables is gamma.
Of course, when $n=1$ then A is also correct.}
\end{frame}

\startframe{5.3 The Poisson Process}
\question{Let $X_1, \ldots, X_n$ be independent exponential random variables
with rates $\lambda_1, \ldots, \lambda_n$, respectively, and $X_{(1)}=\min_{1\le i\le n} X_i$.
What is $P(X_1=X_{(1)})$?}
{$\lambda_1$}
{$\lambda_2+\cdots + \lambda_n$}
{$\lambda_1/(\lambda_1+\cdots + \lambda_n)$}
{$1/(\lambda_2+\cdots + \lambda_n)$}
\notes{Correct answer:  C.  We took a couple minutes in class to think about this one, 
and most got it, possibly with some discussion with neighbors.  We verified quickly that C
is correct using results we have seen already regarding independent exponential
variables.}
\end{frame}

\startframe{5.3 The Poisson Process}
Poisson processes add:
\begin{itemize}
\item
If $N_1(t)$ and $N_2(t)$ are Poisson processes with rates $\lambda_1$ and $\lambda_2$, 
then $M(t) = N_1(t)+N_2(t)$ is a Poisson process with rate $\lambda_1+\lambda_2$.
\item
Similarly, suppose we label each event $M(t)$ as type-1 or type-2 with probabilities
$\lambda_1/(\lambda_1+\lambda_2)$ and 
$\lambda_2/(\lambda_1+\lambda_2)$, respectively.  Then the counting process for each
type of event is separately a Poisson process.
\end{itemize}
\notes{
This is a very important property of Poisson processes, and it is related to the fact on the previous
slide, as we discussed in class. 
}
\end{frame}

\startframe{5.4.1 Nonhomogeneous Poisson Processes}
Notice a slight change from the homogeneous version:
\begin{itemize}
\item Counts of events in disjoint time intervals are independent.
\item $P[N(s+h)-N(s)=1] = \lambda(s) h + o(h)$ as $h\to 0$.
\item $P[N(s+h)-N(s)\ge2] = o(h)$ as $h\to 0$.
\item Technically, we should also say $N(0)=0$.
\end{itemize}
\notes{
We did not discuss this in detail---we will do so in the next class---but here,
the only thing that has changed from the homogeneous Poisson process is the
fact that $\lambda h$ has been replaced by $\lambda(s) h$, which means that
we are no longer assuming stationarity of the process.
}
\end{frame}


%%%%%%%%%%%%%%%%%%%%%%%%%%%%%%%%%%%%%%%%
\end{document}


