\documentclass[handout]{beamer}
\usepackage{beamerthemeshadow}
\usepackage{amssymb}
\usepackage{pgfpages}
\usepackage{tikz}
\pgfpagesuselayout{4 on 1}%[landscape]

\newcommand{\beaa}{\begin{eqnarray*}}
\newcommand{\eeaa}{\end{eqnarray*}}
\newcommand{\bea}{\begin{eqnarray}}
\newcommand{\eea}{\end{eqnarray}}
\def\E{\mathop{\rm E\,}\nolimits}
\def\Var{\mathop{\rm Var\,}\nolimits}
\def\Cov{\mathop{\rm Cov\,}\nolimits}
\def\Corr{\mathop{\rm Corr\,}\nolimits}
\def\logit{\mathop{\rm logit\,}\nolimits}
\newcommand{\eid}{{\stackrel{\cal{D}}{=}}}
\newcommand{\cip}{{\stackrel{{P}}{\to}}}
\def\bR{\mathbb{R}}	% real line
\def\given{\,|\,}

%%%%%%%%%%
% From http://www.disc-conference.org/disc1998/mirror/llncs.sty
\def\vec#1{\mathchoice{\mbox{\boldmath$\displaystyle\bf#1$}}
{\mbox{\boldmath$\textstyle\bf#1$}}
{\mbox{\boldmath$\scriptstyle\bf#1$}}
{\mbox{\boldmath$\scriptscriptstyle\bf#1$}}}
%%%%%%%%%%

\def\choice#1#2
{\tikz 
{\def \lx{-.2} \def \ux{.2} \def \ly{-.2} \def\uy{.2} \filldraw[color=#1] (\lx*.9,\ly*.9) rectangle (\ux*.9,\uy*.9);
\draw(\lx,\ly) rectangle (\ux,\uy); \node at (0,0)[color=white] {\bf {#2}}; \node at (0,0) [color=black] {{#2}};}}

\def\question#1#2#3#4#5{{#1}

\begin{minipage}{2.4in}
\choice{red}{A} {#2}

\choice{yellow}{C} {#4}
\end{minipage}\begin{minipage}{2.4in}
\choice{green}{B} {#3}

\choice{blue}{D}  {#5}
\end{minipage}
}

\def\notes#1{

\vspace{2ex}
{\em Notes: {#1}}}
%
%  Simple macro to declare and use image all at once.
%
\def\figurehere#1#2{\pgfdeclareimage[width={#2}]{{#1}}{figures/#1} \pgfuseimage{{#1}}}

\def\startframe#1{\begin{frame}[t,fragile] \frametitle{\thedate \hfill {#1}} }

%
%  Declare images
%
%\pgfdeclareimage[width=3in]{geyser}{figures/geyser}

\mode<presentation>
{
%  \setbeamertemplate{background canvas}
  \usetheme{boxes}
  \usecolortheme{notblackscreen}

%  \usetheme{default}
%  \setbeamercovered{transparent}
}
%\beamertemplatetransparentcovereddynamic

\usepackage[english]{babel}
\usepackage[latin1]{inputenc}
\usepackage{times}
\usepackage[T1]{fontenc}

\setbeamersize{text margin left=0.1in}
\setbeamersize{text margin right=0.1in}



\begin{document}


\def \thedate{Feb.~15}


%%%%%%%%%%%%%%%%%%%%%%%%%%%%%%%%%%%%%%%%

\startframe{Announcements}
\begin{itemize}
\item
For Friday,
read the parts of Section 5.3 that you haven't yet finished.
\item 
Homework due this Friday; regular office hours Thursday.
\item
The midterm will be \ldots ???
\item 
The final will be Monday, Apr. 30 at 8:00am.
\end{itemize}
\notes{There may have been a misunderstanding about the doodle poll, so
I asked that everyone edit their previous responses to see whether we can find a
midterm time that would acceptable for everyone.}
\end{frame}

\startframe{5.2 The exponential distribution}
Very important fact:  If $X_1$ and $X_2$ are independent 
exponential random variables with rates $\lambda_1$ and
$\lambda_2$, then
\[
P(X_1 < X_2) = \frac{\lambda_1}{\lambda_1 + \lambda_2}
\]
{\em Where did we use the independence?}
\notes{Answer:  We relied heavily on the fact that $X_2$ and $X_2\mid X_1$ have the
same distribution!  This question led to some good discussion.  Among other things, we 
noticed that
\[
P(X_1<X_2 \mid X_1) \quad\mbox{cannot equal}\quad P(X_1 < X_2 \mid X_1=c),
\]
but
\[
P(c<X_2) = X(c<X_2 \mid X_1=c)
\]
is guaranteed by the independence of $X_2$ and $X_1$.
}
\end{frame}

\startframe{5.2 The exponential distribution}
\question{The minimum of a set of $n$ i.i.d.~exponential random variables has what type of distribution?}
{Exponential}
{Lognormal}
{Normal}
{Gamma}
\notes{Answer:  A.  We walked through a proof of this fact, which contains an important
step that is very good to know about:  The probability that the minimum is greater than $m$
is the same as the probability that every one of the variables is greater than $m$.
It is also important to know that the minimum has a rate that is the sum of the rates
of all of the variables.}
\end{frame}

\startframe{5.2 The exponential distribution}
Suppose you arrive at a post office having two clerks at 
a moment when both are busy but nobody else is waiting in line.  
The service times for clerk $i$ are exponential with rate $\lambda_i$.
What is the expected amount of time you have to wait in line?
\notes{The memoryless property tells us that the amount of time each clerk has
already been helping the current customers is irrelevant.  Since you'll join the
next open line, the time until this happens is the minimum of the two times, 
which we know has rate $\lambda_1+\lambda_2$.  So the mean time
is $1/(\lambda_1+\lambda_2)$.}
\end{frame}

\startframe{5.3 The Poisson Process}
Counting processes:
\begin{itemize}
\item $N(t)=$number of ``events'' that occur in $[0, t]$.
\item $N(t)$ is integer-valued and nondecreasing in $t$.
\item Consider an example\ldots
\end{itemize}
\notes{Counting processes are very important stochastic processes.  To
illustrate this point, I showed the first couple of slides from a talk I gave less than a week ago
in which counting processes figure prominently.  If you're interested, the full slides can be found
at \url{http://sites.stat.psu.edu/~dhunter/talks/temple2012_networks.pdf}.}
\end{frame}

\startframe{5.3 The Poisson Process}
A Poisson process is a special type of counting process:
\begin{itemize}
\item Counts of events in disjoint time intervals are independent.
\item For any $0\le s<t$, the number of events in $(s,t]$ is Poisson with parameter
$\lambda(t-s)$.
\item Technically, we should also say $N(0)=0$.
\end{itemize}
\notes{This definition, which basically says that a Poisson process is defined by independent
increments and Poisson increments, is fairly intuitive, but in Friday's class we will see another
equivalent definition that makes the Poisson connection less obvious but that arguably makes the
process more obviously applicable.}
\end{frame}



%%%%%%%%%%%%%%%%%%%%%%%%%%%%%%%%%%%%%%%%
\end{document}


