\documentclass[handout]{beamer}
\usepackage{beamerthemeshadow}
\usepackage{amssymb}
\usepackage{pgfpages}
\usepackage{tikz}
\pgfpagesuselayout{4 on 1}%[landscape]

\newcommand{\beaa}{\begin{eqnarray*}}
\newcommand{\eeaa}{\end{eqnarray*}}
\newcommand{\bea}{\begin{eqnarray}}
\newcommand{\eea}{\end{eqnarray}}
\def\E{\mathop{\rm E\,}\nolimits}
\def\Var{\mathop{\rm Var\,}\nolimits}
\def\Cov{\mathop{\rm Cov\,}\nolimits}
\def\Corr{\mathop{\rm Corr\,}\nolimits}
\def\logit{\mathop{\rm logit\,}\nolimits}
\newcommand{\eid}{{\stackrel{\cal{D}}{=}}}
\newcommand{\cip}{{\stackrel{{P}}{\to}}}
\def\bR{\mathbb{R}}	% real line
\def\given{\,|\,}

%%%%%%%%%%
% From http://www.disc-conference.org/disc1998/mirror/llncs.sty
\def\vec#1{\mathchoice{\mbox{\boldmath$\displaystyle\bf#1$}}
{\mbox{\boldmath$\textstyle\bf#1$}}
{\mbox{\boldmath$\scriptstyle\bf#1$}}
{\mbox{\boldmath$\scriptscriptstyle\bf#1$}}}
%%%%%%%%%%

\def\choice#1#2
{\tikz 
{\def \lx{-.2} \def \ux{.2} \def \ly{-.2} \def\uy{.2} \filldraw[color=#1] (\lx*.9,\ly*.9) rectangle (\ux*.9,\uy*.9);
\draw(\lx,\ly) rectangle (\ux,\uy); \node at (0,0)[color=white] {\bf {#2}}; \node at (0,0) [color=black] {{#2}};}}

\def\question#1#2#3#4#5{{#1}

\begin{minipage}{2.4in}
\choice{red}{A} {#2}

\choice{yellow}{C} {#4}
\end{minipage}\begin{minipage}{2.4in}
\choice{green}{B} {#3}

\choice{blue}{D}  {#5}
\end{minipage}
}

\def\notes#1{

\vspace{2ex}
{\em Notes: {#1}}}
%
%  Simple macro to declare and use image all at once.
%
\def\figurehere#1#2{\pgfdeclareimage[width={#2}]{{#1}}{figures/#1} \pgfuseimage{{#1}}}

\def\startframe#1{\begin{frame}[t,fragile] \frametitle{\thedate \hfill {#1}} }

%
%  Declare images
%
%\pgfdeclareimage[width=3in]{geyser}{figures/geyser}

\mode<presentation>
{
%  \setbeamertemplate{background canvas}
  \usetheme{boxes}
  \usecolortheme{notblackscreen}

%  \usetheme{default}
%  \setbeamercovered{transparent}
}
%\beamertemplatetransparentcovereddynamic

\usepackage[english]{babel}
\usepackage[latin1]{inputenc}
\usepackage{times}
\usepackage[T1]{fontenc}

\setbeamersize{text margin left=0.1in}
\setbeamersize{text margin right=0.1in}



\begin{document}


\def \thedate{Apr.~13}


%%%%%%%%%%%%%%%%%%%%%%%%%%%%%%%%%%%%%%%%

\startframe{Announcements}
\begin{itemize}
\item 
Review Xiaotian's comments from HW \#9.
\item 
Only 1 more homework!  (Only best 10 out of 11 grades will count.)
\item
Tentative take-home final plan:  The exam will be available on Saturday, Apr.~21 and due
at 5:00pm on Wednesday, May~2.
\end{itemize}
\end{frame}


\startframe{Markov chain Monte Carlo}
What do we mean by the transition ``matrix'' $P(x,y)$?
\begin{itemize}
\item For continuous-state case, use instead $P(x, A)$.
\item For {\it Markov transition density} $k(y\mid x)$, 
\[
P(x,A) = \int_A k(u\mid x) \, du.
\]
\end{itemize}
\notes{
Technically, the sets $A\subset \Omega$ have to be measurable; however, this
is not a detail we will focus on in 515.
}
\end{frame}

\startframe{Markov chain Monte Carlo}
For continuous-state case,
\begin{itemize}
\item A stationary {\em density} $\pi(x)$ satisfies
\[
\pi(y) = \int k(y \mid x) \pi(x)\, dx \quad\mbox{for all $y$}
\]
\item Notice analogy to $\pi_j = \sum_i P_{ij} \pi_i$
\end{itemize}
\notes{In Chapter 4, the state space was discrete and so we
dealt with probability mass functions, transition probability matrices, and
sums.  Now, the state space is continuous and so we have densities, transition
densities, and integrals.}
\end{frame}

\startframe{Markov chain Monte Carlo}
If our continuous-time Markov chain with stationary $\pi$ is:
\begin{itemize}
\item $\pi$-irreducible,
\item Aperiodic,
\item Harris recurrent,
\end{itemize}
Then $\| P^n(x, \cdot) - \pi(\cdot) \| \to 0$ for all $x$ and 
$\frac1n\sum_i g(X_i)\stackrel{\rm as}{\to} E_\pi g(X)$.
\notes{
The first convergence is convergence in total variation distance; another
way to write it is
\[
\sup_A | P^n(x, A) - \pi(A) | \to 0 
\]
for all $x$ as $n\to\infty$.  For the second convergence, we need the 
integrability condition $E_\pi |g(X)| <\infty$.
}
\end{frame}

\startframe{Markov chain Monte Carlo}
For concise definitions of\ldots
\begin{itemize}
\item $\pi$-irreducible
\item Aperiodic
\item Harris recurrent
\end{itemize}
\ldots see p.~1711 of Tierney (1994, {\it Annals of Statistics})
\notes{
We actually looked at this page of the Tierney paper, which is a very highly cited paper.
}
\end{frame}

\startframe{Markov chain Monte Carlo}
What about burnin?
\begin{itemize}
\item What is it?  
\item Is it necessary?
\end{itemize}
Check out Jones and Hobert (2001, {\it Stat. Sci.})
\notes{
We have not yet discussed this slide, but we did look at the Jones and Hobert paper,
so I'm including this slide in order to give the reference.
}
\end{frame}

%%%%%%%%%%%%%%%%%%%%%%%%%%%%%%%%%%%%%%%%
\end{document}




