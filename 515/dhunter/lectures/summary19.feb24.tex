\documentclass[handout]{beamer}
\usepackage{beamerthemeshadow}
\usepackage{amssymb}
\usepackage{pgfpages}
\usepackage{tikz}
\pgfpagesuselayout{4 on 1}%[landscape]

\newcommand{\beaa}{\begin{eqnarray*}}
\newcommand{\eeaa}{\end{eqnarray*}}
\newcommand{\bea}{\begin{eqnarray}}
\newcommand{\eea}{\end{eqnarray}}
\def\E{\mathop{\rm E\,}\nolimits}
\def\Var{\mathop{\rm Var\,}\nolimits}
\def\Cov{\mathop{\rm Cov\,}\nolimits}
\def\Corr{\mathop{\rm Corr\,}\nolimits}
\def\logit{\mathop{\rm logit\,}\nolimits}
\newcommand{\eid}{{\stackrel{\cal{D}}{=}}}
\newcommand{\cip}{{\stackrel{{P}}{\to}}}
\def\bR{\mathbb{R}}	% real line
\def\given{\,|\,}

%%%%%%%%%%
% From http://www.disc-conference.org/disc1998/mirror/llncs.sty
\def\vec#1{\mathchoice{\mbox{\boldmath$\displaystyle\bf#1$}}
{\mbox{\boldmath$\textstyle\bf#1$}}
{\mbox{\boldmath$\scriptstyle\bf#1$}}
{\mbox{\boldmath$\scriptscriptstyle\bf#1$}}}
%%%%%%%%%%

\def\choice#1#2
{\tikz 
{\def \lx{-.2} \def \ux{.2} \def \ly{-.2} \def\uy{.2} \filldraw[color=#1] (\lx*.9,\ly*.9) rectangle (\ux*.9,\uy*.9);
\draw(\lx,\ly) rectangle (\ux,\uy); \node at (0,0)[color=white] {\bf {#2}}; \node at (0,0) [color=black] {{#2}};}}

\def\question#1#2#3#4#5{{#1}

\begin{minipage}{2.4in}
\choice{red}{A} {#2}

\choice{yellow}{C} {#4}
\end{minipage}\begin{minipage}{2.4in}
\choice{green}{B} {#3}

\choice{blue}{D}  {#5}
\end{minipage}
}

\def\notes#1{

\vspace{2ex}
{\em Notes: {#1}}}
%
%  Simple macro to declare and use image all at once.
%
\def\figurehere#1#2{\pgfdeclareimage[width={#2}]{{#1}}{figures/#1} \pgfuseimage{{#1}}}

\def\startframe#1{\begin{frame}[t,fragile] \frametitle{\thedate \hfill {#1}} }

%
%  Declare images
%
%\pgfdeclareimage[width=3in]{geyser}{figures/geyser}

\mode<presentation>
{
%  \setbeamertemplate{background canvas}
  \usetheme{boxes}
  \usecolortheme{notblackscreen}

%  \usetheme{default}
%  \setbeamercovered{transparent}
}
%\beamertemplatetransparentcovereddynamic

\usepackage[english]{babel}
\usepackage[latin1]{inputenc}
\usepackage{times}
\usepackage[T1]{fontenc}

\setbeamersize{text margin left=0.1in}
\setbeamersize{text margin right=0.1in}



\begin{document}


\def \thedate{Feb.~24}


%%%%%%%%%%%%%%%%%%%%%%%%%%%%%%%%%%%%%%%%

\startframe{Announcements}
\begin{itemize}
\item 
For Monday, read Section 6.3.
\item 
Midterm exam:  7:00pm on Wednesday, Feb.~29 in 105 Willard.
(NB:  Change of room!)
\item 
HW \#6 will be due next Friday (Mar.~2) at 2:30.
\end{itemize}
\notes{The topics in Chapter 6 are somewhat jumbled relative to the order in which I'll cover them.}
\end{frame}


\startframe{Midterm practice}
\question{Let $X_1$ and $X_2$ be independent exponential random variables with
rates $\lambda_1$ and $\lambda_2$, and let $X_{(1)}$, $X_{(2)}$ be the
order statistics.  What is $\Cov (X_{(1)}, X_{(2)})$?}
{$(\lambda_1+\lambda_2)^2$}
{$\lambda_1\lambda_2$}
{$1/(\lambda_1+\lambda_2)^2$}
{$1/(\lambda_1\lambda_2)$}
\notes{Answer:  C.  Nearly everyone got this (though it was on the homework!).
We discussed the fact that finding $\Cov(X_{(1)}, X_{(2)})$ is not
hard using a direct approach, since it is easy to evaluate 
$E(X_{(1)} X_{(2)})$.}
\end{frame}

\startframe{6.2 Continuous-Time Markov chains}
Consider a continuous-time, discrete-space MC, $\{X(t) : t\in[0,\infty)\}$:
\begin{itemize}
\item The chain stays in each state for a random (real-valued) time.
\item When the chain switches, it picks from the possible states according to
some discrete-time Markov chain.
\item Markov property:  
Conditional on the present, what happens next is independent of the past.
\item Homogeneity:  The chain's behavior at time $t$ does not
depend on $t$.
\end{itemize}
\notes{This is a repeat of a slide shown on Wednesday.  
We will show that the Markov property plus homogeneity
imply together that the times in each state have the memoryless
property.}
\end{frame}

\startframe{6.2 Continuous-Time Markov chains}
Let $T_i$ be the time when the MC leaves state $i$,
given that $X(0)=i$.
\begin{itemize}
\item How is $T_i$ distributed?
\item Claim:  $T_i$ has the memoryless property.
(What does this imply?)
\end{itemize}
\notes{We proved the latter fact, which is not difficult.  Since
the exponential distribution is the only continuous distribution 
with the memoryless property, we conclude that
$T_i$ is exponentially distributed for each $i$.}
\end{frame}

\startframe{6.2 Continuous-Time Markov chains}
Examples:
\begin{itemize}
\item Any Poisson process
\item On-off process
\item Continuous-time random walk
\end{itemize}
\notes{For the Poisson process, the state space is $\{0, 1, 2, \ldots \}$
and we may write $P_{ij} = I{j=i+1}$, where $P$ denotes the
transition probability matrix for moving from the current to the next
state of the Markov chain.
For the on-off process, we have the simplistic
\[
P=\begin{bmatrix}
0 & 1 \\ 1 & 0
\end{bmatrix}.
\]
}
\end{frame}


%%%%%%%%%%%%%%%%%%%%%%%%%%%%%%%%%%%%%%%%
\end{document}


