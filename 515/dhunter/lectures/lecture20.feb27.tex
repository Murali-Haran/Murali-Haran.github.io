\documentclass{beamer}
\usepackage{beamerthemeshadow}
\usepackage{amssymb}
\usepackage{tikz}

\newcommand{\beaa}{\begin{eqnarray*}}
\newcommand{\eeaa}{\end{eqnarray*}}
\newcommand{\bea}{\begin{eqnarray}}
\newcommand{\eea}{\end{eqnarray}}
\def\E{\mathop{\rm E\,}\nolimits}
\def\Var{\mathop{\rm Var\,}\nolimits}
\def\Cov{\mathop{\rm Cov\,}\nolimits}
\def\Corr{\mathop{\rm Corr\,}\nolimits}
\def\logit{\mathop{\rm logit\,}\nolimits}
\newcommand{\eid}{{\stackrel{\cal{D}}{=}}}
\newcommand{\cip}{{\stackrel{{P}}{\to}}}
\def\bR{\mathbb{R}}	% real line
\def\given{\,|\,}

%%%%%%%%%%
% From http://www.disc-conference.org/disc1998/mirror/llncs.sty
\def\vec#1{\mathchoice{\mbox{\boldmath$\displaystyle\bf#1$}}
{\mbox{\boldmath$\textstyle\bf#1$}}
{\mbox{\boldmath$\scriptstyle\bf#1$}}
{\mbox{\boldmath$\scriptscriptstyle\bf#1$}}}
%%%%%%%%%%

\def\choice#1#2{
\tikz {
\def \lx{-.2} \def \ux{.2} \def \ly{-.2} \def\uy{.2} 
\filldraw[color=#1] (\lx*.9,\ly*.9) rectangle (\ux*.9,\uy*.9);
\draw(\lx,\ly) rectangle (\ux,\uy); \node at (0,0)[color=white] {\bf {#2}}; \node at (0,0) [color=black] {{#2}};
} }

\def\question#1#2#3#4#5{{#1}

\begin{minipage}{2.45in}
\begin{enumerate}
\item [{\choice{red}{A}}]\rule{-.25em}{0em}{#2}
\item [{\choice{yellow}{C}}]\rule{-.25em}{0em}{#4}
\end{enumerate}
%\choice{red}{A} {#2}
%
%\choice{yellow}{C} {#4}
\end{minipage}\begin{minipage}{2.45in}
\begin{enumerate}
\item [{\choice{green}{B}}]\rule{-.25em}{0em}{#3}
\item [{\choice{blue}{D}}]\rule{-.25em}{0em}{#5}
\end{enumerate}
%\choice{green}{B} {#3}
%
%\choice{blue}{D}  {#5}
\end{minipage}
}

%
%  Simple macro to declare and use image all at once.
%
\def\figurehere#1#2{\pgfdeclareimage[width={#2}]{{#1}}{figures/#1} \pgfuseimage{{#1}}}

\def\startframe#1{\begin{frame}[t,fragile] \frametitle{\thedate \hfill {#1}} \vspace{-1ex}}


%
%  Declare images
%
%\pgfdeclareimage[width=3in]{geyser}{figures/geyser}

\mode<presentation>
{
%  \setbeamertemplate{background canvas}
  \usetheme{boxes}
  \usecolortheme{blackscreen}

%  \usetheme{default}
%  \setbeamercovered{transparent}
}
%\beamertemplatetransparentcovereddynamic

\usepackage[english]{babel}
\usepackage[latin1]{inputenc}
\usepackage{times}
\usepackage[T1]{fontenc}

\setbeamersize{text margin left=0.1in}
\setbeamersize{text margin right=0.1in}




\begin{document}


\def \thedate{Feb.~27}


%%%%%%%%%%%%%%%%%%%%%%%%%%%%%%%%%%%%%%%%

\startframe{Announcements}
\begin{itemize}
\item 
For Wednesday, work old homework problems and new homework problems.
\item 
Midterm exam:  7:00pm on Wednesday, Feb.~29 in 105 Willard.
(NB:  Change of room!)
\item 
HW \#6 will be due on Friday (Mar.~2) at 2:30.
\end{itemize}
\end{frame}


\startframe{Midterm practice}
\question{
Let $X$ be an exponential random variable.  Without any 
computations, which of the following is correct?  
}
{$E[X^2 \mid X>1] = E[(X+1)^2]$}
{$E[X^2 \mid X>1] = E[X^2]+1$}
{$E[X^2 \mid X>1] = (1+E[X])^2$}
{None of these}
\end{frame}

\startframe{Midterm practice}
\question{A doctor has scheduled two appointments, one at 1:00 and one at 1:30.
The amounts of time that appointments last are independent exponential
random variables with mean 30 minutes.  If both patients are on time, the expected 
amount of time that the 1:30 appointment spends at the doctor's office is\ldots}
{Less than 30 minutes}
{Between 30 and 60 minutes}
{More than 60 minutes}
{}
\end{frame}

\startframe{Midterm practice}
In a two-server queueing system, customers arrive according to a Poisson
process with rate $\lambda$.  If at least one of the two servers is free when
a customer arrives, the customer is immediately served; otherwise, the customer
departs immediately and is lost.  Service times are exponential (independently)
with rate $\mu$.
\end{frame}

\startframe{Midterm practice}
In the two-server system (customer rate=$\lambda$, service rate=$\mu$),
\begin{itemize}
\item Starting from both servers busy, what is the expected time until the next
customer is served?
\item Starting empty, find the expected time until both servers are busy.
\item What is the expected time between two successive lost customers?
\end{itemize}
\end{frame}

\startframe{Midterm practice}
Consider a single server queueing system where customers arrive according to a
Poisson process with rate $\lambda$, service times are exponential with rate $\mu$,
and customers are served in the order of their arrival.  If a customer arrives and finds
$n-1$ others in the system, and $X$ is the number in the system when that
customer departs, find the pmf of $X$.
\end{frame}

\startframe{6.2 Continuous-Time Markov chains}
Given a continuous-time, discrete-space MC, let us define
$P(t)$ to be the implied transition probability matrix at
time $t$.  What properties can we establish for $P(t)$?
(NB:  Not easy to compute $P(t)$ in general.)
\begin{itemize}
\item $P(0) = ??$
\item $\sum_j P_{ij}(t) = ??$
\item $P(s)P(t) = ??$
\end{itemize}
\end{frame}

\startframe{6.8 Computing the Transition Probabilities}
The generator (or rate) matrix $R$:
\begin{itemize}
\item Take $r_{ij} = \mbox{rate of $i\to j$ transitions}$, $i\ne j$.
\item Take $r_{ii} = -\sum_{j\ne i} r_{ij}$.
\item This implies $\sum_{j} r_{ij} = \ldots$
\end{itemize}
What is $R$ for the on-off process?
\end{frame}

%%%%%%%%%%%%%%%%%%%%%%%%%%%%%%%%%%%%%%%%
\end{document}


