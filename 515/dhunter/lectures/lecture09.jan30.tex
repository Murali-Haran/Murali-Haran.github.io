\documentclass{beamer}
\usepackage{beamerthemeshadow}
\usepackage{amssymb}
\usepackage{tikz}

\newcommand{\beaa}{\begin{eqnarray*}}
\newcommand{\eeaa}{\end{eqnarray*}}
\newcommand{\bea}{\begin{eqnarray}}
\newcommand{\eea}{\end{eqnarray}}
\def\E{\mathop{\rm E\,}\nolimits}
\def\Var{\mathop{\rm Var\,}\nolimits}
\def\Cov{\mathop{\rm Cov\,}\nolimits}
\def\Corr{\mathop{\rm Corr\,}\nolimits}
\def\logit{\mathop{\rm logit\,}\nolimits}
\newcommand{\eid}{{\stackrel{\cal{D}}{=}}}
\newcommand{\cip}{{\stackrel{{P}}{\to}}}
\def\bR{\mathbb{R}}	% real line
\def\given{\,|\,}

%%%%%%%%%%
% From http://www.disc-conference.org/disc1998/mirror/llncs.sty
\def\vec#1{\mathchoice{\mbox{\boldmath$\displaystyle\bf#1$}}
{\mbox{\boldmath$\textstyle\bf#1$}}
{\mbox{\boldmath$\scriptstyle\bf#1$}}
{\mbox{\boldmath$\scriptscriptstyle\bf#1$}}}
%%%%%%%%%%

\def\choice#1#2{
\tikz {
\def \lx{-.2} \def \ux{.2} \def \ly{-.2} \def\uy{.2} 
\filldraw[color=#1] (\lx*.9,\ly*.9) rectangle (\ux*.9,\uy*.9);
\draw(\lx,\ly) rectangle (\ux,\uy); \node at (0,0)[color=white] {\bf {#2}}; \node at (0,0) [color=black] {{#2}};
} }

\def\question#1#2#3#4#5{{#1}

\begin{minipage}{2.45in}
\begin{enumerate}
\item [{\choice{red}{A}}]\rule{-.25em}{0em}{#2}
\item [{\choice{yellow}{C}}]\rule{-.25em}{0em}{#4}
\end{enumerate}
%\choice{red}{A} {#2}
%
%\choice{yellow}{C} {#4}
\end{minipage}\begin{minipage}{2.45in}
\begin{enumerate}
\item [{\choice{green}{B}}]\rule{-.25em}{0em}{#3}
\item [{\choice{blue}{D}}]\rule{-.25em}{0em}{#5}
\end{enumerate}
%\choice{green}{B} {#3}
%
%\choice{blue}{D}  {#5}
\end{minipage}
}

%
%  Simple macro to declare and use image all at once.
%
\def\figurehere#1#2{\pgfdeclareimage[width={#2}]{{#1}}{figures/#1} \pgfuseimage{{#1}}}

\def\startframe#1{\begin{frame}[t,fragile] \frametitle{\thedate \hfill {#1}} \vspace{-1ex}}


%
%  Declare images
%
%\pgfdeclareimage[width=3in]{geyser}{figures/geyser}

\mode<presentation>
{
%  \setbeamertemplate{background canvas}
  \usetheme{boxes}
  \usecolortheme{blackscreen}

%  \usetheme{default}
%  \setbeamercovered{transparent}
}
%\beamertemplatetransparentcovereddynamic

\usepackage[english]{babel}
\usepackage[latin1]{inputenc}
\usepackage{times}
\usepackage[T1]{fontenc}

\setbeamersize{text margin left=0.1in}
\setbeamersize{text margin right=0.1in}




\begin{document}


\def \thedate{Jan.~30}


%%%%%%%%%%%%%%%%%%%%%%%%%%%%%%%%%%%%%%%%

\startframe{Announcements}
\begin{itemize}
\item 
Read Sections 4.6 through 4.8 (both editions)
\item 
HW \#3 should go up later today; it will be due Wednesday, Feb.~8
\item
Solutions for HW \#2 use Sweave (source file is provided)
\item
I will not be in class on:  

Friday, Feb.~10;  Wednesday, Mar.~14; Monday, Mar.~12 (probably)
\end{itemize}
\end{frame}


\startframe{4.3 Classification of States}
State $i$ is recurrent if and only if, conditional on $X_0=i$, 
\begin{itemize}
\item $P(\mbox{ever revisiting $i$}) = 1$
\item $E(\#\{T>0:  X_T = i\}) = \infty$.
\item $\sum_{n=1}^\infty P_{ii}^n = \infty$.
\end{itemize}
\end{frame}

\startframe{4.3 Classification of States}
Theorem:  
Recurrence (or transience) is a class property.

How do we prove this?
\end{frame}


\startframe{4.3 Classification of States}
Theorem:  
All states of a finite, irreducible Markov chain are recurrent.

How do we prove this?
\end{frame}


\startframe{4.3 Classification of States}
\question{In a Markov chain, is impossible to move:}
{from a transient state to a transient state}
{from a recurrent state to a transient state}
{from a transient state to a recurrent state}
{from a recurrent state to a recurrent state}
\end{frame}

\startframe{Mean Time Spent in Transient States}
Suppose you have \$2 and you bet on fair games of chance
until you either go broke or have \$5.  
\begin{itemize}
\item What is the expected number of time steps that you have \$2?
\item What is the probability that you will at some point have \$1?
\end{itemize}
\end{frame}

\startframe{Mean Time Spent in Transient States}
Let $s_{ij} \stackrel{\rm def}{=}E(\#\{ T: X_T=j \mid X_0=i \})$
\begin{itemize}
\item For which $i$ and $j$ is $s_{ij}$ meaningful?
\item Let us show $S = I + P_TS$ for correctly defined $S$ and $P_T$.
\end{itemize}
\end{frame}

\startframe{Mean Time Spent in Transient States}
Let $f_{ij} \stackrel{\rm def}{=}P(\mbox{$X_t=j$ for some $t\ge0$} \mid X_0=t)$
\begin{itemize}
\item Demonstrate that $s_{ij} = I\{i=j\} + f_{ij} s_{jj}$.
\end{itemize}
\end{frame}


%%%%%%%%%%%%%%%%%%%%%%%%%%%%%%%%%%%%%%%%
\end{document}


