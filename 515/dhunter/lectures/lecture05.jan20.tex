\documentclass{beamer}
\usepackage{beamerthemeshadow}
\usepackage{amssymb}
\usepackage{tikz}

\newcommand{\beaa}{\begin{eqnarray*}}
\newcommand{\eeaa}{\end{eqnarray*}}
\newcommand{\bea}{\begin{eqnarray}}
\newcommand{\eea}{\end{eqnarray}}
\def\E{\mathop{\rm E\,}\nolimits}
\def\Var{\mathop{\rm Var\,}\nolimits}
\def\Cov{\mathop{\rm Cov\,}\nolimits}
\def\Corr{\mathop{\rm Corr\,}\nolimits}
\def\logit{\mathop{\rm logit\,}\nolimits}
\newcommand{\eid}{{\stackrel{\cal{D}}{=}}}
\newcommand{\cip}{{\stackrel{{P}}{\to}}}
\def\bR{\mathbb{R}}	% real line
\def\given{\,|\,}

%%%%%%%%%%
% From http://www.disc-conference.org/disc1998/mirror/llncs.sty
\def\vec#1{\mathchoice{\mbox{\boldmath$\displaystyle\bf#1$}}
{\mbox{\boldmath$\textstyle\bf#1$}}
{\mbox{\boldmath$\scriptstyle\bf#1$}}
{\mbox{\boldmath$\scriptscriptstyle\bf#1$}}}
%%%%%%%%%%

\def\choice#1#2{
\tikz {
\def \lx{-.2} \def \ux{.2} \def \ly{-.2} \def\uy{.2} 
\filldraw[color=#1] (\lx*.9,\ly*.9) rectangle (\ux*.9,\uy*.9);
\draw(\lx,\ly) rectangle (\ux,\uy); \node at (0,0)[color=white] {\bf {#2}}; \node at (0,0) [color=black] {{#2}};
} }

\def\question#1#2#3#4#5{{#1}

\begin{minipage}{2.45in}
\begin{enumerate}
\item [{\choice{red}{A}}]\rule{-.25em}{0em}{#2}
\item [{\choice{yellow}{C}}]\rule{-.25em}{0em}{#4}
\end{enumerate}
%\choice{red}{A} {#2}
%
%\choice{yellow}{C} {#4}
\end{minipage}\begin{minipage}{2.45in}
\begin{enumerate}
\item [{\choice{green}{B}}]\rule{-.25em}{0em}{#3}
\item [{\choice{blue}{D}}]\rule{-.25em}{0em}{#5}
\end{enumerate}
%\choice{green}{B} {#3}
%
%\choice{blue}{D}  {#5}
\end{minipage}
}

%
%  Simple macro to declare and use image all at once.
%
\def\figurehere#1#2{\pgfdeclareimage[width={#2}]{{#1}}{figures/#1} \pgfuseimage{{#1}}}

\def\startframe#1{\begin{frame}[t,fragile] \frametitle{\thedate \hfill {#1}} \vspace{-1ex}}


%
%  Declare images
%
%\pgfdeclareimage[width=3in]{geyser}{figures/geyser}

\mode<presentation>
{
%  \setbeamertemplate{background canvas}
  \usetheme{boxes}
  \usecolortheme{blackscreen}

%  \usetheme{default}
%  \setbeamercovered{transparent}
}
%\beamertemplatetransparentcovereddynamic

\usepackage[english]{babel}
\usepackage[latin1]{inputenc}
\usepackage{times}
\usepackage[T1]{fontenc}

\setbeamersize{text margin left=0.1in}
\setbeamersize{text margin right=0.1in}




\begin{document}


\def \thedate{Jan.~20}


%%%%%%%%%%%%%%%%%%%%%%%%%%%%%%%%%%%%%%%%

\startframe{Announcements}
\begin{itemize}
\item 
Read Sections 4.2 and 4.3 (both editions) before Monday's class.
\item
I encourage you to:
\begin{itemize}
\item Read ahead!
\item Browse Section 3.6 (both editions)
\end{itemize}
\item I have made an addition to the syllabus.
\end{itemize}
\end{frame}


\startframe{3.4 Computing Expectations by Conditioning}
\begin{itemize}
\item Perform repeated independent Bernoulli$(p)$ experiments.
\item Let $N_k$ be the trial number where $k$ consecutive successes are first observed.
\item Recall: $E(N_k) = E \left[ E(N_k \mid N_{k-1}) \right]$
\end{itemize}
\end{frame}

\startframe{3.5 Computing Probabilities by Conditioning}
\begin{itemize}
\item 
Two players alternate flipping a coin that comes up heads with probability $p$.
\item
The first one to get heads is the winner.
\item 
What is the probability that the first player is the winner?
\end{itemize}
\end{frame}

\startframe{4.1 Introduction to Markov Chains}

In chapter 4, ``time'' $t$ is discrete:  $t=0, 1, 2, \ldots$

\vspace{2ex}
 ``Markov'' means:  
Distribution of $X_t$ depends {\em only} on $X_{t-1}$.

{\em (Not on $t$ nor on $X_{t-2}, X_{t-3}, \ldots$.)}

\end{frame}


\startframe{4.1 Introduction to Markov Chains}
Consider the simple weather example with states ``rain'' and ``no rain'':
\[
P=
\begin{bmatrix}
\alpha & 1-\alpha \\
\beta & 1-\beta
\end{bmatrix}
=
\begin{bmatrix}
0.7 & 0.3 \\
0.4 & 0.6
\end{bmatrix}
\]
\end{frame}

\startframe{4.1 Introduction to Markov Chains}
More complicated rain example with four states:

(0) rain, rain; (1) rain, no rain; (2) no rain, rain; (3) no rain, no rain

\[
P=
\begin{bmatrix}
0.7 & 0 & 0.3 & 0 \\
0.5 & 0 & 0.5 & 0 \\
0 & 0.4 & 0 & 0.6 \\
0 & 0.2 & 0 & 0.8
\end{bmatrix}
\]
\end{frame}

\startframe{4.1 Introduction to Markov Chains}
\question{If $P$ is the transition matrix for a finite-state Markov chain 
$X_0, X_1, X_2, \ldots$, and if the $X_i$ are i.i.d., then:}
{Each (horizontal) row of $P$ is the same.}
{Each (vertical) column of $P$ is the same.}
{Each column of $P$ sums to 1.}
{None of A, B, or C.}
\end{frame}




%%%%%%%%%%%%%%%%%%%%%%%%%%%%%%%%%%%%%%%%
\end{document}


