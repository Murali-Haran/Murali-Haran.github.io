\documentclass{beamer}
\usepackage{beamerthemeshadow}
\usepackage{amssymb}
\usepackage{tikz}

\newcommand{\beaa}{\begin{eqnarray*}}
\newcommand{\eeaa}{\end{eqnarray*}}
\newcommand{\bea}{\begin{eqnarray}}
\newcommand{\eea}{\end{eqnarray}}
\def\E{\mathop{\rm E\,}\nolimits}
\def\Var{\mathop{\rm Var\,}\nolimits}
\def\Cov{\mathop{\rm Cov\,}\nolimits}
\def\Corr{\mathop{\rm Corr\,}\nolimits}
\def\logit{\mathop{\rm logit\,}\nolimits}
\newcommand{\eid}{{\stackrel{\cal{D}}{=}}}
\newcommand{\cip}{{\stackrel{{P}}{\to}}}
\def\bR{\mathbb{R}}	% real line
\def\given{\,|\,}

%%%%%%%%%%
% From http://www.disc-conference.org/disc1998/mirror/llncs.sty
\def\vec#1{\mathchoice{\mbox{\boldmath$\displaystyle\bf#1$}}
{\mbox{\boldmath$\textstyle\bf#1$}}
{\mbox{\boldmath$\scriptstyle\bf#1$}}
{\mbox{\boldmath$\scriptscriptstyle\bf#1$}}}
%%%%%%%%%%

\def\choice#1#2{
\tikz {
\def \lx{-.2} \def \ux{.2} \def \ly{-.2} \def\uy{.2} 
\filldraw[color=#1] (\lx*.9,\ly*.9) rectangle (\ux*.9,\uy*.9);
\draw(\lx,\ly) rectangle (\ux,\uy); \node at (0,0)[color=white] {\bf {#2}}; \node at (0,0) [color=black] {{#2}};
} }

\def\question#1#2#3#4#5{{#1}

\begin{minipage}{2.45in}
\begin{enumerate}
\item [{\choice{red}{A}}]\rule{-.25em}{0em}{#2}
\item [{\choice{yellow}{C}}]\rule{-.25em}{0em}{#4}
\end{enumerate}
%\choice{red}{A} {#2}
%
%\choice{yellow}{C} {#4}
\end{minipage}\begin{minipage}{2.45in}
\begin{enumerate}
\item [{\choice{green}{B}}]\rule{-.25em}{0em}{#3}
\item [{\choice{blue}{D}}]\rule{-.25em}{0em}{#5}
\end{enumerate}
%\choice{green}{B} {#3}
%
%\choice{blue}{D}  {#5}
\end{minipage}
}

%
%  Simple macro to declare and use image all at once.
%
\def\figurehere#1#2{\pgfdeclareimage[width={#2}]{{#1}}{figures/#1} \pgfuseimage{{#1}}}

\def\startframe#1{\begin{frame}[t,fragile] \frametitle{\thedate \hfill {#1}} \vspace{-1ex}}


%
%  Declare images
%
%\pgfdeclareimage[width=3in]{geyser}{figures/geyser}

\mode<presentation>
{
%  \setbeamertemplate{background canvas}
  \usetheme{boxes}
  \usecolortheme{blackscreen}

%  \usetheme{default}
%  \setbeamercovered{transparent}
}
%\beamertemplatetransparentcovereddynamic

\usepackage[english]{babel}
\usepackage[latin1]{inputenc}
\usepackage{times}
\usepackage[T1]{fontenc}

\setbeamersize{text margin left=0.1in}
\setbeamersize{text margin right=0.1in}




\begin{document}


\def \thedate{Feb.~15}


%%%%%%%%%%%%%%%%%%%%%%%%%%%%%%%%%%%%%%%%

\startframe{Announcements}
\begin{itemize}
\item
For Friday,
read the parts of Section 5.3 that you haven't yet finished.
\item 
Homework due this Friday; regular office hours Thursday.
\item
The midterm will be \ldots ???
\item 
The final will be Monday, Apr. 30 at 8:00am.
\end{itemize}
\end{frame}

\startframe{5.2 The exponential distribution}
Very important fact:  If $X_1$ and $X_2$ are independent 
exponential random variables with rates $\lambda_1$ and
$\lambda_2$, then
\[
P(X_1 < X_2) = \frac{\lambda_1}{\lambda_1 + \lambda_2}
\]
{\em Where did we use the independence?}
\end{frame}

\startframe{5.2 The exponential distribution}
\question{The minimum of a set of $n$ i.i.d.~exponential random variables has what type of distribution?}
{Exponential}
{Lognormal}
{Normal}
{Gamma}
\end{frame}

\startframe{5.2 The exponential distribution}
Suppose you arrive at a post office having two clerks at 
a moment when both are busy but nobody else is waiting in line.  
The service times for clerk $i$ are exponential with rate $\lambda_i$.
What is the expected amount of time you have to wait in line?
\end{frame}

\startframe{5.3 The Poisson Process}
Counting processes:
\begin{itemize}
\item $N(t)=$number of ``events'' that occur in $[0, t]$.
\item $N(t)$ is integer-valued and nondecreasing in $t$.
\item Consider an example\ldots
\end{itemize}
\end{frame}

\startframe{5.3 The Poisson Process}
A Poisson process is a special type of counting process:
\begin{itemize}
\item Counts of events in disjoint time intervals are independent.
\item For any $0\le s<t$, the number of events in $(s,t]$ is Poisson with parameter
$\lambda(t-s)$.
\item Technically, we should also say $N(0)=0$.
\end{itemize}
\end{frame}

\startframe{5.3 The Poisson Process}
A Poisson process is a special type of counting process
(equivalent re-definition):
\begin{itemize}
\item Counts of events in disjoint time intervals are independent.
\item $P[N(s+h)-N(s)=1] = \lambda h + o(h)$ as $h\to 0$.
\item $P[N(s+h)-N(s)\ge2] = o(h)$ as $h\to 0$.
\item Technically, we should also say $N(0)=0$.
\end{itemize}
\end{frame}


%%%%%%%%%%%%%%%%%%%%%%%%%%%%%%%%%%%%%%%%
\end{document}


