\documentclass[handout]{beamer}
\usepackage{beamerthemeshadow}
\usepackage{amssymb}
\usepackage{pgfpages}
\usepackage{tikz}
\pgfpagesuselayout{4 on 1}%[landscape]

\newcommand{\beaa}{\begin{eqnarray*}}
\newcommand{\eeaa}{\end{eqnarray*}}
\newcommand{\bea}{\begin{eqnarray}}
\newcommand{\eea}{\end{eqnarray}}
\def\E{\mathop{\rm E\,}\nolimits}
\def\Var{\mathop{\rm Var\,}\nolimits}
\def\Cov{\mathop{\rm Cov\,}\nolimits}
\def\Corr{\mathop{\rm Corr\,}\nolimits}
\def\logit{\mathop{\rm logit\,}\nolimits}
\newcommand{\eid}{{\stackrel{\cal{D}}{=}}}
\newcommand{\cip}{{\stackrel{{P}}{\to}}}
\def\bR{\mathbb{R}}	% real line
\def\given{\,|\,}

%%%%%%%%%%
% From http://www.disc-conference.org/disc1998/mirror/llncs.sty
\def\vec#1{\mathchoice{\mbox{\boldmath$\displaystyle\bf#1$}}
{\mbox{\boldmath$\textstyle\bf#1$}}
{\mbox{\boldmath$\scriptstyle\bf#1$}}
{\mbox{\boldmath$\scriptscriptstyle\bf#1$}}}
%%%%%%%%%%

\def\choice#1#2
{\tikz 
{\def \lx{-.2} \def \ux{.2} \def \ly{-.2} \def\uy{.2} \filldraw[color=#1] (\lx*.9,\ly*.9) rectangle (\ux*.9,\uy*.9);
\draw(\lx,\ly) rectangle (\ux,\uy); \node at (0,0)[color=white] {\bf {#2}}; \node at (0,0) [color=black] {{#2}};}}

\def\question#1#2#3#4#5{{#1}

\begin{minipage}{2.4in}
\choice{red}{A} {#2}

\choice{yellow}{C} {#4}
\end{minipage}\begin{minipage}{2.4in}
\choice{green}{B} {#3}

\choice{blue}{D}  {#5}
\end{minipage}
}

\def\notes#1{

\vspace{2ex}
{\em Notes: {#1}}}
%
%  Simple macro to declare and use image all at once.
%
\def\figurehere#1#2{\pgfdeclareimage[width={#2}]{{#1}}{figures/#1} \pgfuseimage{{#1}}}

\def\startframe#1{\begin{frame}[t,fragile] \frametitle{\thedate \hfill {#1}} }

%
%  Declare images
%
%\pgfdeclareimage[width=3in]{geyser}{figures/geyser}

\mode<presentation>
{
%  \setbeamertemplate{background canvas}
  \usetheme{boxes}
  \usecolortheme{notblackscreen}

%  \usetheme{default}
%  \setbeamercovered{transparent}
}
%\beamertemplatetransparentcovereddynamic

\usepackage[english]{babel}
\usepackage[latin1]{inputenc}
\usepackage{times}
\usepackage[T1]{fontenc}

\setbeamersize{text margin left=0.1in}
\setbeamersize{text margin right=0.1in}



\begin{document}


\def \thedate{Jan.~30}


%%%%%%%%%%%%%%%%%%%%%%%%%%%%%%%%%%%%%%%%

\startframe{Announcements}
\begin{itemize}
\item 
Read Sections 4.6 through 4.8 (both editions)
\item 
HW \#3 should go up later today; it will be due Wednesday, Feb.~8
\item
Solutions for HW \#2 use Sweave (source file is provided)
\item
I will not be in class on:  

Friday, Feb.~10;  Wednesday, Mar.~14; Monday, Mar.~12 (probably)
\end{itemize}
\notes{I highly recommend checking out Sweave!  
To try it, see if you can convert the
.Rnw file for solution set \#2 to a pdf.}
\end{frame}


\startframe{4.3 Classification of States}
State $i$ is recurrent if and only if, conditional on $X_0=i$, 
\begin{itemize}
\item $P(\mbox{ever revisiting $i$}) = 1$
\item $E(\#\{T>0:  X_T = i\}) = \infty$.
\item $\sum_{n=1}^\infty P_{ii}^n = \infty$.
\end{itemize}
\notes{This is a slide from a previous lecture.  We'll use the third fact in the 
following proof.}
\end{frame}

\startframe{4.3 Classification of States}
Theorem:  
Recurrence (or transience) is a class property.

How do we prove this?
\notes{
This is Corollary 4.2, and we followed the proof given in the textbook quite closely.  
It is instructive to walk through the steps, since none of them is too difficult.
}
\end{frame}


\startframe{4.3 Classification of States}
Theorem:  
All states of a finite, irreducible Markov chain are recurrent.

How do we prove this?
\notes{
First, irreducibility means only a single communicating class, which means
we must merely show that a single state is recurrent (since recurrence is a
class property).  If not, then every state is transient, which means that with probability
one, every state has a finite ``last time visited''.  But this is impossible, since this means 
that the entire chain has a ``last time visited'' equal to the maximum of the last times for
every state.
}
\end{frame}


\startframe{4.3 Classification of States}
\question{In a Markov chain, is impossible to move:}
{from a transient state to a transient state}
{from a recurrent state to a transient state}
{from a transient state to a recurrent state}
{from a recurrent state to a recurrent state}
\notes{Correct answer:  B.  This is good practice thinking about some
of the properties of Markov chain states.  As part of our discussion of this
question, we noted that once you enter a recurrent class, you never leave
it.  (This was proved in homework \#2.)
Also, it may be possible to move from one transient class to another, but
then it is never possible to move back.}
\end{frame}

\startframe{Mean Time Spent in Transient States}
Suppose you have \$2 and you bet on fair games of chance
until you either go broke or have \$5.  
\begin{itemize}
\item What is the expected number of time steps that you have \$2?
\item What is the probability that you will at some point have \$1?
\end{itemize}
\notes{These are interesting questions that are answered in Section 4.6.
Because we have time, and because it gives us some more practice with important
concepts like recurrence, transience, and conditioning, we will explore them.}
\end{frame}

\startframe{Mean Time Spent in Transient States}
Let $s_{ij} \stackrel{\rm def}{=}E(\#\{ T: X_T=i \mid X_0=j \})$
\begin{itemize}
\item For which $i$ and $j$ is $s_{ij}$ meaningful?
\item Let us show $S = I + P_TS$ for correctly defined $S$ and $P_T$.
\end{itemize}
\notes{We argued that the answer to the first question is ``only the transient
states''.  The second equation is proven by a conditioning argument, and
it leads to a simple-to-solve matrix equation that will lead us to the matrix $S$,
which immediately answers questions like "What is the expected number of
time steps that you have \$2?''}
\end{frame}



%%%%%%%%%%%%%%%%%%%%%%%%%%%%%%%%%%%%%%%%
\end{document}


