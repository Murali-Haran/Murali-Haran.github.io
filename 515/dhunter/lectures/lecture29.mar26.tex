\documentclass{beamer}
\usepackage{beamerthemeshadow}
\usepackage{amssymb}
\usepackage{tikz}

\newcommand{\beaa}{\begin{eqnarray*}}
\newcommand{\eeaa}{\end{eqnarray*}}
\newcommand{\bea}{\begin{eqnarray}}
\newcommand{\eea}{\end{eqnarray}}
\def\E{\mathop{\rm E\,}\nolimits}
\def\Var{\mathop{\rm Var\,}\nolimits}
\def\Cov{\mathop{\rm Cov\,}\nolimits}
\def\Corr{\mathop{\rm Corr\,}\nolimits}
\def\logit{\mathop{\rm logit\,}\nolimits}
\newcommand{\eid}{{\stackrel{\cal{D}}{=}}}
\newcommand{\cip}{{\stackrel{{P}}{\to}}}
\def\bR{\mathbb{R}}	% real line
\def\given{\,|\,}

%%%%%%%%%%
% From http://www.disc-conference.org/disc1998/mirror/llncs.sty
\def\vec#1{\mathchoice{\mbox{\boldmath$\displaystyle\bf#1$}}
{\mbox{\boldmath$\textstyle\bf#1$}}
{\mbox{\boldmath$\scriptstyle\bf#1$}}
{\mbox{\boldmath$\scriptscriptstyle\bf#1$}}}
%%%%%%%%%%

\def\choice#1#2{
\tikz {
\def \lx{-.2} \def \ux{.2} \def \ly{-.2} \def\uy{.2} 
\filldraw[color=#1] (\lx*.9,\ly*.9) rectangle (\ux*.9,\uy*.9);
\draw(\lx,\ly) rectangle (\ux,\uy); \node at (0,0)[color=white] {\bf {#2}}; \node at (0,0) [color=black] {{#2}};
} }

\def\question#1#2#3#4#5{{#1}

\begin{minipage}{2.45in}
\begin{enumerate}
\item [{\choice{red}{A}}]\rule{-.25em}{0em}{#2}
\item [{\choice{yellow}{C}}]\rule{-.25em}{0em}{#4}
\end{enumerate}
%\choice{red}{A} {#2}
%
%\choice{yellow}{C} {#4}
\end{minipage}\begin{minipage}{2.45in}
\begin{enumerate}
\item [{\choice{green}{B}}]\rule{-.25em}{0em}{#3}
\item [{\choice{blue}{D}}]\rule{-.25em}{0em}{#5}
\end{enumerate}
%\choice{green}{B} {#3}
%
%\choice{blue}{D}  {#5}
\end{minipage}
}

%
%  Simple macro to declare and use image all at once.
%
\def\figurehere#1#2{\pgfdeclareimage[width={#2}]{{#1}}{figures/#1} \pgfuseimage{{#1}}}

\def\startframe#1{\begin{frame}[t,fragile] \frametitle{\thedate \hfill {#1}} \vspace{-1ex}}


%
%  Declare images
%
%\pgfdeclareimage[width=3in]{geyser}{figures/geyser}

\mode<presentation>
{
%  \setbeamertemplate{background canvas}
  \usetheme{boxes}
  \usecolortheme{blackscreen}

%  \usetheme{default}
%  \setbeamercovered{transparent}
}
%\beamertemplatetransparentcovereddynamic

\usepackage[english]{babel}
\usepackage[latin1]{inputenc}
\usepackage{times}
\usepackage[T1]{fontenc}

\setbeamersize{text margin left=0.1in}
\setbeamersize{text margin right=0.1in}




\begin{document}


\def \thedate{Mar.~26}


%%%%%%%%%%%%%%%%%%%%%%%%%%%%%%%%%%%%%%%%

\startframe{Announcements}
\begin{itemize}
\item 
HW \#8 is due on Friday, March 30 at 2:30pm.
\item 
All homework must be turned in electronically from now on.
\item
There is a lot of computing on this assignment; please let me know
early if there will be coding challenges!
\end{itemize}
\end{frame}


\startframe{Monte Carlo methods}
$\hat\mu = \frac1n \sum_{i=1}^n g(X_i)$, where $X_i$ are i.i.d.~from $f$.
\begin{itemize}
\item Why this works:  (Strong or weak) law of large numbers
\item More precise information is provided by the Central Limit Theorem.
\item The above results are true for vectors $X_i$ of arbitrary 
dimension.
\item {\em Thinking about i.i.d.~Monte Carlo $\equiv$ thinking about basic statistics.}
\end{itemize}
\end{frame}

\startframe{Monte Carlo methods}
Toy example from Charlie Geyer via Murali Haran:
\begin{itemize}
\item
Given $X_1, X_2 \stackrel{\rm iid}{\sim} N(0,1)$, what is $P(X_2<X_1^2)$?
\item Try numerical integration.
\item Try Monte Carlo approach.
\item Relative advantages / disadvantages?
\end{itemize}
\end{frame}

\startframe{Monte Carlo methods}
How do we get ``random'' numbers from an arbitrary distribution $F$?
\begin{itemize}
\item Everything starts with uniform pseudo-random numbers.
\item ``Uniform random'' numbers are neither (!!)
\item Refer to {\tt help(Random)} in R.
\item Generating pseudo-random (and pseudo-i.i.d.) numbers is a huge topic
that we will not explore in depth.
\end{itemize}
\end{frame}

\startframe{Monte Carlo methods}
Properties of uniform random number generators:
\begin{itemize}
\item Repetititve
\item Not independent
\item Not even continuous
\end{itemize}
These are not always disadvantages!
\end{frame}

\startframe{Monte Carlo methods}
How do you generate $X\sim F$ given $U\sim\mbox{Unif}(0,1)$?
\begin{itemize}
\item
Some special cases are covered in Ross, section 11.3
\item
General ``inversion'' method based on  quantile function
\[
F^-(u)  \stackrel{\rm def}{=} \inf \{ x : u\le F(x)\}.
\]
Can prove:  $F^-(U)\sim F$, i.e., $P[F^-(U) \le x] = F(x)$ for all $x$.
%In case $F(x)$ is strictly increasing and continuous, $F^-=F^{-1}$.
\end{itemize}
\end{frame}

\startframe{Monte Carlo methods}
Problems with the inversion method:
\begin{itemize}
\item Only works in one dimension
\item Explicit quantile functions are not always available
\item Explicit $F(x)$ may not even be possible; for instance, 
$f(x)$ may be known only up to a constant.
\end{itemize}
\end{frame}

\startframe{Monte Carlo methods}
Rejection sampling works much more generally than inversion method.
\begin{itemize}
\item Given:  An ``easier'' density $g$ and known $K$ such that $Kg(x)\ge f(x)$.
\item Idea:  Sample repeatedly from $g$, but only keep each sampled value
with probability $f(x)/Kg(x)$.
\end{itemize}
\end{frame}


\startframe{Monte Carlo methods}
Rejection sampling notes
\begin{itemize}
\item Only necessary to know $f$ and $g$ up to constants.
\item Often better to take logarithms when computing with ratios.
\item Section 11.2.2 has a couple examples (e.g., beta).
\end{itemize}
%:  If
%$r(x)=\alpha f(x)$ and $s(x)=\beta g(x)$, find $K\ge \sup_x s(x)/r(x)$
\end{frame}


%%%%%%%%%%%%%%%%%%%%%%%%%%%%%%%%%%%%%%%%
\end{document}


