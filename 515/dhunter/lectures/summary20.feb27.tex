\documentclass[handout]{beamer}
\usepackage{beamerthemeshadow}
\usepackage{amssymb}
\usepackage{pgfpages}
\usepackage{tikz}
\pgfpagesuselayout{4 on 1}%[landscape]

\newcommand{\beaa}{\begin{eqnarray*}}
\newcommand{\eeaa}{\end{eqnarray*}}
\newcommand{\bea}{\begin{eqnarray}}
\newcommand{\eea}{\end{eqnarray}}
\def\E{\mathop{\rm E\,}\nolimits}
\def\Var{\mathop{\rm Var\,}\nolimits}
\def\Cov{\mathop{\rm Cov\,}\nolimits}
\def\Corr{\mathop{\rm Corr\,}\nolimits}
\def\logit{\mathop{\rm logit\,}\nolimits}
\newcommand{\eid}{{\stackrel{\cal{D}}{=}}}
\newcommand{\cip}{{\stackrel{{P}}{\to}}}
\def\bR{\mathbb{R}}	% real line
\def\given{\,|\,}

%%%%%%%%%%
% From http://www.disc-conference.org/disc1998/mirror/llncs.sty
\def\vec#1{\mathchoice{\mbox{\boldmath$\displaystyle\bf#1$}}
{\mbox{\boldmath$\textstyle\bf#1$}}
{\mbox{\boldmath$\scriptstyle\bf#1$}}
{\mbox{\boldmath$\scriptscriptstyle\bf#1$}}}
%%%%%%%%%%

\def\choice#1#2
{\tikz 
{\def \lx{-.2} \def \ux{.2} \def \ly{-.2} \def\uy{.2} \filldraw[color=#1] (\lx*.9,\ly*.9) rectangle (\ux*.9,\uy*.9);
\draw(\lx,\ly) rectangle (\ux,\uy); \node at (0,0)[color=white] {\bf {#2}}; \node at (0,0) [color=black] {{#2}};}}

\def\question#1#2#3#4#5{{#1}

\begin{minipage}{2.4in}
\choice{red}{A} {#2}

\choice{yellow}{C} {#4}
\end{minipage}\begin{minipage}{2.4in}
\choice{green}{B} {#3}

\choice{blue}{D}  {#5}
\end{minipage}
}

\def\notes#1{

\vspace{2ex}
{\em Notes: {#1}}}
%
%  Simple macro to declare and use image all at once.
%
\def\figurehere#1#2{\pgfdeclareimage[width={#2}]{{#1}}{figures/#1} \pgfuseimage{{#1}}}

\def\startframe#1{\begin{frame}[t,fragile] \frametitle{\thedate \hfill {#1}} }

%
%  Declare images
%
%\pgfdeclareimage[width=3in]{geyser}{figures/geyser}

\mode<presentation>
{
%  \setbeamertemplate{background canvas}
  \usetheme{boxes}
  \usecolortheme{notblackscreen}

%  \usetheme{default}
%  \setbeamercovered{transparent}
}
%\beamertemplatetransparentcovereddynamic

\usepackage[english]{babel}
\usepackage[latin1]{inputenc}
\usepackage{times}
\usepackage[T1]{fontenc}

\setbeamersize{text margin left=0.1in}
\setbeamersize{text margin right=0.1in}



\begin{document}


\def \thedate{Feb.~27}


%%%%%%%%%%%%%%%%%%%%%%%%%%%%%%%%%%%%%%%%

\startframe{Announcements}
\begin{itemize}
\item 
For Wednesday, work old homework problems and new homework problems.
\item 
Midterm exam:  7:00pm on Wednesday, Feb.~29 in 105 Willard.
(NB:  Change of room!)
\item 
HW \#6 will be due on Friday (Mar.~2) at 2:30.
\end{itemize}
\end{frame}


\startframe{Midterm practice}
\question{
Let $X$ be an exponential random variable.  Without any 
computations, which of the following is correct?  
}
{$E[X^2 \mid X>1] = E[(X+1)^2]$}
{$E[X^2 \mid X>1] = E[X^2]+1$}
{$E[X^2 \mid X>1] = (1+E[X])^2$}
{None of these}
\notes{
Answer:  A,
because $X\mid X>1$ has the same distribution as $X+1$ (this is the memoryless property).
}
\end{frame}

\startframe{Midterm practice}
\question{A doctor has scheduled two appointments, one at 1:00 and one at 1:30.
The amounts of time that appointments last are independent exponential
random variables with mean 30 minutes.  If both patients are on time, the expected 
amount of time that the 1:30 appointment spends at the doctor's office is\ldots}
{Less than 30 minutes}
{Between 30 and 60 minutes}
{More than 60 minutes}
{}
\notes{Answer:  B.  In the case that the first patient is done by 1:30, 
the expected time equals 30 minutes.  In the case that the first patient is not done
by 1:30, the expected time equals 60 minutes (30 minutes for the first patient and
30 minutes for the second).  Therefore, the final answer is a weighted average
of 30 and 60.  
More precisely, the weight on 30 minutes is the probability that the first patient will
be finished by 1:30, which equals $1-e^{-1}$.}
\end{frame}

\startframe{Midterm practice}
In a two-server queueing system, customers arrive according to a Poisson
process with rate $\lambda$.  If at least one of the two servers is free when
a customer arrives, the customer is immediately served; otherwise, the customer
departs immediately and is lost.  Service times are exponential (independently)
with rate $\mu$.
\end{frame}

\startframe{Midterm practice}
In the two-server system (customer rate=$\lambda$, service rate=$\mu$),
\begin{itemize}
\item Starting from both servers busy, what is the expected time until the next
customer is served?
\item Starting empty, find the expected time until both servers are busy.
\item What is the expected time between two successive lost customers?
\end{itemize}
\notes{Answers (we discussed the first two in class; try getting the third on your own---and check 
my algebra!):
\begin{itemize}
\item $\frac{1}{2\mu} + \frac{1}{\lambda}$
\item $(2\lambda + \mu)/\lambda^2$
\item $\frac{1}{\lambda} + \frac{2\mu}{\lambda(\lambda+\mu)}
+ \frac{2\mu^2(2\lambda+\mu)}{\lambda^3(\lambda+\mu)}$
\end{itemize}
}
\end{frame}

\startframe{Midterm practice}
Consider a single server queueing system where customers arrive according to a
Poisson process with rate $\lambda$, service times are exponential with rate $\mu$,
and customers are served in the order of their arrival.  If a customer arrives and finds
$n-1$ others in the system, and $X$ is the number in the system when that
customer departs, find the pmf of $X$.
\notes{
Consider the customer-arrival process and the customer-servicing process, which
are two Poisson processes with rates $\lambda$ and $\mu$, respectively.  The
sum of these processes is Poisson with rate $\lambda+\mu$, and each event in
the summed process is an arrival with probability $\lambda/(\lambda+\mu)$
and a service-finish with probability $\mu/(\lambda+\mu)$.  If you are the $n$th
customer, then you will leave exactly at the $n$th service-finish.  Thus, the question is
how many people arrived (call this a success) before the $n$th service-finish (call this
a failure) in repeated Bernoulli trials.  This means that the total number of trials
is negative binomial random variable, which leads to the pmf.
}
\end{frame}


%%%%%%%%%%%%%%%%%%%%%%%%%%%%%%%%%%%%%%%%
\end{document}


