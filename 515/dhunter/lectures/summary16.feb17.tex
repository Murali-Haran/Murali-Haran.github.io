\documentclass[handout]{beamer}
\usepackage{beamerthemeshadow}
\usepackage{amssymb}
\usepackage{pgfpages}
\usepackage{tikz}
\pgfpagesuselayout{4 on 1}%[landscape]

\newcommand{\beaa}{\begin{eqnarray*}}
\newcommand{\eeaa}{\end{eqnarray*}}
\newcommand{\bea}{\begin{eqnarray}}
\newcommand{\eea}{\end{eqnarray}}
\def\E{\mathop{\rm E\,}\nolimits}
\def\Var{\mathop{\rm Var\,}\nolimits}
\def\Cov{\mathop{\rm Cov\,}\nolimits}
\def\Corr{\mathop{\rm Corr\,}\nolimits}
\def\logit{\mathop{\rm logit\,}\nolimits}
\newcommand{\eid}{{\stackrel{\cal{D}}{=}}}
\newcommand{\cip}{{\stackrel{{P}}{\to}}}
\def\bR{\mathbb{R}}	% real line
\def\given{\,|\,}

%%%%%%%%%%
% From http://www.disc-conference.org/disc1998/mirror/llncs.sty
\def\vec#1{\mathchoice{\mbox{\boldmath$\displaystyle\bf#1$}}
{\mbox{\boldmath$\textstyle\bf#1$}}
{\mbox{\boldmath$\scriptstyle\bf#1$}}
{\mbox{\boldmath$\scriptscriptstyle\bf#1$}}}
%%%%%%%%%%

\def\choice#1#2
{\tikz 
{\def \lx{-.2} \def \ux{.2} \def \ly{-.2} \def\uy{.2} \filldraw[color=#1] (\lx*.9,\ly*.9) rectangle (\ux*.9,\uy*.9);
\draw(\lx,\ly) rectangle (\ux,\uy); \node at (0,0)[color=white] {\bf {#2}}; \node at (0,0) [color=black] {{#2}};}}

\def\question#1#2#3#4#5{{#1}

\begin{minipage}{2.4in}
\choice{red}{A} {#2}

\choice{yellow}{C} {#4}
\end{minipage}\begin{minipage}{2.4in}
\choice{green}{B} {#3}

\choice{blue}{D}  {#5}
\end{minipage}
}

\def\notes#1{

\vspace{2ex}
{\em Notes: {#1}}}
%
%  Simple macro to declare and use image all at once.
%
\def\figurehere#1#2{\pgfdeclareimage[width={#2}]{{#1}}{figures/#1} \pgfuseimage{{#1}}}

\def\startframe#1{\begin{frame}[t,fragile] \frametitle{\thedate \hfill {#1}} }

%
%  Declare images
%
%\pgfdeclareimage[width=3in]{geyser}{figures/geyser}

\mode<presentation>
{
%  \setbeamertemplate{background canvas}
  \usetheme{boxes}
  \usecolortheme{notblackscreen}

%  \usetheme{default}
%  \setbeamercovered{transparent}
}
%\beamertemplatetransparentcovereddynamic

\usepackage[english]{babel}
\usepackage[latin1]{inputenc}
\usepackage{times}
\usepackage[T1]{fontenc}

\setbeamersize{text margin left=0.1in}
\setbeamersize{text margin right=0.1in}



\begin{document}


\def \thedate{Feb.~17}


%%%%%%%%%%%%%%%%%%%%%%%%%%%%%%%%%%%%%%%%

\startframe{Announcements}
\begin{itemize}
\item
For Monday,
read Section 5.4.1
\item 
HW\#5 will be due next Friday.  {\em It must be submitted electronically!}
\item
The midterm will be \ldots ???  (Wednesday evening?)
\end{itemize}
\notes{HW\#5 is somewhat shorter than normal because it must be done electronically.}
\end{frame}

\startframe{5.3 The Poisson Process}
A Poisson process is a special type of counting process:
\begin{itemize}
\item Counts of events in disjoint time intervals are independent.
\item For any $0\le s<t$, the number of events in $(s,s+t]$ is Poisson with parameter
$\lambda t$.
\item Technically, we should also say $N(0)=0$.
\end{itemize}
\notes{This is the ``Poisson-based'' definition of the Poisson process.  It's the one we
discussed on Wednesday.}
\end{frame}

\startframe{5.3 The Poisson Process}
A Poisson process is a special type of counting process
(equivalent re-definition):
\begin{itemize}
\item Counts of events in disjoint time intervals are independent.
\item $P[N(s+h)-N(s)=1] = \lambda h + o(h)$ as $h\to 0$.
\item $P[N(s+h)-N(s)\ge2] = o(h)$ as $h\to 0$.
\item Technically, we should also say $N(0)=0$.
\end{itemize}
\notes{This is the ``first-principles'' definition of the Poisson process.  To understand it,
we had to define the little-o notation:  If $f(x)=o(x)$ as $x\to0$, this means
that $f(x)/x\to0$ as $x\to0$.  Intuitively, $f(x)=o(x)$ as $x\to0$ means that
$f(x)$ goes to zero faster than $x$ does.  An important special case of this notation
is this:  If $f(x)=o(1)$ as $x\to0$, then $f(x)\to0$ as $x\to0$.}
\end{frame}

\startframe{5.3 The Poisson Process}
Why are the two definitions equivalent?
\begin{itemize}
\item How can we check that the first definition (``Poisson'') 
implies the second (``first principles'')?
\item How about the other way around (i.e., first principles implies Poisson)?
\end{itemize}
\notes{The first proof may be done directly (using ``brute force'')
and uses the fact that $e^{-\lambda h}=1-\lambda h + o(h)$ as $h\to 0$.
The second proof requires some type of moment-generating function approach
(the book uses the Laplace transform, which is similar to the MGF).  However,
it is possible to understand the intuition by dividing a time interval of length $t$ into
$k$ equal subintervals, then studying what happens as $k$ goes to infinity.  Basically,
the distribution of the number of events can be shown to be roughly binomial with 
parameters $(k, \lambda t/k)$, and therefore the Poisson approximation to the binomial
(with parameter $t\lambda$) gets more and more accurate as $k\to\infty$.
}
\end{frame}


%%%%%%%%%%%%%%%%%%%%%%%%%%%%%%%%%%%%%%%%
\end{document}


