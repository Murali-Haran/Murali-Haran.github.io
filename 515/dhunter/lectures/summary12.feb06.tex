\documentclass[handout]{beamer}
\usepackage{beamerthemeshadow}
\usepackage{amssymb}
\usepackage{pgfpages}
\usepackage{tikz}
\pgfpagesuselayout{4 on 1}%[landscape]

\newcommand{\beaa}{\begin{eqnarray*}}
\newcommand{\eeaa}{\end{eqnarray*}}
\newcommand{\bea}{\begin{eqnarray}}
\newcommand{\eea}{\end{eqnarray}}
\def\E{\mathop{\rm E\,}\nolimits}
\def\Var{\mathop{\rm Var\,}\nolimits}
\def\Cov{\mathop{\rm Cov\,}\nolimits}
\def\Corr{\mathop{\rm Corr\,}\nolimits}
\def\logit{\mathop{\rm logit\,}\nolimits}
\newcommand{\eid}{{\stackrel{\cal{D}}{=}}}
\newcommand{\cip}{{\stackrel{{P}}{\to}}}
\def\bR{\mathbb{R}}	% real line
\def\given{\,|\,}

%%%%%%%%%%
% From http://www.disc-conference.org/disc1998/mirror/llncs.sty
\def\vec#1{\mathchoice{\mbox{\boldmath$\displaystyle\bf#1$}}
{\mbox{\boldmath$\textstyle\bf#1$}}
{\mbox{\boldmath$\scriptstyle\bf#1$}}
{\mbox{\boldmath$\scriptscriptstyle\bf#1$}}}
%%%%%%%%%%

\def\choice#1#2
{\tikz 
{\def \lx{-.2} \def \ux{.2} \def \ly{-.2} \def\uy{.2} \filldraw[color=#1] (\lx*.9,\ly*.9) rectangle (\ux*.9,\uy*.9);
\draw(\lx,\ly) rectangle (\ux,\uy); \node at (0,0)[color=white] {\bf {#2}}; \node at (0,0) [color=black] {{#2}};}}

\def\question#1#2#3#4#5{{#1}

\begin{minipage}{2.4in}
\choice{red}{A} {#2}

\choice{yellow}{C} {#4}
\end{minipage}\begin{minipage}{2.4in}
\choice{green}{B} {#3}

\choice{blue}{D}  {#5}
\end{minipage}
}

\def\notes#1{

\vspace{2ex}
{\em Notes: {#1}}}
%
%  Simple macro to declare and use image all at once.
%
\def\figurehere#1#2{\pgfdeclareimage[width={#2}]{{#1}}{figures/#1} \pgfuseimage{{#1}}}

\def\startframe#1{\begin{frame}[t,fragile] \frametitle{\thedate \hfill {#1}} }

%
%  Declare images
%
%\pgfdeclareimage[width=3in]{geyser}{figures/geyser}

\mode<presentation>
{
%  \setbeamertemplate{background canvas}
  \usetheme{boxes}
  \usecolortheme{notblackscreen}

%  \usetheme{default}
%  \setbeamercovered{transparent}
}
%\beamertemplatetransparentcovereddynamic

\usepackage[english]{babel}
\usepackage[latin1]{inputenc}
\usepackage{times}
\usepackage[T1]{fontenc}

\setbeamersize{text margin left=0.1in}
\setbeamersize{text margin right=0.1in}



\begin{document}


\def \thedate{Feb.~6}


%%%%%%%%%%%%%%%%%%%%%%%%%%%%%%%%%%%%%%%%

\startframe{Announcements}
\begin{itemize}
\item
Please read Sections 5.1, 5.2.1, 5.2.2, and 5.2.3 (both editions) for Wednesday.
\item 
No class this Friday.
\item 
Office hours will be this {\bf Tuesday} (not Thursday) from 2:00-4:00.
\end{itemize}
\end{frame}

\startframe{4.4 Limiting Probabilities}
Theorem:  For an irreducible ergodic (time-homogeneous) 
Markov chain with transition matrix $P$,
\[
\lim_{n\to\infty} P^n_{ij} 
\]
exists and it does not depend on $i$ (it does generally depend on $j$).
\notes{This was a review of the result from Friday's class, the most important
result in Section 4.4}
\end{frame}

\startframe{4.4 Limiting Probabilities}
Corollary:  For an irreducible ergodic (time-homogeneous) 
Markov chain with transition matrix $P$,
the equations 
\[
\pi_j = \sum_{i=0}^{\mbox{\# states}} \pi_i P_{ij}, \qquad j = 0, 1, 2, \ldots
\]
have solution 
\[
\pi_j=\lim_{n\to\infty} P^n_{ij}.
\]
\notes{In Section 4.4, this ``corollary'' is part of the main theorem.  We sketched a 
proof, which is exactly the same proof found in the textbook.}
\end{frame}

\startframe{4.4 Limiting Probabilities}
Example:
\begin{itemize}
\item
Let $\alpha$ be the probability of rain after a rainy day.
\item
Let $\beta$ be the probability of rain after a non-rainy day.
\end{itemize}
What are the limiting probabilities?
\notes{We argued that this chain must be irreducible and ergodic.   Therefore 
the equations of the corollary give the limiting probabilities.  These equations are
\begin{eqnarray*}
\pi_1 &=& \alpha \pi_1 + \beta \pi_2 \\
\pi_2 &=& (1-\alpha)\pi_1 + (1-\beta) \pi_2 \\
\pi_1 + \pi_2 &=& 1.
\end{eqnarray*}
}
\end{frame}

\startframe{4.4 Limiting Probabilities}
One last bit of terminology:
\begin{itemize}
\item stationary probability
\end{itemize}
\notes{The stationary probabilities are by definition the long-run probabilities.  }
\end{frame}

\startframe{4.4 Limiting Probabilities}
More facts we will not prove:
\begin{itemize}
\item If we assume $\sum_j \pi_j=1$, then the solution in the corollary is unique.
\item The $\pi_j$ from the corollary are the stationary probabilities.
\end{itemize}
\notes{These facts are given in various places in Section 4.4.}
\end{frame}

\startframe{4.4 Limiting Probabilities}
More facts we will not prove:
\begin{itemize}
\item Aperiodicity is not necessary in order for the equations in the corollary to have a 
unique solution.  However, for irreducible chains, positive recurrence is necessary.
\item Stationary probabilities exist even for periodic irreducible chains.
\end{itemize}
\notes{These facts are given in various places in Section 4.4.}
\end{frame}

\startframe{4.8 Time Reversible Markov Chains}
``Consider a stationary ergodic Markov chain (that is, an ergodic Markov chain
that has been in operation for a long time)\ldots''
\begin{itemize}
\item We now know what ``ergodic'' means:

There is a limiting distribution $\pi$ that uniquely solves the
equations $\pi^\top = \pi^\top P$.
\end{itemize}
\end{frame}

\startframe{4.8 Time Reversible Markov Chains}
\begin{itemize}
\item
Define $Q_{ij} = P(X_t = j \mid X_{t+1} = i)$.
\item
Use Bayes' theorem to prove 
\[
Q_{ij} = \frac{\pi_j P_{ji}}{\pi_i}.
\]
\item
What if $P_{ij}=Q_{ij}$ for all $i,j$?
\end{itemize}
\notes{We'll prove the fact about $Q_{ij}$ in the next class.  
When $P_{ij}$ and $Q_{ij}$ are the same, the Markov chain 
has a special property called time-reversibility.}
\end{frame}



%%%%%%%%%%%%%%%%%%%%%%%%%%%%%%%%%%%%%%%%
\end{document}


