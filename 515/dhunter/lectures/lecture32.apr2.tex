\documentclass{beamer}
\usepackage{beamerthemeshadow}
\usepackage{amssymb}
\usepackage{tikz}

\newcommand{\beaa}{\begin{eqnarray*}}
\newcommand{\eeaa}{\end{eqnarray*}}
\newcommand{\bea}{\begin{eqnarray}}
\newcommand{\eea}{\end{eqnarray}}
\def\E{\mathop{\rm E\,}\nolimits}
\def\Var{\mathop{\rm Var\,}\nolimits}
\def\Cov{\mathop{\rm Cov\,}\nolimits}
\def\Corr{\mathop{\rm Corr\,}\nolimits}
\def\logit{\mathop{\rm logit\,}\nolimits}
\newcommand{\eid}{{\stackrel{\cal{D}}{=}}}
\newcommand{\cip}{{\stackrel{{P}}{\to}}}
\def\bR{\mathbb{R}}	% real line
\def\given{\,|\,}

%%%%%%%%%%
% From http://www.disc-conference.org/disc1998/mirror/llncs.sty
\def\vec#1{\mathchoice{\mbox{\boldmath$\displaystyle\bf#1$}}
{\mbox{\boldmath$\textstyle\bf#1$}}
{\mbox{\boldmath$\scriptstyle\bf#1$}}
{\mbox{\boldmath$\scriptscriptstyle\bf#1$}}}
%%%%%%%%%%

\def\choice#1#2{
\tikz {
\def \lx{-.2} \def \ux{.2} \def \ly{-.2} \def\uy{.2} 
\filldraw[color=#1] (\lx*.9,\ly*.9) rectangle (\ux*.9,\uy*.9);
\draw(\lx,\ly) rectangle (\ux,\uy); \node at (0,0)[color=white] {\bf {#2}}; \node at (0,0) [color=black] {{#2}};
} }

\def\question#1#2#3#4#5{{#1}

\begin{minipage}{2.45in}
\begin{enumerate}
\item [{\choice{red}{A}}]\rule{-.25em}{0em}{#2}
\item [{\choice{yellow}{C}}]\rule{-.25em}{0em}{#4}
\end{enumerate}
%\choice{red}{A} {#2}
%
%\choice{yellow}{C} {#4}
\end{minipage}\begin{minipage}{2.45in}
\begin{enumerate}
\item [{\choice{green}{B}}]\rule{-.25em}{0em}{#3}
\item [{\choice{blue}{D}}]\rule{-.25em}{0em}{#5}
\end{enumerate}
%\choice{green}{B} {#3}
%
%\choice{blue}{D}  {#5}
\end{minipage}
}

%
%  Simple macro to declare and use image all at once.
%
\def\figurehere#1#2{\pgfdeclareimage[width={#2}]{{#1}}{figures/#1} \pgfuseimage{{#1}}}

\def\startframe#1{\begin{frame}[t,fragile] \frametitle{\thedate \hfill {#1}} \vspace{-1ex}}


%
%  Declare images
%
%\pgfdeclareimage[width=3in]{geyser}{figures/geyser}

\mode<presentation>
{
%  \setbeamertemplate{background canvas}
  \usetheme{boxes}
  \usecolortheme{blackscreen}

%  \usetheme{default}
%  \setbeamercovered{transparent}
}
%\beamertemplatetransparentcovereddynamic

\usepackage[english]{babel}
\usepackage[latin1]{inputenc}
\usepackage{times}
\usepackage[T1]{fontenc}

\setbeamersize{text margin left=0.1in}
\setbeamersize{text margin right=0.1in}




\begin{document}


\def \thedate{Apr.~2}


%%%%%%%%%%%%%%%%%%%%%%%%%%%%%%%%%%%%%%%%

\startframe{Announcements}
\begin{itemize}
\item 
HW \#9 is due on Friday, April 6 at 2:30pm.
\item 
All homework must be turned in electronically from now on.
\item Problem \#2 is not as straightforward as I had intended, but go ahead and solve it as asked.
This may involve numerically finding the root of a 4th-degree polynomial.
\end{itemize}
\end{frame}

\startframe{Monte Carlo methods}
Ratio Importance Sampling:
\begin{itemize}
\item Goal:  Find $\mu = \int g(x)f(x)\,dx = \E_f g(X)$.
\item Ratio estimator is
\[
\tilde \mu = 
\frac{\frac1n \sum_i  g(X_i)r(X_i)/q(X_i)}{\frac1n \sum_i  r(X_i)/q(X_i)}
\]
where $X_1, \ldots, X_n \stackrel{\rm iid}{\sim} q$ and $r(x) = f(x)/\alpha$.
\end{itemize}
\end{frame}

\startframe{Monte Carlo methods}
Ratio importance sampling.
\begin{itemize}
\item Unlike $\hat\mu=\frac1n \sum_i  g(X_i)f(X_i)/q(X_i)$, $\tilde \mu$
is not unbiased. 
\item However, $\mbox{MSE}(\tilde \mu) \le \mbox{MSE}(\hat\mu)$.

( J. Liu, {\it Monte Carlo Strategies in Scientific Computing})
\item Must assume that $f(x)>0$ implies $q(x)>0$.
\end{itemize}
\end{frame}

\startframe{Monte Carlo methods}
Importance sampling standard errors
\begin{itemize}
\item Plain importance sampling:  Easy
\item Ratio importance sampling:  Use delta method to derive
\[
\Var \left[ \sqrt{n} ( \tilde \mu - \mu)  \right] \approx \tilde\sigma^2,
\]
where $\tilde \sigma^2 = (1,-\mu) \Sigma (1,-\mu)^\top$ and $\Sigma=\ldots$
\end{itemize}
\end{frame}

\startframe{Monte Carlo methods}
Importance Sampling Example:  What is $P(X>4.5)$ for $X\sim N(0,1)$?
\begin{itemize}
\item Try naive approach.
\item Try importance sampling with $q(x) = \exp\{-(x-4.5)\} I\{x>4.5\}$.
\item Try ratio importance sampling.  (Careful!)
\end{itemize}
\end{frame}

\startframe{Monte Carlo methods}
Ratio importance sampling application:
\begin{itemize}
\item Goal:  Estimate $\mu(\theta) = \E_{f_\theta} g(X)$ for $\theta\in\Theta$.
\item Plain importance sampling:  Draw new sample for each $\theta\in\Theta$.
\item Ratio importance sampling:  Draw a single sample from $q(x)$\ldots

(Need: $f_\theta(x)>0$ implies $q(x)>0$ for all $\theta$.)
\end{itemize}
\end{frame}


%%%%%%%%%%%%%%%%%%%%%%%%%%%%%%%%%%%%%%%%
\end{document}


