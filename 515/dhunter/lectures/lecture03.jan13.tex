\documentclass{beamer}
\usepackage{beamerthemeshadow}
\usepackage{amssymb}
\usepackage{tikz}

\newcommand{\beaa}{\begin{eqnarray*}}
\newcommand{\eeaa}{\end{eqnarray*}}
\newcommand{\bea}{\begin{eqnarray}}
\newcommand{\eea}{\end{eqnarray}}
\def\E{\mathop{\rm E\,}\nolimits}
\def\Var{\mathop{\rm Var\,}\nolimits}
\def\Cov{\mathop{\rm Cov\,}\nolimits}
\def\Corr{\mathop{\rm Corr\,}\nolimits}
\def\logit{\mathop{\rm logit\,}\nolimits}
\newcommand{\eid}{{\stackrel{\cal{D}}{=}}}
\newcommand{\cip}{{\stackrel{{P}}{\to}}}
\def\bR{\mathbb{R}}	% real line
\def\given{\,|\,}

%%%%%%%%%%
% From http://www.disc-conference.org/disc1998/mirror/llncs.sty
\def\vec#1{\mathchoice{\mbox{\boldmath$\displaystyle\bf#1$}}
{\mbox{\boldmath$\textstyle\bf#1$}}
{\mbox{\boldmath$\scriptstyle\bf#1$}}
{\mbox{\boldmath$\scriptscriptstyle\bf#1$}}}
%%%%%%%%%%

\def\choice#1#2{
\tikz {
\def \lx{-.2} \def \ux{.2} \def \ly{-.2} \def\uy{.2} 
\filldraw[color=#1] (\lx*.9,\ly*.9) rectangle (\ux*.9,\uy*.9);
\draw(\lx,\ly) rectangle (\ux,\uy); \node at (0,0)[color=white] {\bf {#2}}; \node at (0,0) [color=black] {{#2}};
} }

\def\question#1#2#3#4#5{{#1}

\begin{minipage}{2.45in}
\begin{enumerate}
\item [{\choice{red}{A}}]\rule{-.25em}{0em}{#2}
\item [{\choice{yellow}{C}}]\rule{-.25em}{0em}{#4}
\end{enumerate}
%\choice{red}{A} {#2}
%
%\choice{yellow}{C} {#4}
\end{minipage}\begin{minipage}{2.45in}
\begin{enumerate}
\item [{\choice{green}{B}}]\rule{-.25em}{0em}{#3}
\item [{\choice{blue}{D}}]\rule{-.25em}{0em}{#5}
\end{enumerate}
%\choice{green}{B} {#3}
%
%\choice{blue}{D}  {#5}
\end{minipage}
}

%
%  Simple macro to declare and use image all at once.
%
\def\figurehere#1#2{\pgfdeclareimage[width={#2}]{{#1}}{figures/#1} \pgfuseimage{{#1}}}

\def\startframe#1{\begin{frame}[t,fragile] \frametitle{\thedate \hfill {#1}} \vspace{-1ex}}


%
%  Declare images
%
%\pgfdeclareimage[width=3in]{geyser}{figures/geyser}

\mode<presentation>
{
%  \setbeamertemplate{background canvas}
  \usetheme{boxes}
  \usecolortheme{blackscreen}

%  \usetheme{default}
%  \setbeamercovered{transparent}
}
%\beamertemplatetransparentcovereddynamic

\usepackage[english]{babel}
\usepackage[latin1]{inputenc}
\usepackage{times}
\usepackage[T1]{fontenc}

\setbeamersize{text margin left=0.1in}
\setbeamersize{text margin right=0.1in}




\begin{document}


\def \thedate{Jan.~13}


%%%%%%%%%%%%%%%%%%%%%%%%%%%%%%%%%%%%%%%%

\startframe{Reading for next class}
\begin{itemize}
\item 
Read Sections 3.4 and 3.5 (both editions) before Wednesday's class.
\end{itemize}
\end{frame}




\startframe{3.2 The Discrete Case}
\question{Suppose $X_1$ and $X_2$ are independent, with
$X_i \sim \mbox{binom}(n_i,p)$.  Given $X_1+X_2=m$, the distribution of $X_1$ is}
{binomial}{Poisson}{uniform}{hypergeometric}
\end{frame}

\startframe{3.2 The Discrete Case}
\question{Suppose $X_1$ and $X_2$ are independent, with
$X_i \sim \mbox{Poisson}(\lambda_i)$.  Given $X_1+X_2=m$, the distribution of $X_1$ is}
{binomial}{Poisson}{uniform}{hypergeometric}
\end{frame}

\startframe{3.3 The Continuous Case}
\question{Suppose $X_1$ and $X_2$ are independent, with
$X_i \sim \mbox{Exponential}(\lambda)$.  Given $X_1+X_2=m$, the distribution of $X_1$ is}
{binomial}{Poisson}{uniform}{hypergeometric}
\end{frame}

\startframe{3.4 Computing Expectations by Conditioning}
\question{If $X\sim \mbox{Geom}(p)$, then $E(X) = $?}
{$\frac1p$}
{$\frac1{p^2}$}
{$\frac1{1-p}$}
{$\frac1{(1-p)^2}$}
\end{frame}



%\startframe{1.2\quad Properties of Probability}
%
%\begin{columns}
%\begin{column}{4cm}
%
%\tikz{
%\def \lx{-1.4} \def \ux{2.4} \def \ly{-1.2} \def\uy{1.2}
%\draw (\lx,\ly) rectangle (\ux,\uy);
%% grid with dot at origin for helping place objects in rectangle:
%%\draw[step=.5, very thin] (\lx,\ly) grid (\ux,\uy); \filldraw (0,0) circle (.05);
%\draw (0,0) ellipse (1 and .8);
%\draw (1,0) ellipse (1 and .8);
%\draw (0,.9) node {\small$A$};
%\draw (1,.9) node {\small$B$};
%\draw (2.2,1) node{$S$};
%}
%\end{column}
%\begin{column}{6cm}
%Venn diagram depicting events $A$ and $B$.  In the picture, they appear to 
%have a nonempty intersection.
%\end{column}
%\end{columns}
%\end{frame}





%%%%%%%%%%%%%%%%%%%%%%%%%%%%%%%%%%%%%%%%
\end{document}


