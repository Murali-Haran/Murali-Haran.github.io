\documentclass{beamer}
\usepackage{beamerthemeshadow}
\usepackage{amssymb}
\usepackage{tikz}

\newcommand{\beaa}{\begin{eqnarray*}}
\newcommand{\eeaa}{\end{eqnarray*}}
\newcommand{\bea}{\begin{eqnarray}}
\newcommand{\eea}{\end{eqnarray}}
\def\E{\mathop{\rm E\,}\nolimits}
\def\Var{\mathop{\rm Var\,}\nolimits}
\def\Cov{\mathop{\rm Cov\,}\nolimits}
\def\Corr{\mathop{\rm Corr\,}\nolimits}
\def\logit{\mathop{\rm logit\,}\nolimits}
\newcommand{\eid}{{\stackrel{\cal{D}}{=}}}
\newcommand{\cip}{{\stackrel{{P}}{\to}}}
\def\bR{\mathbb{R}}	% real line
\def\given{\,|\,}

%%%%%%%%%%
% From http://www.disc-conference.org/disc1998/mirror/llncs.sty
\def\vec#1{\mathchoice{\mbox{\boldmath$\displaystyle\bf#1$}}
{\mbox{\boldmath$\textstyle\bf#1$}}
{\mbox{\boldmath$\scriptstyle\bf#1$}}
{\mbox{\boldmath$\scriptscriptstyle\bf#1$}}}
%%%%%%%%%%

\def\choice#1#2{
\tikz {
\def \lx{-.2} \def \ux{.2} \def \ly{-.2} \def\uy{.2} 
\filldraw[color=#1] (\lx*.9,\ly*.9) rectangle (\ux*.9,\uy*.9);
\draw(\lx,\ly) rectangle (\ux,\uy); \node at (0,0)[color=white] {\bf {#2}}; \node at (0,0) [color=black] {{#2}};
} }

\def\question#1#2#3#4#5{{#1}

\begin{minipage}{2.45in}
\begin{enumerate}
\item [{\choice{red}{A}}]\rule{-.25em}{0em}{#2}
\item [{\choice{yellow}{C}}]\rule{-.25em}{0em}{#4}
\end{enumerate}
%\choice{red}{A} {#2}
%
%\choice{yellow}{C} {#4}
\end{minipage}\begin{minipage}{2.45in}
\begin{enumerate}
\item [{\choice{green}{B}}]\rule{-.25em}{0em}{#3}
\item [{\choice{blue}{D}}]\rule{-.25em}{0em}{#5}
\end{enumerate}
%\choice{green}{B} {#3}
%
%\choice{blue}{D}  {#5}
\end{minipage}
}

%
%  Simple macro to declare and use image all at once.
%
\def\figurehere#1#2{\pgfdeclareimage[width={#2}]{{#1}}{figures/#1} \pgfuseimage{{#1}}}

\def\startframe#1{\begin{frame}[t,fragile] \frametitle{\thedate \hfill {#1}} \vspace{-1ex}}


%
%  Declare images
%
%\pgfdeclareimage[width=3in]{geyser}{figures/geyser}

\mode<presentation>
{
%  \setbeamertemplate{background canvas}
  \usetheme{boxes}
  \usecolortheme{blackscreen}

%  \usetheme{default}
%  \setbeamercovered{transparent}
}
%\beamertemplatetransparentcovereddynamic

\usepackage[english]{babel}
\usepackage[latin1]{inputenc}
\usepackage{times}
\usepackage[T1]{fontenc}

\setbeamersize{text margin left=0.1in}
\setbeamersize{text margin right=0.1in}




\begin{document}


\def \thedate{Apr.~9}


%%%%%%%%%%%%%%%%%%%%%%%%%%%%%%%%%%%%%%%%

\startframe{Announcements}
\begin{itemize}
\item 
All homework must be turned in electronically from now on.
\item 
Only 2 more homeworks!  (Only best 10 out of 11 grades will count.)
\end{itemize}
\end{frame}

\startframe{Monte Carlo methods}
Recall:  Monte Carlo MLE
\begin{itemize}
\item
Given $X\sim f_\theta(x) = \exp\{\theta^\top s(x)\} / c(\theta)$,
write  log-likelihood as
\[
\ell(\theta) = \ell(\theta_0) + (\theta-\theta_0)^\top s(X) - 
\log E_{\theta_0} \exp\{( \theta-\theta_0)^\top s(Y)\}.
\]
\item Use $Y_1, \ldots, Y_m\sim f_{\theta_0}$ to approximate
$\ell(\theta)-\ell(\theta_0)$.
\end{itemize}
\end{frame}

\startframe{Monte Carlo methods}
\begin{itemize}
\item Regular importance sampling:  %With $q(x) = f_{\theta_0}(x)$,
Estimate $E_{f_{\theta_0}} g(X)f_\theta(X)/f_{\theta_0}(X)$ by
\[
\left( \sum_i\frac{g(X_i)r_\theta(X_i)}{r_{\theta_0}(X_i)} \right)
\Bigg/
\left( \sum_i \frac{r_\theta(X_i)}{r_{\theta_0}(X_i)} \right).
\]
\item
In the case of Monte Carlo MLE, $g$ is irrelevant and we only
need the denominator.
\end{itemize}
\end{frame}

\startframe{Markov chain Monte Carlo}
\begin{itemize}
\item Goal:  Estimate $\mu=E_\pi g(X)$ but
cannot %use Monte Carlo directly since we cannot 
sample $X_i\sim \pi$ directly.
\item 
Importance sampling may be difficult, particularly as dimension increases.
\item 
MCMC solution:  Take $\hat\mu = \frac1n\sum g(X_i)$, where
$X_1, X_2, \ldots$ is a simulated Markov chain with stationary distribution $\pi$.
\end{itemize}
\end{frame}

\startframe{Markov chain Monte Carlo}
Metropolis-Hastings algorithm (recall HW\#4, problem 5):
\begin{itemize}
\item Start with $X_0=x_0$.  
\item For $i=0, 1, \ldots$, 
Generate $Y \sim q(y\mid x_0)$.
\item Define $\alpha(x,y) = \min\{1, \pi(y)q(x\mid y) / [\pi(x)q(y\mid x)]$.
\item Let $X_{i+1}=Y$ with probability $\alpha(x_i,Y)$.
\end{itemize}
\end{frame}

\startframe{Markov chain Monte Carlo}
Metropolis-Hastings algorithm (recall HW\#4, problem 5):
\begin{itemize}
\item $X_{i+1} = 
\begin{cases}
Y & \mbox{with probability $\min\{1, \pi(Y)q(Y\mid X_i)/[\pi(X_i)q(X_i\mid Y)\}$} \\
X_i & \mbox{otherwise}
\end{cases}$
\item 
What are the requirements of $q(x\mid y)$?
\item 
What can we say about the 
transition probabilities $P(x,y)$?
\end{itemize}
\end{frame}

\startframe{Markov chain Monte Carlo}
Metropolis-Hastings algorithm example:
\begin{itemize}
\item Suppose we observe $Y_1, \ldots, Y_n$
\item $Y_i \mid \theta \sim N(\theta, 1)$, conditionally independent.
\item $\theta \sim \mbox{lognormal}(\mu, \sigma)$.
\item Wanted: A sample from the posterior $\pi(\theta\mid \vec Y)$.
\end{itemize}
\end{frame}

\startframe{Markov chain Monte Carlo}
Metropolis algorithm (``random walk'' M-H):
\begin{itemize}
\item Take $q$ so that $q(x\mid y)=q(y\mid x)$.
\item In this case, $\alpha(x,y) = \min\{1, \pi(y)/\pi(x)\}$.
\item Example:  Take $Y \mid X_i=x \sim N(x, \tau^2)$.
\item Consider the tradeoff involved in choosing $\tau^2$.
\end{itemize}
\end{frame}

%%%%%%%%%%%%%%%%%%%%%%%%%%%%%%%%%%%%%%%%
\end{document}




