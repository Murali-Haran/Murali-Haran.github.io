\documentclass{beamer}
\usepackage{beamerthemeshadow}
\usepackage{amssymb}
\usepackage{tikz}

\newcommand{\beaa}{\begin{eqnarray*}}
\newcommand{\eeaa}{\end{eqnarray*}}
\newcommand{\bea}{\begin{eqnarray}}
\newcommand{\eea}{\end{eqnarray}}
\def\E{\mathop{\rm E\,}\nolimits}
\def\Var{\mathop{\rm Var\,}\nolimits}
\def\Cov{\mathop{\rm Cov\,}\nolimits}
\def\Corr{\mathop{\rm Corr\,}\nolimits}
\def\logit{\mathop{\rm logit\,}\nolimits}
\newcommand{\eid}{{\stackrel{\cal{D}}{=}}}
\newcommand{\cip}{{\stackrel{{P}}{\to}}}
\def\bR{\mathbb{R}}	% real line
\def\given{\,|\,}

%%%%%%%%%%
% From http://www.disc-conference.org/disc1998/mirror/llncs.sty
\def\vec#1{\mathchoice{\mbox{\boldmath$\displaystyle\bf#1$}}
{\mbox{\boldmath$\textstyle\bf#1$}}
{\mbox{\boldmath$\scriptstyle\bf#1$}}
{\mbox{\boldmath$\scriptscriptstyle\bf#1$}}}
%%%%%%%%%%

\def\choice#1#2{
\tikz {
\def \lx{-.2} \def \ux{.2} \def \ly{-.2} \def\uy{.2} 
\filldraw[color=#1] (\lx*.9,\ly*.9) rectangle (\ux*.9,\uy*.9);
\draw(\lx,\ly) rectangle (\ux,\uy); \node at (0,0)[color=white] {\bf {#2}}; \node at (0,0) [color=black] {{#2}};
} }

\def\question#1#2#3#4#5{{#1}

\begin{minipage}{2.45in}
\begin{enumerate}
\item [{\choice{red}{A}}]\rule{-.25em}{0em}{#2}
\item [{\choice{yellow}{C}}]\rule{-.25em}{0em}{#4}
\end{enumerate}
%\choice{red}{A} {#2}
%
%\choice{yellow}{C} {#4}
\end{minipage}\begin{minipage}{2.45in}
\begin{enumerate}
\item [{\choice{green}{B}}]\rule{-.25em}{0em}{#3}
\item [{\choice{blue}{D}}]\rule{-.25em}{0em}{#5}
\end{enumerate}
%\choice{green}{B} {#3}
%
%\choice{blue}{D}  {#5}
\end{minipage}
}

%
%  Simple macro to declare and use image all at once.
%
\def\figurehere#1#2{\pgfdeclareimage[width={#2}]{{#1}}{figures/#1} \pgfuseimage{{#1}}}

\def\startframe#1{\begin{frame}[t,fragile] \frametitle{\thedate \hfill {#1}} \vspace{-1ex}}


%
%  Declare images
%
%\pgfdeclareimage[width=3in]{geyser}{figures/geyser}

\mode<presentation>
{
%  \setbeamertemplate{background canvas}
  \usetheme{boxes}
  \usecolortheme{blackscreen}

%  \usetheme{default}
%  \setbeamercovered{transparent}
}
%\beamertemplatetransparentcovereddynamic

\usepackage[english]{babel}
\usepackage[latin1]{inputenc}
\usepackage{times}
\usepackage[T1]{fontenc}

\setbeamersize{text margin left=0.1in}
\setbeamersize{text margin right=0.1in}



\begin{document}


\def \thedate{Aug. 25}


%%%%%%%%%%%%%%%%%%%%%%%%%%%%%%%%%%%%%%%%


%\startframe{Announcements}
%\begin{itemize}
%\item Undergrad stat major reception 3:30pm Friday in 328 Thomas
%\item How to get into this class:  Show up and\ldots
%\item HW\#1 will be due Sept. 1 (nothing this Friday)
%\item Small changes to syllabus
%\end{itemize}
%\end{frame}

\begin{frame}[t]
\frametitle{\thedate \hfill  Desired Outcomes from last class}
\vspace{-1ex}
Students will be able to:
\pause
\begin{itemize}
\item identify the sample space of a probability experiment;
\pause
\item define a random variable and describe simple examples of RVs;
%\pause
%\item apply set notation to simplify expressions involving events;
%\pause
%\item apply Venn diagrams to the same.
\end{itemize}
\end{frame}


\startframe{1.2\quad Properties of Probability }
\begin{itemize}
\item Event:  A subset of the sample space (technical restrictions unimportant for now)
\item Often use early-alphabet capital letters (e.g., $A, B, C$) for events.
\pause
\item Difference between an {\em outcome} and an {\em event}?
\pause
\item After an experiment, event $A$ has occurred if \ldots
\item Special random variable:  The indicator $I_A$ or $I(A)$ or $1_A$
\end{itemize}
\end{frame}

\startframe{1.2\quad Properties of Probability }
\begin{itemize}
\item Algebra of sets and Venn diagrams
	\begin{itemize}
	\item Null event and full sample space: $\emptyset$ and $S$
	\item Subset notation: $A\subset B$ or, alternatively, $B\supset A$ or $A$ implies $B$
	\item Union: $A\cup B$
	\item Intersection: $A\cap B$
	\item Complement: $A'$ or, alternatively, $A^c$
	\end{itemize}
\end{itemize}
\end{frame}

\startframe{1.2\quad Properties of Probability}

\begin{columns}
\begin{column}{4cm}

\tikz{
\def \lx{-1.4} \def \ux{2.4} \def \ly{-1.2} \def\uy{1.2}
\draw (\lx,\ly) rectangle (\ux,\uy);
% grid with dot at origin for helping place objects in rectangle:
%\draw[step=.5, very thin] (\lx,\ly) grid (\ux,\uy); \filldraw (0,0) circle (.05);
\draw (0,0) ellipse (1 and .8);
\draw (1,0) ellipse (1 and .8);
\draw (0,.9) node {\small$A$};
\draw (1,.9) node {\small$B$};
\draw (2.2,1) node{$S$};
}
\end{column}
\begin{column}{6cm}
Venn diagram depicting events $A$ and $B$.  In the picture, they appear to 
have a nonempty intersection.
\end{column}
\end{columns}
\end{frame}

\startframe{1.2\quad Properties of Probability }
Example from SOA/CAS Sample Exam P:

\vspace{2ex}
You are given $P(A\cup B)= 0.7$ and $P(A\cup B') = 0.9$.  Find $P(A)$.
\end{frame}

\startframe{1.2\quad Properties of Probability }
\begin{itemize}
\item {\em Mutually exclusive} events:  Have empty pairwise intersection
\item {\em Exhaustive} events:  Have union equal to $S$
\pause
\item {\em Partition}:  A group of mutually exclusive and exhaustive events
\end{itemize}
\end{frame}

\startframe{1.2\quad Properties of Probability }
An example:  Roll two dice, so $S$ has 36 elements or outcomes.
\begin{itemize}
\item Let $A=\{ s\in S: \mbox{sum of dice results in $s$ is even}\}$
\item \quad\ \ $B=\{\mbox{first die is even}\}$ \qquad {\em (using abbreviated notation)}
\item \quad\ \ $C=\{\mbox{second die is $\ge5$}\}$
\item \quad\ \  $D=\{\mbox{sum is prime}\}$
\end{itemize}
\pause
What is $A\cap D$? 
\pause
\qquad$(B\cup C' \cup A)'$?
\pause
\qquad$(A'\cap B)\cup (A\cap B)$?
\end{frame}

\startframe{ Appendix A.1\quad Algebra of sets }
\begin{itemize}
\item Commutative property:  $A\cup B = B\cup A$
\item Associative property:  $(A\cup B)\cup C = A\cup(B\cup C)$
\item Distributive property:  $A\cup(B\cap C) = (A\cup B)\cap(A\cup C)$
\item DeMorgan's Law: $(A\cup B)' = A' \cap B'$ or, more generally,
\[\left(\bigcup_{i=1}^n A_i\right)' = \bigcap_{i=1}^n A_i'\]
\end{itemize}
\end{frame}

\startframe{Appendix A.1\quad Algebra of sets }
\begin{itemize}
\item Commutative property:  $A\cap B = B\cap A$
\item Associative property:  $(A\cap B)\cap C = A\cap(B\cap C)$
\item Distributive property:  $A\cap(B\cup C) = (A\cap B)\cup(A\cap C)$
\item DeMorgan's Law: $(A\cap B)' = A' \cup B'$ or, more generally,
\[\left(\bigcap_{i=1}^n A_i\right)' = \bigcup_{i=1}^n A_i'\]
\end{itemize}
\end{frame}

%%% This is where I stopped on Aug. 25

\startframe{1.2\quad Properties of Probability }
\underline{Probability}:  
Real-valued set function, $P$, satisfying:
\begin{itemize}
\item $P(A)\ge0$ \qquad {\em (Nonnegativity)}
\item If $A$ is the whole sample space $S$ then $P(A)=1$
\item Whenever $A_1, A_2, \ldots$ are mutually exclusive,
\[ P\left( \bigcup_{i=1}^\infty A_i \right) = \sum_{i=1}^\infty P(A_i) 
\qquad\qquad \mbox{\em (Countable additivity)}
\]
\end{itemize}
\end{frame}

\startframe{1.2\quad Properties of Probability }
Some theorems:
	\begin{overprint}
	\onslide<1>
	\begin{itemize}
	\item[(1)] For any $A$, $P(A)=1-P(A')$.
	\end{itemize}
	\onslide<2>
	\begin{itemize}
	\item[(2)] $P(\emptyset)=0$.
	\end{itemize}
	\onslide<3>
	\begin{itemize}
	\item[(3)] If $A$ implies $B$ (i.e., $A\subset B$), then $P(A)\le P(B)$.
	\end{itemize}
	\onslide<4>
	\begin{itemize}
	\item[(4)] For any $A$, $P(A)\le1$.
	\end{itemize}
	\onslide<5>
	\begin{itemize}
	\item[(5)] For any $A$ and $B$,
	\[ P(A\cup B) = P(A) + P(B) - P(A\cap B) \]
	\end{itemize}
	\onslide<6>
	\begin{itemize}
	\item[(5$^*$)] For any $A$ and $B$ and $C$,
	\beaa
	P(A\cup B \cup C) &=& P(A) + P(B) + P(C)  \\
	&&- P(A\cap B) - P(A\cap C) - P(B\cap C) \\
	&& + P(A\cap B\cap C)
	\eeaa
	\end{itemize}
	\onslide<7>
	\begin{itemize}
	\item[(5$^{**}$)] For any $A_1, \ldots, A_n$,
	\beaa 
	P\left(\bigcup_{i=1}^n A_i \right) &=& 
	\sum_{i=1}^n P(A_i) - \mathop{\sum\sum}_{i<j} P(A_i\cap A_j) + \cdots \\
	&&+(-1)^{n+1} P(A_1\cap \cdots \cap A_n)
	\eeaa
	\end{itemize}
	\end{overprint}
\end{frame}


\startframe{1.2\quad Properties of Probability }
Question from SOA/CAS Sample Exam P:

\vspace{2ex}
A survey of a group's viewing habits of gymnastics (G), baseball (B), and
soccer (S) revealed that:  


\qquad 28\% watched 
G
\qquad 29\% watched B \qquad 
19\% watched S

\qquad\qquad14\% watched G and B \qquad
12\% watched B and S \qquad

\qquad 10\% watched G and S
\qquad 8\% watched all three sports

What percentage of the group watched none of the three sports?
\end{frame}

\startframe{1.2\quad Properties of Probability }
Very important special case: Outcomes in $S$ are equally likely.
\pause
\begin{itemize}
\item In this case, for all $A$, $\displaystyle P(A) = \frac{|A|}{|S|}$
\item Counting the elements in $A$ and $S$ becomes vitally important!
\item Combinatorics:  ``Counting without counting''
\end{itemize}
\end{frame}

\startframe{Desired Outcomes}
Students will be able to:
\begin{itemize}
\item define a random variable and describe simple examples of RVs;
\pause
\item apply set notation to simplify expressions involving events;
\pause
\item apply Venn diagrams to the same;
%%% This is where I stopped on Aug. 25
\pause
\item prove simple facts based on the three axioms of probability;
\pause
\item use set notation or Venn diagrams for certain probability problems.
\end{itemize}
\end{frame}




%%%%%%%%%%%%%%%%%%%%%%%%%%%%%%%%%%%%%%%%
\end{document}


