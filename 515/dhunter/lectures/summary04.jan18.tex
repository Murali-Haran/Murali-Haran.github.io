\documentclass[handout]{beamer}
\usepackage{beamerthemeshadow}
\usepackage{amssymb}
\usepackage{pgfpages}
\usepackage{tikz}
\pgfpagesuselayout{4 on 1}%[landscape]

\newcommand{\beaa}{\begin{eqnarray*}}
\newcommand{\eeaa}{\end{eqnarray*}}
\newcommand{\bea}{\begin{eqnarray}}
\newcommand{\eea}{\end{eqnarray}}
\def\E{\mathop{\rm E\,}\nolimits}
\def\Var{\mathop{\rm Var\,}\nolimits}
\def\Cov{\mathop{\rm Cov\,}\nolimits}
\def\Corr{\mathop{\rm Corr\,}\nolimits}
\def\logit{\mathop{\rm logit\,}\nolimits}
\newcommand{\eid}{{\stackrel{\cal{D}}{=}}}
\newcommand{\cip}{{\stackrel{{P}}{\to}}}
\def\bR{\mathbb{R}}	% real line
\def\given{\,|\,}

%%%%%%%%%%
% From http://www.disc-conference.org/disc1998/mirror/llncs.sty
\def\vec#1{\mathchoice{\mbox{\boldmath$\displaystyle\bf#1$}}
{\mbox{\boldmath$\textstyle\bf#1$}}
{\mbox{\boldmath$\scriptstyle\bf#1$}}
{\mbox{\boldmath$\scriptscriptstyle\bf#1$}}}
%%%%%%%%%%

\def\choice#1#2
{\tikz 
{\def \lx{-.2} \def \ux{.2} \def \ly{-.2} \def\uy{.2} \filldraw[color=#1] (\lx*.9,\ly*.9) rectangle (\ux*.9,\uy*.9);
\draw(\lx,\ly) rectangle (\ux,\uy); \node at (0,0)[color=white] {\bf {#2}}; \node at (0,0) [color=black] {{#2}};}}

\def\question#1#2#3#4#5{{#1}

\begin{minipage}{2.4in}
\choice{red}{A} {#2}

\choice{yellow}{C} {#4}
\end{minipage}\begin{minipage}{2.4in}
\choice{green}{B} {#3}

\choice{blue}{D}  {#5}
\end{minipage}
}

\def\notes#1{

\vspace{2ex}
{\em Notes: {#1}}}
%
%  Simple macro to declare and use image all at once.
%
\def\figurehere#1#2{\pgfdeclareimage[width={#2}]{{#1}}{figures/#1} \pgfuseimage{{#1}}}

\def\startframe#1{\begin{frame}[t,fragile] \frametitle{\thedate \hfill {#1}} }

%
%  Declare images
%
%\pgfdeclareimage[width=3in]{geyser}{figures/geyser}

\mode<presentation>
{
%  \setbeamertemplate{background canvas}
  \usetheme{boxes}
  \usecolortheme{notblackscreen}

%  \usetheme{default}
%  \setbeamercovered{transparent}
}
%\beamertemplatetransparentcovereddynamic

\usepackage[english]{babel}
\usepackage[latin1]{inputenc}
\usepackage{times}
\usepackage[T1]{fontenc}

\setbeamersize{text margin left=0.1in}
\setbeamersize{text margin right=0.1in}



\begin{document}


\def \thedate{Jan.~18}


%%%%%%%%%%%%%%%%%%%%%%%%%%%%%%%%%%%%%%%%

\startframe{Announcements}
\begin{itemize}
\item 
Read Section 4.1  before Friday's class.
\item
Office hours tomorrow will be 12:45 to 2:45 due to a comprehensive exam I have
to attend. 
\item There is a new dropbox that you can use to turn in homework, if you wish.
\end{itemize}
\notes{Reminder:  You are not required (yet!) to write up your homework assignments
electronically.}
\end{frame}

\startframe{3.4 Computing Expectations by Conditioning}
\question{If $X\sim \mbox{Geom}(p)$, then $E(X) = $?}
{$\frac1p$}
{$\frac1{p^2}$}
{$\frac1{1-p}$}
{$\frac1{(1-p)^2}$}
\notes{Answer:  A.  We discussed how this could be computed directly, though
today in class we'll use a different approach based on conditioning.}
\end{frame}

\startframe{3.4 Computing Expectations by Conditioning}
\begin{itemize}
\item Perform repeated independent Bernoulli$(p)$ experiments.
\item Let $N_k$ be the trial number where $k$ consecutive successes are first observed.
\item What is $E(N_k)$?
\end{itemize}
\notes{The answer may be found by conditioning on $N_{k-1}$, and also on the 
outcome of the $(N_{k-1}+1)$th experiment.}
\end{frame}

\startframe{3.4 Computing Expectations by Conditioning}
\begin{itemize}
\item Suppose that adult males' heights have mean $\mu_M$ and variance
$\sigma^2_M$.
\item Similarly, females' heights have mean $\mu_F$ and variance $\sigma^2_F$.
\item If some proportion $p$ of the population is female, what is the variance of
the height of an adult selected at random from the population?
\end{itemize}
\notes{The answer is a straightforward application of the 
conditioning formula for the variance.}
\end{frame}

\startframe{3.4 Computing Expectations by Conditioning}
\question{$\Var(X)$ equals which of the following?}
{$E \left[ E(X^2 \mid Y) \right]$}
{$E \left[ \Var(X \mid Y) \right]$}
{$\Var \left[ E(X \mid Y) \right]$}
{$E \left[ \Var(X \mid Y) \right] + \Var\left[ E(X\mid Y) \right]$}
\notes{Answer:  D.  This is not hard to remember; it's worth memorizing.}
\end{frame}

\startframe{3.5 Computing Probabilities by Conditioning}
\begin{columns}
\begin{column}{1.25in}
\tikz{
\def \lx{-1.4} \def \ux{2.4} \def \ly{-1.2} \def\uy{1.2}
%\draw (\lx,\ly) rectangle (\ux,\uy);
% grid with dot at origin for helping place objects in rectangle:
%\draw[step=.5, very thin] (\lx,\ly) grid (\ux,\uy); \filldraw (0,0) circle (.05);
%
%\draw (-1.2, 1) node {(c)};
\filldraw (-1,0) circle (.05);
\filldraw (2,0) circle (.05);
\draw (-1,.2) node {$A$};
\draw (2,.2) node {$B$};
\draw (-1,0) -- (-.75,0) -- (-.75, .75) -- (1.75,.75) -- (1.75,0) -- (2,0);
\draw (-1,0) -- (-.75,0) -- (-.75, -.75) -- (1.75,-.75) -- (1.75,0) -- (2,0);
\draw (.125,.75) -- (.125,0) -- (.875, 0) -- (.875, -.75);
%
\foreach \x / \y / \label  in {-.375 / .75 / 1, -.375 / -.75 / 2, 1.125 / .75 / 3, 1.125 / -.75 / 4, .375 / 0 / 5} 
  {\filldraw (\x, \y) circle (.05); \draw (\x, \y) -- (\x + .25,\y + .25); 
   \draw[white, thick] (\x + .05, \y) -- (\x + .25, \y); \draw (\x + .15, \y - .2) node {\small \label};
  }
}
\end{column}
\begin{column}{3.5in}
\begin{itemize}
\item
Switch $i$ is closed (i.e., current will flow) with probability $p_i$,
independently of others.
\item What is the probability that current will flow from $A$ to $B$?
\end{itemize}
\end{column}
\end{columns}
\notes{We derived the answer by conditioning on $I\{\mbox{switch 5 is closed}\}$.}
\end{frame}

\startframe{3.5 Computing Probabilities by Conditioning}
\begin{itemize}
\item
At a party, $n$ men take off their hats, which are then mixed up.  
\item 
Each man randomly selects one of the hats.
\item 
What is the probability that nobody gets his own hat?
\end{itemize}
\notes{This is a famous problem that is considered in chapters 2 and 3 in the textbook.
In class, we watched a simulation using cards that is equivalent to the hat problem with
$n=52$.  I will eventually make that R code available.}
\end{frame}

\startframe{3.5 Computing Probabilities by Conditioning}
\begin{itemize}
\item 
Two players alternate flipping a coin that comes up heads with probability $p$.
\item
The first one to get heads is the winner.
\item 
What is the probability that the first player is the winner?
\end{itemize}
\notes{We ran out of time before tackling this question, but we did establish
two things:  First, as $p\to 1$, we expect the probability to tend to 1.  Second, 
as $p\to 0$, we expect the probability to tend to $1/2$.  If you work on this question,
make sure that your answer fulfills these two criteria.}
\end{frame}

%\startframe{1.2\quad Properties of Probability}
%
%\begin{columns}
%\begin{column}{4cm}
%
%\tikz{
%\def \lx{-1.4} \def \ux{2.4} \def \ly{-1.2} \def\uy{1.2}
%\draw (\lx,\ly) rectangle (\ux,\uy);
%% grid with dot at origin for helping place objects in rectangle:
%%\draw[step=.5, very thin] (\lx,\ly) grid (\ux,\uy); \filldraw (0,0) circle (.05);
%\draw (0,0) ellipse (1 and .8);
%\draw (1,0) ellipse (1 and .8);
%\draw (0,.9) node {\small$A$};
%\draw (1,.9) node {\small$B$};
%\draw (2.2,1) node{$S$};
%}
%\end{column}
%\begin{column}{6cm}
%Venn diagram depicting events $A$ and $B$.  In the picture, they appear to 
%have a nonempty intersection.
%\end{column}
%\end{columns}
%\end{frame}





%%%%%%%%%%%%%%%%%%%%%%%%%%%%%%%%%%%%%%%%
\end{document}


