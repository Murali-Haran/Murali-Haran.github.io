\documentclass{article}

\setlength{\topmargin}{-.75in}
\setlength{\oddsidemargin}{.0in}
\setlength{\evensidemargin}{.0in}
\setlength{\textheight}{9.5in}
\setlength{\textwidth}{6.5in}
\setlength{\parindent}{0in}
%\parskip=.125in

\usepackage{amsmath,bm}%
\usepackage{amsfonts}%


\newcommand{\beaa}{\begin{eqnarray*}}
\newcommand{\eeaa}{\end{eqnarray*}}
\newcommand{\bea}{\begin{eqnarray}}
\newcommand{\eea}{\end{eqnarray}}
\newcommand{\svskip}{\vspace{.2in}}
\newcommand{\mvskip}{\vspace{.25in}}
\newcommand{\lvskip}{\vspace{.5in}}
\def\E{\mathop{\rm E\,}\nolimits}
\def\Var{\mathop{\rm Var\,}\nolimits}
\def\Cov{\mathop{\rm Cov\,}\nolimits}
\def\Cor{\mathop{\rm Corr\,}\nolimits}
\def\Tr{\mathop{\rm Tr\,}\nolimits}
\def\diag{\mathop{\rm diag\,}\nolimits}
\def\midd{\mathop{\,|\,}\nolimits}
\def\cip{\mathop{\stackrel{P}{\rightarrow}}\nolimits}
\def\cid{\mathop{\stackrel{d}{\rightarrow}}\nolimits}
\def\ciqm{\mathop{\stackrel{\mbox{\scriptsize qm}}{\rightarrow}}\nolimits}
\def\defn{{\stackrel{\mbox{\scriptsize def}}{=}}}
\def\eid{{\stackrel{{\cal D}}{=}}}
\def\rvseq{\mathop{X_1, X_2, \ldots}\nolimits}
\def\rvseqn{\mathop{X_1, \ldots, X_n}\nolimits}
\def\u#1{{\underline{#1}}}
\def\o#1{{\overline{#1}}}
\def\n#1{^{(#1)}}
\newcommand{\qed}{\rule{2mm}{2mm}}


\def\cas{\mathop{\stackrel{\mbox{\scriptsize as}}{\rightarrow}}\nolimits}

\pagestyle{empty}

%%-------------------------------------------------------------------

\begin{document}
        \hrule
        \begin{center}
        \Large\bf Stat 515: Stochastic Processes I \hfill Spring 2012\\
        Midterm  \hfill WITH SOLUTIONS
        \end{center}
        \hrule

%\mvskip {\bf Name:  \rule{4in}{.01in}}

\mvskip 
This midterm is worth 20 points.  You have 60 minutes.  
{\bf For full credit, you must explain all of your work!}
Naturally, you may use any results that you know.

\mvskip 
{\bf Problem 1. [8 points]\ } 
A Markov chain $\{X_t: t=0, 1, \ldots\}$ with state space $\{0, 1, 2\}$ has transition probability matrix
\[
P=
\begin{bmatrix}
%\frac13 & \frac13 & \frac13 \\ 
%\frac13 & \frac23 & 0 \\
%0 & 1 & 0
1/3 & 1/3 & 1/3 \\ 
1/3 & 2/3 & 0 \\
0 & 1 & 0
\end{bmatrix}.
\]

\svskip
{\bf(a) [2 points]\ }
Suppose $P(X_0=0) = P(X_0=2) = \frac12$.  Find $E(X_2)$.  Show all work.
  \begin{quotation}{\bf Solution:}
  \[
  \begin{bmatrix}
  \frac12 & 0 & \frac12
  \end{bmatrix}
  \times
  \begin{bmatrix}
  1/3 & 1/3 & 1/3 \\ 
  1/3 & 2/3 & 0 \\
  0 & 1 & 0
  \end{bmatrix}
  \times
  \begin{bmatrix}
  1/3 & 1/3 & 1/3 \\  
  1/3 & 2/3 & 0 \\
  0 & 1 & 0
  \end{bmatrix} =
  \begin{bmatrix}
  \frac16 & \frac23 & \frac16
  \end{bmatrix}
  \times
  \begin{bmatrix}
  1/3 & 1/3 & 1/3 \\ 
  1/3 & 2/3 & 0 \\
  0 & 1 & 0
  \end{bmatrix}
  =
  \begin{bmatrix}
  \frac{5}{18} & \frac{12}{18} & \frac{1}{18}
  \end{bmatrix}.
  \]
  This is the distribution on the three states $\{0, 1, 2\}$ after two steps of the chain.
  Therefore, $E(X_2) = (12/18) + 2\times(1/18) = 7/9$.
  \end{quotation}
  
\svskip 
{\bf(b) [2 points]\ }  
Straightforward calculation shows that
$P^3$ consists only of nonzero entries.
%\[
%P^3= \frac{1}{27} \times
%\begin{bmatrix}
%8 & 17 & 2 \\ 
%8 & 16 & 3 \\
%9 & 15 & 3
%\end{bmatrix}.
%\]
Using this fact, explain how the Chapman-Kolmogorov equations (the equations that relate
$P_{ij}^{n+m}$ to $P^n$ and $P^m$ for each $i$ and $j$) imply that
 $P^n$ consists only of nonzero entries for $n\ge 3$.
  \begin{quotation}{\bf Solution:}
  {\em NB:  Fixed mistake in original solution.}
  For any $n\ge 3$, the Chapman-Kolmogorov equations state that
  for all $i$ and $j$,
  \[
  P_{ij}^{n} = \sum_{k=0}^2 P_{ik}^{n-3}P_{kj}^3.  
  \]
  Since the $P_{ik}^{n-3}$ values sum to one, at least one of them must be nonzero.  Since each of the 
  $P_{kj}^3$ values is nonzero, this means that the sum is nonzero, which is what we had to prove.
  \end{quotation}
  

\svskip 
{\bf(c) [2 points]\ }  
Prove that in the long run, no matter which state the chain starts in, the number of steps 
that the chain spends in states 0, 1, and 2, respectively, will approach the
ratios $3:6:1$.
  \begin{quotation}{\bf Solution:}
  Since all entries in $P^3$ are positive, this chain is ergodic (because 
  all states communicate and are 
  aperiodic and there are only finitely many of them).  Therefore, it suffices to 
  check that
  $\pi^\top P=\pi^\top$, where $\pi$ has its entries in the ratio $3:6:1$,
  since this will imply that $\pi$ is the unique stationary (and limiting)
  probability vector.   We find immediately that
  \[
  \begin{bmatrix}
  \frac3{10} & \frac6{10} & \frac1{10}
  \end{bmatrix}
  \times
  \begin{bmatrix}
  1/3 & 1/3 & 1/3 \\ 
  1/3 & 2/3 & 0 \\
  0 & 1 & 0
  \end{bmatrix}
  =
  \begin{bmatrix}
  \frac3{10} & \frac6{10} & \frac1{10}
  \end{bmatrix}.
  \]
  \end{quotation}

\svskip 
{\bf(d) [2 points]\ }  
Is this Markov chain time-reversible? Explain your answer.
  \begin{quotation}{\bf Solution:}
  Detailed balance is a necessary condition for time-reversibility, and in this case
  detailed balance is obviously not satisfied by $\pi$ of part (c), since for example
  $P_{13}=0$ but $P_{31}\ne 0$, which means that
  \[
  \pi_1P_{13} \ne \pi_3P_{31}.
  \]
  We conclude that the chain is not time-reversible.
  \end{quotation}

\mvskip 
{\bf Problem 2.  [4 points]\ } 
A Markov chain has transition probability matrix
\[
P=
\begin{bmatrix}
0 & 0.5 & 0 & 0 & 0 & 0.5 \\
0 & 0 & 1 & 0 & 0 & 0 \\
0 & 0 & 0 & 1 & 0 & 0 \\
1 & 0 & 0 & 0 & 0 & 0 \\
0 & 0 & 0 & 0 & 0 & 1 \\
0.5 & 0 & 0 & 0 & 0.5 & 0 \\
\end{bmatrix}.
\]

\svskip 
{\bf(a) [2 points]\ }  
Classify each state as transient, null recurrent, or positive recurrent.  Explain.
  \begin{quotation}{\bf Solution:}
  By inspection, we find that it is possible to go from 
  $1\to2\to3\to4\to1$ and from $1\to6\to5\to6\to1$.  Therefore, all states communicate.
  Therefore they all have the same classification, and because there are only finitely
  many of them, they cannot all be transient nor null recurrent.  We conclude that
  all states are positive recurrent.
  \end{quotation}

\svskip 
{\bf(b) [2 points]\ }  
Is the chain ergodic?  Explain.
  \begin{quotation}{\bf Solution:}
  If the chain starts at time zero in, say, state 1, then states $\{1, 3, 6\}$ can only be 
  visited at even times, whereas $\{2, 4, 5\}$ can only be visited at odd times.
  This means that the chain is periodic, so it is not ergodic.
  \end{quotation}

\mvskip
{\bf Problem 3.  [8 points]\ } 
Buses arrive at a certain bus stop according to a Poisson process with rate
2 per hour.  
%The number of hours between successive bus arrivals at a certain stop is uniformly 
%distributed on $(0,1)$.  
Passengers arrive according to an independent Poisson process with rate
10 per hour.  The instant a bus arrives, all passengers at the stop at that instant board the
bus and the bus departs.  

\svskip
{\em Fact:  An exponential random variable with rate $\lambda$ has
mean $1/\lambda$ and variance $1/\lambda^2$.}

\mvskip
{\bf(a) [2 points]\ }  
Assume that there are currently no passengers at the bus stop.
What is the probability that the next bus will pick up no passengers?  Explain.
  \begin{quotation}{\bf Solution:}
  The sum of the two Poisson processes has rate $10+2=12$, and each event in 
  the combined process is a bus with probability $2/12$.  So the probability
  that the next event is a bus (which is equivalent to the next bus picking up no passengers)
  is 2/12.
  \end{quotation}

\svskip
{\bf(b) [2 points]\ }  
If you arrive at the stop at noon, what is the expected amount of time you will have to wait until
the next arrival of any type (bus or passenger)?  Explain.
  \begin{quotation}{\bf Solution:}
  The rate of the combined process is 12, so we have to wait on average 1/12 hour, or 5 minutes.
  By the memoryless property, we do not have to know how long before noon the last event happened.
  \end{quotation}

\svskip
{\bf(c) [2 points]\ }  
At 2:00, there are 2 passengers waiting for the bus.
Given this information, what is the expected arrival time of the next bus after 2:00?
Explain.
  \begin{quotation}{\bf Solution:}
  The next bus will arrive in 1/2 hour on average, or at 2:30.  By the memoryless property,
  the information about the number of passengers present at 2:00 is irrelevant.
  \end{quotation}

\svskip 
{\bf(d) [2 points]\ }  
Assume that there are currently no passengers at the bus stop.
Let $X$ be the number of people at the stop when the next bus arrives.
Find $E(X)$ and $\Var(X)$, showing all your work.
  \begin{quotation}{\bf Solution:}
  Let $T$ be the time (in hours from now) when the next bus will arrive.  Then
  $T$ is exponential with mean $1/2$ and variance $1/4$.  Also, 
  given $T$, $X$ is Poisson with mean $10T$, which means that
  $E(X\mid T)=\Var(X\mid T)=10T$.
  Therefore, conditioning on $T$ gives
  \begin{eqnarray*}
  E(X) &=& E[ E(X\mid T) ] = E[ 10T] = 10 E[T] = 5 \\
  \Var(X) &=& E[ \Var(X\mid T)] + \Var[E(X \mid T)] = 
  E[10T] + \Var[10T] = 10E[T] + 100\Var[T] = 5+25=30.
  \end{eqnarray*}
  \end{quotation}


\end{document}
