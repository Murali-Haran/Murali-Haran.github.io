\documentclass[12pt]{article}
\usepackage[sort,longnamesfirst]{natbib}
\usepackage{amsbsy,amsmath,amsthm,amssymb,graphicx}
\usepackage{subfigure}
\usepackage{geometry}
\geometry{hmargin=2.5cm,vmargin={2.5cm,2.5cm},footskip=0.5in}
\renewcommand{\baselinestretch}{1.25}
\setlength{\baselineskip}{0.3in} \setlength{\parskip}{.05in}

\begin{document}

\title{A Silly LaTeX document}
%\date{}
\maketitle
\section{Introduction}\label{sec:intro}
This is my introduction to blahdiblah which I will tell you about in the main section.

\section{Main Work}\label{sec:main}
There is where I woof about everything I did, including talking about symbols like $\alpha,\beta, \theta$ and writing pointless equations like: 
\begin{equation}\label{eqn:woof} 
E_P f = \frac{E_H\{\sum_{t=0}^Lf(X_t)
  I_{N_t=1}\}} {E_H\{\sum_{t=0}^L I_{N_t=1}\}}.  
\end{equation} 
Later on I can refer to this equation by saying, why don't you look at
equation (\ref{eqn:woof}) which comes right after Section
(\ref{sec:intro}). I can also confuse you with an illustrative figure
of my research results.

Sometimes, you may decide to create an equation without a label, which can either be done as: 
$$ x^2 = y^2 + z^2$$ or 
\begin{equation*} 
  x^2 = y^2 + z^2 
\end{equation*} and occasionally you may decide to have multiple lines of equations such as: 
\begin{equation}
  \begin{split}
    P(X \leq x | X \mbox{accepted}) = & P\left(X \leq x | U \leq \frac{h(X)}{K q(X)}\right)\\
  = & \frac{P\left(X \leq x, U \leq \frac{h(X)}{K q(X)}\right)}{P(U \leq \frac{h(X)}{K q(X)})}\\
  = & \frac{E_q\left(P(X \leq x, U \leq \frac{h(X)}{K q(X)}| X\right)}{E_q\left(P(U \leq \frac{h(X)}{K q(X)} | X\right)}
  \end{split}
\end{equation}

\begin{figure}\label{fig:foo}
\begin{center}
  \rotatebox{0}{\includegraphics[height=3.5in,width=3.0in]{southpark.jpg}}
\end{center}
\end{figure}

Later on, I can refer to the figure as Figure \ref{fig:foo}. 
% You can comment out lines by putting a percent sign in front of the lines.
Sometimes it is also nice to summarize results into a table such as Table \ref{tab:results} below.
\begin{table}
  \caption{\label{tab:results} My Results}
  \centering
    \begin{tabular}{|c|c|c|c|c|}
      \hline
      data set
%      & \multicolumn{2}{c|} {samples/sec} & \multicolumn{2}{c|} {acceptance rates}\\
%      \cline{2-5}
      & perfect & rejection & perfect & rejection\\
      \hline
      breast cancer      & 0.91 & 0.90 & 0.009 & 0.008\\
      colo-rectal cancer & 4.3 & 4.5 & 0.043 & 0.043\\
      \hline
    \end{tabular}
\end{table}

\section{Summary}\label{sec:model}
I told you everything in Section \ref{sec:main}. If that makes no sense, refer to (add reference).

\bibliographystyle{apalike} 
%\bibliography{samplebib}
\end{document}
