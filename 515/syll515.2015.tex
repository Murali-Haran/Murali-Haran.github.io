\documentclass[11pt]{article}
\usepackage[sort,longnamesfirst]{natbib}
%\usepackage{psfig}
\usepackage{amsmath}%
\usepackage{hyperref}
\begin{document}
%{\bf TENTATIVE SCHEDULE}\\
\pagestyle{empty}
\vspace{-0.5in}
\Large
\begin{center}
{\bf  Syllabus for Stat515 (Spring 2015)}\\
{\bf  Stochastic Processes and Monte Carlo Methods}\\
\end{center}
\normalsize {\bf Instructor}: Murali Haran\\ Office: 421B Thomas
Building \:Phone: 863-8126 \: email: {\tt mharan}\footnote{@stat.psu.edu}\\ Office
Hours: Monday: 3:30-4:30, Tuesday: 2:30-3:30\\\\ 
{\bf Teaching Assistant}: Yawen Guan \\ Office: 330B Thomas Building \:Phone: 863-2314
\: email: {\tt yig5031}\footnote{@psu.edu}\\  Office
Hours: Tuesdays 9:30-10:30am, Thursdays: 10:00am-11:00am.\\\\ %Tue: 1-3pm.\\\\
% email:\:\:\\ Office Hours: TBA\\\\ 
{\bf Class Times}: MWF 2:30-3:20 in 009 Life Sciences\\\\
{\bf Textbook}: {\it Introduction to Probability Models}, Sheldon
Ross, {\bf 8th/9th/10th/11th} Edition, Academic Press. Plus notes for
the simulation part of the course. \\Suggested reference for Monte Carlo: {\it Introducing Monte Carlo Methods with R}, Robert, Christian P. and Casella, George (2010), Springer.\\\\
%{\bf Important}: The course schedule may change according to how the class progresses. Check the website for all announcements regularly.\\\\
{\bf Targeted Coverage}:
\begin{itemize}
\item Review of conditional probability and expectations (Chapter 3)
\item Markov chains (Chapter 4)
\item Poisson processes (Chapter 5)
\item Continuous time Markov chains (Chapter 6) 
\item Brownian motion, martingale basics (Chapter 10/other) 
\item Classical Monte Carlo: rejection, importance sampling
\item Markov chain Monte Carlo: Gibbs, Metropolis-Hastings algorithms
  % \item This course is also oriented towards providing students with
  %   some basic useful computing skills, programming habits and the
  %   ability to write up short reports on simulations.
\end{itemize}
{\bf Course Requirements:}
\begin{itemize}
%\item  Assigned Readings
\item Weekly homework (30\%). You may discuss them but they must be written up independently.
\item Midterm exam in early March (25\%).
\item Final exam: in class (30\%) + take home (15\%) (total: 45\%)
\end{itemize} 
{\bf Course Website}: \url{http://www.stat.psu.edu/~mharan/515/515.html}\\\\
{\bf Academic Integrity:} All Penn State and Eberly
College of Science policies regarding
academic integrity apply to this course.  Please see\\
{\tt http://www.science.psu.edu/academic/Integrity/index.html}.
\\\\
\noindent {\bf Policy on Late Homework:}\\
Unless you inform me ahead of time ({\it at least 1 day in advance}),
you will lose 50\% of the grade for late homework turned in within 24
hours of the due date. You will receive no credit for homework turned
in after that.\\\\
\noindent {\bf Important}: Check the website for all announcements regularly.\\\\
\noindent {\bf Computing:}\\
All {\it graduate students in statistics} are required to use the statistical
computing language {\tt R} through most of the latter half of the
course. If you are not a statistics graduate student, you are still
encouraged to write programs in {\tt R} since it is widely used. You
are free to use any other language you like but please note that I may
not be able to help you with your computing if the language is
unfamiliar to me.
\\\\
{\bf LaTeX:}\\
All {\it graduate students in statistics} are required to use {\tt LaTeX} to
write up their computing assignments in the latter half of the course.
All other students are expected to hand in typed computing assignments
even if they choose not to use {\tt LaTeX}.
\end{document}
